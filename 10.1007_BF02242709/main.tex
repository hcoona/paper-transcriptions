\PassOptionsToPackage{unicode=true}{hyperref} % options for packages loaded elsewhere
\PassOptionsToPackage{hyphens}{url}
\documentclass[a4paper,11pt,notitlepage,twoside,openright]{article}

\usepackage{ifxetex}
\ifxetex
\else
\errmessage{Must be built with XeLaTeX}
\fi

\usepackage{amssymb,amsmath}
\usepackage{fourier}
\usepackage{inconsolata}
\usepackage{enumitem}
\usepackage{footnote}

% Table
\usepackage{tabu}
\usepackage{longtable}
\usepackage{booktabs}
\usepackage{multirow}

% Verbatim & Source code
\usepackage{fancyvrb}
\usepackage{minted}

% Beauty
\usepackage[protrusion]{microtype}
\usepackage[all]{nowidow}
\usepackage{upquote}
\usepackage{parskip}
\usepackage[strict]{changepage}

\usepackage{hyperref}

% Graph
\usepackage{graphicx}
\usepackage{grffile}
\usepackage{tikz}


\hypersetup{
  bookmarksnumbered,
  pdfborder={0 0 0},
  pdfpagemode=UseNone,
  pdfstartview=FitH,
  breaklinks=true}
\urlstyle{same}  % don't use monospace font for urls

\usetikzlibrary{arrows.meta,shapes.geometric,shapes.misc}

\newminted{java}{%
  autogobble,
  breakbytokenanywhere,
  breaklines,
  fontsize=\footnotesize,
}
\newmintinline{java}{%
  autogobble,
  breakbytokenanywhere,
  breaklines,
  fontsize=\footnotesize,
}

\makeatletter
\def\maxwidth{\ifdim\Gin@nat@width>\linewidth\linewidth\else\Gin@nat@width\fi}
\def\maxheight{\ifdim\Gin@nat@height>\textheight\textheight\else\Gin@nat@height\fi}
\makeatother

% Scale images if necessary, so that they will not overflow the page
% margins by default, and it is still possible to overwrite the defaults
% using explicit options in \includegraphics[width, height, ...]{}
\setkeys{Gin}{width=\maxwidth,height=\maxheight,keepaspectratio}
\setlength{\emergencystretch}{3em}  % prevent overfull lines
\setcounter{secnumdepth}{3}

% Redefines (sub)paragraphs to behave more like sections
\ifx\paragraph\undefined\else
\let\oldparagraph\paragraph
\renewcommand{\paragraph}[1]{\oldparagraph{#1}\mbox{}}
\fi
\ifx\subparagraph\undefined\else
\let\oldsubparagraph\subparagraph
\renewcommand{\subparagraph}[1]{\oldsubparagraph{#1}\mbox{}}
\fi

% set default figure placement to htbp
\makeatletter
\def\fps@figure{htbp}
\makeatother


\title{Practical Uses of Synchronized Clocks in Distributed Systems}
\author{Barbara Liskov\\%
MIT Laboratory for Computer Science\\
Cambridge, MA 02139%
}
\date{%
Received June 1991\\
Accepted January 1993%
}

\begin{document}
\maketitle

\begin{abstract}
  Synchronized clocks are interesting because they can be used to improve
  performance of a distributed system by reducing communication. Since
  they have only recently become a reality in distributed systems, their
  use in distributed algorithms has received relatively little attention.
  This paper discusses a number of distributed algorithms that make use of
  synchronized clocks and analyzes how clocks are used in these
  algorithms.
\end{abstract}

\hypertarget{introduction}{%
\section{Introduction}\label{introduction}}

Synchronized clocks are quickly becoming a reality in distributed
systems. For example, the network time protocol NTP {[}14{]} synchronizes
clocks of nodes on geographically distributed networks. It does this at
low cost and provides clocks that are synchronized to within a few
milliseconds of one another. NTP is running on the internet today and is
used to synchronize clocks of nodes throughout the United States,
Canada, and various places in Europe.

Synchronized clocks are interesting because they can be used to improve
the performance of distributed algorithms. They make it possible to
replace communication with local computation. Instead of node N asking
another node M whether some property holds, it can deduce the answer
based on some information about M from the past together with the
current time on N' s clock.

Since the practical availability of synchronized clocks is a recent
phenomenon, their use in distributed algorithms has not received much
attention. This paper describes the role of synchronized clocks in
several distributed algorithms. The focus is on practical algorithms
that either are in use in systems today or that will be used in the near
future. The algorithms differ in their synchronization requirements;
some require clocks synchronized to within a few minutes of one another,
while others require closer synchronization. All of them have much less
stringent requirements on synchronization than current clock algorithms
provide.

There is a considerable literature on clock synchronization algorithms
{[}21{]}. It is not the goal of this paper to explain how clock
synchronization works; instead the paper assumes the clocks exist and
discusses how to use them. The ability of NTP to synchronize clocks in
the internet with small clock skews and low cost is taken as evidence
that relying on synchronized clocks in distributed algorithms is a
reasonable thing to do, both in local area networks and geographically
distributed networks.

Clock synchronization algorithms are based on probabilistic assumptions
about clock rate and message delay. Therefore, clocks are only
synchronized with some (very high) probability. Since clock
synchronization can fail occasionally, it is most desirable for
algorithms to depend on synchronization for performance but not for
correctness. Depending on synchronization for performance is reasonable;
since clocks will be synchronized most of the time, performance will
only degrade rarely. Some of the algorithms to be discussed depend on
synchronization only for performance, but others depend on it for
correctness. Depending on synchronization for correctness is more
problematic but it is sometimes appropriate. In practical systems,
performance is very important. Furthermore, the correctness of an
algorithm may depend on the non-occurrence of other low-probability
events, so that having it also depend on synchronized clocks has little
impact. Also there may be recovery mechanisms at a higher level to
compensate for failures of the algorithm.

The remainder of the paper is organized as follows. It begins in Section
2 with a few remarks about synchronized clocks. Then it describes
several distributed algorithms that use synchronized clocks. It
concludes with a discussion of how to incorporate synchronized clocks
into new algorithms.

\hypertarget{synchronized-clocks}{%
\section{Synchronized Clocks}\label{synchronized-clocks}}

Clock synchronization algorithms synchronize clocks with some skew e:
They guarantee that if cl and c2 are the clocks at two nodes of a
network, then at any instant the time at cl differs from the time at c2
by no more than e. As mentioned, the synchronization property cannot be
provided absolutely, but only with some very high probability.


It is worth noting that practical clock synchronization algorithms must
provide efficient engineering solutions to a number of problems. Some of
these are technical problems, e.g., how to avoid being misled about the
time when a message containing a time value is delayed in the network.
The algorithms that exist today are robust in the face of problems such
as network congestion and links with widely varying delays. They are
less likely to be robust, however, in the face of operator and software
errors. Most systems allow a manual override to set the clock but such
an override clearly allows the clock to be set incorrectly, i.e., in a
way that violates the clock skew assumption. Clock synchronization
algorithms need to be thought of as part of a total system, and care
must be taken to limit the damage caused by operator error. Furthermore,
the algorithm must accomplish its task without consuming much of the
bandwidth of the network and without requiring that every node be
equipped with expensive devices.

A property closely related to synchronized clocks is synchronized clock
rates. Several of the algorithms to be described depend only on clock
rates being synchronized rather than on clocks being synchronized. This
means that the clocks at the different nodes run at approximately the
same rates although the times on these clocks may be different. The
interest in clock rates dates from the time when clock synchronization
was thought to be practically unattainable. Clock rates were assumed to
be "naturally" synchronized so that no algorithm was required to keep
them synchronized; in fact, to ensure that rates are truly synchronized
requires algorithms similar to clock synchronization algorithms. The
focus on algorithms that depend only on synchronized rates will likely
diminish once synchronized clocks are available. Nevertheless, it is
interesting to understand what enables an algorithm to depend on rates
instead of time; this question is discussed in Section 8.

Clock synchronization algorithms typically synchronize clocks with
"real" time, i.e., at any moment a node's clock differs from real time
by no more than €/2. At the root of such algorithms is a dependence on
devices that sample universal time; such devices are attached to time
servers, and the algorithm spreads the information about the current
time from the servers to other nodes in the network. Having clocks close
to real time is obviously important; for example, the "time last
modified" for files ought to be close to the time the modification
actually occurred, or users may notice strange behavior. Also, the
presence of an algorithm that synchronizes nodes' clocks to real time
obviates the need for operators to set the clock manually.

None of the algorithms to be discussed depends on the clock times being
close to real time. They do depend on clock rates being close to real
clock rates, however. This is because each algorithm makes use of time
intervals that have been chosen based on assumptions about how users of
the system behave, e.g., what kinds of delays a user is willing to
tolerate. If internal clock rates aren't close to real clock rates these
assumptions will not be honored.


\hypertarget{at-most-once-messages}{%
\section{At-most-once Messages}\label{at-most-once-messages}}


The first example of the use of synchronized clocks is the SCMP protocol
{[}12{]}, which guarantees at-most-once delivery of messages. Many
networks do not guarantee at-most-once delivery; instead they may
duplicate messages and furthermore duplicates may arrive very late. In
addition, since networks may lose messages, higher levels of a system
re-send them, which may also lead to duplicates.

Implementing at-most-once semantics is typically done by having each
message receiver maintain a table containing information about "active"
senders that have communicated with the receiver recently. When a
message arrives, if there is information about the sender in the table
it is used to determine whether or not the message is a duplicate. If
there is no information, there are two choices: either accept the
message or reject it. If the message is accepted, there is a chance of
accepting a duplicate. This chance can be made arbitrarily small by
keeping information about senders long enough, but it is difficult to
determine how long to keep this information in the presence of sender
retransmission and networks with probabilistic delay.

The alternative of rejecting the message guarantees that no duplicates
will ever be accepted. However, it gives rise to a problem. When a
message is sent, we want to be reasonably certain that the receiver will
accept it. Therefore we need to know that the receiver has information
about the sender in its table. If it is unlikely to have such
information, e.g., because this is the first time the sender has
communicated with it in a while, then it is necessary to set up the
information before sending the message. This can be done by means of a
handshake in which a pair of messages is exchanged between the sender
and receiver in advance of the at-most-once message. If the sender then
sends many messages over the connection established by the handshake,
the cost of the handshake is amortized across all of them. If there are
only a few messages, the overhead is high relative to useful work. In
the worst case, the sender fransmits only one message per handshake. Yet
this case may be quite common; it corresponds to a client that performs
a single operation at each of many servers.

The SCMP protocol avoids the handshake between the sender and receiver
by using synchronized clocks. The idea is that the receiver remembers
all "recent" communications. If a message from a particular sender is
"recent," the receiver will be able to compare it with the stored
information and decide accurately whether the message is a duplicate. If
the message from the sender is "old," it will be tagged as a duplicate
even though it may not be, but this case is very unlikely. Thus the
system will never accept a duplicate but it may occasionally reject a
non-duplicate.

For the scheme to work, receivers need to know whether a message is
"recent." When a node sends a message, it timestamps the message with
the current time of its clock. When the message arrives at the receiver,
it is considered recent if its timestamp is later than the receiver's
local time minus the \emph{message lifetime interval} p; otherwise it is
old. The message lifetime interval must be big enough (e.g., ten
minutes) so that almost all messages will arrive within p time units of
when they were sent; it is much larger than e. The characteristics of p
are discussed further in {[}12{]}.

The protocol works as follows. Every module G has a \emph{current time},
G.time; this is read from the clock belonging to its node. Every message
m contains a timestamp, m.ts; this is G.time of the sending module at
the time m is created. Even though a particular message may be
duplicated either by the network or by the software that carries out a
higher-level protocol, all copies of the message contain the same m.ts.
Each message also contains a \emph{connection identifier}, m,conn; this
is selected by the sender without consultation with the receiver, unlike
other protocols, e.g., TCP {[}19{]}. The connection id must be distinct
from the ids used in other senders, e.g., at other nodes; in addition,
if the sender has several outstanding messages to the same receiver, it
should use a separate connection id for each.\footnote{Thus in a system
  supporting lightweight threads within processes a connection id might
  be a triple \textless{}node-id, process-ids, thread-id\textgreater{},
  where the node-id is a unique name of the sender's node, the
  process-id identifies the process within the node, and the thread-id
  identifies the thread within the process.} Every message sent to a
particular receiver and containing a particular connection id should
contain a distinct timestamp; thus the connection id and the timestamp
together constitute a unique message id with respect to that receiver.

Each receiver maintains a \emph{connection table}, G.CT, that maps
connection ids to connection information including the timestamp of the
last message accepted on that connection. Not all connections have an
entry in $G.CT$. $G$ is free to remove an entry for connection C from its
connection table provided $G.CT[C].ts \leq G.time - \rho - \varepsilon$; such an entry is
considered to be "old." (Recall that p is the message lifetime
interval.) A receiver also maintains an upper bound, G.upper, on the
timestamps that have been removed from the table. Since only old
timestamps are removed from the table, G.upper G.time - p - E.

The algorithm works by determining a per-connection bound that
distinguishes ``new'' from ``old'', or potentially duplicate, messages, and
comparing the timestamp of the newly arrived message with that bound. If
the message's connection has an entry in the table G.CT, the bound is
the timestamp of the most recent previously accepted message. If there
is no table entry, the global bound G.upper is used. G.upper is an
appropriate bound because if there is no information for the connection
in G.CT, this means the last message on the connection (if any)
contained a timestamp t G.upper. Therefore, if a message arrives whose
timestamp is later than this, it must be a new message. Since $G.upper \leq
G.time - \rho - \varepsilon$ there is little chance
of incorrectly flagging a message as a duplicate, provided p is large
enough. Messages with timestamps less than the bound are discarded; if a
message is accepted, its timestamp is stored in the G.CT entry for its
connection.

Receivers that survive crashes need a way to determine whether a message
that arrives after crash recovery is a duplicate of a message that
arrived before the crash. SMTP uses time to solve this problem also. It
maintains on stable storage {[}8{]} a timestamp G.latest that is larger
than the timestamps of all messages accepted so far. A message that
arrives too early (i.e., its timestamp is greater than G.latest) is
discarded or delayed. After a crash, G.upper is initialized to G.latest.
This will cause all potential duplicates to be rejected because only
messages with timestamps less than this new G.upper could have been
accepted before the crash. G.latest is maintained by writing G.time + to
stable storage periodically; G.latest is the most recent value written
to stable storage. is some increment (e.g., a few seconds) that is large
enough so that stable storage isn't written often, but small enough so
that not many messages must be rejected after a crash. For many
persistent servers, G.latest can simply be written to stable storage as
part of the records that are being written there anyway to record
information about the server's persistent state.

Synchronized clocks allow the protocol to establish a system-wide notion
of "recent." Clocks are used to avoid communication (to establish a
connection) and to save storage at receivers (only timestamps of recent
messages need be saved). Timestamps identify messages that have already
arrived. The identification is only approximate, since a single
timestamp G.upper stands for all earlier messages, and therefore
sometimes a message that is not a duplicate will be rejected.

If clocks get out of synch, there is no danger of a duplicate message
being accepted, but recent messages may be flagged as duplicates. If a
node's clock is slow, its messages are more likely to be flagged as
duplicates by other modules; if its clock is fast, it is more likely to
flag messages from other modules as duplicates. The algorithm does
depend on the values stored for G.latest being monotonic, but this is
easy to guarantee in software each time a new value for G.latest is
written to stable storage.


\hypertarget{authentication-tickets-in-kerberos}{%
\section{Authentication Tickets in
Kerberos}\label{authentication-tickets-in-kerberos}}


The next example is taken from the Kerberos system {[}22{]}. Kerberos
provides a means for modules to communicate using secure, authenticated
connections. It makes use of private keys using the DES encryption
technique {[}15{]}, and is based on the Needham and Schroeder
authentication protocol {[}16{]}. Systems that use Kerberos make use of
authenticated connections between every client-server pair. Kerberos
uses synchronized clocks in two ways: to limit the use of particular
keys, and to help servers detect replayed messages.

Every communication between a particular client C and server S is
controlled by a ticket. The client obtains a ticket by applying to the
ticket granting service, TGS. If the client is authorized to use S, the
TGS gives it a ticket Tc s for S and also a secret "session" key, Kc s,
that can be used during future communication between C and S'. The
message from the TGS to the client containing Tc,s and Kc s is encrypted
using C's private key; this ensures both that the message really comes
from the TGS (since only the TGS and C know C's private key) and also
that the key cannot be stolen by an intruder that intercepts the
message. When the client receives this message, it can decrypt it to
obtain the key and the ticket.


C sends the ticket to S every time it communicates with S. The ticket Tc
S identifies C and S and also contains the session key Kc,s. The ticket
is encrypted using S 's private key (again to prevent forgery and
theft); S can decrypt the ticket to obtain the key. Thus both (and only)
C and S know the session key and therefore they can use it to exchange
private information.

Since the ticket is what allows C to talk to S, its use must be
controlled. For example, suppose C runs at a public workstation and the
ticket was obtained on behalf of some person who was using that
workstation. If that person leaves the workstation without logging out,
someone else could obtain access to his or her tickets. Therefore, each
ticket also contains an expiration time, E; a server S will only accept
communications from C using ticket Tc s provided the expiration time it
contains is less than the server's clock - e.

This use of synchronized clocks saves a communication between S and TGS.
Without synchronized clocks, S could ask the TGS whether a ticket were
still valid and the TGS could determine validity by comparing the
expiration time of the ticket with its local clock. In doing the test at
S, we rely on the time at S's clock being a close approximation to the
time at the TGS's clock.

The correcmess condition for the expiration time is: S must not use the
ticket after it expires at the TGS. Therefore, correctness will fail if
S's clock is slow (or the TGS's clock is fast). But, S's clock must be
very slow for this to be a problem. The lifetime of a ticket is
typically much larger than e. If S 's clock is just a little slow, it
doesn't matter at all. Furthermore, the problems arising from using a
ticket a little too long are less than those arising from other threats,
such as tickets being stolen at unattended workstations.

One point that is interesting about this use of synchronized clocks is
the following. A system like Kerberos is designed to permit secure
communication in the face of various threats such as users u•ying to
steal keys or trick the system into telling them about keys. To base a
system like Kerberos on synchronized clocks requires that the clock
synchronization algorithm be secure against similar threats. For
example, if it were possible for a malicious user to "spoof' the
synchronization algorithm in a way that causes S 's clock to become very
slow, ückets could be used long after they should have expired.

The second use of clocks in Kerberos, to avoid replays, makes use of
authenticators. An authenticator is basically just a timestamp that has
been encrypted. The timestamp is produced by the client reading its
clock; the client then encrypts it by using the session key Kc s. Since
Kc s is known only to C and S, it is not possible (with very high
probability) for an intruder to create a valid authenticator on its own;
instead all an intruder can do is to re-use an authenticator. Messages
containing old authenticators are discarded. If desired, a server can in
addition retain the timestamps of all recent messages and discard any
new messages that contain these timestamps.

The use of authenticators is similar to what occurs in the at-most-once
message protocol. However, at-most-once delivery is not the point of the
protocol. For example, if the client does not receive a reply to a
message, it is free to send it again with a different authenticator.


If clocks get out of synch, no harm will occur if the server is
maintaining a list of current messages; otherwise, if the server's clock
is slow, it might accept a replay of an earlier message. As was the case
with the at-most-once protocol, clocks are used to reduce storage at
servers (only timestamps of current messages need be saved) and avoid
communication (checking with a client about the status of a newly
received message).


\hypertarget{cache-consistency}{%
\section{Cache Consistency}\label{cache-consistency}}


The next example concerns systems in which servers provide persistent
storage for objects and programs that use those objects run at client
workstations. To provide reasonable response time to clients, copies of
persistent objects are cached at the workstations so that clients can
use them locally when there is a cache hit.

As is the case in any system with cached copies, we need to be concerned
with how to maintain cache consistency. One possibility is to use
"leases" as discussed below. This idea first appeared in {[}5{]}; it is
also used in the Echo system {[}13{]}. In either case the concept is
used in a file system, so the objects in question are files. In the
initial use of leases, the caches were writethrough; in Echo, the caches
are write-behind. This difference in cache behavior leads to a
difference in how leases are used. Below I explain how these systems
work given synchronized clocks. The systems in fact only require
synchronized clock rates as discussed in Section 8.

In the case of the write-through cache, leases work as follows. Each
client workstation obtains a lease for a file when the file is copied
into its cache. The lease contains an expiration time E; when E has been
reached, i.e., when E \textgreater{} time(client) - e, the client stops
using the file. The client can request that a lease can e renewed by
asking the server for a new expiration time.

When a client modifies a file, the modification goes directly to the
server (since this is a write-through cache). The server can do the
modification immediately if there are no other outstanding leases on the
file. Otherwise it communicates with the clients holding the leases,
requesting them to give them up. The modification is done when all
leases have been relinquished.

Of course, it is possible that a current holder of a lease might not
respond, either because of a network problem, or because of a crash of
its node. In this case, the system will wait until the expiration time
of the lease, and then do the modification.

The idea of leases requires only a small extension to work with
write-behind caches. Now there are two kinds of leases, read leases and
write leases, and a client must use the file in accordance with its
lease. Thus a client with a read lease can only read the file, while a
client with a write lease can both read and write the file. There are
the usual rules concerning readers and writers: Many clients can
simultaneously have read leases for a file, but if a client has a write
lease for some file, no other clients can have read or write leases for
that file.

Each lease has an expiration time as discussed above. The only
difference is that competition for leases now occurs when a client
requests a lease (rather than when a file is written). If a client needs
a lease that conflicts with leases held by other clients, the server
sends messages to the other clients requesting them to relinquish their
leases. For example, if some client needs a write lease, the system will
notify all holders of read leases to remove that filc from their caches.
The new lease is granted when all old leases are either relinquished or
expired.


The invariant of interest in a system with leases is: each time a client
uses a file, it has a valid lease for that file. Validity is determined
by using the client's clock as an approximation of the time of the
server's clock. If the client's clock is slow, or the server's clock is
fast, the invariant will not hold. In this case, the client may continue
to use the file after its lease has expired at the server.


In the absence of the use of synchronized clocks there are two
possibilities for maintaining cache consistency, neither of which is
desirable. One alternative is for the client to check the validity of
each file use. This alternative is not much better than not having
caches at all. In particular, to read a file in its cache, a client
would have to communicate with the server to determine if the cached
copy is valid; if it is valid, the server need not send back a copy, so
this part of the communication is saved, but in any case there would be
long delays. This is effectively a system in which all leases have a
lifetime of zero.

The other alternative is for the server to not invalidate a client's
lease until it hears from the client. This is roughly what happens in
cache consistency protocols in multi-processors. Such protocols are
based on the assumption that nodes and communication never fail (or that
all fail together); these assumptions ensure that the wait to release
the lease will be very short. Such an assumption is less attractive in a
distributed environment; here nodes md the network can fail
independently, so that the wait can be long. In fact, requiring the
server to wait for the client to give up leases is equivalent to having
a lease with an infinite lifetime.

Thus we can see that the designer of such a system is presented with two
unattractive choices: either depend on assumptions such as synchronized
clocks that might fail causing inconsistencies, or sacrifice
performance. Choosing to improve performance is a valid position, given
the low probability of clocks getting out of synch.


\hypertarget{atomicity}{%
\section{Atomicity}\label{atomicity}}


Although the decision to use leases is justifiable, still it would be
nice if there were a higher level mechanism to take care of cache
consistency problems due to unsynchronized clocks. Transactions are such
a mechanism. For example, in the Thor object-oriented database {[}9,
10{]}, all accesses to objects occur within atomic transactions. Objects
in Thor are not just files; instead they belong to various types,
including user-defined types. Objects are stored at servers, which
provide persistent storage for them. Clients run at workstations, and
caching is used to reduce delay to the clients.

Optimistic concurrency control {[}7{]} is used to provide serialization
of transactions. Each object has a version number that is copied into a
client cache along with the object. As a transaction runs, it uses the
objects cached at its workstation without communicating with the
serv\textsuperscript{r}ers. When the transaction commits, the version
numbers of all the objects it used are sent to the servers along with
the new versions of all objects it modified. The servers compare these
version numbers with those stored with the objects; if the numbers do
not match, the transaction must abort.

To reduce the likelihood that transactions will abort due to stale data,
servers notify clients when objects in their caches arc modified by a
committing transaction. Leases can be used to limit server
responsibility for notifying clients about stale information and to
delay commits of transactions that might cause transactions running at
other clients to abort. Since Thor is a write-through system (writes are
delayed until transactions commit, but really happen at this point),
there is just one kind of lease. When a transaction that modified object
x attempts to commit, the server can check with all other clients
holding unexpired leases on x, asking them to give up their Icascs, and
allow the transaction to commit only if all clients relinquish their
leases, or when all leases expire.

The invariant in this system is: each time a client uses an object, it
holds a valid lease for that object. If clocks get out of synch, the
invariant might not be preserved. However, in Thor the worst that will
happen is that some transaction may have to abort. No damage will have
been done to the consistency of the persistent objects. Thus, the higher
level mechanism (transactions) provides the safety that was missing in
the lower level mechanism (leases).


\hypertarget{commit-windows}{%
\section{Commit Windows}\label{commit-windows}}


The final example concerns the use of synchronized clocks within a
replication algorithm. The algorithm to be described is used in the Harp
file system {[}11{]}; a similar technique is used in Echo {[}13{]}.

Harp is a replicated Unix file system that provides highly reliable and
highly available storage for files. It supports the virtual file system
(VPS) {[}6{]} interface and is intended to be used within a file service
in a distributed
network, such as NFS {[}20, 23{]}. The idea is that clients continue to
use the file service just as they always did. However, the server code
of the file service calls Harp and achieves higher reliability and
availability as a result. Harp runs each file operation as an
independent atomic operation; as is usual in file systems, there is no
mechanism to run transactions consisting of sequences of operations.

Harp uses the primary copy replication technique {[}1, 17, 18{]}. In a
primary copy method, client requests are sent to just one server called
the primary; the other servers are backups. The primary decides what to
do and records any new information at a sub-majority of backups before
committing the operation. A sub-majority is one less than a majority,
e.g., if there are five servers, the operation would be recorded at two
backups before it commits. Since the primary also knows about the
operation, this means a committed operation is known to a majority of
replicas.

The primary and backups always run within a view; a view is simply the
group of replicas that are currently cooperating to provide service.
When there is a failure, or a recovery from a failure, the system
reconfigures itself by performing a view change {[}2, 3{]}, which leads
to the creation of a new view. A view always contains a majority of
servers; this ensures that the new view intersects with the old one in
at least one replica, which in turn can be used to ensure that the new
view starts in a state that reflects all committed operations from
earlier views. The primary of the new view may be a different node than
the primary of the old view.

In Harp, each replica maintains a log in which it records information
about client modification operations.\footnote{In Harp, the log is kept
  in volatile memory, which is backed-up by an uninterruptible power
  supply. The power supply allows the server enough time to copy the log
  to disk in the event of a power failure.} To carry out a modification
operation, the primary creates an event record that describes the
modification, appends it to the log, and sends the logged information to
the backups. As new log entries arrive in messages from the primary, a
backup appends them to its log and sends an acknowledgement message to
the primary. When acknowledgments have arrived from a sub-majority of
backups, the primary commits the operation and responds to the client.

Read operations (e.g., to determine file status) could be handled
similarly to modification operations by making entries in the log and
waiting for the ack from the backup, but this seems unnecessary because
read operations don't change anything. Therefore read operations are
performed entirely at the primary. This can lead to a problem if the
network partitions. For example, suppose a partition separates the
primary from the backups, and the backups form a new view with a new
primary. If the old primary processes a read operation at this point,
the result returned might not reflect a write operation that has already
committed in the new view. Such a situation does not compromise the
state of the file system, but it can lead to a loss of external
consistency {[}4{]}. (A violation -of external consistency occurs when
the ordering of operations Inside a system does not agree with the order
a user expects.)

Synchronized clocks can be used to reduce the probability of having a
violation of internal consistency. Essentially the primary holds leases,
but the object in question is the entire replica group. Each message
sent by a backup to the primary gives the primary a lease. The primary
can do a read operation unilaterally if it holds unexpired leases from a
sub-majority of backups. When a new view starts, its new primary cannot
reply to any client requests until all leases given to the old primary
by replicas in the new view have expired. Leases are short, e.g., to
create the expiration time in a lease a backup might add one second to
its current time; therefore a new view is unlikely to be delayed since
by the time the view change has finished, the old leases will all have
expired.

The invariant in this system is: whenever a primary performs a read it
holds valid leases from a sub-majority of backups. This invariant will
not be preserved if clocks get out of synch. However, the impact of
violating the invariant is small. At worse there will be a violation of
external consistency, but in fact this is unlikely. For the violation to
happen, there must be two clients Cl and C2, with Cl doing a read at the
old primary, and C2 doing a write at the new one. Here is the scenario:


\begin{enumerate}
\def\labelenumi{\arabic{enumi}.}
\item

  C2 performs a write at the new primary.

\item

  C2 informs Cl (e.g., by performing a remote procedure call to Cl)
  about the update.

\item

  Cl reads the modified file (at the old primary) and does not see the
  update.

\end{enumerate}


It is unlikely that clocks would become unsynchronized enough for this
to happen.

A similar technique could be used to avoid two-phase commit for
read-only transactions.
The technique works as follows. Suppose a transaction starts at some
node and visits other nodes by making remote procedure calls (RPCs).
Each node that it visits contains objects that the transaction can read
or modify. The reply to an RPC contains an indication of whether the
transaction read or modified objects during that call and a time during
which the node promises to not release any read locks. When the
transaction attempts to commit, if it is read-only (all RPC replies
indicated that it only read objects), and if the time at the coordinator
is less than the promised release time (minus E) for all read locks, the
coordinator can commit the transaction without communicating with any
other nodes. Note that as in the commit window scheme, a node must not
provide service when it recovers from a failure until it is certain that
all promises have expired.

If clocks get out of synch, this algorithm may result in nonserializable
behavior. For example, if one of the participants releases a transaction
Tl 's read locks early because its clock is fast, this may allow some
other transaction T2 to modify objects at P after Tl read them, and to
modify objects at some other node Q before Tl reads them. As was the
case in the Harp file system, a situation in which this occurs is highly
unlikely. Furthermore, out-of-synch clocks affect the serial order only
of read-only transactions; all transactions that modify objects would be
serialized properly with respect to one another. Although it is probably
not a good idea to adopt such transactions as the only choice, they
might be a useful option. Certain applications might be willing to
settle for a slight danger of seeing an inconsistent state in a
read-only transaction to gain the improved performance that would
result.


\hypertarget{synchronized-rates}{%
\section{Synchronized Rates}\label{synchronized-rates}}


Several of the algorithms discussed above can be implemented using
synchronized clock rates rather than synchronized clocks. This section
discusses how rates are used instead of clocks, and when rates are
adequate.

Suppose we want to use the cache-consistency lease mechanism, but based
on synchronized rates instead of synchronized clocks. Then instead of
the message from the server containing an expiration time, it would
contain a time to expiration, i.e., an interval such as "five seconds."
A client always receives a lease in response to some message it sent; it
simply adds the expiration interval of the lease to the time of its
clock when it sent the request for the lease, obtaining a local
expiration time TC. The server does the same thing, but its local
expiration time Ts depends on the time of its clock when it sent the
response. Provided the clock rate differences are bounded by some skew,
and that this skew is used at the client to determine when the leases
expire, we can be sure that the lease will expire at the client no later
than it expires at the server. This is true because the response at the
server must have occurred after the request was sent by the client.


The use of leases illustrates the kind of situation in which
synchronized clock rates can be used instead of synchronized clocks. In
all these algorithms, there is some event of interest, e, such as the
expiration of a lease at the server or the expiration of a ticket at the
TGS. Call the node where this event happens the owner, O, of the event.
The event e occurs at time T ; here Te is the real time at which e
occurs rather than a time of some node's clock. Some other node (or
nodes) depends on event e and must make a conservative judgment about
when it occurs. Call this other node the dependent node, D. The
dependent node makes the approximation by means of an event d of its
own; d happens at absolute time Td, and we require that T T . For
example, the expiration of a lease at the client is such an event d and
it must happen no later than the expiration of that lease at the server.

The event of interest e is always preceded by some other event f that
leads to it and that also occurs at e's owner. For example, the TGS
granting a ticket is such an event f. Note that the dependent node will
find out about f via a message that arrives from the owner (possibly
sent via intermediate nodes); for example, this is how the client finds
out about the granting of the lease. Rates can be used instead of clocks
if there is some still earlier event g that happens at the dependent
node D and that leads to f. For example, a lease is granted at the
server because a client requested it. The situation is illustrated in
\autoref{fig-8-1}.

\begin{figure}
  \centering
  \includegraphics{fig-8-1.png}
  \caption{Using rates. Time increases going down.\label{fig-8-1}}
\end{figure}


Rates work because this communication pattern enables D to make the
necessary conservative judgment. The event d will happen at time T + -
e. Here T is the absolute time at which event g occurred is the
expiration interval, and e is the appropriate bound on the skew based on
how rates can vary over an interval l. Furthermore, e will happen at
time Tf + X. Thus we have:

\begin{align*}
  T_d &= T_g + \lambda - \varepsilon\\
  T_e &= T_f + \lambda
\end{align*}

Since $T_g \leq T_f$ and the skew in the rates of the clocks at $D$ and $O$ during the
interval $\lambda$ is bounded by $\varepsilon$, we know that $T_d \leq T_e$.

Thus rates can be used when there is a communication already happening
that allows the dependent node to estimate approximately when the event
of interest happens. Clocks are needed when there is no such
communication; in fact, clocks permit the communication to be avoided.
For example, in the at-most-once protocol the owner O is the sender of
the message; the receiver is the dependent node D. Since the message is
sent autonomously by the sender, without a prior communication from the
receiver, there is no way to use rates. In the case of Kerberos tickets,
the request of a client for a ticket might appear to be event g, but the
client cannot be trusted to abide by any rules, so that the server is
the one that enforces the expiration time. Thus the server is actually
the dependent node D. If the TGS communicated with the server before
granting the ticket, it would be possible to use rates: The server's
response in this communication would be the event g that leads to event
f (the TGS's granting of the ticket). However, this communication does
not in fact take place, which speeds up the ticket-granting process.

The focus on algorithms that depend only on rates is likely to diminish
now that synchronized clocks exist. Note that synchronized clocks are
more powerful than synchronized rates; they support all algorithms that
depend on rates, and some other algorithms besides.

\hypertarget{discussion}{%
\section{Discussion}\label{discussion}}


Earlier sections of this paper have looked at how synchronized clocks
are used in a number of distributed algorithms. In each case, clocks
were used to provide improved performance by avoiding communication. In
some algorithms, communication could also have been avoided by retaining
state (e.g., in the use of authenticators in Kerberos, or in the
at-most-once message protocol); for these, the use of clocks can also be
thought of as a way of using garbage collection of "old" information to
reduce storage requirements.

The algorithms differ in the consequences of clocks getting out of
synch. There are the following possibilities:


\begin{enumerate}
\def\labelenumi{\arabic{enumi}.}
\item
  No effect on correctness. This is the case with the at-most-once
  message protocol and also with the authenticators in Kerberos provided
  the server keeps track of the timestamps of all recent messages.
\item
  Compensation. Even when clock synchronization gives rise to errors,
  there may be some other part of the system that compensates. The use
  of atomic transactions is an example of how a problem at one level of
  a system may be resolved at a higher level.
\item
  Domination by other failures. In some systems, failures that arise
  because clocks are out of synch are dominated by other possible
  failures. This is what happens with Kerberos tickets. Tickets might be
  used too long if servers' clocks are slow. However, using a ticket
  that was supposed to last for several hours for an few minutes is not
  a serious matter, especially compared to other difficulties such as
  stolen tickets.
\item
  Trade-off for performance. Finally, it is reasonable to choose a
  mechanism that works improperly when clocks fail when the alternatives
  are unacceptable. For example, the alternatives to leases are either
  high overhead on each read, or very long periods in which certain
  files cannot be used.
\end{enumerate}


Although none of the algorithms depends on clocks approximating "real"
time, all require that clock rates approximate real time passing. For
example, a Kerberos ticket is supposed to have a lifetime that
approximates a real timc interval; a ticket that is intended to last for
one hour should last for about that long. The algorithms rely on real
clock rates because they all depend on lifetime intervals that are
chosen based on expectations about user requirements. Thus, the lifetime
of a ticket is chosen based on an analysis of thc likelihood that it
will be stolen within that time and the seriousness of the consequences
of such an event.

To convert a distributed algorithm to one that uses synchronized clocks,
there are two places to look. By examining the messages that are being
exchanged, it may be possible to identify some that could be avoided by
using timestamps. ()r, in the case where message exchange is already
reduced by maintaining state, it may be possible to find a way to save
storage by using timestamps as a garbage collection technique. After
finding a place to use timestamps, the next step is to analyze the
consequences of using synchronized clocks, both on normal behavior (when
clocks are in synch) and during clock failures. During this analysis the
time interval that will be used is selected (all algorithms have such an
interval, e.g., the message lifetime interval p in the at-most-once
protocol). An algorithm based on timestamps is a good idea if ultimately
the worst case behavior is sufficiently unlikely or sufficiently benign
so as to represent a good tradeoff for improved performance.

\section*{Acknowledgments}

The author gratefully acknowledges the help given by readers of earlier
drafts of this paper, including Dorothy Curtis, Robert Gruber, John
Guttag, Paul Johnson, Andrew Myers, Liuba Shrira, Raymie Stata, Greg
Troxel, and John Wroclawski.

\hypertarget{references}{%
\section{References}\label{references}}

\begin{enumerate}
\def\labelenumi{\arabic{enumi}.}
\item
  Alsberg, P., and Day, J. A Principle for Resilient Sharing of
  Distributed Resources. Proc. of the 2nd International Conference on
  Software Engineering, October, 1976, pp. 627-644. Also available in
  unpublished form as CAC Document number 202 Center for Advanced
  Computation University of Illinois, Urbana-Champaign, Illinois 61801
  by Alsberg, Benford, Day, and Grapa..
\item
  El-Abbadi, A., and Toueg, S. Maintaining Availability in Partitioned
  Replicated Databases. Proc. of the Fifth Symposium on Principles of
  Database Systems, ACM. 1986. pp. 240-251.
\item
  El-Abbadi, A., Skeen, D., and Cristian, F. An Efficient Fault-tolerant
  Protocol for Replicated Data Management. Proc. of the Fourth Symposium
  on Principles of Database Systems, ACM, 1985, pp. 215-229.
\item
  Gifford, D.K. Information Storage in a Decentralized Computer System.
  Technical Report CSL-81-8, Xerox Corporation, March, 1983.
\item
  Gray, C., and Cheriton, D. Leases: An Efficient Fault-Tolerant
  Mechanism for Distributed File Cache Consistency. Proc. of the Twelfth
  ACM Symposium on Operating Systems Principles, 1989, pp. 202-210.
\item
  Kleiman, S. Vnodes: An Architecture for Multiple File System Types in
  Sun UNIX. USENTX summer '86 Conference Proceedings, 1986, pp. 238-247.
\item
  Kung, H. T., and Robinson, J. T. "On Optimistic Methods for
  Concurrency Control". ACM Trans. on Database Systems 6, 2 (June 1981),
  213-226.
\item
  Lampson, B. W., and Sturgis, H. E. Crash Recovery in a Distributed
  Data Storage System. Xerox Research Center, Palo Alto, ca., 1979.
\item
  Liskov, B., et al. Preliminary Design of the Thor Object-Oriented
  Database System. In preparation.
\item
  Liskov, B., Gruber, R., Johnson, P., and Shrira, L. A Highly Available
  Object Repository for use in a Heterogeneous Distributed System. Proc.
  of the Fourth International Workshop on
  Persistent Object Systems Design,
  Implementation and Use, Martha's Vineyard, MA, September, 1990.
\item
  Liskov, B., Ghemawat, S., Gruber, R., Johnson, P., Shrira, L., and
  Williams, M. Replication in the Harp File System. Submitted for
  publication.
\item
  Liskov, B., Shrira, L., and Wroclawski, J. Efficient At-Most-Once
  Messages Based on Synchronized Clocks. To appear in ACM Trans. on
  Computers.
\item
  Mann, T., Hisgen, A., and Swart, G. An Algorithm for Data Replication.
  Report 46, DEC Systems Research Center, Palo Alto, CA, June, 1989.
\item
  Mills, D.L. Network Time Protocol (Version 1) Specification and
  Implementation. DARPA-Internet Report RFC-1059. July 1988.
\item
  National Bureau of Standards. Data Encryption Standard. Government
  Printing Office, Washington, DC, 1977.
\item
  Needham, R., and Schroeder, M. "Using Encryption for Authentication in
  Large Networks of Computers". Comm. of the ACM 21, 12 (December 1978),
  993-999.
\item
  Oki, B. M., and Liskov, B. View-stamped Replication: A New Primary
  Copy Method to Support Highly-Available Distributed Systems. Proc. of
  the 7th ACM Symposium on Principles of Distributed Computing, ACM,
  August, 1988.
\item
  Oki, B. M. View-stamped Replication for Highly Available Distributed
  Systems. Technical Report MIT/LCSÆR-423, M.I.T. Laboratory for
  Computer Science, Cambridge, MA, August, 1988.
\item
  Postel, J. DOD Standard Transmition Control Protocol. DARPA-Internet
  RFC-793. September, 1981.
\item
  Sandberg, R., et al. Design and Implementation of the Sun Network
  Filesystem. Proc. of the Summer 1985 USENIX Conference, June, 1985,
  pp. 119-130.
\item
  Simons, Be, Welch, J., and Lynch, N. An Overview of Clock
  Synchronization. Research Report RJ 6505, IBM Almaden Research Center,
  1988.
\item
  Steiner, J.G., Neuman, C., Schiller, J.I. Kerberos: An Authentication
  Service for Open Network Systems. Project Athena, MIT, Cambridge, MA,
  March, 1988.
\item
  Sun Microsystems, Inc. NFS: Network File System Protocol
  Specification. Tech. Rept. RFC 1094, Network Information Center, SRI
  International, March, 1989.
\end{enumerate}

\end{document}
