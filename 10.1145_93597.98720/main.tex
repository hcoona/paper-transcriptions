\PassOptionsToPackage{unicode=true}{hyperref} % options for packages loaded elsewhere
\PassOptionsToPackage{hyphens}{url}
%
\documentclass[a4paper,11pt,notitlepage,twoside,openright]{article}

\usepackage{ifxetex}
\ifxetex
\else
  \errmessage{Must be built with xelatex}
\fi

\usepackage{amssymb,amsmath}
\usepackage{fourier}
\usepackage{inconsolata}
\usepackage{enumitem}
\usepackage{footnote}

% Table
\usepackage{tabu}
\usepackage{longtable}
\usepackage{booktabs}
\usepackage{multirow}

% Verbatim & Source code
\usepackage{fancyvrb}
\usepackage{minted}

% Beauty
\usepackage[protrusion]{microtype}
\usepackage[all]{nowidow}
\usepackage{upquote}
\usepackage{parskip}
\usepackage[strict]{changepage}

\usepackage{hyperref}

% Graph
\usepackage{graphicx}
\usepackage{grffile}
\usepackage{tikz}


\hypersetup{
  bookmarksnumbered,
  pdfborder={0 0 0},
  pdfpagemode=UseNone,
  pdfstartview=FitH,
  breaklinks=true}
\urlstyle{same}  % don't use monospace font for urls

\usetikzlibrary{arrows.meta,shapes.geometric,shapes.misc}

\newminted{java}{%
  autogobble,
  breakbytokenanywhere,
  breaklines,
  fontsize=\footnotesize,
}
\newmintinline{java}{%
  autogobble,
  breakbytokenanywhere,
  breaklines,
  fontsize=\footnotesize,
}

\makeatletter
\def\maxwidth{\ifdim\Gin@nat@width>\linewidth\linewidth\else\Gin@nat@width\fi}
\def\maxheight{\ifdim\Gin@nat@height>\textheight\textheight\else\Gin@nat@height\fi}
\makeatother

% Scale images if necessary, so that they will not overflow the page
% margins by default, and it is still possible to overwrite the defaults
% using explicit options in \includegraphics[width, height, ...]{}
\setkeys{Gin}{width=\maxwidth,height=\maxheight,keepaspectratio}
\setlength{\emergencystretch}{3em}  % prevent overfull lines
\setcounter{secnumdepth}{3}

% Redefines (sub)paragraphs to behave more like sections
\ifx\paragraph\undefined\else
\let\oldparagraph\paragraph
\renewcommand{\paragraph}[1]{\oldparagraph{#1}\mbox{}}
\fi
\ifx\subparagraph\undefined\else
\let\oldsubparagraph\subparagraph
\renewcommand{\subparagraph}[1]{\oldsubparagraph{#1}\mbox{}}
\fi

% set default figure placement to htbp
\makeatletter
\def\fps@figure{htbp}
\makeatother


\title{Encapsulation of Parallelism in the Volcano Query Processing System}
\author{Goetz Graefe}
\date{}

\begin{document}

\maketitle

\begin{abstract}
  Volcano is a new dataflow query processing system we have developed for database systems research and education.
  The uniform interface between operators makes Volcano extensible by new operators. All operators are designed and coded as
  if they were meant for a single-process system only. When attempting to parallelize Volcano, we had to choose between two
  models of parallelization, called here the \emph{bracket} and \emph{operator} models. We describe the reasons for not choosing the bracket
  model, introduce the novel operator model, and provide details of Volcano's \emph{exchange} operator that parallelizes all other operators.
  It allows intra-operator parallelism on partitioned datasets and both vertical and horizontal inter-operator parallelism. The
  exchange operator encapsulates all parallelism issues and therefore makes implementation of parallel database algorithms significantly
  easier and more robust. Included in this encapsulation is the translation between demand-driven dataflow within
  processes and data-driven dataflow between processes. Since the interface between Volcano operators is similar to the one
  used in ``real'', commercial systems, the techniques described here can be used to parallelize other query processing engine.
\end{abstract}

\section{Introduction}

In order to provide a testbed for database systems
education and research, we decided to implement an extensible
and modular query processing system. One important
goal was to achieve flexibility and extensibility without
sacrificing efficiency. The result is a small system, consisting
of less than two dozen core modules with a total of about
\(15,000\) lines of C code. These modules include a file system,
buffer management, sorting, top-down B+-trees, and
two algorithms each for natural join, semi-join, outer join,
anti-join, aggregation, duplicate elimination, division, union,
intersection, difference, anti-difference, and Cartesian produce.
Moreover, a single module allows parallel processing
of all algorithms listed above.

The last module, called the \emph{exchange} module, is the
focus of this paper. It was designed and implemented \emph{after}
most of the other query processing modules. The design
goal was to parallelize all existing query processing algorithms
without modifying their implementations.
Equivalently, the goal was to allow parallelizing new algorithms
not yet invented without requiring that these algorithms
be implemented with concern for parallelism. This
goal was met almost entirely, the only change to the existing
modules concerned device names and numbers to allow
horizontal partitioning over multiple disks, also called disk
striping~{[25]}.

Parallelizing a query evaluation engine using an
operator is a novel idea earlier research projects used
template processes that encompass specific operators. We
call the new method of parallelizing the \emph{operator model}. In
this paper, we describe this new method and contrast it
with the method used in GAMMA and Bubba, which we
call the \emph{bracket model}. Since we developed, implemented,
and tested the operator model within the framework of the
Volcano system, we will describe it as realized in Volcano.

Volcano was designed to be extensible, its design and
implementation follows many of the ideas outlined by
Batory et al.\ for the GENESIS design~{[5]}. In this paper,
we do not focus on or substantiate the claim to extensibility
and instead refer the reader to~{[17]}, suffice it to point out
that if new operators use and provide Volcano's standard
interface between operators, they can easily be included in a
Volcano query evaluation plan and parallelized by the
exchange operator.

Volcano's mechanism to synchronize multiple operators
in complex query trees within a single process and to
exchange data items between operators are very similar to
many commercial database systems, e.g., Ingres and the
System R family of database systems. Therefore, it seems
fairly straightforward to apply the techniques developed for
Volcano's exchange operator and outlined in this paper to
parallelize the query processing engines of such systems.

THis paper is organized as follows. In the following
section, we briefly review previous work that influenced our
design, and introduce the \emph{bracket} model of parallelization.
In Section 3, we provide a more detailed description of
Volcano. The \emph{operator} model of parallelization and
Volcano's \emph{exchange} operator are described in Section 4.
We present experimental performance measurements in Section 5
that show the \emph{exchange} operator's low overhead.
Section 6 contains a summary and our conclusions from this effort.

\section{Previous Work}

Since so many different system have been developed
to process large dataset efficiently, we only survey the systems
that have strongly influenced the design of Volcano.

At the start in 1987, we felt that some decisions in
WiSS~{[11]} and GAMMA~{[12]} were not optimal for performance
or generality. For instance, the decisions to protect
WiSS's buffer space by copying a data record in or out for
each request and to re-request a buffer page for every
record during a scan seemed to inflict too much overhead.\footnote{
  This statement only pertains to the original version of
  WiSS as described in~{[11]}. Both decisions were reconsidered for
  the version of WiSS used in GAMMA.
}
However, many of the design decisions in Volcano were
strongly influenced by experiences with WiSS and
GAMMA.The design of the data exchange mechanism
between operators, the focus of this paper, is one of the
few radical departures from GAMMA's design.

During the design of the EXODUS storage manager~{[10]},
many of these issues were revisited. Lessons learned
and tradeoffs explored in these discussions certainly helped
form the ideas behind Volcano. The development of E~{[24]}
influenced the strong emphasis on iterators for query processing.
The design of GENESIS~{[5]} emphasized the
importance of a uniform iterator interface.

Finally, a number of conventional (relational) and
extensible systems have influenced our design. Without
further discussion, we mention Ingres~{[27]}, System R~{[3]},
Bubba~{[2]}, Starburst~{[26]}, Postgres~{[28]}, and XPRS~{[29]}.
Furthermore, there has been a large amount of research and
development in the database machine area, such that there
is an international workshop on the topic. ALmost all database
machine proposals and implementations utilize parallelism
in some form. We certainly have learned from this
work and tried to include its lessons in the design and
implementation of Volcano. In particular, we have strived
for simplicity in the design, \emph{mechanisms} that can support a
multitude of \emph{policies}, and efficiency in all details. We
believe that the query execution engine should provide
mechanisms, and that the query optimizer should incorporate
and decide on policies.

Independently of our work, Tandern Computers has
designed an operator called the \emph{parallel operator} which is
very similar to Volcano's exchange operator. It has proven
useful in Tandern's query execution engine~{[14]}, but is not
yet documented in the open literature. We learned about
this operator through one of the referees. Furthermore, the
distributed database system R* used a technique similar to
ours to transfer data between nodes~{[31]}. However, this
operation was used only to effect data transfer and did not
support data or intra-operator parallelism.

\section{The Bracket Model of Parallelization}

When attempting to parallelize existing single-process,
Volcano software, we considered two paradigms or models
of parallelization. The first one, which we call the \emph{bracket model},
has been used in a number of systems, for example,
GAMMA~{[12]} and Bubba~{[2]}. The second one, which we
call the \emph{operator model}, is novel and is described in detail
in Section 4.

In the bracket model, there is a generic process template
that can receive and send data and can execute
exactly one operator at any point of time. A schematic
diagram of such a template process is shown in \autoref{fig1}
with two possible operators, join and aggregation. The
code that makes up the generic template invokes the operator
which then controls execution, network I/O on the
receiving and sending sides are performed as service to the
operator on request, implemented as procedures to be called
by the operator. The number of inputs that can be active
at any point of time is limited to two since there are only
unary and binary operators in most database systems. The
operator is surrounded by generic template code which
shields it from its environment, for example the operator(s)
that produce its input and consume its output.

\begin{figure}
  \centering
  \includegraphics{fig1.png}
  \caption{Bracket Model of Parallelization.\label{fig1}}
\end{figure}

One problem with the bracket model is that each
locus of control needs to be created. This is typically done
by a separate scheduler process, requiring software development
beyond the actual operators, both initially and for each
extension to the set of query processing algorithms. Thus,
the bracket model seems unsuitable for an extensible system.

In a query processing system using the bracket
model, operators are coded in such a way that network I/O
is their only means of obtaining input and delivering output
(with the exception of scan and store operators). The reason
is that each operator is its own locus of control and
network flow control must be used to coordinate multiple
operators, e.g., to match two operators' speed in a
producer-consumer relationship. Unfortunately, this also
means that passing a data item from one operator to
another always involves expensive inter-process communication
(IPC) system calls, even in the cases when an entire
query is evaluated on a single machine (and could therefore
be evaluated without IPC in a single process) or when data
do not need to be repartitioned among nodes in a network.
An example for the latter is the three-way join query
``joinCselAselB'' in the Wisconsin Benchmark~{[6,9]} which
uses the same join attribute for both two-way joins. Thus,
in queries with multiple operators (meaning almost all
queries), IPS and its overhead are mandatory rather than optional.

In most (single-process) query processing engines,
operators schedule each other much more efficiently by
means of procedure calls rather the system calls. The concepts
and methods needed for operators to schedule each
other using procedure calls are the subject of the next section.

\section{Volcano System Design}

In this section, we provide an overview of the
modules in Volcano. Volcano's file system is rather conventional.
It includes a modules to manage devices, buffer
pools, files, records, and B+-trees. For a detailed discussion,
we refer to~{[17]}.

The file system routines are used by the query processing
routines to evaluate complex query plans. Queries
are expressed as complex algebra expressions, the operators
of this algebra are query processing algorithms. ALl algebra
operators are implemented as \emph{iterators}, i.e., they support a
simple \emph{open-next-close} protocol similar to conventional file scans.

Associated with each algorithm is a \emph{state record}.
The arguments for the algorithms are kept in the state
record. All operations on records, e.g., comparisons and
hashing, are performed by \emph{support functions} which are given
in the state records as arguments to the iterators. Thus, the
query processing modules could be implemented without
knowledge or constraint on the internal structure of data objects.

In queries involving more than one operator (i.e.,
almost all queries), state records are linked together by
means of \emph{input} pointers. The input pointers are also kept
in the state records. They are pointers to a \emph{QEP} structure
that consists of four pointers to the entry points of the
three procedures implementing the operator (\emph{open}, \emph{next}, and
\emph{close}) and a state record. All state information for an
iterator is kept in its state record, thus, an algorithm may
be used multiple times in a query by including more than
one state record in the query. An operator does not need
to know what kind of operator procedures its input, and
whether its input comes from a complex query tree or from
a simple file scan. We call this concept \emph{anonymous inputs}
or \emph{streams}. Streams are a simple but powerful abstraction
that allows combining any number of operators to evaluate
a complex query. Together with the iterator control paradigm,
streams represent the most efficient execution model
in terms of time (overhead for synchronizing operators) and
space (number of records that must reside in memory at
any point of time) for single process query evaluation.

Calling \emph{open} for the top-most operator results in
instantiations for the associated state record, e.g., allocation
of a hash table, and in \emph{open} calls for all inputs. In this
way, all iterators in a query are initiated recursively. In
order to process the query, \emph{next} for the top-most operator is
called repeatedly until it fails with an \emph{end of stream} indicator.
Finally, the \emph{close} call recursively ``shuts down'' all
iterators in the query. This model of query execution
matches very closely the one being included in the E programming
language design~{[24]} and the algebraic query
evaluation system of the Starburst extensible relational database system~{[22]}.

The tree-structured query evaluation plan is used to
execute queries by demand-driven dataflow. The return
value of \emph{next} is, besides a status value, a structure called
\texttt{NEXT\_RECORD} that consists of a record identifier and a
record address in the buffer pool. This record is pinned
(fixed) in the buffer. The protocol about fixing and unfixing
records is as follows. Each record pinned in the buffer
is \emph{owned} by exactly one operator at any point in time.
After receiving a record, the operator can hold on to it for
a while, e.g., in a hash table, unfix it, e.g., when a predicate
fails, or pass it on to the next operators. Complex
operations like join that create new records have to fix
them in the buffer before passing them on, and have to
unfix their input records.

For intermediate results, Volcano uses \emph{virtual devices}.
Pages of such a device exist only in the buffer, and are
discarded when unfixed. Using this mechanism allows
assigning unique RID's to intermediate result records, and
allows managing such records in all operators as if they
resided on a real (disk) device. The operators are not
affected by the use of virtual devices, and can be programmed
as if all input comes from a disk-resident file and
output is written to a disk file.

\section{The Operator Model of Parallelization}

When porting Volcano to a multi-processor machine,
we felt it desirable to use the single-process query processing
code described above \emph{without any change}. The result is
very clean, self-scheduling parallel processing. We call this
novel approach the \emph{operator model} of parallelizing a query
evaluation engine. In this model, all issues of control are
localized in one operator that uses and provides the standard
iterator interface to the operators above and below in a
query tree.

The model responsible for parallel execution and
synchronization is called the \emph{exchange} iterator in Volcano.
Notice that it is an iterator with \emph{open}, \emph{next} and \emph{close}
procedures, therefore, it can be inserted at any one place or at
multiple places in a complex query tree. \autoref{fig2} shows a
complex query execution plan that includes data processing
operators, e.g. file scan and join, and exchange operators.

\begin{figure}
  \centering
  \includegraphics{fig2.png}
  \caption{Operator Model of Parallelization\label{fig2}}
\end{figure}

This section describes how the \emph{exchange} iterator
implements vertical and horizontal parallelism followed by a
detailed example and a discussion of alternative modes of
operation of Volcano's \emph{exchange} operator.

\subsection{Vertical Parallelism}

The first function of exchange is to provide \emph{vertical
parallelism} or pipelining between processes. The \emph{open}
procedure creates a new process after creating a data structure
in shared memory called a \emph{port} for synchronization and data
exchange. The child process, created using the UNIX \emph{fork}
system call, is an exact duplicate of the parent process.
The exchange operator then takes different paths in the
parent and child process.

The parent process serves as the \emph{consumer} and the
child process as the \emph{producer} in Volcano. The exchange
operator in the consumer process acts as a normal iterator,
the only difference from other iterators is that it receives its
input via inter-process communication rather than iterator
(procedure) calls. After creating the child process,
\texttt{open\_exchange} in the consumer is done. \texttt{next\_exchange}
waits for data to arrive via the port and returns them a
record at a time. \texttt{close\_exchange} informs the producer that
it can close, waits for an acknowledgement, and returns.

The exchange operator in the producer process
becomes the \emph{driver} for the query tree below the exchange
operator using \emph{open}, \emph{next}, and \emph{close} on its input. The output
of \emph{next} is collected in \emph{packets}, which are arrays of
\texttt{NEXT\_RECORD} structures. The packet size is an argument
in the exchange iterator's state record, and can be set
between \(1\) and \(32,000\) records. When a packet is filled, it
is inserted into a linked list originating in the \emph{port} and a
semaphore is used to inform the consumer about the new
packet. Records in packets are fixed in the shared buffer
and must be unfixed by a consuming operator.

When its input is exhausted, the exchange operator in
the producer process marks the last packet with an \emph{end-of-stream}
tag, passes it to the consumer, and waits until the
consumer allows closing all open files. This delay is
necessary in Volcano because files on virtual devices must
not be closed before all their records are unpinned in the
buffer. In other words, it is a peculiarity due to other
design decisions in Volcano rather than inherent in the
exchange iterator or the operator model of parallelization.

The alert reader has noticed that the exchange model
uses a different dataflow paradigm than all other operators.
While all other modules are based on demand-driven
dataflow (iterators, lazy evaluation), the producer-consumer
relationship of exchange uses data-driven dataflow (eager
evaluation). There are two reasons for this change in paradigms.
First, we intend to use the exchange operator also
for \emph{horizontal parallelism}, to be described below, which is
easier to implement with data-driven dataflow. Second, this
scheme removes the need for request messages. Even
though a scheme with request messages, e.g., using a semaphore,
would probably perform acceptably on a shared-memory
machine, we felt that it creates unnecessary control
overhead and delays. Since we believe that very hight
degrees of parallelism and very high-performance query
evaluation require a closely tied network, e.g., a hypercube,
of shared-memory machines, we decided to use a paradigm
for data exchange that has has been proven to perform well
in a shared-nothing database machine~{[12,13]}.

A run-time switch of exchange enables \emph{flow control}
or \emph{back pressure} using an additional semaphore. If the
producer is significantly faster than the consumer, the producer
may pin a significant portion of the buffer, thus impeding
overall system performance. If flow control is enabled,
after a producer has inserted a new packet into the port, it
must request the flow control semaphore. After a consumer
has removed a packet from the port, it releases the flow
control semaphore. The initial value of the flow control
semaphore, e.g., $4$, determines how many packets the
producers may get ahead of the consumers.

Notice that flow control and demand-driven dataflow
are not the same. One significant difference is that flow
control allows some ``slack'' in the synchronization of
producer and consumer and therefore truly overlapped execution,
while demand-driven dataflow is a rather rigid structure
of request and delivery in which the consumer waits
while the producer works on its next output. The second
significant difference is that data-driven dataflow is easier to
combine efficiently with horizontal parallelism and partitioning.

\subsection{Horizontal Parallelism}

There are two forms of horizontal parallelism which
we call \emph{bushy parallelism} and \emph{intra-operator} parallelism. In
bushy parallelism, different CPU's execute different subtrees
of a complex query tree. Bushy parallelism and vertical
parallelism are forms of \emph{inter-operator} parallelism. Intra-operator
parallelism means that several CPU's perform the
same operator on different subsets of a stored dataset or an
intermediate result\footnote{%
A fourth form of parallelism is inter-query parallelism,
i.e., the ability of a database management system to work on
several queries concurrently. In the current version, Volcano
does not support inter-query parallelism. A fifth and sixth form
of parallelism that can be used for database operations involve
hardware vector processing~{[30]} and pipelining in the instruction
execution. Since Volcano is a software architecture and following
the analysis in~{[8]}, we do not consider hardware parallelism further.
}.

Bushy parallelism can easily be implemented by
inserting one or two exchange operators into a query tree.
For example, in order to sort two inputs into a merge-join
in parallel, the first or both inputs are separated from the
merge-join by an exchange operation\footnote{%
In general, sorted streams can be piped directly into the
join, both in the single-process and the multi-process case
Volcano's sort operator includes a parameter ``final merge fan-in''
that allows sharing the merge space by two sort operators performing
the final merge in an interleaved fashion as requested by
the merge join operator.
}. The parent process
turns to the second sort immediately after forking the child
process that will produce the first input in sorted order.
Thus, the two sort operations are working in parallel.

Intra-operator parallelism requires data partitioning.
Partitioning of stored datasets is achieved by using multiple
files, preferably on different devices. Partitioning of intermediate
results is implemented by including multiple queues
in a port. If there are multiple consumer processes, each
uses its own input queue. The producers use a support
function to decide into which of the queues (or actually.
into which of the packets being filled by the producer) an
output record must go. Using a support function allows
implementing round-robin-, key-range-, or hash-partitioning.

If an operator or an operator subtree is executed in
parallel by a \emph{group} of processes, one of them is designated
the \emph{master}. When a query tree is \emph{opened}, only one process
is running, which is naturally the master. When a master
forks a child process in a producer-consumer relationship,
the child process becomes the master within its group. The
first action of the master producer is to determine how
many slaves are needed by calling an appropriate support
function. If the producer operation is to run in parallel, the
master producer forks the other producer processes.

Gerber pointed out that such a centralized scheme is
suboptimal for high degrees of parallelism~{[15]}. When we
changed our initial implementation from forking all producer
processes by the master to using a \emph{propagation tree} scheme,
we observed significant performance improvements. In such
a scheme, the master forks one slave, then both fork a new
slave each, then all four fork a new slave each, etc. This
scheme has been used very effectively for broadcast communication
and synchronization in binary hypercubes.

Even after optimizing the forking scheme, its overhead
is not negligible. We have considered using \emph{primed}
processes, i.e., processes that are always present and wait
for work packets. Primed processes are used in many commercial
database systems. Since portable distribution of
complied code (for support functions) is not trivial, we
delayed this change and plan on using primed processes
only when we move to an environment with multiple
shared-memory machines\footnote{%
In fact, this work is currently under way.
}. Others have also observed the
high cost of process creation and have provided alternatives,
in particular ``light-weight'' processes in various forms. e.g.,
in Mach~{[1]}.

After all producer processes are forked. they run
without further synchronization among themselves. with two
exceptions. First. when accessing a shared data structure,
e.g., the port to the consumers or a buffer table, short-term
locks must be acquired for the duration of one liked-list
insertion. Second, when a producer group is also a consumer
group, i.e., there are at least two exchange operators
and three process groups involved in a vertical pipeline, the
processes that are both consumers and producers synchronize
twice. During the (very short) interval between synchronizations,
the master of this group creates a port which serves
aU processes in its group.

When a \emph{close} request is propagated down the tree
and reaches the first exchange operator, the master
consumer's \texttt{close\_exchange} procedure informs all producer
processes that they are allowed to close down using the
semaphore mentioned above in the discussion on vertical
parallelism. If the producer processes are also consumers,
the master of the process group informs its producers, etc.
In this way, all operators are shut down in an orderly
fashion, and the entire query evaluation is self-scheduling.

\subsection{An Example}

Let us consider an example. Assume a query with
four operators, \(A\), \(B\), \(C\), and \(D\) such that \(A\) calls \(B\)'s, \(B\)
calls \(C\)'s, and \(C\) calls \(D\)'s \emph{open}, \emph{close}, and \emph{next} procedures.
Now assume that this query plan is to be run in
three process groups, called \(A\), \(BC\), and \(D\). This requires
an exchange operator between operators \(A\) and \(B\), say \(X\),
and one between \(C\) and \(D\), say \(Y\). \(B\) and \(C\) continue to
pass records via a simple procedure call to the \(C\)'s \emph{next}
procedure without crossing process boundaries. Assume
further that \(A\) runs as a single process, \(A_0\), while \(BC\) and
\(D\) run in parallel in processes \(BC_0\) to \(BC_2\) and \(D_0\) to \(D_3\),
for a total of eight processes.

A calls $X$'s \emph{open}, \emph{close}, and \emph{next} procedures instead
of $B$'s~(Figure 3a), without knowledge that a process boundary
will be crossed, a consequence of anonymous inputs in
Volcano When $X$ is opened, it creates a port with one
mput queue for $A_0$ and forks $BC_0$~(Figure 3b), which in
turn forks $BC_1$ and $BC_2$~(Figure 3c). When the $BC$ group
\emph{opens} $Y$, $BC_0$ to $BC_2$ synchronize, and wait until the $Y$
operator in process $BC_0$ has initialized a port with three
input queues. $BC_0$ creates the port and stores its location
at an address known only to the $BC$ processes. Then $BC_0$
to $BC_2$ synchronize again, and $BC_1$ and $BC_2$ get the port
information from its location. Next, $BC_0$ forks $D_0$~(Figure 3d)
which in turn forks $D_1$ to $D_3$~(Figure 3e).

\begin{figure}
  \centering
  \includegraphics{fig3.png}
\end{figure}

When the $D$ operators have exhausted their inputs in
$D_0$ to $D_3$, they return an \emph{end-of-stream} indicator to the
driver parts of $Y$. In each $D$ process, $Y$ flags its last
packets to each of the $BC$ processes (i.e. a total of \(3\times{}4=12\)
flagged packets) with an \emph{end-of-stream} tag and then waits
on a semaphore for permission to \emph{close}. The copies of the
$Y$ operator in the $BC$ processes count the number of tagged
packets, after four tags (the number of producers or $D$
processes), they have exhausted their inputs, and a call by
$C$ to $Y$'s \emph{next} procedure will return an \emph{end-of-stream}
indicator. In effect, the \emph{end-of-stream} indicator has been
propagated from the $D$ operators to the $C$ operators. In due
turn, $C$, $B$, and then the driver part of $X$ will receive an
\emph{end-of-stream} indicator. After receiving three tagged packets,
$X$'s \emph{next} procedure in $A_0$ will indicate \emph{end-of-stream}
to $A$.

When \emph{end-of-stream} reaches the root operator of the
query, $A$, the query tree is closed. Closing the exchange
operator $X$ includes releasing the semaphore that allows the
$BC$ processes to shut down~(Figure 3f). The $X$ driver in
each $BC$ process \emph{closes} its input, operator $B$. $B$ \emph{closes} $C$,
and $C$ \emph{closes} $Y$. Closing $Y$ in $BC_1$ and $BC_2$ is an empty
operation. When the process $BC_0$ \emph{closes} the exchange
operator $Y$, $Y$ permits the $D$ processes to shut down by
releasing a semaphore. After the processes of the $D$ group
have closed all files and deallocated all temporary data
structures, e.g., hash tables, they indicate the fact to $Y$ in
$BC_0$ using another semaphore, and $Y$'s \emph{close} procedure
returns to its caller, $C$'s \emph{close} procedure, while the $D$
processes terminate~(Figure 3g). When all $BC$ processes
have \emph{closed} down, $X$'s \emph{close} procedure indicates the fact to
$A_0$ and query evaluation terminates~(Figure 3h).

\subsection{Variants of the Exchange Operator}

There are a number of situations for which the
\emph{exchange} operator described so far required some modifications
or extensions. In this section, we outline additional
capabilities implemented in Volcano’s exchange operator.

\end{document}
