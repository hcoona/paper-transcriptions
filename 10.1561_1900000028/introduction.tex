\hypertarget{introduction}{%
\chapter{Introduction}\label{introduction}}

Less than 10 years after Bayer and McCreight {[}7{]} introduced B-trees,
and now more than a quarter century ago, Comer called B-tree indexes
ubiquitous {[}27{]}. Gray and Reuter asserted that ``B-trees are by far
the most important access path structure in database and file systems''
{[}59{]}. B-trees in various forms and variants are used in databases,
information retrieval, and file systems. It could be said that the
world's information is at our fingertips because of B-trees.

\hypertarget{perspectives-on-b-trees}{%
\section{Perspectives on B-trees}\label{perspectives-on-b-trees}}

Figure 1.1 shows a very simple B-tree with a root node and four leaf
nodes. Individual records and keys within the nodes are not shown. The
leaf nodes contain records with keys in disjoint key ranges. The root
node contains pointers to the leaf nodes and separator keys that divide
the key ranges in the leaves. If the number of leaf nodes exceeds the
number of pointers and separator keys that fit in the root node, an
intermediate layer of ``branch'' nodes is introduced. The separator keys
in the root node divide key ranges covered by the branch nodes (also
known as internal, intermediate, or interior nodes), and separator keys
in the branch nodes divide key ranges in the leaves. For very large data
collections, B-trees with multiple layers of branch nodes are used. One
or two branch levels are common in B-trees used as database indexes.

Fig. 1.1 A simple B-tree with root node and four leaf nodes.

Complementing this ``data structures perspective'' on B-trees is the
following ``algorithms perspective.'' Binary search in a sorted array
permits efficient search with robust performance characteristics. For
example, a search among 10\textsuperscript{9} or 2\textsuperscript{30}
items can be accomplished with only 30 comparisons. If the array of data
items is larger than memory, however, some form of paging is required,
typically relying on virtual memory or on a buffer pool. It is fairly
inefficient with respect to I/O, however, because for all but the last
few comparisons, entire pages containing tens or hundreds of keys are
fetched but only a single key is inspected. Thus, a cache might be
introduced that contains the keys most frequently used during binary
searches in the large array. These are the median key in the sorted
array, the median of each resulting half array, the median of each
resulting quarter array, etc., until the cache reaches the size of a
page. In effect, the root of a B-tree is this cache, with some
flexibility added in order to enable array sizes that are not powers of
two as well as efficient insertions and deletions. If the keys in the
root page cannot divide the original large array into sub-arrays smaller
than a single page, keys of each sub-array are cached, forming branch
levels between the root page and page-sized sub-arrays.

B-tree indexes perform very well for a wide variety of operations that
are required in information retrieval and database management, even if
some other index structure is faster for some individual index
operations. Perhaps the ``B'' in their name ``B-trees'' should stand for
their balanced performance across queries, updates, and utilities.
Queries include exact-match queries (``='' and ``in'' predicates), range
queries (``\emph{\textless{}}'' and ``between'' predicates), and full
scans, with sorted output if required. Updates include insertion,
deletion, modifications of existing data associated with a specific key
value, and ``bulk'' variants of those operations, for example bulk
loading new information and purging out-of-date records. Utilities
include creation and removal of entire indexes, defragmentation, and
consistency checks. For all of those operations, including incremental
and online variants of the utilities, B-trees also enable efficient
concurrency control and recovery.

\hypertarget{purpose-and-scope}{%
\section{Purpose and Scope}\label{purpose-and-scope}}

Many students, researchers, and professionals know the basic facts about
B-tree indexes. Basic knowledge includes their organization in nodes
including one root and many leaves, the uniform distance between root
and leaves, their logarithmic height and logarithmic search effort, and
their efficiency during insertions and deletions. This survey briefly
reviews the basics of B-tree indexes but assumes that the reader is
interested in more detailed and more complete information about modern
B-tree techniques.

Commonly held knowledge often falls short when it comes to deeper topics
such as concurrency control and recovery or to practical topics such as
incremental bulk loading and structural consistency checking. The same
is true about the many ways in which B-trees assist in query processing,
e.g., in relational databases. The goal here is to make such knowledge
readily available as a survey and as a reference for the advanced
student or professional.

The present survey goes beyond the ``classic'' B-tree references {[}7,
8, 27, 59{]} in multiple ways. First, more recent techniques are
covered, both research ideas and proven implementation techniques.
Whereas the first twenty years of B-tree improvements are covered in
those references, the last twenty years are not. Second, in addition to
core data structure and algorithms, the present survey also discusses
their usage, for example in query processing and in efficient update
plans. Finally, auxiliary algorithms are covered, for example
defragmentation and consistency checks.

During the time since their invention, the basic design of B-trees has
been improved upon in many ways. These improvements pertain to
additional levels in the memory hierarchy such as CPU caches, to
multi-dimensional data and multi-dimensional queries, to concurrency
control techniques such as multi-level locking and key range locking, to
utilities such as online index creation, and to many more aspects of
B-trees. Another goal here is to gather many of these improvements and
techniques in a single document.

The focus and primary context of this survey are B-tree indexes in
database management systems, primarily in relational databases. This is
reflected in many specific explanations, examples, and arguments.
Nonetheless, many of the techniques are readily applicable or at least
transferable to other possible application domains of B-trees, in
particular to information retrieval {[}83{]}, file systems {[}71{]}, and
``No SQL'' databases and key-value stores recently popularized for web
services and cloud computing {[}21, 29{]}.

A survey of techniques cannot provide a comprehensive performance
evaluation or immediate implementation guidance. The reader still must
choose what techniques are required or appropriate for specific
environments and requirements. Issues to consider include the expected
data size and workload, the anticipated hardware and its memory
hierarchy, expected reliability requirements, degree of parallelism and
needs for concurrency control, the supported data model and query
patterns, etc.

\hypertarget{new-hardware}{%
\section{New Hardware}\label{new-hardware}}

Flash memory, flash devices, and other solid state storage technology
are about to change the memory hierarchy in computer systems in general
and in data management in particular. For example, most current software
assumes two levels in the memory hierarchy, namely RAM and disk, whereas
any further levels such as CPU caches and disk caches are hidden by
hardware and its embedded control software. Flash memory might also
remain hidden, perhaps as large and fast virtual memory or as fast disk
storage. The more likely design for databases, however, seems to be
explicit modeling of a memory hierarchy with three or even more levels.
Not only algorithms such as external merge sort but also storage
structures such as B-tree indexes will need a re-design and perhaps a
re-implementation.

Among other effects, flash devices with their very fast access latency
are about to change database query processing. They likely will shift
the break-even point toward query execution plans based on
index-to-index navigation, away from large scans and large set
operations such as sort and hash join. With more index-to-index
navigation, tuning the set of indexes including automatic incremental
index creation, growth, optimization, etc. will come more into focus in
future database engines.

As much as solid state storage will change tradeoffs and optimizations
for data structures and access algorithms, many-core processors will
change tradeoffs and optimizations for concurrency control and recovery.
High degrees of concurrency can be enabled only by appropriate
definitions of consistent states and of transaction boundaries, and
recovery techniques for individual transactions and for the system state
must support them. These consistent intermediate states must be defined
for each kind of index and data structure, and B-trees will likely be
first index structure for which such techniques are implemented in
production-ready database systems, file systems, and key-value stores.

In spite of future changes for databases and indexes on flash devices
and other solid state storage technology, the present survey often
mentions tradeoffs or design choices appropriate for traditional disk
drives, because much of the presently known and implemented techniques
have been invented and designed in this context. The goal is to provide
comprehensive background knowledge about B-trees for those researching
and implementing techniques appropriate for the new types of storage.

\hypertarget{overview}{%
\section{Overview}\label{overview}}

The next section (Section 2) sets out the basics as they may be found in
a college level text book. The following sections cover implementation
techniques for mature database management products. Their topics are
implementation techniques for data structures and algorithms (Section
3), transactional techniques (Section 4), query processing using B-trees
(Section 5), utility operations specific to B-tree indexes (Section 6),
and B-trees with advanced key structures (Section 7). These sections
might be more suitable for an advanced course on data management
implementation techniques and for a professional developer desiring
in-depth knowledge about B-tree indexes.
