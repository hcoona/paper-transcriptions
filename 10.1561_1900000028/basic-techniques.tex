\hypertarget{basic-techniques}{%
\chapter{Basic Techniques}\label{basic-techniques}}

B-trees enable efficient retrieval of records in the native sort order
the index because, in a certain sense, B-trees capture and preserve the
result of a sort operation. Moreover, they preserve the sort effort in a
representation that can accommodate insertions, deletions, and updates.
The relationship between B-trees and sorting can be exploited in many
ways; the most common ones are that a sort operation can be avoided if
an appropriate B-tree exists and that the most efficient algorithm for
B-tree creation eschews random ``insert'' operations and instead pays
the cost of an initial sort for the benefit of efficient ``append''
operations.

Figure 2.1 illustrates how a B-tree index can preserve or cache the sort
effort. With the output of a sort operation, the B-tree with root, leaf
nodes, etc. can be created very efficiently. A subsequent scan can
retrieve data sorted without additional sort effort. In addition to
preserving the sort effort over an arbitrary length of time, B-trees
also permit efficient insertions and deletions, retaining their native
sort order and enabling efficient scans in sorted order at any time.

Fig. 2.1 Caching the sort effort in a B-tree.

Ordered retrieval aids many database operations, in particular
subsequent join and grouping operations. This is true if the list of
sort keys required in the subsequent operation is precisely equal to or
a prefix of that in the B-tree. It turns out, however, that B-trees can
save a lot of sort effort in many more cases. A later section will
consider the relationship between B-tree indexes and database query
operations in detail.

B-trees share many of their characteristics with binary trees, raising
the question why binary trees are commonly used for in-memory data
structures and B-trees for on-disk data. The reason is quite simple:
disk drives have always been block-access devices, with a high overhead
per access. B-trees exploit disk pages by matching the node size to the
page size, e.g., 4 KB. In fact, B-trees on today's high-bandwidth disks
perform best with nodes of multiple pages, e.g., 64 KB or 256 KB.
Inasmuch as main memory should be treated as a block-access device when
accessed through CPU caches and their cache lines, B-trees in memory
also make sense. Later sections will resume this discussion of memory
hierarchies and their effect on optimal data structures and algorithm
for indexes and specifically B-trees.

B-trees are more similar to 2-3-trees, in particular as both data
structures have a variable number of keys and child pointers in a node.
In fact, B-trees can be seen as a generalization of 2-3-trees. Some
books treat them both as special cases of ($a,b$)-trees with
$a \geq 2$ and $b \geq 2a − 1$ {[}92{]}. The number of child
pointers in a canonical B-tree node varies between $N$ and
$2N − 1$. For a small page size and a particularly large key size,
this might indeed be the range between $2$ and $3$. The representation of a
single node in a 2-3-tree by linking two binary nodes also has a
parallel in B-trees, discussed later as B\textsuperscript{link}-trees.

Figure 2.2 shows a ternary node in a 2-3-tree represented by two binary
nodes, one pointing to the other half of the ternary node rather than a
child. There is only one pointer to this ternary node from a parent
node, and the node has three child pointers.

In a perfectly balanced tree such as a B-tree, it makes sense to count
the levels of nodes not from the root but from the leaves. Thus, leaves
are sometimes called level-0 nodes, which are children of level-1 nodes,
etc. In addition to the notion of child pointers, many family terms are
used in connection with B-trees: parent, grandparent, ancestor,
descendent, sibling, and cousin. Siblings are children of the same
parent node. Cousins are nodes in the same B-tree level with different
parent nodes but the same grandparent node. If the first common ancestor
is a greatgrandparent, the nodes are second cousins, etc. Family
analogies are not used throughout, however. Two siblings or cousins with
adjoining key ranges are called neighbors, because there is no commonly
used term for such siblings in families. The two neighbor nodes are
called left and right neighbors; their key ranges are called the
adjacent lower and upper key ranges.

Fig. 2.2 A ternary node in a 2-3-tree represented with binary nodes.

In most relational database management systems, the B-tree code is part
of the access methods module within the storage layer, which also
includes buffer pool management, lock manager, log manager, and more.
The relational layer relies on the storage layer and implements query
optimization, query execution, catalogs, and more. Sorting and index
maintenance span those two layers. For example, large updates may use an
update execution plan similar to a query execution plan to maintain each
B-tree index as efficiently as possible, but individual B-tree
modifications as well as read-ahead and write-behind may remain within
the storage layer. Details of such advanced update and prefetch
strategies will be discussed later.

In summary:

\begin{itemize}
\item
  B-trees are indexes optimized for paged environments, i.e., storage
  not supporting byte access. A B-tree node occupies a page or a set of
  contiguous pages. Access to individual records requires a buffer pool
  in byte-addressable storage such as RAM.
\item
  B-trees are ordered; they effectively preserve the effort spent on
  sorting during index creation. Differently than sorted arrays, B-trees
  permit efficient insertions and deletions.
\item
  Nodes are leaves or branch nodes. One node is distinguished as root
  node.
\item
  Other terms to know: parent, grandparent, ancestor, child, descendent,
  sibling, cousin, neighbor.
\item
  Most implementations maintain the sort order within each node, both
  leaf nodes and branch nodes, in order to enable efficient binary
  search.
\item
  B-trees are balanced, with a uniform path length in root-to-leaf
  searches. This guarantees uniformly efficient search.
\end{itemize}

\hypertarget{data-structures}{%
\section{Data Structures}\label{data-structures}}

In general, a B-tree has three kinds of nodes: a single root, a lot of
leaf nodes, and as many branch nodes as required to connect the root and
the leaves. The root contains at least one key and at least two child
pointers; all other nodes are at least half full at all times. Usually
all nodes have the same size, but this is not truly required.

The original design for B-trees has user data in all nodes. The design
used much more commonly today holds user data only in the leaf nodes.
The root node and the branch nodes contain only separator keys that
guide the search algorithm to the correct leaf node. These separator
keys may be equal to keys of current or former data, but the only
requirement is that they can guide the search algorithm.

This design has been called B\textsuperscript{+}-tree but it is nowadays
the default design when B-trees are discussed. The value of this design
is that deletion can affect only leaf nodes, not branch nodes and that
separator keys in branch nodes can be freely chosen within the
appropriate key range. If variable-length records are supported as
discussed later, the separator keys can often be very short. Short
separator keys increase the node fanout, i.e., the number of child
pointers per node, and decrease the B-tree height, i.e., the number of
nodes visited in a root-to-leaf search.

The records in leaf nodes contain a search key plus some associated
information. This information can be all the columns associated with a
table in a database, it can be a pointer to a record with all those
columns, or it can be anything else. In most parts of this survey, the
nature, contents, and semantics of this information are not important
and not discussed further.

Fig. 2.3 B-tree with root, branch nodes, and leaves.

In both branch nodes and leaves, the entries are kept in sorted order.
The purpose is to enable fast search within each node, typically using
binary search. A branch node with $N$ separator keys contains
$N$ + 1 child pointers, one for each key range between neighboring
separator keys, one for the key, range below the smallest separator key,
and one for the key range above the largest separator key.

Figure 2.3 illustrates a B-tree more complex than the one in Figure 1.1,
including one level of branch nodes between the leaves and the root. In
the diagram, the root and all branch nodes have fan-outs of 2 or 3. In a
B-tree index stored on disk, the fan-out is determined by the sizes of
disk pages, child pointers, and separator keys. Keys are omitted in
Figure 2.3 and in many of the following figures unless they are required
for the discussion at hand.

Among all possible node-to-node pointers, only the child pointers are
truly required. Many implementations also maintain neighbor pointers,
sometimes only between leaf nodes and sometimes only in one direction.
Some rare implementations have used parent pointers, too, e.g., a
Siemens product {[}80{]}. The problem with parent pointers is that they
force updates in many child nodes when a parent node is moved or split.
In a disk-based B-tree, all these pointers are represented as page
identifiers.

B-tree nodes may include many additional fields, typically in a page
header. For consistency checking, there are table or index identifier
plus the B-tree level, starting with 0 for leaf pages; for space
management, there is a record count; for space management with
variable-size records, there are slot count, byte count, and lowest
record offset; for data compression, there may be a shared key prefix
including its size plus information as required for other compression
techniques; for write-ahead logging and recovery, there usually is a
Page LSN (log sequence number) {[}95{]}; for concurrency control, in
particular in shared-memory systems, there may be information about
current locks; and for efficient key range locking, consistency
checking, and page movement as in defragmentation, there may be fence
keys, i.e., copies of separator keys in ancestor pages. Each field, its
purpose, and its use are discussed in a subsequent section.

\begin{itemize}
\item
  Leaf nodes contain key values and some associated information. In most
  B-trees, branch nodes including the root node contain separator keys
  and child pointers but no associated information.
\item
  Child pointers are essential. Sibling pointers are often implemented
  but not truly required. Parent pointers are hardly ever employed.
\item
  B-tree nodes usually contain a fixed-format page header, a
  variable-size array of fixed-size slots, and a variable-size data
  area. The header contains a slot counter, information pertaining to
  compression and recovery, and more. The slots serve space management
  for variable-size records.
\end{itemize}

\hypertarget{sizes-tree-height-etc.}{%
\section{Sizes, Tree Height, etc.}\label{sizes-tree-height-etc.}}

In traditional database designs, the typical size of a B-tree node is
4--8 KB. Larger B-tree nodes might seem more efficient for today's disk
drives based on multiple analyses {[}57, 86{]} but nonetheless are
rarely used in practice. The size of a separator key can be as large as
a record but it can also be much smaller, as discussed later in the
section on prefix B-trees. Thus, the typical fan-out, i.e., the number
of children or of child pointers, is sometimes only in the tens,
typically in the hundreds, and sometimes in the thousands.

If a B-tree contains $N$ records and $L$ records per leaf, the
B-tree requires $N/L$ leaf nodes. If the average number of children
per parent is $F$, the number of branch levels is
$\log_F(N/L)$. For example, the B-tree in
Figure 2.3 has 9 leaf nodes, a fan-out $F = 3$, and thus
$\log_3 9 = 2$ branch levels. Depending on the convention,
the height of this B-tree is $2$ (levels above the leaves) or $3$ (levels
including the leaves). In order to reflect the fact that the root node
usually has a different fan-out, this expression is rounded up. In fact,
after some random insertions and deletions, space utilization in the
nodes will vary among nodes. The average space utilization in B-trees is
usually given as about 70\% {[}75{]}, but various policies used in
practice and discussed later may result in higher space utilization. Our
goal here is not to be precise but to show crucial effects, basic
calculations, and the orders of magnitude of various choices and
parameters.

If a single branch node can point to hundreds of children, then the
distance between root and leaves is usually very small and 99\% or more
of all B-tree nodes are leaves. In other words, great-grandparents and
even more distant ancestors are rare in practice. Thus, for the
performance of random searches based on root-to-leaf B-tree traversals,
treatment of only 1\% of a B-tree index and thus perhaps only 1\% of a
database determine much of the performance. For example, keeping the
root of a frequently used B-tree index in memory benefits many searches
with little cost in memory or cache space.

\begin{itemize}
\item
  The B-tree depth (nodes along a root-to-leaf path) is logarithmic in
  the number of records. It is usually small.
\item
  Often more than 99\% of all nodes in a B-tree are leaf nodes.
\item
  B-tree pages are filled between 50\% and 100\%, permitting insertions
  and deletions as well as splitting and merging nodes. Average
  utilization after random updates is about 70\%.
\end{itemize}

\hypertarget{algorithms}{%
\section{Algorithms}\label{algorithms}}

The most basic, and also the most crucial, algorithm for B-trees is
search. Given a specific value for the search key of a B-tree or for a
prefix thereof, the goal is to find, correctly and as efficiently as
possible, all entries in the B-tree matching the search key. For range
queries, the search finds the lowest key satisfying the predicate.

A search requires one root-to-leaf pass. In each branch node, the search
finds the pair of neighboring separator keys smaller and larger than the
search key, and then continues by following the child pointer between
those two separator keys.

The number of comparisons during binary search among $L$ records in
a leaf is $\log_2(L)$, ignoring rounding effects.
Similarly, binary search among $F$ child pointers in a branch node
requires $\log_2(F)$ comparisons. The number of leaf
nodes in a B-tree with $N$ records and $L$ records per leaf is
$N/L$. The depth of a B-tree is
$\log_F(N/L)$, which is also the number of
branch nodes visited in a root-to-leaf search. Together, the number of
comparisons in a search inspecting both branch nodes and a leaf node is
$\log_F(N/L) \times \log_2(F)
+ \log_2(L)$. By elementary rules for algebra with
logarithms, the product term simplifies to
$\log_2(N/L)$ and then the entire expression
simplifies to $\log_2(N)$. In other words, node size
and record size may produce secondary rounding effects in this
calculation but the record count is the only primary influence on the
number of comparisons in a root-to-leaf search in a B-tree.

Figure 2.4 shows parts of a B-tree including some key values. A search
for the value 31 starts at the root. The pointer between the key values
7 and 89 is followed to the appropriate branch node. As the search key
is larger than all keys in that node, the right-most pointer is followed
to a leaf. A search within that node determines that the key value 31
does not exist in the B-tree. A search for key value 23 would lead to
the center node in Figure 2.4, assuming a convention that a separator
key serves as inclusive upper bound for a key range. The search cannot
terminate when the value 23 is found at the branch level. This is
because the purpose of most B-tree searches is retrieving the
information attached to each key and information contents exists only in
leaves in most B-tree implementations. Moreover, as can be seen for key
value 15 in Figure 2.4, a key that might have existed at some time in a
valid leaf entry may continue to serve as separator key in a nonleaf
node even after the leaf entry has been removed.

Fig. 2.4 B-tree with root-to-leaf search.

An exact-match query is complete after the search, but a range query
must scan leaf nodes from the low end of the range to the high end. The
scan can employ neighbor pointers if they exist in the B-tree.
Otherwise, parent and grandparent nodes and their child pointers must be
employed. In order to exploit multiple asynchronous requests, e.g., for
a B-tree index stored in a disk array or in network-attached storage,
parent and grandparent nodes are needed. Range scans relying on neighbor
pointers are limited to one asynchronous prefetch at-a-time and
therefore unsuitable for arrays of storage devices or for virtualized
storage.

Insertions start with a search for the correct leaf to place the new
record. If that leaf has the required free space, the insertion is
complete. Otherwise, a case called ``overflow,'' the leaf needs to be
split into two leaves and a new separator key must be inserted into the
parent. If the parent is already full, the parent is split and a
separator key is inserted into the appropriate grandparent node. If the
root node needs to be split, the B-tree grows by one more level, i.e., a
new root node with two children and only one separator key. In other
words, whereas the depth of many tree data structures grows at the
leaves, the depth of B-trees grows at the root. This is what guarantees
perfect balance in a B-tree. At the leaf level, B-trees grow only in
width, enabled by the variable number of child nodes in each parent
node. In some implementations of B-trees, the old root page becomes the
new root page and the old root contents are distributed into the two
newly allocated nodes. This is a valuable technique if modifying the
page identifier of the root node in the database catalogs is expensive
or if the page identifier is cached in compiled query execution plans.

Figure 2.5 shows the B-tree of Figure 2.4 after insertion of the key 22
and a resulting leaf split. Note that the separator key propagated to
the parent node can be chosen freely; any key value that separates the
two nodes resulting from the split is acceptable. This is particularly
useful for variable-length keys: the shortest possible separator key can
be employed in order to reduce space requirements in the parent node.

Fig. 2.5 B-tree with insertion and leaf split.

Some implementations of B-trees delay splits as much as possible, for
example by load balancing among siblings such that a full node can make
space for an insertion. This design raises the code complexity but also
the space utilization. High space utilization enables high scan rates if
data transfer from storage devices is the bottleneck. Moreover, splits
and new page allocations may force additional seek operations during a
scan that are expensive in disk-based B-trees.

Deletions also start with a search for the correct leaf that contains
the appropriate record. If that leaf ends up less than half full, a case
called ``underflow,'' either load balancing or merging with a sibling
node can ensure the traditional B-tree invariant that all nodes other
than the root be at least half full. Merging two sibling nodes may
result in underflow in their parent node. If the only two children of
the root node merge, the resulting node becomes the root and the old
root is removed. In other words, the depth of B-trees both grows and
shrinks at the root. If the page identifier of the root node is cached
as mentioned earlier, it might be practical to move all contents to the
root node and de-allocate the two children of the root node.

Figure 2.6 shows the B-tree from Figure 2.5 after deletion of key value
23. Due to underflow, two leaves were merged. Note that the separator
key 23 was not removed because it still serves the required function.

Many implementations, however, avoid the complexities of load balancing
and of merging and simply let underflows persist. A subsequent insertion
or defragmentation will presumably resolve it later. A recent study of
worst case and average case behaviors of B-trees concludes that ``adding
periodic rebuilding of the tree,... the data structure... is
theoretically superior to standard B\textsuperscript{+}-trees in many
ways {[}and{]}... rebalancing on deletion can be considered harmful''
{[}116{]}.

Fig. 2.6 Deletion with load balancing.

Updates of key fields in B-tree records often require deletion in one
place and insertion in another place in the B-tree. Updates of nonkey
fixed-length fields happen in place. If records contain variable-length
fields, a change in the record size might force overflow or underflow
similar to insertion or deletion.

The final basic B-tree algorithm is B-tree creation. Actually, there are
two algorithms, characterized by random insertions and by prior sorting.
Some database products used random insertions in their initial releases
but their customers found creation of large indexes very slow. Sorting
the future index entries prior to B-tree creation permits many
efficiency gains, from massive I/O savings to various techniques saving
CPU effort. As the future index grows larger than the available buffer
pool, more and more insertions require reading, updating, and writing a
page. A database system might also require logging each such change in
the recovery log, whereas most systems nowadays employ non-logged index
creation, which is discussed later. Finally, a stream of append
operations also encourages a B-tree layout on disk that permits
efficient scans with a minimal number of disk seeks. Efficient sort
algorithms for database systems have been discussed elsewhere {[}46{]}.

\begin{itemize}
\item
  If binary search is employed in each node, the number of comparisons
  in a search is independent of record and node sizes except for
  rounding effects.
\end{itemize}

\begin{itemize}
\item
  B-trees support both equality (exact-match) predicates and range
  predicate. Ordered scans can exploit neighbor pointers or ancestor
  nodes for deep (multi-page) read-ahead.
\item
  Insertions use existing free space or split a full node into two
  half-full nodes. A split requires adding a separator key and a child
  pointer to the parent node. If the root node splits, a new root node
  is required and the B-tree grows by one level.
\item
  Deletions may merge half-full nodes. Many implementations ignore this
  case and rely on subsequent insertions or defragmentation
  (reorganization) of the B-tree.
\item
  Loading B-trees by repeated random insertion is very slow; sorting
  future B-tree entries permits efficient index creation.
\end{itemize}

\hypertarget{b-trees-in-databases}{%
\section{B-trees in Databases}\label{b-trees-in-databases}}

Having reviewed the basics of B-trees as a data structure, it is also
required to review the basics of B-trees as indexes, for example in
database systems, where B-tree indexes have been essential and
ubiquitous for decades. Recent developments in database query processing
have focused on improvements of large scans, e.g., by sharing scans
among concurrent queries {[}33, 132{]}, by a columnar data layout that
reduces the scan volume in many queries {[}17, 121{]}, or by predicate
evaluation by special hardware, such as FPGAs. The advent of flash
devices in database servers will likely result in more index usage in
database query processing --- their fast access times encourage small
random accesses whereas traditional disk drives with high capacity and
high bandwidth favor large sequential accesses. With the B-tree index
the default choice in most systems, the various roles and usage patterns
of B-tree indexes in databases deserve attention. We focus here on
relational databases because their conceptual model is fairly close to
the records and fields used in the storage layer of all database systems
as well as other storage services.

In a relational database, all data is logically organized in tables with
columns identified by name and rows identified by unique values in
columns forming the table's primary key. Relationships among tables are
captured in foreign key constraints. Relationships among rows are
expressed in foreign key columns, which contain copies of primary key
values elsewhere. Other forms of integrity constraints include
uniqueness of one or more columns; uniqueness constraints are often
enforced using a B-tree index.

The simplest representation for a database table is a heap, a collection
of pages holding records in no particular order, although often in the
order of insertion. Individual records are identified and located by
means of page identifier and slot number (see below in Section 3.3),
where the page identifier may include a device identifier. When a record
grows due to an update, it might need to move to a new page. In that
case, either the original location retains ``forwarding'' information or
all references to the old location, e.g., in indexes, must be updated.
In the former case, all future accesses incur additional overhead,
possibly the cost of a disk read; in the latter case, a seemingly simple
change may incur a substantial unforeseen and unpredictable cost.

Figure 2.7 shows records (solid lines) and pages (dashed lines) within a
heap file. Records are of variable size. Modifications of two records
have changed their sizes and forced moving the record contents to
another page with forwarding pointers (dotted lines) left behind in the
original locations. If an index points to records in this file,
forwarding does not affect the index contents but does affect the access
times in subsequent queries.

If a B-tree structure rather than a heap is employed to store all
columns in a table, it is called a primary index here. Other commonly
used names include clustered index or index-organized table. In a
secondary index, also commonly called a non-clustered index, each entry
must contain a reference to a row or a record in the primary index. This
reference can be a search key in a primary index or it can be a record
identifier including a page identifier. The term ``reference'' will
often be used below to refer to either one. References to records in a
primary index are also called bookmarks in some contexts.

\begin{quote}
\includegraphics[width=2.37667in,height=0.49in]{./media/image1.png}
\end{quote}

Fig. 2.7 Heap file with variable-length records and forwarding pointers.

Fig. 2.8 Index navigation with a search key.

Both designs, reference by search key and reference by record
identifier, have advantages and disadvantages {[}68{]}; there is no
perfect design. The former design requires a root-to-leaf search in the
primary index after each search in the secondary index. Figure 2.8
illustrates the double index search when the primary data structure for
a table is a primary index and references in secondary indexes use
search keys in the primary index. The search key extracted from the
query requires an index search and root-to-leaf traversal in the
secondary index. The information associated with a key in the secondary
index is a search key for the primary index. Thus, after a successful
search in the secondary index, another B-tree search is required in the
primary index including a root-to-leaf traversal there.

The latter design permits faster access to records in the primary index
after a search in the secondary index. When a leaf in the primary index
splits, however, this design requires many updates in all relevant
secondary indexes. These updates are expensive with many I/O operations
and B-tree searches, they are infrequent enough to be always surprising,
they are frequent enough to be disruptive, and they impose a substantial
penalty due to concurrency control and logging for these updates.

A combination is also possible, with the page identifier as a hint and
the search key as the fallback. This design has some intrinsic
difficulties, e.g., when a referenced page is de-allocated and later
reallocated for a different data structure. Finally, some systems employ
clustering indexes over heap files; their goal is to keep the heap file
sorted if possible but nonetheless enable fast record access via record
identifiers.

In databases, all B-tree keys must be unique, even if the user-defined
B-tree key columns are not. In a primary index, unique keys are required
for correct retrieval. For example, each reference found in a secondary
index must guide a query to exactly one record in the primary index ---
therefore, the search keys in the primary index must be unambiguous in
the reference-by-search-key design. If the user-defined search key for a
primary index is a person's lastname, values such as ``Smith'' are unlikely
to safely identify individual records in the primary index.

In a secondary index, unique keys are required for correct deletion.
Otherwise, deletion of a logical row and its record in a primary index
might be followed by deletion of the wrong entry in a non-unique
secondary index. For example, if the user-defined search in a secondary
index is a person's first name, deletion of a record in the primary
index containing ``Bob Smith'' must delete only the correct matching
record in the secondary index, not all records with search key ``Bob''
or a random such record.

If the search key specified during index creation is unique due to
uniqueness constraints, the user-defined search key is sufficient. Of
course, once a logical integrity constraint is relied upon in an index
structure, dropping the integrity constraint must be prevented or
followed by index reorganization. Otherwise, some artificial field must
be added to the user-defined index key. For primary indexes, some
systems employ a globally unique ``database key'' such as a microsecond
timestamp, some use an integer value unique within the table, and some
use an integer value unique among B-tree entries with the same value in
the user-defined index key. For secondary indexes, most systems simply
add the reference to the search key of the primary index.

B-tree entries are kept sorted on their entire unique key. In a primary
index, this aids efficient retrieval; in a secondary index, it aids
efficient deletion. Moreover, the sorted lists of references for each
unique search key enable efficient list intersection and union. For
example, for a query predicate ``A = 5 and B = 15,'' sorted lists of
references can be obtained from indexes on columns A and B and their
intersection computed by a simple merge algorithm.

The relationships between tables and indexes need not be as tight and
simple as discussed so far. A table may have calculated columns that are
not stored at all, e.g., the difference (interval) between two date
(timestamp) columns. On the other hand, a secondary index might be
organized on such a column, and in this case necessarily store a copy of
the column. An index might even include calculated columns that
effectively copy values from another table, e.g., the table of order
details might include a customer identifier (from the table of orders)
or a customer name (from the table of customers), if the appropriate
functional dependencies and foreign key constraints are in place.

Another relationship that is usually fixed, but need not be, is the
relationship between uniqueness constraints and indexes. Many systems
automatically create an index when a uniqueness constraint is defined
and drop the index when the constraint is dropped. Older systems did not
support uniqueness constraints at all but only unique indexes. The index
is created even if an index on the same column set already exists, and
the index is dropped even if it would be useful in future queries. An
alternative design merely requires that some index with the appropriate
column set exists while a uniqueness constraint is active. For instant
definition of a uniqueness constraint with existing useful index, a
possible design counts the number of unique keys during each insertion
and deletion in any index. In a sorted index such as a B-tree, a count
should be maintained for each key prefix, i.e., for the first key field
only, the first and second key fields together, etc. The required
comparisons are practically free as they are a necessary part of
searching for the correct insertion or deletion point. A new uniqueness
constraint is instantly verified if the count of unique key values is
equal to the record count in the index.

Finally, tables and indexes might be partitioned horizontally (into sets
of rows) or vertically (into sets of columns), as will be discussed
later. Partitions usually are disjoint but this is not truly required.
Horizontal partitioning can be applied to a table such that all indexes
of that table follow the same partitioning rule, sometimes called
``local indexes.'' Alternatively, partitioning can be applied to each
index individually, with secondary indexes partitioned with their own
partitioning rule different from the primary index, which is sometimes
called ``global indexes.'' In general, physical database design or the
separation of logical tables and physical indexes remains an area of
opportunity and innovation.

\begin{itemize}
\item
  B-trees are ubiquitous in databases and information retrieval.
\item
  If multiple B-trees are related, e.g., the primary index and the
  secondary index of a database table, pointers can be physical
  addresses (record identifiers) or logical references (search keys in
  the primary index). Neither choice is perfect, both choices have been
  used.
\item
  B-tree entries must be unique in order to ensure correct updates and
  deletions. Various mechanisms exist to force uniqueness by adding an
  artificial key value.
\item
  Traditional database design rigidly connects tables and B-trees, much
  more rigidly then truly required.
\end{itemize}

\hypertarget{b-trees-versus-hash-indexes}{%
\section{B-trees Versus Hash
Indexes}\label{b-trees-versus-hash-indexes}}

It might seem surprising that B-tree indexes have become ubiquitous
whereas hash indexes have not, at least not in database systems. Two
arguments seem to strongly favor hash indexes. First, hash indexes
should save I/O costs due to a single I/O per look-up, whereas B-trees
require a complete root-to-leaf traversal for each search. Second, hash
indexes and hash values should also save CPU effort due to efficient
comparisons and address calculations. Both of these arguments have only
very limited validity, however, as explained in the following
paragraphs. Moreover, B-trees have substantial advantages over hash
indexes with respect to index creation, range predicates, sorted
retrieval, phantom protection in concurrency control, and more. These
advantages, too, are discussed in the following paragraphs. All
techniques mentioned here are explained in more depth in subsequent
sections.

With respect to I/O savings, it turns out that fairly simple
implementation techniques can render B-tree indexes competitive with
hash indexes in this regard. Most B-trees have a fan-out of 100s or
1,000s. For example, for node of 8 KB and records of 20 bytes, 70\%
utilization means 140 child nodes per parent node. For larger node sizes
(say 64 KB), good defragmentation (enabling run-length encoding of child
pointers, say 2 bytes on average), key compression using prefix and
suffix truncation (say 4 bytes on average per entry), 70\% utilization
means 5,600 child nodes per parent node. Thus, root-to-leaf paths are
short and more than 99\% or even 99.9\% of pages in a B-tree are leaf
nodes. These considerations must be combined with the traditional rule
that many database servers run with memory size equal to 1--3\% of
storage size. Today and in the future, the percentage might be higher,
up to 100\% for in-memory databases. In other words, for any B-tree
index that is ``warm'' in the buffer pool, all branch nodes will be
present in the buffer pool. Thus, each B-tree search only requires a
single I/O, the leaf page. Moreover, the branch nodes could be fetched
into the buffer pool in preparation of repeated look-up operations,
perhaps even pinned in the buffer pool. If they are pinned, further
optimizations could be applied, e.g., spreading separator keys into a
separate array such that interpolation search is most effective,
replacing or augmenting child pointers in form of page identifiers with
child pointers in form of memory pointers, etc.

Figure 2.9 illustrates the argument. All B-tree levels but the leaf
nodes easily fit into the buffer pool in RAM memory. For leaf pages, a
buffer pool might employ the least-recently-used (LRU) replacement
policy. Thus, for searches with random search keys, only a single I/O
operation is required, similar to a hash index if one is available in a
database system.

With respect to CPU savings, B-tree indexes can compete with hash
indexes using a few simple implementation techniques. B-tree indexes
support a wide variety of search keys, but they also support very simple
ones such as hash values. Where hash indexes can be used, a B-tree on
hash values will also provide sufficient functionality. In other cases,
a ``poor man's normalized key'' can be employed and even be sufficient,
rendering all additional comparison effort unnecessary. Later sections
discuss normalized keys, poor man's normalized keys, and caching poor
man's normalized keys in the ``indirection vector'' that is required for
variable-size records. In sum, poor man's normalized keys and the
indirection vector can behave similarly to hash values and hash buckets.

Fig. 2.9 B-tree levels and buffering.

B-trees also permit direct address calculation. Specifically,
interpolation search may guide the search faster than binary search. A
later section discusses interpolation search including avoiding
worst-case behavior of pure interpolation by switching to binary search
after two interpolation steps, and more.

While B-tree indexes can be competitive with hash indexes based on a few
implementation techniques, B-trees also have distinct advantages over
hash indexes. For example, space management in B-trees is very
straightforward. In the simplest implementations, full nodes are split
into two halves and empty nodes are removed. Multiple schemes have been
invented for hash indexes to grow gracefully, but none seems quite as
simple and robust. Algorithms for graceful shrinking of hash indexes are
not widely known.

Probably the strongest arguments for B-trees over hash indexes pertain
to multi-field indexes and to nonuniform distributions of key values. A
hash index on multiple fields requires search keys for all those fields
such that a hash value can be calculated. A B-tree index, on the other
hand, can efficiently support exact-match queries for a prefix of the
index key, i.e., any number of leading index fields. In this way, a
B-tree with $N$ search keys can be as useful as $N$ hash
indexes. In fact, B-tree indexes can support many other forms of
queries; it is not even required that the restricted fields are leading
fields in the B-tree's sort order {[}82{]}.

With respect to nonuniform (``skewed'') distributions of key values,
imagine a table with 10\textsuperscript{9} rows that needs a secondary
index on a column with the same value in 10\% of the rows. A hash index
requires introduction of overflow pages, with additional code for index
creation, insertion, search, concurrency control, recovery, consistency
checks, etc.

For example, when a row in the table is deleted, an expensive search is
required before the correct entry in the secondary index can be found
and removed, whereupon overflow pages might need to be merged. In a
B-tree, entries are always unique, if necessary by appending a field to
the search key as discussed earlier. In hash indexes, the additional
code requires additional execution time as well as additional effort for
testing and maintenance. Due to the well-defined sort order in B-trees,
neither special code nor extra time is required in any of the index
functions.

Another strong argument in favor of B-trees is index creation. After
extracting future index entries and sorting them, B-tree creation is
simple and very efficient, even for the largest data collections. An
efficient, general-purpose sorting algorithm is readily available in
most systems managing large data. Equally efficient index creation for
hash indexes would require a special-purpose algorithm, if it is
possible at all. Index creation by repeated random insertions is
extremely inefficient for both B-trees and hash indexes. Techniques for
online index creation (with concurrent database updates) are well known
and widely implemented for B-trees but not for hash indexes.

An obvious advantage of B-trees over hash indexes is the support for
ordered scans and for range predicates. Ordered scans are important for
key columns and set operations such as merge join and grouping; range
predicates are usually more important for nonkey columns. In other
words, B-trees are superior to hash indexes for both key columns and
nonkey columns in relational databases, also known as dimensions and
measures in online analytical processing. Ordering also has advantages
for concurrency control, in particular phantom protection by means of
key range locking (covered in detail later) rather than locking key
values only.

Taken together, these arguments favor B-trees over hash indexes as a
general indexing technique for databases and many other data
collections. Where hash indexes seem to have an advantage, appropriate
B-tree implementation techniques minimize it. Thus, very few database
implementation teams find hash indexes among the opportunities or
features with a high ratio of benefit and effort, in particular if
B-tree indexes are required in any case in order to support range
queries and ordered scans.

While nodes of 10 KB likely result in B-trees with multiple levels of
branch nodes, nodes of 1 MB probably do not. In other words, the
considerations above may apply to B-tree indexes on flash storage but
probably not on disks. For disks, it is probably best to cache all
branch nodes in memory and to employ fairly small leaf nodes such that
neither transfer bandwidth nor buffer space is wasted on unwanted
records.

\begin{itemize}
\item
  B-tree indexes are ubiquitous, whereas hash indexes are not, even
  though hash indexes promise exact-match look-up with direct address
  calculation in the hash directory and a single I/O.
\item
  B-tree software can provide similar benefits if desired. In addition,
  B-trees support efficient index creation based on sorting, support for
  exact match predicates and for partial predicates, graceful
  degradation in case of duplicate or distribution skew among the key
  values, and ordered scans.
\end{itemize}

\hypertarget{summary}{%
\section{Summary}\label{summary}}

In summary of this section on the basic data structure, B-trees are
ordered, balanced search trees optimized for block-access devices such
as disks. They guarantee good performance for various types of searches
well as for insertions, deletions, and updates. Thus, they are
particularly suitable to databases and in fact have been ubiquitous in
databases for decades.

Over time, many techniques have been invented and implemented beyond the
basic algorithms and data structures. These practical improvements are
covered in the next few sections.

The present section focuses on data structures and algorithms found in
mature data management systems but usually not in college-level text
books; the subsequent sections cover transactional techniques, B-trees
and their usage in database query processing, and B-tree utilities.

While only a single sub-section below is named ``data compression,''
almost all sub-sections pertain to compression in some form: storing
fewer bytes per record, describing multiple records together, comparing
fewer bytes in each search, modifying fewer bytes in each update, and
avoiding fragmentation and wasted space. Efficiency in space and time is
the theme of this section.

The following sub-sections are organized such that the first group
pertains to the size and internal structure of nodes, the next group to
compression specific to B-trees, and the last group to management of
free space. Most of the techniques in the individual sub-sections are
independent of others, although certain combinations may ease their
implementation. For example, prefix- and suffix-truncation require
detailed and perhaps excessive record keeping unless key values are
normalized into binary strings.
