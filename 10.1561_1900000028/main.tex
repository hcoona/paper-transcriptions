\PassOptionsToPackage{unicode=true}{hyperref} % options for packages loaded elsewhere
\PassOptionsToPackage{hyphens}{url}
%
\documentclass[a4paper,11pt,twoside,openright]{book}

\usepackage{ifxetex}
\ifxetex{}
\else
  \errmessage{Must be built with xelatex}
\fi

\usepackage{amssymb,amsmath}
\usepackage{fourier}
\usepackage{inconsolata}
% Math
\usepackage[binary-units]{siunitx}

\usepackage{caption}
\usepackage{authblk}
\usepackage{enumitem}
\usepackage{footnote}

% Table
\usepackage{tabu}
\usepackage{longtable}
\usepackage{booktabs}
\usepackage{multirow}

% Verbatim & Source code
\usepackage{fancyvrb}
\usepackage{minted}

% Beauty
\usepackage[protrusion]{microtype}
\usepackage[all]{nowidow}
\usepackage{upquote}
\usepackage{parskip}
\usepackage[strict]{changepage}

\usepackage{hyperref}

% Graph
\usepackage{graphicx}
\usepackage{grffile}
\usepackage{tikz}


\hypersetup{
  bookmarksnumbered,
  pdfborder={0 0 0},
  pdfpagemode=UseNone,
  pdfstartview=FitH,
  breaklinks=true}
\urlstyle{same}  % don't use monospace font for urls

\usetikzlibrary{arrows.meta,calc,shapes.geometric,shapes.misc}

\setminted{
  autogobble,
  breakbytokenanywhere,
  breaklines,
  fontsize=\footnotesize,
}
\setmintedinline{
  autogobble,
  breakbytokenanywhere,
  breaklines,
  fontsize=\footnotesize,
}

\makeatletter
\def\maxwidth{\ifdim\Gin@nat@width>\linewidth\linewidth\else\Gin@nat@width\fi}
\def\maxheight{\ifdim\Gin@nat@height>\textheight\textheight\else\Gin@nat@height\fi}
\makeatother

% Scale images if necessary, so that they will not overflow the page
% margins by default, and it is still possible to overwrite the defaults
% using explicit options in \includegraphics[width, height, ...]{}
\setkeys{Gin}{width=\maxwidth,height=\maxheight,keepaspectratio}
\setlength{\emergencystretch}{3em}  % prevent overfull lines
\setcounter{secnumdepth}{3}

% Redefines (sub)paragraphs to behave more like sections
\ifx\paragraph\undefined\else
\let\oldparagraph\paragraph{}
\renewcommand{\paragraph}[1]{\oldparagraph{#1}\mbox{}}
\fi
\ifx\subparagraph\undefined\else
\let\oldsubparagraph\subparagraph{}
\renewcommand{\subparagraph}[1]{\oldsubparagraph{#1}\mbox{}}
\fi

% set default figure placement to htbp
\makeatletter
\def\fps@figure{htbp}
\makeatother

\newcommand{\centeringcell}[1]{\multicolumn{1}{c}{#1}}

\title{Modern B-Tree Techniques}
\author{Goetz Graefe}
\affil{Hewlett-Packard Laboratories, USA, goetz.graefe@hp.com}

\begin{document}

\maketitle
\thispagestyle{empty}

\cleardoublepage{}
\pagenumbering{roman}
\tableofcontents{}
\cleardoublepage{}

\hypertarget{abstract}{%
\chapter*{Abstract}\label{abstract}}
\addcontentsline{toc}{chapter}{Abstract}
\markboth{Abstract}{}
\pagenumbering{arabic}

Invented about 40 years ago and called ubiquitous less than 10 years
later, B-tree indexes have been used in a wide variety of computing
systems from handheld devices to mainframes and server farms. Over the
years, many techniques have been added to the basic design in order to
improve efficiency or to add functionality. Examples include separation
of updates to structure or contents, utility operations such as
non-logged yet transactional index creation, and robust query processing
such as graceful degradation during index-to-index navigation.

This survey reviews the basics of B-trees and of B-tree indexes in
databases, transactional techniques and query processing techniques
related to B-trees, B-tree utilities essential for database operations,
and many optimizations and improvements. It is intended both as a survey
and as a reference, enabling researchers to compare index innovations
with advanced B-tree techniques and enabling professionals to select
features, functions, and tradeoffs most appropriate for their data
management challenges.

\cleardoublepage{}

\hypertarget{introduction}{%
\chapter{Introduction}\label{introduction}}

Less than 10 years after Bayer and McCreight {[}7{]} introduced B-trees,
and now more than a quarter century ago, Comer called B-tree indexes
ubiquitous {[}27{]}. Gray and Reuter asserted that ``B-trees are by far
the most important access path structure in database and file systems''
{[}59{]}. B-trees in various forms and variants are used in databases,
information retrieval, and file systems. It could be said that the
world's information is at our fingertips because of B-trees.

\hypertarget{perspectives-on-b-trees}{%
\section{Perspectives on B-trees}\label{perspectives-on-b-trees}}

\autoref{fig-1-1} shows a very simple B-tree with a root node and four leaf
nodes. Individual records and keys within the nodes are not shown. The
leaf nodes contain records with keys in disjoint key ranges. The root
node contains pointers to the leaf nodes and separator keys that divide
the key ranges in the leaves. If the number of leaf nodes exceeds the
number of pointers and separator keys that fit in the root node, an
intermediate layer of ``branch'' nodes is introduced. The separator keys
in the root node divide key ranges covered by the branch nodes (also
known as internal, intermediate, or interior nodes), and separator keys
in the branch nodes divide key ranges in the leaves. For very large data
collections, B-trees with multiple layers of branch nodes are used. One
or two branch levels are common in B-trees used as database indexes.

\begin{figure}
  \centering
  \includegraphics[width=\columnwidth]{./media/fig-1-1.png}

  \caption{A simple B-tree with root node and four leaf nodes.\label{fig-1-1}}
\end{figure}

Complementing this ``data structures perspective'' on B-trees is the
following ``algorithms perspective.'' Binary search in a sorted array
permits efficient search with robust performance characteristics. For
example, a search among $10^9$ or $2^30$
items can be accomplished with only $30$ comparisons. If the array of data
items is larger than memory, however, some form of paging is required,
typically relying on virtual memory or on a buffer pool. It is fairly
inefficient with respect to I/O, however, because for all but the last
few comparisons, entire pages containing tens or hundreds of keys are
fetched but only a single key is inspected. Thus, a cache might be
introduced that contains the keys most frequently used during binary
searches in the large array. These are the median key in the sorted
array, the median of each resulting half array, the median of each
resulting quarter array, etc., until the cache reaches the size of a
page. In effect, the root of a B-tree is this cache, with some
flexibility added in order to enable array sizes that are not powers of
two as well as efficient insertions and deletions. If the keys in the
root page cannot divide the original large array into sub-arrays smaller
than a single page, keys of each sub-array are cached, forming branch
levels between the root page and page-sized sub-arrays.

B-tree indexes perform very well for a wide variety of operations that
are required in information retrieval and database management, even if
some other index structure is faster for some individual index
operations. Perhaps the ``B'' in their name ``B-trees'' should stand for
their balanced performance across queries, updates, and utilities.
Queries include exact-match queries (``='' and ``in'' predicates), range
queries (``\textless{}'' and ``between'' predicates), and full
scans, with sorted output if required. Updates include insertion,
deletion, modifications of existing data associated with a specific key
value, and ``bulk'' variants of those operations, for example bulk
loading new information and purging out-of-date records. Utilities
include creation and removal of entire indexes, defragmentation, and
consistency checks. For all of those operations, including incremental
and online variants of the utilities, B-trees also enable efficient
concurrency control and recovery.

\hypertarget{purpose-and-scope}{%
\section{Purpose and Scope}\label{purpose-and-scope}}

Many students, researchers, and professionals know the basic facts about
B-tree indexes. Basic knowledge includes their organization in nodes
including one root and many leaves, the uniform distance between root
and leaves, their logarithmic height and logarithmic search effort, and
their efficiency during insertions and deletions. This survey briefly
reviews the basics of B-tree indexes but assumes that the reader is
interested in more detailed and more complete information about modern
B-tree techniques.

Commonly held knowledge often falls short when it comes to deeper topics
such as concurrency control and recovery or to practical topics such as
incremental bulk loading and structural consistency checking. The same
is true about the many ways in which B-trees assist in query processing,
e.g., in relational databases. The goal here is to make such knowledge
readily available as a survey and as a reference for the advanced
student or professional.

The present survey goes beyond the ``classic'' B-tree references {[}7,
8, 27, 59{]} in multiple ways. First, more recent techniques are
covered, both research ideas and proven implementation techniques.
Whereas the first twenty years of B-tree improvements are covered in
those references, the last twenty years are not. Second, in addition to
core data structure and algorithms, the present survey also discusses
their usage, for example in query processing and in efficient update
plans. Finally, auxiliary algorithms are covered, for example
defragmentation and consistency checks.

During the time since their invention, the basic design of B-trees has
been improved upon in many ways. These improvements pertain to
additional levels in the memory hierarchy such as CPU caches, to
multi-dimensional data and multi-dimensional queries, to concurrency
control techniques such as multi-level locking and key range locking, to
utilities such as online index creation, and to many more aspects of
B-trees. Another goal here is to gather many of these improvements and
techniques in a single document.

The focus and primary context of this survey are B-tree indexes in
database management systems, primarily in relational databases. This is
reflected in many specific explanations, examples, and arguments.
Nonetheless, many of the techniques are readily applicable or at least
transferable to other possible application domains of B-trees, in
particular to information retrieval {[}83{]}, file systems {[}71{]}, and
``No SQL'' databases and key-value stores recently popularized for web
services and cloud computing {[}21, 29{]}.

A survey of techniques cannot provide a comprehensive performance
evaluation or immediate implementation guidance. The reader still must
choose what techniques are required or appropriate for specific
environments and requirements. Issues to consider include the expected
data size and workload, the anticipated hardware and its memory
hierarchy, expected reliability requirements, degree of parallelism and
needs for concurrency control, the supported data model and query
patterns, etc.

\hypertarget{new-hardware}{%
\section{New Hardware}\label{new-hardware}}

Flash memory, flash devices, and other solid state storage technology
are about to change the memory hierarchy in computer systems in general
and in data management in particular. For example, most current software
assumes two levels in the memory hierarchy, namely RAM and disk, whereas
any further levels such as CPU caches and disk caches are hidden by
hardware and its embedded control software. Flash memory might also
remain hidden, perhaps as large and fast virtual memory or as fast disk
storage. The more likely design for databases, however, seems to be
explicit modeling of a memory hierarchy with three or even more levels.
Not only algorithms such as external merge sort but also storage
structures such as B-tree indexes will need a re-design and perhaps a
re-implementation.

Among other effects, flash devices with their very fast access latency
are about to change database query processing. They likely will shift
the break-even point toward query execution plans based on
index-to-index navigation, away from large scans and large set
operations such as sort and hash join. With more index-to-index
navigation, tuning the set of indexes including automatic incremental
index creation, growth, optimization, etc., will come more into focus in
future database engines.

As much as solid state storage will change tradeoffs and optimizations
for data structures and access algorithms, many-core processors will
change tradeoffs and optimizations for concurrency control and recovery.
High degrees of concurrency can be enabled only by appropriate
definitions of consistent states and of transaction boundaries, and
recovery techniques for individual transactions and for the system state
must support them. These consistent intermediate states must be defined
for each kind of index and data structure, and B-trees will likely be
first index structure for which such techniques are implemented in
production-ready database systems, file systems, and key-value stores.

In spite of future changes for databases and indexes on flash devices
and other solid state storage technology, the present survey often
mentions tradeoffs or design choices appropriate for traditional disk
drives, because much of the presently known and implemented techniques
have been invented and designed in this context. The goal is to provide
comprehensive background knowledge about B-trees for those researching
and implementing techniques appropriate for the new types of storage.

\hypertarget{overview}{%
\section{Overview}\label{overview}}

The next section (Section 2) sets out the basics as they may be found in
a college level text book. The following sections cover implementation
techniques for mature database management products. Their topics are
implementation techniques for data structures and algorithms (Section
3), transactional techniques (Section 4), query processing using B-trees
(Section 5), utility operations specific to B-tree indexes (Section 6),
and B-trees with advanced key structures (Section 7). These sections
might be more suitable for an advanced course on data management
implementation techniques and for a professional developer desiring
in-depth knowledge about B-tree indexes.


\hypertarget{basic-techniques}{%
\chapter{Basic Techniques}\label{basic-techniques}}

B-trees enable efficient retrieval of records in the native sort order
the index because, in a certain sense, B-trees capture and preserve the
result of a sort operation. Moreover, they preserve the sort effort in a
representation that can accommodate insertions, deletions, and updates.
The relationship between B-trees and sorting can be exploited in many
ways; the most common ones are that a sort operation can be avoided if
an appropriate B-tree exists and that the most efficient algorithm for
B-tree creation eschews random ``insert'' operations and instead pays
the cost of an initial sort for the benefit of efficient ``append''
operations.

\autoref{fig-2-1} illustrates how a B-tree index can preserve or cache the sort
effort. With the output of a sort operation, the B-tree with root, leaf
nodes, etc. can be created very efficiently. A subsequent scan can
retrieve data sorted without additional sort effort. In addition to
preserving the sort effort over an arbitrary length of time, B-trees
also permit efficient insertions and deletions, retaining their native
sort order and enabling efficient scans in sorted order at any time.

\begin{figure}
  \centering
  \includegraphics[width=\columnwidth]{./media/fig-2-1.png}

  \caption{Caching the sort effort in a B-tree.\label{fig-2-1}}
\end{figure}

Ordered retrieval aids many database operations, in particular
subsequent join and grouping operations. This is true if the list of
sort keys required in the subsequent operation is precisely equal to or
a prefix of that in the B-tree. It turns out, however, that B-trees can
save a lot of sort effort in many more cases. A later section will
consider the relationship between B-tree indexes and database query
operations in detail.

B-trees share many of their characteristics with binary trees, raising
the question why binary trees are commonly used for in-memory data
structures and B-trees for on-disk data. The reason is quite simple:
disk drives have always been block-access devices, with a high overhead
per access. B-trees exploit disk pages by matching the node size to the
page size, e.g., 4 KB. In fact, B-trees on today's high-bandwidth disks
perform best with nodes of multiple pages, e.g., 64 KB or 256 KB.
Inasmuch as main memory should be treated as a block-access device when
accessed through CPU caches and their cache lines, B-trees in memory
also make sense. Later sections will resume this discussion of memory
hierarchies and their effect on optimal data structures and algorithm
for indexes and specifically B-trees.

B-trees are more similar to 2-3-trees, in particular as both data
structures have a variable number of keys and child pointers in a node.
In fact, B-trees can be seen as a generalization of 2-3-trees. Some
books treat them both as special cases of ($a,b$)-trees with
$a \geq 2$ and $b \geq 2a - 1$ {[}92{]}. The number of child
pointers in a canonical B-tree node varies between $N$ and
$2N - 1$. For a small page size and a particularly large key size,
this might indeed be the range between $2$ and $3$. The representation of a
single node in a 2-3-tree by linking two binary nodes also has a
parallel in B-trees, discussed later as B\textsuperscript{link}-trees.

\autoref{fig-2-2} shows a ternary node in a 2-3-tree represented by two binary
nodes, one pointing to the other half of the ternary node rather than a
child. There is only one pointer to this ternary node from a parent
node, and the node has three child pointers.

\begin{figure}
  \centering
  \includegraphics[width=0.3\columnwidth]{./media/fig-2-2.png}

  \caption{A ternary node in a 2-3-tree represented with binary nodes.\label{fig-2-2}}
\end{figure}

In a perfectly balanced tree such as a B-tree, it makes sense to count
the levels of nodes not from the root but from the leaves. Thus, leaves
are sometimes called level-0 nodes, which are children of level-1 nodes,
etc. In addition to the notion of child pointers, many family terms are
used in connection with B-trees: parent, grandparent, ancestor,
descendent, sibling, and cousin. Siblings are children of the same
parent node. Cousins are nodes in the same B-tree level with different
parent nodes but the same grandparent node. If the first common ancestor
is a greatgrandparent, the nodes are second cousins, etc. Family
analogies are not used throughout, however. Two siblings or cousins with
adjoining key ranges are called neighbors, because there is no commonly
used term for such siblings in families. The two neighbor nodes are
called left and right neighbors; their key ranges are called the
adjacent lower and upper key ranges.

In most relational database management systems, the B-tree code is part
of the access methods module within the storage layer, which also
includes buffer pool management, lock manager, log manager, and more.
The relational layer relies on the storage layer and implements query
optimization, query execution, catalogs, and more. Sorting and index
maintenance span those two layers. For example, large updates may use an
update execution plan similar to a query execution plan to maintain each
B-tree index as efficiently as possible, but individual B-tree
modifications as well as read-ahead and write-behind may remain within
the storage layer. Details of such advanced update and prefetch
strategies will be discussed later.

In summary:

\begin{itemize}
\item
  B-trees are indexes optimized for paged environments, i.e., storage
  not supporting byte access. A B-tree node occupies a page or a set of
  contiguous pages. Access to individual records requires a buffer pool
  in byte-addressable storage such as RAM.
\item
  B-trees are ordered; they effectively preserve the effort spent on
  sorting during index creation. Differently than sorted arrays, B-trees
  permit efficient insertions and deletions.
\item
  Nodes are leaves or branch nodes. One node is distinguished as root
  node.
\item
  Other terms to know: parent, grandparent, ancestor, child, descendent,
  sibling, cousin, neighbor.
\item
  Most implementations maintain the sort order within each node, both
  leaf nodes and branch nodes, in order to enable efficient binary
  search.
\item
  B-trees are balanced, with a uniform path length in root-to-leaf
  searches. This guarantees uniformly efficient search.
\end{itemize}

\hypertarget{data-structures}{%
\section{Data Structures}\label{data-structures}}

In general, a B-tree has three kinds of nodes: a single root, a lot of
leaf nodes, and as many branch nodes as required to connect the root and
the leaves. The root contains at least one key and at least two child
pointers; all other nodes are at least half full at all times. Usually
all nodes have the same size, but this is not truly required.

The original design for B-trees has user data in all nodes. The design
used much more commonly today holds user data only in the leaf nodes.
The root node and the branch nodes contain only separator keys that
guide the search algorithm to the correct leaf node. These separator
keys may be equal to keys of current or former data, but the only
requirement is that they can guide the search algorithm.

This design has been called B\textsuperscript{+}-tree but it is nowadays
the default design when B-trees are discussed. The value of this design
is that deletion can affect only leaf nodes, not branch nodes and that
separator keys in branch nodes can be freely chosen within the
appropriate key range. If variable-length records are supported as
discussed later, the separator keys can often be very short. Short
separator keys increase the node fanout, i.e., the number of child
pointers per node, and decrease the B-tree height, i.e., the number of
nodes visited in a root-to-leaf search.

The records in leaf nodes contain a search key plus some associated
information. This information can be all the columns associated with a
table in a database, it can be a pointer to a record with all those
columns, or it can be anything else. In most parts of this survey, the
nature, contents, and semantics of this information are not important
and not discussed further.

In both branch nodes and leaves, the entries are kept in sorted order.
The purpose is to enable fast search within each node, typically using
binary search. A branch node with $N$ separator keys contains
$N + 1$ child pointers, one for each key range between neighboring
separator keys, one for the key, range below the smallest separator key,
and one for the key range above the largest separator key.

\autoref{fig-2-3} illustrates a B-tree more complex than the one in \autoref{fig-1-1},
including one level of branch nodes between the leaves and the root. In
the diagram, the root and all branch nodes have fan-outs of 2 or 3. In a
B-tree index stored on disk, the fan-out is determined by the sizes of
disk pages, child pointers, and separator keys. Keys are omitted in
\autoref{fig-2-3} and in many of the following figures unless they are required
for the discussion at hand.

\begin{figure}
  \centering
  \includegraphics[width=\columnwidth]{./media/fig-2-3.png}

  \caption{B-tree with root, branch nodes, and leaves.\label{fig-2-3}}
\end{figure}

Among all possible node-to-node pointers, only the child pointers are
truly required. Many implementations also maintain neighbor pointers,
sometimes only between leaf nodes and sometimes only in one direction.
Some rare implementations have used parent pointers, too, e.g., a
Siemens product {[}80{]}. The problem with parent pointers is that they
force updates in many child nodes when a parent node is moved or split.
In a disk-based B-tree, all these pointers are represented as page
identifiers.

B-tree nodes may include many additional fields, typically in a page
header. For consistency checking, there are table or index identifier
plus the B-tree level, starting with 0 for leaf pages; for space
management, there is a record count; for space management with
variable-size records, there are slot count, byte count, and lowest
record offset; for data compression, there may be a shared key prefix
including its size plus information as required for other compression
techniques; for write-ahead logging and recovery, there usually is a
Page LSN (log sequence number) {[}95{]}; for concurrency control, in
particular in shared-memory systems, there may be information about
current locks; and for efficient key range locking, consistency
checking, and page movement as in defragmentation, there may be fence
keys, i.e., copies of separator keys in ancestor pages. Each field, its
purpose, and its use are discussed in a subsequent section.

\begin{itemize}
\item
  Leaf nodes contain key values and some associated information. In most
  B-trees, branch nodes including the root node contain separator keys
  and child pointers but no associated information.
\item
  Child pointers are essential. Sibling pointers are often implemented
  but not truly required. Parent pointers are hardly ever employed.
\item
  B-tree nodes usually contain a fixed-format page header, a
  variable-size array of fixed-size slots, and a variable-size data
  area. The header contains a slot counter, information pertaining to
  compression and recovery, and more. The slots serve space management
  for variable-size records.
\end{itemize}

\hypertarget{sizes-tree-height-etc.}{%
\section{Sizes, Tree Height, etc.}\label{sizes-tree-height-etc.}}

In traditional database designs, the typical size of a B-tree node is
4--8 KB. Larger B-tree nodes might seem more efficient for today's disk
drives based on multiple analyses {[}57, 86{]} but nonetheless are
rarely used in practice. The size of a separator key can be as large as
a record but it can also be much smaller, as discussed later in the
section on prefix B-trees. Thus, the typical fan-out, i.e., the number
of children or of child pointers, is sometimes only in the tens,
typically in the hundreds, and sometimes in the thousands.

If a B-tree contains $N$ records and $L$ records per leaf, the
B-tree requires $N/L$ leaf nodes. If the average number of children
per parent is $F$, the number of branch levels is
$\log_F(N/L)$. For example, the B-tree in
\autoref{fig-2-3} has $9$ leaf nodes, a fan-out $F = 3$, and thus
$\log_3 9 = 2$ branch levels. Depending on the convention,
the height of this B-tree is $2$ (levels above the leaves) or $3$ (levels
including the leaves). In order to reflect the fact that the root node
usually has a different fan-out, this expression is rounded up. In fact,
after some random insertions and deletions, space utilization in the
nodes will vary among nodes. The average space utilization in B-trees is
usually given as about $70\%$ {[}75{]}, but various policies used in
practice and discussed later may result in higher space utilization. Our
goal here is not to be precise but to show crucial effects, basic
calculations, and the orders of magnitude of various choices and
parameters.

If a single branch node can point to hundreds of children, then the
distance between root and leaves is usually very small and $99\%$ or more
of all B-tree nodes are leaves. In other words, great-grandparents and
even more distant ancestors are rare in practice. Thus, for the
performance of random searches based on root-to-leaf B-tree traversals,
treatment of only $1\%$ of a B-tree index and thus perhaps only $1\%$ of a
database determine much of the performance. For example, keeping the
root of a frequently used B-tree index in memory benefits many searches
with little cost in memory or cache space.

\begin{itemize}
\item
  The B-tree depth (nodes along a root-to-leaf path) is logarithmic in
  the number of records. It is usually small.
\item
  Often more than 99\% of all nodes in a B-tree are leaf nodes.
\item
  B-tree pages are filled between 50\% and 100\%, permitting insertions
  and deletions as well as splitting and merging nodes. Average
  utilization after random updates is about 70\%.
\end{itemize}

\hypertarget{algorithms}{%
\section{Algorithms}\label{algorithms}}

The most basic, and also the most crucial, algorithm for B-trees is
search. Given a specific value for the search key of a B-tree or for a
prefix thereof, the goal is to find, correctly and as efficiently as
possible, all entries in the B-tree matching the search key. For range
queries, the search finds the lowest key satisfying the predicate.

A search requires one root-to-leaf pass. In each branch node, the search
finds the pair of neighboring separator keys smaller and larger than the
search key, and then continues by following the child pointer between
those two separator keys.

The number of comparisons during binary search among $L$ records in
a leaf is $\log_2(L)$, ignoring rounding effects.
Similarly, binary search among $F$ child pointers in a branch node
requires $\log_2(F)$ comparisons. The number of leaf
nodes in a B-tree with $N$ records and $L$ records per leaf is
$N/L$. The depth of a B-tree is
$\log_F(N/L)$, which is also the number of
branch nodes visited in a root-to-leaf search. Together, the number of
comparisons in a search inspecting both branch nodes and a leaf node is
$\log_F(N/L) \times \log_2(F)
+ \log_2(L)$. By elementary rules for algebra with
logarithms, the product term simplifies to
$\log_2(N/L)$ and then the entire expression
simplifies to $\log_2(N)$. In other words, node size
and record size may produce secondary rounding effects in this
calculation but the record count is the only primary influence on the
number of comparisons in a root-to-leaf search in a B-tree.

\autoref{fig-2-4} shows parts of a B-tree including some key values. A search
for the value 31 starts at the root. The pointer between the key values
7 and 89 is followed to the appropriate branch node. As the search key
is larger than all keys in that node, the right-most pointer is followed
to a leaf. A search within that node determines that the key value 31
does not exist in the B-tree. A search for key value 23 would lead to
the center node in \autoref{fig-2-4}, assuming a convention that a separator
key serves as inclusive upper bound for a key range. The search cannot
terminate when the value 23 is found at the branch level. This is
because the purpose of most B-tree searches is retrieving the
information attached to each key and information contents exists only in
leaves in most B-tree implementations. Moreover, as can be seen for key
value 15 in \autoref{fig-2-4}, a key that might have existed at some time in a
valid leaf entry may continue to serve as separator key in a nonleaf
node even after the leaf entry has been removed.

\begin{figure}
  \centering
  \includegraphics[width=\columnwidth]{./media/fig-2-4.png}

  \caption{B-tree with root-to-leaf search.\label{fig-2-4}}
\end{figure}

An exact-match query is complete after the search, but a range query
must scan leaf nodes from the low end of the range to the high end. The
scan can employ neighbor pointers if they exist in the B-tree.
Otherwise, parent and grandparent nodes and their child pointers must be
employed. In order to exploit multiple asynchronous requests, e.g., for
a B-tree index stored in a disk array or in network-attached storage,
parent and grandparent nodes are needed. Range scans relying on neighbor
pointers are limited to one asynchronous prefetch at-a-time and
therefore unsuitable for arrays of storage devices or for virtualized
storage.

Insertions start with a search for the correct leaf to place the new
record. If that leaf has the required free space, the insertion is
complete. Otherwise, a case called ``overflow,'' the leaf needs to be
split into two leaves and a new separator key must be inserted into the
parent. If the parent is already full, the parent is split and a
separator key is inserted into the appropriate grandparent node. If the
root node needs to be split, the B-tree grows by one more level, i.e., a
new root node with two children and only one separator key. In other
words, whereas the depth of many tree data structures grows at the
leaves, the depth of B-trees grows at the root. This is what guarantees
perfect balance in a B-tree. At the leaf level, B-trees grow only in
width, enabled by the variable number of child nodes in each parent
node. In some implementations of B-trees, the old root page becomes the
new root page and the old root contents are distributed into the two
newly allocated nodes. This is a valuable technique if modifying the
page identifier of the root node in the database catalogs is expensive
or if the page identifier is cached in compiled query execution plans.

\autoref{fig-2-5} shows the B-tree of \autoref{fig-2-4} after insertion of the key 22
and a resulting leaf split. Note that the separator key propagated to
the parent node can be chosen freely; any key value that separates the
two nodes resulting from the split is acceptable. This is particularly
useful for variable-length keys: the shortest possible separator key can
be employed in order to reduce space requirements in the parent node.

\begin{figure}
  \centering
  \includegraphics[width=\columnwidth]{./media/fig-2-5.png}

  \caption{B-tree with insertion and leaf split.\label{fig-2-5}}
\end{figure}

Some implementations of B-trees delay splits as much as possible, for
example by load balancing among siblings such that a full node can make
space for an insertion. This design raises the code complexity but also
the space utilization. High space utilization enables high scan rates if
data transfer from storage devices is the bottleneck. Moreover, splits
and new page allocations may force additional seek operations during a
scan that are expensive in disk-based B-trees.

Deletions also start with a search for the correct leaf that contains
the appropriate record. If that leaf ends up less than half full, a case
called ``underflow,'' either load balancing or merging with a sibling
node can ensure the traditional B-tree invariant that all nodes other
than the root be at least half full. Merging two sibling nodes may
result in underflow in their parent node. If the only two children of
the root node merge, the resulting node becomes the root and the old
root is removed. In other words, the depth of B-trees both grows and
shrinks at the root. If the page identifier of the root node is cached
as mentioned earlier, it might be practical to move all contents to the
root node and de-allocate the two children of the root node.

\autoref{fig-2-6} shows the B-tree from \autoref{fig-2-5} after deletion of key value
23. Due to underflow, two leaves were merged. Note that the separator
key 23 was not removed because it still serves the required function.

Many implementations, however, avoid the complexities of load balancing
and of merging and simply let underflows persist. A subsequent insertion
or defragmentation will presumably resolve it later. A recent study of
worst case and average case behaviors of B-trees concludes that ``adding
periodic rebuilding of the tree,\ldots the data structure\ldots is
theoretically superior to standard B\textsuperscript{+}-trees in many
ways {[}and{]}\ldots rebalancing on deletion can be considered harmful''
{[}116{]}.

\begin{figure}
  \centering
  \includegraphics[width=\columnwidth]{./media/fig-2-6.png}

  \caption{Deletion with load balancing.\label{fig-2-6}}
\end{figure}

Updates of key fields in B-tree records often require deletion in one
place and insertion in another place in the B-tree. Updates of nonkey
fixed-length fields happen in place. If records contain variable-length
fields, a change in the record size might force overflow or underflow
similar to insertion or deletion.

The final basic B-tree algorithm is B-tree creation. Actually, there are
two algorithms, characterized by random insertions and by prior sorting.
Some database products used random insertions in their initial releases
but their customers found creation of large indexes very slow. Sorting
the future index entries prior to B-tree creation permits many
efficiency gains, from massive I/O savings to various techniques saving
CPU effort. As the future index grows larger than the available buffer
pool, more and more insertions require reading, updating, and writing a
page. A database system might also require logging each such change in
the recovery log, whereas most systems nowadays employ non-logged index
creation, which is discussed later. Finally, a stream of append
operations also encourages a B-tree layout on disk that permits
efficient scans with a minimal number of disk seeks. Efficient sort
algorithms for database systems have been discussed elsewhere {[}46{]}.

\begin{itemize}
\item
  If binary search is employed in each node, the number of comparisons
  in a search is independent of record and node sizes except for
  rounding effects.
\end{itemize}

\begin{itemize}
\item
  B-trees support both equality (exact-match) predicates and range
  predicate. Ordered scans can exploit neighbor pointers or ancestor
  nodes for deep (multi-page) read-ahead.
\item
  Insertions use existing free space or split a full node into two
  half-full nodes. A split requires adding a separator key and a child
  pointer to the parent node. If the root node splits, a new root node
  is required and the B-tree grows by one level.
\item
  Deletions may merge half-full nodes. Many implementations ignore this
  case and rely on subsequent insertions or defragmentation
  (reorganization) of the B-tree.
\item
  Loading B-trees by repeated random insertion is very slow; sorting
  future B-tree entries permits efficient index creation.
\end{itemize}

\hypertarget{b-trees-in-databases}{%
\section{B-trees in Databases}\label{b-trees-in-databases}}

Having reviewed the basics of B-trees as a data structure, it is also
required to review the basics of B-trees as indexes, for example in
database systems, where B-tree indexes have been essential and
ubiquitous for decades. Recent developments in database query processing
have focused on improvements of large scans, e.g., by sharing scans
among concurrent queries {[}33, 132{]}, by a columnar data layout that
reduces the scan volume in many queries {[}17, 121{]}, or by predicate
evaluation by special hardware, such as FPGAs. The advent of flash
devices in database servers will likely result in more index usage in
database query processing --- their fast access times encourage small
random accesses whereas traditional disk drives with high capacity and
high bandwidth favor large sequential accesses. With the B-tree index
the default choice in most systems, the various roles and usage patterns
of B-tree indexes in databases deserve attention. We focus here on
relational databases because their conceptual model is fairly close to
the records and fields used in the storage layer of all database systems
as well as other storage services.

In a relational database, all data is logically organized in tables with
columns identified by name and rows identified by unique values in
columns forming the table's primary key. Relationships among tables are
captured in foreign key constraints. Relationships among rows are
expressed in foreign key columns, which contain copies of primary key
values elsewhere. Other forms of integrity constraints include
uniqueness of one or more columns; uniqueness constraints are often
enforced using a B-tree index.

The simplest representation for a database table is a heap, a collection
of pages holding records in no particular order, although often in the
order of insertion. Individual records are identified and located by
means of page identifier and slot number (see below in Section 3.3),
where the page identifier may include a device identifier. When a record
grows due to an update, it might need to move to a new page. In that
case, either the original location retains ``forwarding'' information or
all references to the old location, e.g., in indexes, must be updated.
In the former case, all future accesses incur additional overhead,
possibly the cost of a disk read; in the latter case, a seemingly simple
change may incur a substantial unforeseen and unpredictable cost.

\autoref{fig-2-7} shows records (solid lines) and pages (dashed lines) within a
heap file. Records are of variable size. Modifications of two records
have changed their sizes and forced moving the record contents to
another page with forwarding pointers (dotted lines) left behind in the
original locations. If an index points to records in this file,
forwarding does not affect the index contents but does affect the access
times in subsequent queries.

\begin{figure}
  \centering
  \includegraphics[width=\columnwidth]{./media/fig-2-7.png}

  \caption{Heap file with variable-length records and forwarding pointers.\label{fig-2-7}}
\end{figure}

If a B-tree structure rather than a heap is employed to store all
columns in a table, it is called a primary index here. Other commonly
used names include clustered index or index-organized table. In a
secondary index, also commonly called a non-clustered index, each entry
must contain a reference to a row or a record in the primary index. This
reference can be a search key in a primary index or it can be a record
identifier including a page identifier. The term ``reference'' will
often be used below to refer to either one. References to records in a
primary index are also called bookmarks in some contexts.

Both designs, reference by search key and reference by record
identifier, have advantages and disadvantages {[}68{]}; there is no
perfect design. The former design requires a root-to-leaf search in the
primary index after each search in the secondary index. \autoref{fig-2-8}
illustrates the double index search when the primary data structure for
a table is a primary index and references in secondary indexes use
search keys in the primary index. The search key extracted from the
query requires an index search and root-to-leaf traversal in the
secondary index. The information associated with a key in the secondary
index is a search key for the primary index. Thus, after a successful
search in the secondary index, another B-tree search is required in the
primary index including a root-to-leaf traversal there.

\begin{figure}
  \centering
  \includegraphics[width=\columnwidth]{./media/fig-2-8.png}

  \caption{Index navigation with a search key.\label{fig-2-8}}
\end{figure}

The latter design permits faster access to records in the primary index
after a search in the secondary index. When a leaf in the primary index
splits, however, this design requires many updates in all relevant
secondary indexes. These updates are expensive with many I/O operations
and B-tree searches, they are infrequent enough to be always surprising,
they are frequent enough to be disruptive, and they impose a substantial
penalty due to concurrency control and logging for these updates.

A combination is also possible, with the page identifier as a hint and
the search key as the fallback. This design has some intrinsic
difficulties, e.g., when a referenced page is de-allocated and later
reallocated for a different data structure. Finally, some systems employ
clustering indexes over heap files; their goal is to keep the heap file
sorted if possible but nonetheless enable fast record access via record
identifiers.

In databases, all B-tree keys must be unique, even if the user-defined
B-tree key columns are not. In a primary index, unique keys are required
for correct retrieval. For example, each reference found in a secondary
index must guide a query to exactly one record in the primary index ---
therefore, the search keys in the primary index must be unambiguous in
the reference-by-search-key design. If the user-defined search key for a
primary index is a person's lastname, values such as ``Smith'' are unlikely
to safely identify individual records in the primary index.

In a secondary index, unique keys are required for correct deletion.
Otherwise, deletion of a logical row and its record in a primary index
might be followed by deletion of the wrong entry in a non-unique
secondary index. For example, if the user-defined search in a secondary
index is a person's first name, deletion of a record in the primary
index containing ``Bob Smith'' must delete only the correct matching
record in the secondary index, not all records with search key ``Bob''
or a random such record.

If the search key specified during index creation is unique due to
uniqueness constraints, the user-defined search key is sufficient. Of
course, once a logical integrity constraint is relied upon in an index
structure, dropping the integrity constraint must be prevented or
followed by index reorganization. Otherwise, some artificial field must
be added to the user-defined index key. For primary indexes, some
systems employ a globally unique ``database key'' such as a microsecond
timestamp, some use an integer value unique within the table, and some
use an integer value unique among B-tree entries with the same value in
the user-defined index key. For secondary indexes, most systems simply
add the reference to the search key of the primary index.

B-tree entries are kept sorted on their entire unique key. In a primary
index, this aids efficient retrieval; in a secondary index, it aids
efficient deletion. Moreover, the sorted lists of references for each
unique search key enable efficient list intersection and union. For
example, for a query predicate ``A = 5 and B = 15,'' sorted lists of
references can be obtained from indexes on columns A and B and their
intersection computed by a simple merge algorithm.

The relationships between tables and indexes need not be as tight and
simple as discussed so far. A table may have calculated columns that are
not stored at all, e.g., the difference (interval) between two date
(timestamp) columns. On the other hand, a secondary index might be
organized on such a column, and in this case necessarily store a copy of
the column. An index might even include calculated columns that
effectively copy values from another table, e.g., the table of order
details might include a customer identifier (from the table of orders)
or a customer name (from the table of customers), if the appropriate
functional dependencies and foreign key constraints are in place.

Another relationship that is usually fixed, but need not be, is the
relationship between uniqueness constraints and indexes. Many systems
automatically create an index when a uniqueness constraint is defined
and drop the index when the constraint is dropped. Older systems did not
support uniqueness constraints at all but only unique indexes. The index
is created even if an index on the same column set already exists, and
the index is dropped even if it would be useful in future queries. An
alternative design merely requires that some index with the appropriate
column set exists while a uniqueness constraint is active. For instant
definition of a uniqueness constraint with existing useful index, a
possible design counts the number of unique keys during each insertion
and deletion in any index. In a sorted index such as a B-tree, a count
should be maintained for each key prefix, i.e., for the first key field
only, the first and second key fields together, etc. The required
comparisons are practically free as they are a necessary part of
searching for the correct insertion or deletion point. A new uniqueness
constraint is instantly verified if the count of unique key values is
equal to the record count in the index.

Finally, tables and indexes might be partitioned horizontally (into sets
of rows) or vertically (into sets of columns), as will be discussed
later. Partitions usually are disjoint but this is not truly required.
Horizontal partitioning can be applied to a table such that all indexes
of that table follow the same partitioning rule, sometimes called
``local indexes.'' Alternatively, partitioning can be applied to each
index individually, with secondary indexes partitioned with their own
partitioning rule different from the primary index, which is sometimes
called ``global indexes.'' In general, physical database design or the
separation of logical tables and physical indexes remains an area of
opportunity and innovation.

\begin{itemize}
\item
  B-trees are ubiquitous in databases and information retrieval.
\item
  If multiple B-trees are related, e.g., the primary index and the
  secondary index of a database table, pointers can be physical
  addresses (record identifiers) or logical references (search keys in
  the primary index). Neither choice is perfect, both choices have been
  used.
\item
  B-tree entries must be unique in order to ensure correct updates and
  deletions. Various mechanisms exist to force uniqueness by adding an
  artificial key value.
\item
  Traditional database design rigidly connects tables and B-trees, much
  more rigidly then truly required.
\end{itemize}

\hypertarget{b-trees-versus-hash-indexes}{%
\section{B-trees Versus Hash
Indexes}\label{b-trees-versus-hash-indexes}}

It might seem surprising that B-tree indexes have become ubiquitous
whereas hash indexes have not, at least not in database systems. Two
arguments seem to strongly favor hash indexes. First, hash indexes
should save I/O costs due to a single I/O per look-up, whereas B-trees
require a complete root-to-leaf traversal for each search. Second, hash
indexes and hash values should also save CPU effort due to efficient
comparisons and address calculations. Both of these arguments have only
very limited validity, however, as explained in the following
paragraphs. Moreover, B-trees have substantial advantages over hash
indexes with respect to index creation, range predicates, sorted
retrieval, phantom protection in concurrency control, and more. These
advantages, too, are discussed in the following paragraphs. All
techniques mentioned here are explained in more depth in subsequent
sections.

With respect to I/O savings, it turns out that fairly simple
implementation techniques can render B-tree indexes competitive with
hash indexes in this regard. Most B-trees have a fan-out of 100s or
1,000s. For example, for node of 8 KB and records of 20 bytes, $70\%$
utilization means 140 child nodes per parent node. For larger node sizes
(say 64 KB), good defragmentation (enabling run-length encoding of child
pointers, say 2 bytes on average), key compression using prefix and
suffix truncation (say 4 bytes on average per entry), $70\%$ utilization
means 5,600 child nodes per parent node. Thus, root-to-leaf paths are
short and more than $99\%$ or even $99.9\%$ of pages in a B-tree are leaf
nodes. These considerations must be combined with the traditional rule
that many database servers run with memory size equal to 1--3\% of
storage size. Today and in the future, the percentage might be higher,
up to $100\%$ for in-memory databases. In other words, for any B-tree
index that is ``warm'' in the buffer pool, all branch nodes will be
present in the buffer pool. Thus, each B-tree search only requires a
single I/O, the leaf page. Moreover, the branch nodes could be fetched
into the buffer pool in preparation of repeated look-up operations,
perhaps even pinned in the buffer pool. If they are pinned, further
optimizations could be applied, e.g., spreading separator keys into a
separate array such that interpolation search is most effective,
replacing or augmenting child pointers in form of page identifiers with
child pointers in form of memory pointers, etc.

\autoref{fig-2-9} illustrates the argument. All B-tree levels but the leaf
nodes easily fit into the buffer pool in RAM memory. For leaf pages, a
buffer pool might employ the least-recently-used (LRU) replacement
policy. Thus, for searches with random search keys, only a single I/O
operation is required, similar to a hash index if one is available in a
database system.

\begin{figure}
  \centering
  \includegraphics[width=\columnwidth]{./media/fig-2-9.png}

  \caption{B-tree levels and buffering.\label{fig-2-9}}
\end{figure}

With respect to CPU savings, B-tree indexes can compete with hash
indexes using a few simple implementation techniques. B-tree indexes
support a wide variety of search keys, but they also support very simple
ones such as hash values. Where hash indexes can be used, a B-tree on
hash values will also provide sufficient functionality. In other cases,
a ``poor man's normalized key'' can be employed and even be sufficient,
rendering all additional comparison effort unnecessary. Later sections
discuss normalized keys, poor man's normalized keys, and caching poor
man's normalized keys in the ``indirection vector'' that is required for
variable-size records. In sum, poor man's normalized keys and the
indirection vector can behave similarly to hash values and hash buckets.

B-trees also permit direct address calculation. Specifically,
interpolation search may guide the search faster than binary search. A
later section discusses interpolation search including avoiding
worst-case behavior of pure interpolation by switching to binary search
after two interpolation steps, and more.

While B-tree indexes can be competitive with hash indexes based on a few
implementation techniques, B-trees also have distinct advantages over
hash indexes. For example, space management in B-trees is very
straightforward. In the simplest implementations, full nodes are split
into two halves and empty nodes are removed. Multiple schemes have been
invented for hash indexes to grow gracefully, but none seems quite as
simple and robust. Algorithms for graceful shrinking of hash indexes are
not widely known.

Probably the strongest arguments for B-trees over hash indexes pertain
to multi-field indexes and to nonuniform distributions of key values. A
hash index on multiple fields requires search keys for all those fields
such that a hash value can be calculated. A B-tree index, on the other
hand, can efficiently support exact-match queries for a prefix of the
index key, i.e., any number of leading index fields. In this way, a
B-tree with $N$ search keys can be as useful as $N$ hash
indexes. In fact, B-tree indexes can support many other forms of
queries; it is not even required that the restricted fields are leading
fields in the B-tree's sort order {[}82{]}.

With respect to nonuniform (``skewed'') distributions of key values,
imagine a table with 10\textsuperscript{9} rows that needs a secondary
index on a column with the same value in 10\% of the rows. A hash index
requires introduction of overflow pages, with additional code for index
creation, insertion, search, concurrency control, recovery, consistency
checks, etc.

For example, when a row in the table is deleted, an expensive search is
required before the correct entry in the secondary index can be found
and removed, whereupon overflow pages might need to be merged. In a
B-tree, entries are always unique, if necessary by appending a field to
the search key as discussed earlier. In hash indexes, the additional
code requires additional execution time as well as additional effort for
testing and maintenance. Due to the well-defined sort order in B-trees,
neither special code nor extra time is required in any of the index
functions.

Another strong argument in favor of B-trees is index creation. After
extracting future index entries and sorting them, B-tree creation is
simple and very efficient, even for the largest data collections. An
efficient, general-purpose sorting algorithm is readily available in
most systems managing large data. Equally efficient index creation for
hash indexes would require a special-purpose algorithm, if it is
possible at all. Index creation by repeated random insertions is
extremely inefficient for both B-trees and hash indexes. Techniques for
online index creation (with concurrent database updates) are well known
and widely implemented for B-trees but not for hash indexes.

An obvious advantage of B-trees over hash indexes is the support for
ordered scans and for range predicates. Ordered scans are important for
key columns and set operations such as merge join and grouping; range
predicates are usually more important for nonkey columns. In other
words, B-trees are superior to hash indexes for both key columns and
nonkey columns in relational databases, also known as dimensions and
measures in online analytical processing. Ordering also has advantages
for concurrency control, in particular phantom protection by means of
key range locking (covered in detail later) rather than locking key
values only.

Taken together, these arguments favor B-trees over hash indexes as a
general indexing technique for databases and many other data
collections. Where hash indexes seem to have an advantage, appropriate
B-tree implementation techniques minimize it. Thus, very few database
implementation teams find hash indexes among the opportunities or
features with a high ratio of benefit and effort, in particular if
B-tree indexes are required in any case in order to support range
queries and ordered scans.

While nodes of 10 KB likely result in B-trees with multiple levels of
branch nodes, nodes of 1 MB probably do not. In other words, the
considerations above may apply to B-tree indexes on flash storage but
probably not on disks. For disks, it is probably best to cache all
branch nodes in memory and to employ fairly small leaf nodes such that
neither transfer bandwidth nor buffer space is wasted on unwanted
records.

\begin{itemize}
\item
  B-tree indexes are ubiquitous, whereas hash indexes are not, even
  though hash indexes promise exact-match look-up with direct address
  calculation in the hash directory and a single I/O.
\item
  B-tree software can provide similar benefits if desired. In addition,
  B-trees support efficient index creation based on sorting, support for
  exact match predicates and for partial predicates, graceful
  degradation in case of duplicate or distribution skew among the key
  values, and ordered scans.
\end{itemize}

\hypertarget{summary}{%
\section{Summary}\label{summary}}

In summary of this section on the basic data structure, B-trees are
ordered, balanced search trees optimized for block-access devices such
as disks. They guarantee good performance for various types of searches
well as for insertions, deletions, and updates. Thus, they are
particularly suitable to databases and in fact have been ubiquitous in
databases for decades.

Over time, many techniques have been invented and implemented beyond the
basic algorithms and data structures. These practical improvements are
covered in the next few sections.

The present section focuses on data structures and algorithms found in
mature data management systems but usually not in college-level text
books; the subsequent sections cover transactional techniques, B-trees
and their usage in database query processing, and B-tree utilities.

While only a single sub-section below is named ``data compression,''
almost all sub-sections pertain to compression in some form: storing
fewer bytes per record, describing multiple records together, comparing
fewer bytes in each search, modifying fewer bytes in each update, and
avoiding fragmentation and wasted space. Efficiency in space and time is
the theme of this section.

The following sub-sections are organized such that the first group
pertains to the size and internal structure of nodes, the next group to
compression specific to B-trees, and the last group to management of
free space. Most of the techniques in the individual sub-sections are
independent of others, although certain combinations may ease their
implementation. For example, prefix- and suffix-truncation require
detailed and perhaps excessive record keeping unless key values are
normalized into binary strings.


\hypertarget{data-structures-and-algorithms}{%
\chapter{Data Structures and
Algorithms}\label{data-structures-and-algorithms}}

The present section focuses on data structures and algorithms found in
mature data management systems but usually not in college-level text
books; the subsequent sections cover transactional techniques, B-trees
and their usage in database query processing, and B-tree utilities.

While only a single sub-section below is named ``data compression,''
almost all sub-sections pertain to compression in some form: storing
fewer bytes per record, describing multiple records together, comparing
fewer bytes in each search, modifying fewer bytes in each update, and
avoiding fragmentation and wasted space. Efficiency in space and time is
the theme of this section.

The following sub-sections are organized such that the first group
pertains to the size and internal structure of nodes, the next group to
compression specific to B-trees, and the last group to management of
free space. Most of the techniques in the individual sub-sections are
independent of others, although certain combinations may ease their
implementation. For example, prefix- and suffix-truncation require
detailed and perhaps excessive record keeping unless key values are
normalized into binary strings.

\hypertarget{node-size}{%
\section{Node Size}\label{node-size}}

Even the earliest papers on B-trees discussed the optimal node size for
B-trees on disk {[}7{]}. It is governed primarily by access latency and
transfer bandwidth as well as the record size. High latency and high
bandwidth both increase the optimal node size; therefore, the optimal
node size for modern disks approaches 1 MB and the optimal on flash
devices is just a few KB {[}50{]}. A node size with equal access latency
and transfer time is a promising heuristic --- it guarantees a sustained
transfer bandwidth at least half of the theoretical optimum as well as
an I/O rate at least half of the theoretical optimum. It is calculated
by multiplying access latency and transfer bandwidth. For example, for a
disk with 5 ms access latency and 200 MB/s transfer bandwidth, this
leads to 1 MB. An estimated access latency of 0.1 ms and a transfer
bandwidth of 100 MB/s lead to 10 KB as a promising node size for B-trees
on flash devices.

For a more precise optimization, the goal is to maximize the number of
comparisons per unit of I/O time. Examples for this calculation can
already be found in the original B-tree papers {[}7{]}. This
optimization assumes that the goal is to optimize root-to-leaf searches
and not large range scans, I/O time and not CPU effort is the
bottleneck, binary search is used within nodes, and a fixed total number
of comparisons in a root-to-leaf B-tree search independent of the node
size as discussed above.

\autoref{fig-3-1} shows a calculation similar to those in {[}57{]}. It assumes
pages filled to $70\%$ with records of 20 bytes, typical in secondary
indexes. For example, in a page of 4 KB holding 143 records, binary
search performs a little over 7 comparisons on average. The number of
comparisons is termed the utility of the node with respect to searching
the index. I/O times in \autoref{fig-3-1} are calculated assuming 5 ms access
time and 200 MB/s (burst) transfer bandwidth. The heuristic above would
suggest a page size of 5 ms × 200 MB/s = 1,000 KB. B-tree nodes of 128
KB enable the most comparisons (in binary search) relative to the disk
device time. Historically common disk pages of 4 KB are far from optimal
for B-tree indexes on traditional disk drives. Different record sizes
and different devices will result in different optimal page sizes for
B-tree indexes. Most importantly, devices based on flash devices may
achieve 100 times faster access times without substantially different
transfer bandwidth. Optimal B-tree node sizes will be much smaller,
e.g., 2 KB {[}50{]}.

\begin{figure}
  \centering
  \begin{tabular}{rrrrr}
    \centeringcell{Page size} & \centeringcell{Records} & \centeringcell{Node} & \centeringcell{I/O time} & \centeringcell{Utility} \\
    \centeringcell{{[}KB{]}} & \centeringcell{/ page} & \centeringcell{utility} & \centeringcell{{[}ms{]}} & \centeringcell{/ time} \\
    4 & 143 & 7.163 & 5.020 & 1.427 \\
    16 & 573 & 9.163 & 5.080 & 1.804 \\
    64 & 2,294 & 11.163 & 5.320 & 2.098 \\
    128 & 4,588 & 12.163 & 5.640 & 2.157 \\
    256 & 9,175 & 13.163 & 6.280 & 2.096 \\
    1,024 & 36,700 & 15.163 & 10.120 & 1.498 \\
    4,096 & 146,801 & 17.163 & 25.480 & 0.674 \\
  \end{tabular}
  \caption{Utility for pages sizes one a traditional disk.\label{fig-3-1}}
\end{figure}

\begin{itemize}
\item
  The node size should be optimized based on latency and bandwidth of
  the underlying storage. For example, the optimal page size differs for
  traditional disks and semiconductor storage.
\end{itemize}

\hypertarget{interpolation-search}{%
\section{Interpolation Search}\label{interpolation-search}}

\footnote{Much of this section is derived from {[}45{]}.}Like binary
search, interpolation search employs the concept of a remaining search
interval, initially comprising the entire page. Instead of inspecting
the key in the center of the remaining interval like binary search,
interpolation search estimates the position of the sought key value,
typically using a linear interpolation based on the lowest and highest
key value in the remaining interval. For some keys, e.g., artificial
identifier values generated by a sequential process such as invoice
numbers in a business operation, interpolation search works extremely
well.

In the best case, interpolation search is practically unbeatable.
Consider an index on the column Order-Number in the table Orders given
that order numbers and invoice numbers are assigned sequentially. Since
each order number exists precisely once, interpolation among hundreds or
even thousands of records within a B-tree node instantly guides the
search to the correct record.

In the worst case, however, the performance of pure interpolation search
equals that of linear search due to a nonuniform distribution of key
values. The theoretical complexity is $\mathcal{O}(\log\log N)$ for search among
$N$ keys {[}36, 107{]}, or 2 to 4 steps for practical page sizes.
Thus, if the sought key has not yet been found after 3 or 4 steps, the
actual key distribution is not uniform and it might be best to perform
the remaining search using binary search.

Rather than switching from pure interpolation search to pure binary
search, a gradual transition may pay off. If interpolation search has
guided the search to one end of the remaining interval but not directly
to the sought key value, the interval remaining for binary search may be
very small or very large. Thus, it seems advisable to bias the last
interpolation step in such a way to make it very likely that the sought
key is in the smaller remaining interval.

The initial interpolation calculation might use the lowest and highest
possible values in a page, the lowest and highest actual values, or a
regression line based on all current values. The latter technique may be
augmented with a correlation calculation that guides the initial search
steps toward interpolation or binary search. Sums and counts required to
quickly derive regression and correlation coefficients can easily be
maintained incrementally during updates of individual records in a page.

\autoref{fig-3-2} shows two B-tree nodes and their key values. In the upper
one, the correlation between slot numbers and key values is very high
($> 0.998$). Slope and intercept are $3.1$ and $5.9$,
respectively (slot numbers start with 0). An interpolation search for
key value 12 immediately probes slot number $(12 − 5.9) \div
3.1 = 2$ (rounded), which is where key value 12 indeed can be
found. In other words, if the correlation between position and key value
is very strong, interpolation search is promising. In the lower B-tree
node shown in \autoref{fig-3-2}, slope and intercept are −64 and 31,
respectively. More importantly, the correlation coefficient is much
lower ($< 0.75$). Not surprisingly, interpolation search
for key value 97 starts probing at slot $(97 − 64) \div 31 = 5$ whereas the
correct slot number of key value 97 is 8. Thus, if the correlation
between position and key value is weak, binary search is the more
promising approach.

\begin{figure}
  \centering
  \begin{tabular}{l}
    \hline
    \multicolumn{1}{|l|}{5, 9, 12, 16, 19, 21, 25, 27, 31, 34, 36} \\
    \hline\\
    \\
    \hline
    \multicolumn{1}{|l|}{5, 6, 15, 15, 16, 43, 95, 96, 97, 128, 499} \\
    \hline
  \end{tabular}
  \caption{Sample key values.\label{fig-3-2}}
\end{figure}

\begin{itemize}
\item
  If the key value distribution within a page is close to uniform,
  interpolation search requires fewer comparisons and incurs fewer cache
  faults than binary search. Artificial identifiers such as order
  numbers are ideal cases for interpolation search.
\item
  For cases on non-uniform key value distributions, various techniques
  can prevent repeated erroneous interpolation.
\end{itemize}

\hypertarget{variable-length-records}{%
\section{Variable-length Records}\label{variable-length-records}}

While B-trees are usually explained for fixed-length records in the
leaves and fixed-length separator keys in the branch nodes, B-trees in
practically all database systems support variable-length records and
variable-length separator keys. Thus, space management within B-tree
nodes is not trivial.

The standard design for variable-length records in fixed-length pages,
both in B-trees and in heap files, employs an indirection vector (also
known as slot array) with entries of fixed size. Each entry represents
one record. An entry must contain the byte offset of the record and may
contain additional information, e.g., the size of the record.

\autoref{fig-3-3} shows the most important parts of a disk page in a database.
The page header, shown far left within the page, contains index
identifier, B-tree level (for consistency checks), record count, etc.
This is followed in \autoref{fig-3-3} by the indirection vector. In heap files,
slots remain unused after a record deletion in order to ensure that the
remaining valid records retain their record identifier. In B-trees,
insertions or deletions require shifting some slot entries in order to
ensure that binary search can work correctly. (Figure 4.7 in Section 4.2
shows an alternative to this traditional design with less shifting due
to intentional gaps in the sequence of records.) Each used slot contains
a pointer (in form of a byte offset within the page) to a record. In the
diagram, the indirection vector grows from left to right and the set of
records grows from right to left. The opposite design is also possible.
Letting two data structures grow toward each other enables equally well
many small records or fewer large records.

\begin{figure}
  \centering
  \includegraphics[width=\columnwidth]{./media/image1.png}
  \caption{A database page with page header, indirection vector, and
  variable-length records.\label{fig-3-3}}
\end{figure}

For efficient binary search, the entries in the indirection vector are
sorted on their search key. It is not required that the entries be
sorted on their offsets, i.e., the placement of records. For example,
the sequence of slots in the left half of \autoref{fig-3-3} differs from the
sequence of records in the right half. A sort order on offsets is needed
only temporarily for consistency checks and for compaction or free space
consolidation, which may be invoked by a record insertion, by a
size-changing record update, or by a defragmentation utility.

Record insertion requires free space both for the record and for the
entry in the indirection vector. In the standard design, the indirection
vector grows from one end of the page and the data space occupied by
records grows from the opposite end. Free space for the record is
usually found very quickly by growing the data space into the free space
in the middle. Free space for the entry requires finding the correct
placement in the sorted indirection vector and then shifting entries as
appropriate. On average, half of the indirection must shift by one
position.

Record deletion is fast as it typically just leaves a gap in the data
space. However, it must keep the indirection vector dense and sorted,
and thus requires shifting just like insertion. Some recent designs
require less shifting {[}12{]}. Some designs also separate separator
keys and child pointers in branch nodes in order to achieve more
effective compression as well as more efficient search within each
branch node. Those techniques are also discussed below.

\begin{itemize}
\item
  Variable-size records can be supported efficiently by a level of
  indirection within a page.
\item
  Shift operations in the indirection vector can be minimized by gaps
  (invalid entries).
\end{itemize}

\hypertarget{normalized-keys}{%
\section{Normalized Keys}\label{normalized-keys}}

In order to reduce the cost of comparisons, many implementations of
B-trees transform keys into a binary string such that simply binary
comparisons suffice to sort the records during index creation and to
guide a search in the B-tree to the correct record. The key sequence for
the sort order of the original key and for the binary string are the
same, and all comparisons equivalent. This binary string may encode
multiple columns, their sort direction (e.g., descending) and collation
including local characters (e.g., case-insensitive German), string
length or string termination, etc.

Key normalization is a very old technique. It is already mentioned by
Singleton {[}118{]} without citation, presumably because it seemed a
well-known or trivial concept: ``integer comparisons were used to order
normalized floating-point numbers.''

\autoref{fig-3-4} illustrates the idea based on an integer column followed by
two string columns. The initial single bit (shown underlined) indicates
whether the leading key column contains a valid value. Using 0 for null
values and 1 for other values ensures that a null value ``sorts lower''
than all other values. If the integer column value is not null, it is
stored in the next 32 bits. Signed integers require reversing some bits
to ensure the proper sort order, just like floating point values require
proper treatment of exponent, mantissa, and the two sign bits. \autoref{fig-3-4}
assumes that the first column is unsigned. The following single bit
(also shown underlined) indicates whether the first string column
contains a valid value. This value is shown here as text but really
ought to be stored in a binary format as appropriate for the desired
international collation sequence. A string termination symbol (shown as
\textbackslash{}0) marks the end of the string. A termination symbol is
required to ensure the proper sort order. A length indicator, for
example, would destroy the main value of normalized keys, namely sorting
with simple binary comparisons. If the string termination symbol can
occur as a valid character in some strings, the binary representation
must offer one more symbol than the alphabet contains. Notice the
difference in representations between a missing value in a string column
(in the third row) and an empty string (in the fourth row).

\begin{figure}
  \centering
  \begin{tabular}{llll}
Integer & First string & Second string & Normalized key \\
2 & ``flow'' & ``error'' & \underline{1} 0…0 0000 0000 0010 \underline{1} flow\textbackslash{}0 \underline{1} error\textbackslash{}0 \\
3 & ``flower'' & ``rare'' & \underline{1} 0…0 0000 0000 0011 \underline{1} flower\textbackslash{}0 \underline{1} rare\textbackslash{}0 \\
1024 & Null & ``brush'' & \underline{1} 0…0 0100 0000 0000 \underline{0} \underline{1} brush\textbackslash{}0 \\
Null & ``'' & Null & \underline{0} \underline{1} \textbackslash{}0 \underline{0}
  \end{tabular}
  \caption{Normalized keys.\autoref{fig-3-4}}
\end{figure}

For some collation sequences, ``normalized keys'' lose information. A
typical example is a language with lower and upper case letters sorted
and indexed in a case-insensitive order. In that case, two different
original strings might map to the same normalized key, and it is
impossible from the normalized key to decide which original style was
used. One solution for this problem is to store both the normalized key
and the original string value. A second solution is to append to the
normalized key the minimal information that enables a precise recovery
of the original writing style. A third solution, specific to B-tree
indexes, is to employ normalized keys only in branch nodes; recall that
key values in branch nodes merely guide the search to the correct child
but do not contain user data.

In many operating systems, appropriate functions are provided to compute
a normalized key from a localized string value, date value, or time
value. This functionality is used, for example, to list files in a
directory as appropriate for the local language. Adding normalization
for numeric data types is relatively straightforward, as is
concatenation of multiple normalized values. Database code must not rely
on such operating system code, however. The problem with relying on
operating systems support for database indexes is the update frequency.
An operating system might update its normalization code due to an error
or extension in the code or in the definition of a local sort order; it
is unacceptable, however, if such an update silently renders existing
large database indexes incorrect.

Another issue with normalized keys is that they tend to be longer than
the original string values, in particular for some languages and their
complex rules for order, sorting, and index look-up. Compression of
normalized keys seems quite possible but a detailed description seems to
be missing yet in the literature. Thus, normalized keys are currently
used primarily in internal B-tree nodes, where they simplify the
implementation of prefix and suffix truncation but never require
recovery of original key values.

\begin{itemize}
\item
  Normalized keys enable comparisons by traditional hardware
  instructions, much faster than column-by-column interpolation of
  metadata about international sort order, ascending versus descending
  sort order, etc.
\item
  Normalized keys can be longer than a traditional representation but
  are amenable to compression.
\item
  Some systems employ normalized keys in branch nodes but not in leaf
  nodes.
\end{itemize}

\hypertarget{prefix-b-trees}{%
\section{Prefix B-trees}\label{prefix-b-trees}}

Once keys have been normalized into a simple binary string, another
B-tree optimization becomes much easier to implement, namely prefix and
suffix truncation or compression {[}10{]}. Without key normalization,
these techniques would require a fair bit of bookkeeping, even if they
were applied only to entire key fields rather than to individual bytes;
with key normalization, their implementation is relatively
straightforward.

Prefix truncation analyzes the keys in a B-tree node and stores the
common prefix only once, truncating it from all keys stored in the node.
Saving storage space permits increasing the number of records per leaf
and increasing the fan-out of branch nodes. In addition, the truncated
key bytes do not need to be considered in comparisons during a search.

\autoref{fig-3-5} shows the same records within a B-tree node represented
without and with prefix truncation. It is immediately obvious that the
latter representation is more efficient. It is possible to combine
prefix truncation with some additional compression technique, e.g., to
eliminate symbols from the birthdates given. Of course, it is always
required to weigh gains in run-time performance and storage efficiency
against implementation complexity including testing effort.

\begin{figure}
  \centering
  \includegraphics[width=0.6\columnwidth]{./media/fig-3-5.png}

  \caption{A B-tree node without and with prefix truncation.\label{fig-3-5}}
\end{figure}

Prefix truncation can be applied to entire nodes or to some subset of
the keys within a node. Code simplicity argues for truncating the same
prefix from all entries in a B-tree node. Moreover, one can apply prefix
truncation based on the actual keys currently held in a node or based on
the possible key range as defined by the separator keys in parent (and
possibly other ancestor) pages. Code simplicity, in particular for
insertions, argues for prefix truncation based on the maximal possible
key range, even if prefix truncation based on actual keys might produce
better compression {[}87{]}. If prefix truncation is based on actual
keys, insertion of a new key might force reformatting all existing keys.
In an extreme case, a new record might be much smaller than the free
space in a B-tree page yet its insertion might force a page split.

The maximal possible key range for a B-tree page can be captured by
retaining two fence keys in each node, i.e., copies of separator keys
posted in parent nodes while splitting nodes. Figure 4.11 (in Section
4.4) illustrates fence keys in multiple nodes in a B-tree index. Fence
keys have multiple benefits in B-tree implementations, e.g., for key
range locking. With respect to prefix truncation, the leading bytes
shared by the two fence keys of a page define the bytes by all current
and future key values in the page. At the same time, prefix truncation
reduces the overhead imposed by fence keys, and suffix truncation
(applied when leaf nodes are split) ensures that separator keys and thus
fence keys are always as short as possible.

Prefix truncation interacts with interpolation search. In particular, if
the interpolation calculation uses fixed and limited precision,
truncating common prefixes enables more accurate interpolation. Thus,
normalized keys, prefix truncation, suffix truncation, and interpolation
search are a likely combination in an implementation.

A very different approach to prefix truncation technique is offsetvalue
coding {[}28{]}. It is used in high-performance implementations of
sorting, in particular in sorted runs and in the merge logic {[}72{]}.
In this representation, each string is compared to its immediate
predecessor in the sorted sequence and the shared prefix is replaced by
an indication of its length. The sign bit is reserved to make the
indicator order-preserving, i.e., a short shared prefix sorts later than
a long shared prefix. The result is combined with the data at this
offset such that a single machine instruction can compare both offset
and value. This representation saves more space than prefix truncation
applied uniformly to an entire page. It is very suitable to sequential
scans and merging but not to binary search or interpolation search.
Instead, a trie representation could attempt to combine the advantages
of prefix truncation and binary search, but it is used in very few
database systems. The probable reasons are code complexity and update
overhead.

Even if prefix truncation is not implemented in a B-tree and its page
format, it can be exploited for faster comparisons and thus faster
search. The following technique might be called dynamic prefix
truncation. While searching for the correct child pointer in a parent
node, the two keys flanking the child pointer will be inspected. If they
agree on some leading bytes, all keys found by following the child
pointer must agree on the same bytes, which can be skipped in all
subsequent comparisons. It is not necessary to actually compare the two
neighboring separator keys with each other, because the required
information is readily available from the necessary comparisons of these
separator keys with the search key. In other words, dynamic prefix
truncation can be exploited without adding comparison steps to a
root-to-leaf search in a B-tree.

For example, assume a binary search within the B-tree node shown on the
left side of \autoref{fig-3-5}, with the remaining search interval from
``Smith, Jack'' to ``Smith, Jason.'' Thus, the search argument must be
in that range and also start with ``Smith, Ja.'' For all remaining
comparisons, this prefix may be assumed and thus skipped in all
remaining comparisons within this search. Note that dynamic prefix
truncation also applies to B-tree nodes stored with prefix truncation.
In this example, the string ``a'' beyond the truncated prefix ``Smith,
J'' may be skipped in all remaining comparisons.

While prefix truncation can be employed to all nodes in a B-tree, suffix
truncation pertains specifically to separator keys in branch nodes
{[}10{]}. Prefix truncation is most effective in leaf nodes whereas
suffix truncation primarily affects branch nodes and the root node. When
a leaf is split into two neighbor leaves, a new separator key is
required. Rather than taking the highest key from the left neighbor or
the lowest key from the right neighbor, the separator is chosen as the
shortest string that separates those two keys in the leaves.

For example, assume the key values shown in \autoref{fig-3-6} are in the middle
of a node that needs to be split. The precise center is near the long
arrow. The minimal key splitting the node there requires at least 9
letters, including the first letter of the given name. If, on the other
hand, a split point anywhere between the short arrows is acceptable, a
single letter suffices. A single comparison of the two keys defining the
range of acceptable split points can determine the shortest possible
separator key. For example, in \autoref{fig-3-6}, a comparison between
``Johnson, Lucy'' and ``Smith, Eric'' shows their first difference in
the first letter, indicating that a separator key with a single letter
suffices. Any letter can be chosen that is larger than J and not larger
than S. It is not required that the letter actually occurs in the
current key values.

\begin{figure}
  \centering
  \includegraphics[width=0.3\columnwidth]{./media/fig-3-6.png}

  \caption{Finding a separator key during a leaf split.\label{fig-3-6}}
\end{figure}

It is tempting to apply suffix truncation not only when splitting leaf
nodes but also when splitting branch nodes. The problem with this idea,
however, is that a separator key in a grandparent node must guide the
search not only to the correct parent but also to the correct leaf. In
other words, applying suffix truncation again might guide a search to
the highest node in the left sub-tree rather than to the lowest node in
the right sub-tree, or vice versa. Fortunately, if 99\% of all B-tree
nodes are leaves and 99\% of the remaining nodes are immediate parents
of leaves, additional truncation could benefit at most 1\% of 1\% of all
nodes. Thus, this problematic idea, even if it worked flawlessly, would
probably never have a substantial effect on B-tree size or search
performance.

\autoref{fig-3-7} illustrates the problem. The set of separator keys in the
upper B-tree is split by the shortened key ``g,'' but the set of leaf
entries is not. Thus, a root-to-leaf search for the key ``gh'' will be
guided to the right sub-tree and thus fail, obviously incorrectly. The
correct solution is to guide searches based on the original separator
key ``gp.'' In other words, when the branch node is split, no further
suffix truncation must be applied. The only choice when splitting a
branch node is the split point and the key found there.

\begin{figure}
  \centering
  \includegraphics[width=\columnwidth]{./media/fig-3-7.png}

  \caption{Incorrect suffix truncation.\label{fig-3-7}}
\end{figure}

\begin{itemize}
\item
  A simple technique for compression, particularly effective in leaf
  pages, is to identify the prefix shared by all key values and to store
  the prefix only once.
\item
  Alternatively, or in addition, binary search and interpolation search
  can ignore key bytes shared by the lower and upper bounds of the
  remaining search interval. In a root-to-leaf search, such dynamic
  prefix truncation carries from parent to child.
\item
  Key values in branch pages need not be actual key values. They merely
  need to guide root-to-leaf searching. When posting a separator key
  while splitting a leaf page, a good choice is the shortest value that
  splits near the middle.
\item
  Offset-value coding compares each key value with its immediate
  neighbor and truncates the shared prefix. It achieves better
  compression than page-wide prefix truncation but disables efficient
  binary search and interpolation search.
\item
  Normalized keys significantly reduce the implementation complexity of
  prefix and suffix truncation as well as of offsetvalue coding.
\end{itemize}

\hypertarget{cpu-caches}{%
\section{CPU Caches}\label{cpu-caches}}

Cache faults contribute a substantial fraction to the cost of searching
within a B-tree page. If a B-tree needs to be searched with many keys
and the sequence of search operations may be modified, temporal locality
may be exploited {[}128{]}. Otherwise, optimization of data structures
is required. Cache faults for instructions can be reduced by use of
normalized keys --- comparisons of individual fields with international
sort order, collation sequence, etc., plus interpretation of schema
information, can require a large amount of code whereas two normalized
keys can be compared by a single hardware instruction. Moreover,
normalized keys simplify the implementation not only of prefix and
suffix truncation but also of optimizations targeted at reducing cache
faults for data accesses. In fact, many optimizations seem practical
only if normalized keys are used.

After prefix truncation has been applied, many comparisons in a binary
search are decided by the first few bytes. Even where normalized keys
are not used in the records, e.g., in B-tree leaves, storing a few bytes
of the normalized key can speed up comparisons. If only those few bytes
are stored, not the entire normalized key, such that they can decide
many but not all comparisons, they are called ``poor man's normalized
keys'' {[}41{]}.

In order to enable key comparisons and search without cache faults for
data records, poor man's normalized keys can be an additional field in
the elements of the indirection vector. This design has been employed
successfully in the implementation of AlphaSort {[}101{]} and can be
equally beneficial in B-tree pages {[}87{]}.

On the other hand, it is desirable to keep each element in the
indirection vector small. While traditional designs often include the
record size in the elements of the indirection vector as was mentioned
in the discussion of \autoref{fig-3-3}, the record length is hardly ever
accessed without access to the related record. Thus, the field
indicating the record length might as well be placed with the main
record rather than in the indirection vector.

\autoref{fig-3-8} illustrates such a B-tree page with keys indicating three
European countries. On the left are page header and indirection vector,
on the right are the variable-size records. The poor man's normalized
key, indicated here by a single letter, is kept in the indirection
vector. The main record contains the total record size and the remaining
bytes of the key. A search for ``Denmark'' can eliminate all records by
the poor man's normalized keys without incurring cache faults for the
main records. A search for ``Finland,'' on the other hand, can rely on
the poor man's normalized key for the binary search but eventually must
access the main record for ``France.'' While the poor man's normalized
key in \autoref{fig-3-8} comprises only a single letter, 2 or 4 bytes seem more
appropriate, depending on the page size. For example, in a small
database page optimized for flash storage and its fast access latency, 2
bytes might be optimal; whereas in large database pages optimized for
traditional disks and their fast transfer bandwidth, 4 bytes might be
optimal.

\begin{figure}
  \centering
  \includegraphics[width=\columnwidth]{./media/fig-3-8.png}

  \caption{Poor man's normalized keys in the indirection vector.\label{fig-3-8}}
\end{figure}

An alternative design organizes the indirection vector not as a linear
array but as a B-tree of cache lines. The size of each node in this
B-tree is equal to a single cache line or a small number of them
{[}64{]}. Root-to-leaf navigation in this B-tree might employ pointers
or address calculations {[}110, 87{]}. Search time and cache faults
within a B-tree page may be cut in half compared to node formats not
optimized for CPU caches {[}24{]}. A complementary, more theoretical
design of cache efficient B-tree formats is even more complex but
achieves optimal asymptotic performance independently of the sizes of
disk page and cache line {[}11{]}. Both organizations of B-tree nodes,
i.e., linear arrays as shown in \autoref{fig-3-8} and B-trees within B-tree
nodes, can benefit from ghost slots, i.e., entries with valid key values
but marked invalid, which will be discussed shortly.

\begin{itemize}
  \item
A cache fault may waste 100s of CPU cycles. B-tree pages can be
optimized to reduce cache faults just like B-trees are optimized
(compared to binary trees) to reduce page faults.
\end{itemize}

\hypertarget{duplicate-key-values}{%
\section{Duplicate Key Values}\label{duplicate-key-values}}

Duplicate values in search keys are fairly common. Duplicate records are
less common but do occur in some databases, namely if there is confusion
between relation and table and if a primary key has not been defined.
For duplicate records, the standard representations are either multiple
copies or a single copy with a counter. The former method might seem
simpler to implement as the latter method requires maintenance of
counters during query operations, e.g., a multiplication in the
calculation of sums and averages, setting the counter to one during
duplicate elimination, and a multiplication of two counters in joins.

Duplicate key values in B-tree indexes are not desirable because they
may lead to ambiguities, for example during navigation from a secondary
index to a primary index or during deletion of B-tree entries pertaining
to a specific row in a table. Therefore, all B-tree entries must be made
unique as discussed earlier. Nonetheless, duplicate values in the
leading fields of a search key can be exploited to reduce storage space
as well as search effort.

The most obvious way to store non-unique keys and their associated
information combines each key value with an array representing the
information. In non-unique secondary indexes with record identifiers as
the information associated with a key, this is a traditional format. For
efficient search, for example during deletion, the list of record
identifiers is kept sorted. Some simple forms of compression might be
employed. One such scheme stores differences between neighboring values
using the minimal number of bytes instead of storing full record
identifiers. A similar scheme has been discussed above as an alternative
scheme for prefix B-trees. For efficient sequential search, offset-value
coding {[}28{]} can be adapted.

A more sophisticated variant of this scheme permits explicit control
over the key prefix stored only once and the record remainder stored in
an array. If, for example, the leading key fields are large with few
distinct values, and the final key field is small with very many
distinct values, then storing values of those leading fields once can
save storage space.

An alternative representation of non-unique secondary index employs
bitmaps. There are various forms and variants. These will be discussed
below.

The rows in \autoref{fig-3-9} show alternative representations of the same
information: (a) shows individual records repeating the duplicate key
value for each distinct record identifier associated with the key value,
which is a simple scheme that requires the most space. Example (b) shows
a list of record identifiers with each unique key value, and (c) shows a
combination of these two techniques suitable for breaking up extremely
long lists, e.g., those spanning multiple pages. Example (d) shows a
simple compression based on truncation of shared prefixes. For example,
``(9)2'' indicates that this entry is equal to the preceding one in its
first 9 letters or ``Smith, 471,'' followed by the string ``2.'' Note
that this is different from prefix B-trees, which truncate the same
prefix from all records in a page or B-tree nodes. Example (e) shows
another simple compression schemes based on run-length encoding. The
encoding ``4711(2)'' indicates a contiguous series with 2 entries
starting with 4711. Example (f) shows a bitmap as might be used in a
bitmap index. The leading value 4708 indicates the integer represented
by the first bit in the bitmap; the ``1'' bits in the bitmap represent
the values 4711, 4712, and 4723. Bitmaps themselves are often compressed
using some variant of run-length encoding. Without doubt, many readers
could design additional variations and combinations.

\begin{figure}
  \centering
  \includegraphics[width=\columnwidth]{./media/fig-3-9.png}

  \caption{Alternative representations of duplicates.\label{fig-3-9}}
\end{figure}

Each of these schemes has its own strengths and weaknesses. For example,
(d) seems to combine the simplicity of (a) with space efficiency
comparable to that of (b), but it might require special considerations
for efficient search, whether binary or interpolation search is
employed. In other words, there does not seem to be a perfect scheme.
Perhaps the reason is that compression techniques focus on sequential
access rather than random access within the compressed data structure.

These schemes can be extended for multi-column B-tree keys. For example,
each distinct value in the first field may be paired with a list of
values of the second field, and each of those has a list of detail
information. In a relational database about students and courses, as a
specific example, an index for a many-to-many relationship may have many
distinct values for the first foreign key (e.g., student identifier),
each with a list of values for the second foreign key (e.g., course
number), and additional attributes about the relationship between pair
of key values (e.g., the semester when the student took the course). For
information retrieval, a full-text index might have many distinct
keywords, each with a list of documents containing a given keyword, each
document entry having a list of occurrences of keywords with documents.
Ignoring compression, this is the basic format of many text indexes.

Duplicate key values pertain not only to representation choices but also
to integrity constraints in relational databases. B-trees are often used
to prevent violations of unique constraints by insertion of a duplicate
key value. Another technique, not commonly used, employs existing B-tree
indexes for instant creation and verification of newly defined
uniqueness constraints. During insertion of new key values, the search
for the appropriate insertion location could indicate the longest shared
prefix with either of the future neighbor keys. The required logic is
similar to the logic in dynamic prefix truncation. Based on the lengths
of such shared prefixes, the metadata of a B-tree index may include a
counter of distinct values. In multi-column B-trees, multiple counters
can be maintained. When a uniqueness constraint is declared, these
counters immediately indicate whether the candidate constraint is
already violated.

\begin{itemize}
\item
  Even if each B-tree entry is unique, keys might be divided into prefix
  and suffix such that there are many suffix values for each prefix
  value. This enables many compression techniques.
\item
  Long lists may need to be broken up into segments, with each segment
  smaller than a page.
\end{itemize}

\hypertarget{bitmap-indexes}{%
\section{Bitmap Indexes}\label{bitmap-indexes}}

The term bitmap index is commonly used, but it is quite ambiguous
without explanation of the index structure. Bitmaps can be used in
B-trees just as well as in hash indexes and other forms of indexes. As
seen in \autoref{fig-3-9}, bitmaps are one or many representation techniques
for a set of integers. Wherever a set of integers is associated with
each index key, the index can be a bitmap index. In the following,
however, a non-unique secondary B-tree index is assumed.

Bitmaps in database indexes are a fairly old idea {[}65, 103{]} that
gained importance with the rise of relational data warehousing. The only
requirement is a one-to-one mapping between information associated with
index keys and integers, i.e., the positions of bits in a bitmap. For
example, record identifiers consisting of device number, page number,
and slot number can be interpreted as a single large integer and thus
can be encoded in bitmaps and bitmap indexes.

In addition, bitmaps can be segmented and compressed. For segmentation,
the domain of possible bit positions is divided into ranges. These
ranges are numbered and a separate bitmap is created for each non-empty
range. The search key is repeated for each segment and extended by the
range number. An example for breaking lists into segments is shown in
\autoref{fig-3-9}, albeit with lists of references rather than with bitmaps.

A segment size with $2^15$ bit positions ensures that the
bitmap for any segment easily fits into a database page; a segment size
with $2^30$ bit positions ensures that standard integer
values can be used in compression by run-length encoding. Dividing
bitmaps into segments of $2^15$ or $2^30$
bit positions also enables reasonably efficient updates. For example,
insertion of a single record require decompression and re-compression of
only a single bitmap segment, and space management very similar to
changing the length of a traditional B-tree record.

For bitmap compression, most schemes rely primarily on run-length
encoding. For example, WHA {[}126{]} divides a bitmap into sections of
31 bits and replaces multiple neighboring sections with a count. In the
compressed image, a 32-bit word contains an indicator bit plus either a
literal bitmap of 31 bits or a run of constant values. In each run, a
30-bit count leaves one bit to indicate whether the replaced sections
contain ``0'' bits or ``1'' bits. Bitmap compression schemes based on
bytes rather than words tend to achieve tighter compression but require
more expensive operations {[}126{]}. This is true in particular if run
lengths are encoded in variable-length integers.

\autoref{fig-3-10} illustrates this compression technique. Example (a) shows a
bitmap similar to the ones in \autoref{fig-3-10} although with a different
third value. Example (b) shows WAH compression. Commas indicate word
boundaries in the compressed representation, underlined bit values
indicate the word usage. The bitmap starts with 151 groups of 31 ``0''
bits. The following two words show literal bitmaps; two are required
because bit positions 4711 and 4712 fall into different groups of 31 bit
positions. Five more groups of 31 ``0'' bits then skip forward toward
bit position 4923, which is shown as a single ``1'' bit in the final
literal group of 31 bits.

\begin{figure}
  \centering
  \includegraphics[width=\columnwidth]{./media/fig-3-10.png}

  \caption{A WAH-compressed bitmap.\label{fig-3-10}}
\end{figure}

Without compression, bitmap indexes are space-efficient only for very
few distinct key values in the index. With effective compression, the
size of bitmap indexes is about equal to that of traditional indexes
with lists of references broken into segments, as shown in \autoref{fig-3-10}.
For example, with WAH compression, each reference requires at most one
run of ``0'' sections plus a bitmap of 31 bits. A traditional
representation with record identifiers might also require 64 bits per
reference. Thus, bitmap indexes are useful for both sparse and dense
bitmaps, i.e., for both low- and high-cardinality attributes {[}125,
126{]}.

Bitmaps are used primarily for read-only or read-mostly data, not for
update-intensive databases and indexes. This is due to the perceived
difficulty of updating compressed bitmaps, e.g., insertion of a new
value in run-length encoding schemes such as WAH. On the other hand,
lists of record identifiers compressed using numeric differences are
very similar to the counters in run-length encoding. Update costs should
be very similar in these two compressed storage formats.

The primary operations on bitmaps are creation, intersection, union,
difference, and scanning. Bitmap creation occurs during index creation,
and, when bitmaps are used to represent intermediate query results,
during query execution. Bitmap intersection aids in conjunctive
(``and'') query predicates, union in disjunctive (``or'') predicates.
Note that range queries on integer keys can often be translated into
disjunctions, e.g., ``\ldots between 3 and 5'' is equivalent to
``$\ldots = 3$ or $\ldots = 4$ or $\ldots = 5$.'' Thus, even if
most query predicates are written as conjunctions rather than
disjunctions, union operations are important for bitmaps and lists of
references.

Using a bitmap representation for an intermediate query result
implicitly sorts the data. This is particularly useful when retrieving
an unpredictable number of rows from a table using references obtained
from a secondary index. Gathering references in a bitmap and than
fetching the required database rows in sorted order is often more
efficient then fetching the rows without sorting. A traditional sort
operation might require more memory and more effort than a bitmap.

In theory, bitmaps can be employed for any Boolean property. In other
words, a bit in a bitmap indicates whether or not a certain record has
the property of interest. The discussion above and the example in \autoref{fig-3-9}
implicitly assume that this property is equality with a certain key
value. Thus, there is a bitmap for each key value in an index indicating
the records with those key values. Another scheme is based on modulo
operations {[}112{]}. For example, if a column to be indexed is a 32-bit
integer, there are 32 bitmaps. The bitmap for bit position $k$
indicates the records in which the key value modulo
$2^k$ is nonzero. Queries need to perform
intersection and union operations. Many other schemes, e.g., based on
range predicates, could also be designed. O'Neil et al. {[}102{]} survey
many of the design choices.

Usually, bitmap indexes represent one-to-many relationships, e.g.,
between key values and references. In these cases, a specific bit
position is set to ``1'' in precisely one of the bitmaps in the index
(assuming there is a row corresponding to the bit position). In some
cases, however, a bitmap index may represent a many-to-many
relationship. In those cases, the same bit position may be set to ``1''
in multiple bitmaps. For example, if a table contains two foreign keys
to capture a many-to-many relationship, one of the foreign key columns
might provide the key values in a secondary index and the other foreign
key column is represented by bitmaps. As a more specific example, the
many-to-many relationship enrollment between students and courses might
be represented by a B-tree on student identifier. A student's courses
can be captured in a bitmap. The same bit position representing a
specific course is set to ``1'' in many bitmaps, namely in the bitmaps
of all students enrolled in that course.

\begin{itemize}
\item
  Bitmaps require a one-to-one relationship between values and bit
  positions.
\item
  Bitmaps and compressed bitmaps are just another format to represent
  duplicate (prefix) values.
\item
  Bitmaps can be useful to represent all suffix values associated with
  distinct prefix values.
\item
  Run-length encoding as a compression technique for bitmaps is similar
  to compressing a list of integer values by sorting the list and
  storing the differences between neighbors. Based on this similarity,
  both techniques for representing duplicate values can be similarly
  space efficient.
\end{itemize}

\hypertarget{data-compression}{%
\section{Data Compression}\label{data-compression}}

Data compression reduces the expense of purchasing storage devices. It
also reduces the cost to house, connect, power, and cool these devices.
Moreover, it can improve the effective scan bandwidth as well as the
bandwidths of utilities such as defragmentation, consistency checks,
backup, and restore. Flash devices, due to their high cost per unit of
storage space, are likely to increase the interest in data compression
for file systems, databases, etc.

B-trees are the primary data structure in databases, justifying
compression techniques tuned specifically for B-tree indexes.
Compression in B-tree indexes can be divided into compression of key
values, compression of node references (primarily child pointers), and
representation of duplicates. Duplicates have been discussed above; the
other two topics are surveyed here.

For key values, prefix and suffix truncation have already been
mentioned, as has single storage of non-unique key values. Compression
of normalized keys has also been mentioned, albeit as a problem without
published techniques. Another desirable form of compression is
truncation of zeroes and spaces, with careful attention to
order-preserving truncation in keys {[}2{]}.

Other order-preserving compression methods seem largely ignored in
database systems, for example order-preserving Huffman coding or
arithmetic coding. Order-preserving dictionary codes received initial
attention {[}127{]}. Their potential usage in sorting, in particular
sorting in database query processing, is surveyed elsewhere {[}46{]};
many of the considerations there also apply to B-tree indexes.

For both compression and de-compression, order-preserving Huffman codes
rely on binary trees. For static codes, the tree is similar to the tree
for nonorder-preserving techniques. Construction of a Huffman code
starts with each individual symbol forming a singleton set and then
repeatedly merges two sets of symbols. For a standard Huffman code, the
two sets with the lowest frequencies are merged. For an order-preserving
Huffman code, the pair of immediate neighbors with the lowest combined
frequency is chosen. Both techniques support static and adaptive codes.
Adaptive methods start with a tree created as for a static method but
modify it according to the actual, observed frequency of symbols in the
uncompressed stream. Each such modification rotates nodes in the binary
tree.

\autoref{fig-3-11}, copied from {[}46{]}, shows a rotation in the binary tree
central to encoding and decoding in order-preserving Huffman
compression. The leaf nodes represent symbols and the root-to-leaf paths
represent the encodings. With a left branch encoded by a 0 and a right
branch by a 1, the symbols ``A,'' ``B,'' and ``C'' have encodings ``0,''
``10,'' and ``11,'' respectively. The branch nodes of the tree contain
separator keys, very similar to separator keys in B-trees. The left tree
in \autoref{fig-3-11} is designed for relatively frequent ``A'' symbols. If the
symbol ``C'' is particularly frequent, the encoding tree can be rotated
into the right tree, such that the symbols ``A,'' ``B,'' and ``C'' have
encodings ``00,'' ``01,'' and ``1,'' respectively. The rotation from the
left tree in \autoref{fig-3-11} to the right tree is worthwhile if the
accumulated weight in leaf node C is higher than that in leaf node A,
i.e., if effective compression is more important for leaf node C than
for leaf node A. Note that the frequency of leaf node B is not relevant
and the size of its encoding is not affected by the rotation, and that
this tree transformation is not suitable to minimize the path to node B
or the representation of B.

\begin{figure}
  \centering
  \includegraphics[width=\columnwidth]{./media/fig-3-11.png}

  \caption{Tree rotation in adaptive order-preserving Huffman
  compression.\label{fig-3-11}}
\end{figure}

Compression of B-tree child pointers may exploit the fact that
neighboring nodes are likely to have been allocated in neighboring
locations while a B-tree is created from a sorted stream of future index
entries. In this case, child pointers in a parent page can be compressed
by storing not the absolute values of pointers but their numeric
differences, and by storing those in the fewest words possible
{[}131{]}. In the extreme case, a form of run-length encoding can be
employed that simply indicates a starting node location and the number
of neighbor nodes allocated contiguously. Since careful layout of B-tree
nodes can improve scan performance, such allocation of B-tree nodes is
often created and maintained using appropriate space management
techniques. Thus, this compression technique often applies and it is
used in products. In addition to child pointers within B-tree indexes, a
variant can also be applied to a list of references associated with a
key value in a non-unique secondary index.

Compression using numeric differences is also a mainstay technique in
document retrieval, where ``an inverted index... records, for each
distinct word or term, the list of documents that contain the term, and
depending on the query modalities that are being supported, may also
incorporate the frequencies and impacts of each term in each document,
plus a list of the positions in each document at which that word
appears. For effective compression, the lists of document and position
numbers are usually sorted and transformed to the corresponding sequence
of differences (or gaps) between adjacent values.'' {[}1{]}. Research
continues to optimize compression effectiveness, i.e., the bits required
for values and length indicators for the values, and decompression
bandwidth. For example, Anh and Moffat {[}1{]} evaluate schemes in which
a single length indicator applies to all differences encoded in a single
machine word. Many more ideas and techniques can be found in dedicated
books and surveys, e.g., {[}124, 129{]}.

\begin{itemize}
\item
  Various data compression schemes exist for separator keys and child
  pointers in branch nodes and for key values and their associated
  information in leaf nodes.
\item
  Standard techniques are truncation of blank spaces and zeroes,
  representing values by their difference from a base value, and
  representing a sorted list of numbers by their differences.
  Offset-value coding is particularly effective for sorted runs in a
  merge sort but can also be used in B-trees.
\item
  Order-preserving, dynamic variants exist for Huffman compression,
  dictionary compression, and arithmetic compression.
\end{itemize}

\hypertarget{space-management}{%
\section{Space Management}\label{space-management}}

It is sometimes said that, in contrast to heap files, B-trees have space
management for records built-in. On the other hand, one could also say
that record placement in B-trees offers no choice even if multiple pages
have some free space; instead, a new record must be placed where its key
belongs and cannot be placed anywhere else.

There are some opportunities for good space management, however. First,
when an insertion fails due to insufficient space in the appropriate
node, a choice is required among compaction (reclamation of free space
within the page), compression (re-coding keys and their associated
information), load balancing (among sibling nodes), and splitting. As
simple and local operations are preferable, the sequence given indicates
the best approach. Load balancing among two neighbors is rarely
implemented; load balancing among more than two neighbors hardly ever.
Some defragmentation utilities, however, might be invoked for specific
key ranges only rather than for an entire B-tree.

Second, when splitting and thus page allocation are required, the
location of the new page offers some opportunities for optimization. If
large range scans and index-order scans are frequent, and if the B-tree
is stored on disks with expensive seek operations, it is important to
allocate the new page near the existing page.

Third, during deletion, similar choices exist. Load balancing among two
neighbors can be required during deletion in order to avoid underflow,
whereas it is an optional optimization for insertion. A commonly used
alternative to the ``text book'' design for deletion in B-trees ignores
underflows and, in the extreme cases, permits even empty pages in a
B-tree. Space reclamation is left to future insertions or to a
defragmentation utility.

In order to avoid or at least delay node splits, many database systems
permit leaving some free space in every page during index creation, bulk
loading, and defragmentation. For example, leaving 10\% free space in
all branch nodes hardly affects their fan-out or the height of the tree,
but it reduces the overhead of node splits during transaction
processing. In addition, some systems permit leaving free pages on disk.
For example, if the unit of I/O in large scans contains multiple B-tree
nodes, it can be advantageous to leave a few pages unallocated in each
such unit. If a node splits, a nearby page is readily available for
allocation. Until many nodes in the B-tree have been split due to many
insertions, scan performance is not affected.

An interesting approach to free space management on disk relies on the
core logic of B-trees. O'Neil's SB-trees {[}104{]} allocate disk space
in large contiguous extents of many pages, leaving some free pages in
each extent during index creation and defragmentation. When a node
splits, a new node is allocated within the same extent. If that is not
possible because the entire extent is allocated, the extent is split
into two extents, each half full. This split is quite similar to a node
split in a B-tree. While simple and promising, this idea has not been
widely adopted. This pattern of ``self-similar'' data structures and
algorithms can be applied at multiple levels of the memory hierarchy.

\autoref{fig-3-12} shows the two kinds of nodes in an SB-tree. Both extents and
pages are nodes in the sense that they may overflow and then are split
in half. The child pointers in page 75.2 contain very similar values for
page identifiers and thus are amenable to compression. When, for
example, page 93.4 must be split in response to an insertion, the entire
extent 93 is split and multiple pages, e.g., 93.3--93.5, moved to a new
extent.

\begin{figure}
  \centering
  \includegraphics[width=\columnwidth]{./media/fig-3-12.png}

  \caption{Nodes in an SB-tree.\label{fig-3-12}}
\end{figure}

\begin{itemize}
\item
  B-trees rigidly place a new record according to its sort key but
  handle space management gracefully, e.g., by load balancing among
  neighbor nodes.
\item
  B-tree concepts apply not only to placement of records in pages but
  also to placement of pages in contiguous clusters of pages on the
  storage media.
\end{itemize}

\hypertarget{splitting-nodes}{%
\section{Splitting Nodes}\label{splitting-nodes}}

After a leaf node is split into two, a new separator key must be posted
in the parent node. This might cause an overflow in the parent,
whereupon the parent node must be split into two and a separator key
must be posted in the grandparent node. In the extreme case, nodes from
a leaf to the root must be split and a new root must be added to the
B-tree.

The original B-tree algorithms called for leaf-to-root splitting as just
described. If, however, multiple threads or transactions share a B-tree,
then the bottom-up (leaf-to-root) splits in one thread might conflict
with a top-down (root-to-leaf) search of the other thread. The earliest
design relied on the concept of a ``safe'' node, i.e., one with space
for on more insertion, and retained locks from the last safe node during
a root-to-leaf search {[}9{]}. A more drastic approach restricts each
B-tree to only one structural change at a time {[}93{]}. Three other,
less restrictive solutions have been used for this problem.

First, since only few insertions require split operations, one can force
such an insertion to perform an additional root-to-leaf traversal. The
first traversal determines the level at which a split is required. The
second traversal performs a node split at the appropriate level. If it
is unable to post the separator key as required, it stops and instead
invokes another root-to-leaf pass that performs a split at the next
higher level. This additional root-to-leaf traversal can be optimized.
For example, if the upper B-tree nodes have not been changed in the
meantime, there is no need to repeat binary search with known outcomes.

Second, the initial root-to-leaf search of an insertion operation may
verify that all visited nodes have sufficient free space for one more
separator key. A branch node without sufficient free space is split
preventively {[}99{]}. Thus, a single root-to-leaf search promises to
perform all insertions and node splits. If each node can hold hundreds
of separator keys, splitting a little earlier than truly required does
not materially affect B-tree space utilization, node fan-out, or tree
height.

Unfortunately, variable-length separator keys present a problem; either
the splitting decision must be extremely conservative or there may be
rare cases in which a second root-to-leaf pass is required as in the
first solution described in the preceding paragraph. In other words, an
implementation of the first solution might be required in any case. If
node splits are rare, adding a heuristic code path with its own test
cases, regression tests, etc. might not provide a worthwhile or even
measurable performance gain.

Third, splitting a B-tree node and posting a new separator key in the
node's parent is divided into two steps {[}81{]}. During the
intermediate state, which may last a long time but ideally does not, the
B-tree node looks similar to the ternary node in a 2-3-tree as shown in
\autoref{fig-2-2}. In other words, two separate steps split a full node in two
and post a separator key in the parent node. For a short time, the new
node is linked to the old neighbor, not its parent, giving rise to the
name B\textsuperscript{link}-trees. As soon as convenient, e.g., during
the next root-to-leaf traversal, the separator key and the pointer are
copied from the formerly overflowing sibling node to the parent node.

\begin{itemize}
  \item
  Some variations of the original B-tree structure enable high
concurrency and efficient concurrency control.
B\textsuperscript{link}-trees seem particularly promising although they
seem to have been overlooked in products.
\end{itemize}

\hypertarget{summary}{%
\section{Summary}\label{summary}}

In summary, the basic B-tree design, both data structure and algorithms,
have been refined in many ways in decades of research and implementation
efforts. Many industrial implementations employ many of the techniques
reviewed so far. Research that ignores or even contradicts these
techniques may be perceived as irrelevant to commercial database
management products.


\end{document}
