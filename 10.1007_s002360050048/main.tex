\PassOptionsToPackage{unicode=true}{hyperref} % options for packages loaded elsewhere
\PassOptionsToPackage{hyphens}{url}
%
\documentclass[a4paper,11pt,notitlepage,twoside,openright]{article}

\usepackage{ifxetex}
\ifxetex{}
\else
  \errmessage{Must be built with xelatex}
\fi

\usepackage{amssymb,amsmath}
\usepackage{fourier}
\usepackage{inconsolata}
% Math
\usepackage[binary-units]{siunitx}

\usepackage{caption}
\usepackage{authblk}
\usepackage{enumitem}
\usepackage{footnote}

% Table
\usepackage{tabu}
\usepackage{longtable}
\usepackage{booktabs}
\usepackage{multirow}

% Verbatim & Source code
\usepackage{fancyvrb}
\usepackage{minted}

% Beauty
\usepackage[protrusion]{microtype}
\usepackage[all]{nowidow}
\usepackage{upquote}
\usepackage{parskip}
\usepackage[strict]{changepage}

\usepackage{hyperref}

% Graph
\usepackage{graphicx}
\usepackage{grffile}
\usepackage{tikz}


\hypersetup{
  bookmarksnumbered,
  pdfborder={0 0 0},
  pdfpagemode=UseNone,
  pdfstartview=FitH,
  breaklinks=true}
\urlstyle{same}  % don't use monospace font for urls

\usetikzlibrary{arrows.meta,calc,shapes.geometric,shapes.misc}

\setminted{
  autogobble,
  breakbytokenanywhere,
  breaklines,
  fontsize=\footnotesize,
}
\setmintedinline{
  autogobble,
  breakbytokenanywhere,
  breaklines,
  fontsize=\footnotesize,
}

\makeatletter
\def\maxwidth{\ifdim\Gin@nat@width>\linewidth\linewidth\else\Gin@nat@width\fi}
\def\maxheight{\ifdim\Gin@nat@height>\textheight\textheight\else\Gin@nat@height\fi}
\makeatother

% Scale images if necessary, so that they will not overflow the page
% margins by default, and it is still possible to overwrite the defaults
% using explicit options in \includegraphics[width, height, ...]{}
\setkeys{Gin}{width=\maxwidth,height=\maxheight,keepaspectratio}
\setlength{\emergencystretch}{3em}  % prevent overfull lines
\setcounter{secnumdepth}{3}

% Redefines (sub)paragraphs to behave more like sections
\ifx\paragraph\undefined\else
\let\oldparagraph\paragraph{}
\renewcommand{\paragraph}[1]{\oldparagraph{#1}\mbox{}}
\fi
\ifx\subparagraph\undefined\else
\let\oldsubparagraph\subparagraph{}
\renewcommand{\subparagraph}[1]{\oldsubparagraph{#1}\mbox{}}
\fi

% set default figure placement to htbp
\makeatletter
\def\fps@figure{htbp}
\makeatother


\title{The log-structured merge-tree (LSM-tree)}
\author[1]{Patrick O'Neil}
\affil[1]{Department of Mathematics and Computer Science,
University of Massachusetts/Boston,
Boston, MA 02125--3393, USA (e-mail: \{poneil/eoneil\}@cs.umb.edu)}
\author[2]{Edward Cheng}
\affil[2]{Digital Equipment Corporation, Palo Alto, CA 94301,
USA (e-mail: edwardc@pa.dec.com)}
\author[3]{Dieter Gawlick}
\affil[3]{Oracle Corporation, Redwood Shores, CA, USA (e-mail:
dgawlick@us.oracle.com)}
\author[1]{Elizabeth O'Neil}

\date{Received July 6, 1992 / Apri1 11, 1995}

\begin{document}

\maketitle

\begin{abstract}
High-performance transaction system applications typically insert rows
in a History table to provide an activity trace; at the same time the
transaction system generates log records for purposes of system
recovery. Both types of generated information can benefit from emcient
indexing. An example in a well-known setting is the TPC-A benchmark
application, modified to support emcient queries on the history for
account activity for specific accounts. This requires an index by
account-id on the fast-growing History table. Unfortunately, standard
disk-based index structures such as the B-tree will effectively double
the I/O cost of the transaction to maintain an index such as this in
real time, increasing the total system cost up to fifty percent. Clearly
a method for maintaining a real-time index at low cost is desirable. The
log-structured mergetree (LSM-tree) is a disk-based data structure
designed to provide low-cost indexing for a file experiencing a high
rate of record inserts (and deletes) over an extended period. The
LSM-tree uses an algorithm that defers and batches index changes,
cascading the changes from a memory-based component through one or more
disk components in an emcient manner reminiscent of merge sort. During
this process all index values are continuously accessible to retrievals
(aside from very short locking periods), either through the memory
component or one of the disk components. The algorithm has greatly
reduced disk arm movements compared to a traditional access methods such
as B-trees, and will improve cost-performance in domains where disk arm
costs for inserts with traditional access methods overwhelm storage
media costs. The LSM-tree approach also generalizes to operations other
than insert and delete. However, indexed finds requiring immediate
response will lose I/O emciency in some cases, so the LSM-tree is most
useful in applications where index inserts are more common than finds
that retrieve the entries. This seems to be a common property for
history tables and log files, for example. The conclusions of Sect. 6
compare the hybrid use of memory and disk components in the LSM-tree
access method with the commonly understood advantage of the hybrid
method to buffer disk pages in memory.
\end{abstract}

\hypertarget{introduction}{%
\section{Introduction}\label{introduction}}


As long-lived transactions in activity flow management systems become
commercially available {[}10--12, 19, 24, 27{]}, there will be
increased need to provide indexed access to transactional log records.
Traditionally, transactional logging has focused on aborts and recovery,
and has required the system to refer back to a relatively short-term
history in normal processing with occasional transaction rollback, while
recovery was performed using batched sequential reads. However, as
systems take on responsibility for more complex activities, the duration
and number of events that make up a single long-lived activity will
increase to a point where there is sometimes a need to review past
transactional steps in real time to remind users of what has been
accomplished. At the same time, the total number of active events known
to a system will increase to the point where memory-resident data
structures now used to keep track of active logs are no longer feasible,
notwithstanding the continuing decrease in memory cost to be expected.
The need to answer queries about a vast number of past activity logs
implies that indexed log access will become more and more important.

Even with current transactional systems there is clear value in
providing indexing to support queries on history tables with high insert
volume. Networking, electronic mail, and other nearly-transactional
systems produce huge logs often to the detriment of their host systems.
To start from a concrete and well-known example, we explore a modified
TPC-A benchmark in the following Examples 1.1 and 1.2. Note that
examples presented in this paper deal with specific numeric parametric
values for ease of presentation; it is a simple task to generalize these
results. Note too that although both history tables and logs involve
time-series data, the index entries of the LSM-tree are not assumed to
have identical temporal key order. The only assumption for improved
emciency is high update rates compared to retrieval rates.


\hypertarget{the-five-minute-rule}{%
\subsection{The five minute rule}\label{the-five-minute-rule}}


The following two examples both depend on the five minute rule {[}13{]}.
This basic result states that we can reduce system costs by purchasing
memory buffer space to keep pages in memory, thus avoiding disk I/O,
when page reference frequency exceeds about once every 60 s. The time
period of 60 s is approximate, a ratio between the amortized cost for a
disk arm providing one I/O per second and memory cost to buffer a disk
page of 4 KBytes amortized over one second. In terms of the notation of
Sect. 3, the ratio is COSTp/COSTm divided by the page size in Mbytes.
Here we are simply trading off disk accesses for memory buffers while
the tradeoff gives economic gain. Note that the 60 s time period is
expected to grow over the years as memory prices come down faster than
disk arms. The reason it is smaller now in 1995 than when defined in
1987 when it was 5 min, is partly technical (different buffering
assumptions) and partly due to the intervening introduction of extremely
inexpensive mass-produced disks.

Example 1.1. Consider the multi-user application envisioned by the TPC-A
benchmark {[}26{]} running 1000 transactions per second (this rate can
be scaled, but we will consider only 1000 TPS in what follows). Each
transaction updates a column value, withdrawing an amount Delta from a
Balance column, in a randomly chosen row containing 100 bytes, from each
of three tables: the Branch table, with 1000 rows, the Teller table with
10000 rows, and the account table, with 100000000 rows; the transaction
then writes a 50 byte row to a History table before committing, with
columns: Account-ID, Branch-ID, Teller-ID, Delta, and Timestamp.

Accepted calculations projecting disk and memory costs shows that
account table pages will not be memory resident for a number of years to
come {[}6{]}, while the Branch and Teller table should be entirely
memory resident now. Under the assumptions given, repeated references to
the same disk page of the Accounts table will be about 2500 s, apart
well below the frequency needed to justify buffer residerice by the five
minute rule. Now each transaction requires about two disk I/Os, one to
read in the desired account record (we treat the rare case where the
page accessed is already in buffer as insignificant), and one to write
out a prior dirty account page to make space in buffers for a read
(necessary for steady-state behavior). Thus 1000 TPS will correspond to
about 2000 1/Os per second. This requires 80 disk arms (actuators) at
the nominal rate of 25 1/Os per disk-armsecond assumed in {[}13{]}. In
the 8 yr since then (1987 to 1995) the rate has climbed by less than
10\%/year so that the nominal rate is now about 40 1/Os per second, or
50 disk arms for 20001/Os per second. The cost of disk for the TPC
application was calculated to be about half the total cost of the system
in {[}6{]}, although it is somewhat less on IBM mainframe systems.
However, the cost for supporting I/O is clearly a growing component of
the total system cost as the cost of both memory and CPU drop faster
than disk.

Example 1.2. Now we consider an index on the high insert volume History
table, and demonstrate that such an index essentially doubles the disk
cost for the TPC application. An index on "account-ID concatenated with
timestamp" (AcctID Il Timestamp) for the History table is crucial to
support emcient queries on recent account activity such as:

Select * from History

where History.Acct-ID = \%custacctid

(1.1) and History, Timestamp \textgreater{} \%custdatetime;

If an Acc-ID" Timestamp index is not present, such a query requires a
direct search of all rows of the history table, and thus becomes
impractical. An index on Acct-ID alone provides most of the benefit, but
cost considerations that follow do not change if the Timestamp is left
out, so we assume here the more useful concatenated index. What
resources are required to maintain such a secondary B-tree index in real
time? We see that the entries in the B-tree are generated 1000 per
second, and assuming a 20 day period of accumulation, with eight hour
days and 16 byte index entries, this implies 576 000000 entries on 9.2
GBytes of disk, or about 2.3 million pages needed on the index leaf
level, even if there is no wasted space. Since transactional Acct-ID
values are randomly chosen, each transaction will require at least one
page read from this index, and in the steady state a page write as well.
By the five minute rule these index pages will not be buffer resident
(disk page reads about 2300 s apart), so all I/Os are to disk. This
addition of 2000 I/Os per second to the 2000 1/Os already needed for
updating the Account table, requires a purchase of an additional 50 disk
arms, doubling our disk requirements. The figure optimistically assumes
that deletes needed to keep the log file index only 20 days in length
can be performed as a batch job during slack use times.

We have considered a B-tree for the Acct-ID Il Timestamp index on the
History file because it is the most common disk-based access method used
in commercial systems, and in fact no classical disk indexing structure
consistently gives superior I/O cost/performance. We will discuss the
considerations that lead us to this conclusion in Sect. 5.

The LSM-tree access method presented in this paper enables us to perform
the frequent index inserts for the Account-ID I l Timestamp index with
much less disk arm use, therefore at an order of magnitude lower cost.
The LSM-tree uses an algorithm that defers and batches index changes,
migrating the changes out to disk in a particularly effcient way
reminiscent of merge sort. As we shall see in Sect. 5, the function of
deferring index entry placement to an ultimate disk position is of
fundamental importance, and in the general LSM-tree case there is a
cascaded series of such deferred placements. The LSM-tree structure also
supports other operations of indexing such as deletes, updates, and even
long latency find operations with the same deferred emciency. Only finds
that require immediate response remain relatively costly. A major area
of effective use for the LSM-tree is in applications such as Example 1.2
where retrieval is much less frequent than insert (most people do not
ask for recent account activity nearly as often as they write a check or
make a deposit). In such a situation, reducing the cost of index inserts
is of paramount importance; at the same time, find access is frequent
enough that an index of some kind must be maintained, because a
sequential search through all the records is out of the question.

Here is the plan of the paper. In Sect. 2, we introduce the
two-component LSM-tree algorithm. In Sect. 3, we analyze the performance
of the LSM-tree, and motivate the multi-component LSM-tree. In Sect. 4
we sketch the concepts of concurrency and recovery for the LSM-tree. In
Sect. 5 we consider competing access methods and their performance for
applications of interest. Section 6 contains conclusions, where we
evaluate some implications of the LSM-tree, and provide a number of
suggestions for extensions.


\hypertarget{the-two-component-lsm-tree-algorithm}{%
\section{The two component LSM-tree
algorithm}\label{the-two-component-lsm-tree-algorithm}}


An LSM-tree is composed of two or more tree-like component data
structures. We deal in this section with the simple two component case
and assume in what follows that the LSM-tree is indexing rows in a
History table as in Example 1.2 (see Fig. 1, below).

A two component LSM-tree has a smaller component which is entirely
memory resident, known as the Co tree (or Co component), and a larger
component which is resident on disk, known as the Cl tree (or Cl
component). Although the Cl component is disk resident, frequently
referenced page nodes in Cl will remain in memory buffers as usual
(buffers not shown), so that popular high level directory nodes of Cl
can be counted on to be memory resident.


C, tree Co tree


\includegraphics[width=2.82817in,height=0.70359in]{./media/image2.jpg}


Disk Memory


Fig. 1. Schematic picture of an LSM-tree of two components

As each new History row is generated, a log record to recover this
insert is first written to the sequential log file in the usual way. The
index entry for the History row is then inserted into the memory
resident Co tree, after which it will in time migrate out to the Cl tree
on disk; any search for an index entry will look first in Co and then in
C 1 . There is a certain amount of latency (delay) before entries in the
Co tree migrate out to the disk resident Cl tree, implying a need for
recovery of index entries that do not get out to disk prior to a crash.
Recovery is discussed in Sect. 4, but for now we simply note that the
log records that allow us to recover new inserts of History rows can be
treated as logical logs; during recovery we can reconstruct the History
rows that have been inserted and simultaneously recreate any needed
entries to index these rows to recapture the lost content of Co.

The operation of inserting an index entry into the memory resident Co
tree has no I/O cost. However, the cost of memory capacity to house the
Co component is high compared to disk, and this imposes a limit on its
size. We need an effcient way to migrate entries out to the Cl tree that
resides on the lower cost disk medium. To achieve this, whenever the Co
tree as a result of an insert reaches a threshold size near the maximum
allotted, an ongoing rolling merge process serves to delete some
contiguous segment of entries from the Co tree and merge it into the Cl
tree on disk. Figure 2 depicts a conceptual picture of the rolling merge
process.

The Cl tree has a comparable directory structure to a B-tree, but is
optimized for sequential disk access, with nodes 100\% full, and
sequences of single-page nodes on each level below the root packed
together in contiguous multi-page disk blocks for emcient arm use; this
optimization was also used in the SB-tree {[}21{]}. Multi-page block I/O
is used during the rolling merge and for long range retrievals, while
single-page nodes are used for matching indexed finds to minimize
buffering requirements. Multi-page block sizes of 256 KBytes are
envisioned to contain nodes below the root; the root node is always a
single page by definition.

\includegraphics[width=3.2584in,height=2.63762in]{./media/image3.jpg}

Fig. 2. Conceptual picture of rolling merge steps, with result written
back to disk

The rolling merge acts in a series of merge steps. A read of a
multi-page block containing leaf nodes of the Cl tree makes a range of
entries in Cl buffer resident. Each merge step then reads a disk page
sized leaf node of the C 1 tree buffered in this block, merges entries
from the leaf node with entries taken from the leaf level of the Co
tree, thus decreasing the size of Co, and creates a newly merged leaf
node of the Cl tree.

The buffered multi-page block containing old Cl tree nodes prior to
merge is called the emptying block, and new leaf nodes are written to a
different buffered multi-page block called the filling block. When this
filling block has been packed full with newly merged leaf nodes of Cl,
the block is written to a new free area on disk. The new multi-page
block containing merged results is pictured in Fig. 2 as lying on the
right of the former nodes. Subsequent merge steps bring together
increasing index value segments of the Co and Cl components until the
maximum values are reached and the rolling merge starts again from the
smallest values.

Newly merged blocks are written to new disk positions, so that the old
blocks will not be overwritten and will be available for recovery in
case of a crash. The parent directory nodes in Cl, also buffered in
memory, are updated to reflect this new leaf structure, but usually
remain in buffer for longer periods to minimize I/O; the old leaf nodes
from the Cl component are invalidated after the merge step is complete
and are then deleted from the Cl directory. In general, there will be
leftover leaf-level entries for the merged Cl component following each
merge step, since a merge step is unlikely to result in a new node just
as the old leaf node empties. The same consideration holds for
multi-page blocks, since in general when the filling block has filled
with newly merged nodes, there will be numerous nodes containing entries
still in the shrinking block. These leftover entries, as well as updated
directory node information, remain in block memory buffers for a time
without being written to disk. Techniques to provide concurrency during
the merge step and recovery from lost memory during a crash are covered
in detail in Sect. 4. To reduce reconstruction time in recovery,
checkpoints of the merge process are taken periodically, forcing all
buffered information to disk.


\hypertarget{how-a-two-component-lsm-tree-grows}{%
\subsection{How a two component LSM-tree
grows}\label{how-a-two-component-lsm-tree-grows}}


To trace the metamorphosis of an LSM-tree from the beginning of its
growth, let us begin with a first insertion to the Co tree component in
memory. Unlike the Cl tree, the Co tree is not expected to have a
B-tree-like structure. For one thing, the nodes could be any size: there
is no need to insist on disk page size nodes since the Co tree never
sits on disk, and so we need not sacrifice CPU emciency to minimize
depth. Thus a (2---3) tree or AVL-tree (as explained, for example, in
are possible alternative structures for a Co tree. When the growing Co
tree first reaches its threshold size, a leftmost sequence of entries is
deleted from the Co tree (this should be done in an emcient batch manner
rather than one entry at a time) and reorganized into a Cl tree leaf
node packed 100\% full. Successive leaf nodes are placed left-to-right
in the initial pages of a buffer resident multi-page block until the
block is full; then this block is written out to disk to become the
first part of the Cl tree disk-resident leaf level. A directory node
structure of the Cl tree is created in memory buffers as successive leaf
nodes are added, with details explained below.

Successive multi-page blocks of the Cl tree leaf level in ever
increasing keysequence order are written out to disk to keep the Co tree
threshold size from exceeding its threshold. Upper level Cl tree
directory nodes are maintained in separate multi-page block buffers, or
else in single page buffers, whichever makes more sense from a
standpoint of total memory and disk arm cost; entries in these directory
nodes contain separators that channel access to individual single-page
nodes below, as in a B-tree. The intention is to provide emcient
exact-match access along a path of single page index nodes down to the
leaf level, avoiding multi-page block reads in such a case to minimize
memory buffer requirements. Thus we read and write multi-page blocks for
the rolling merge or for long range retrievals, and single-page nodes
for indexed find (exact-match) access. A somewhat different architecture
that supports such a dichotomy is presented in {[}21{]}. Partially full
multi-page blocks of Cl directory nodes are usually allowed to remain in
buffer while a sequence of leaf node blocks are written out. Cl
directory nodes are forced to new positions on disk when:


\begin{itemize}
\item
  A multi-page block buffer containing directory nodes becomes full.
\item
  The root node splits, increasing the depth of the Cl tree (to a depth
  greater than two).
\item
  A checkpoint is performed.
\end{itemize}


In the first case, the single multi-page block which has filled is
written out to disk. In the latter two cases, all multi-page block
buffers and directory node buffers are flushed to disk.

After the rightmost leaf entry of the Co tree is written out to the Cl
tree for the first time, the process starts over on the left end of the
two trees, except that now and with successive passes multi-page
leaf-level blocks of the Cl tree must be read into buffer and merged
with the entries in the Co tree, thus creating new multi-page leaf
blocks of Cl to be written to disk.

Once the merge starts, the situation is more complex. We picture the
rolling merge process in a two component LSM-tree as having a conceptual
cursor which slowly circulates in quantized steps through equal key
values of the Co tree and Cl tree components, drawing indexing data out
from the Co tree to the Cl tree on disk. The rolling merge cursor has a
position at the leaf level of the Cl tree and within each higher
directory level as well. At each level, all currently merging multi-page
blocks of the Cl tree will in general be split into two blocks: the
"emptying" block whose entries have been depleted but which retains
information not yet reached by the merge cursor, and the "filling" block
which reflects the result of the merge up to this moment. There will be
an analogous "filling node" and "emptying node" defining the cursor
which will certainly be buffer resident. For concurrent access purposes,
both the emptying block and the filling block on each level contain an
integral number of page-sized nodes of the Cl tree, which simply happen
to be buffer resident. (During the merge step that restructures
individual nodes, other types of concurrent access to the entries on
those nodes are blocked.) Whenever a complete flush of all buffered
nodes to disk is required, all buffered information at each level must
be written to new positions on disk (with positions reflected in
superior directory information, and a sequential log entry for recovery
purposes). At a later point, when the filling block in buffer on some
level of the Cl tree fills and must be flushed again, it goes to a new
position. Old information that might still be needed during recovery is
never overwritten on disk, only invalidated as new writes succeed with
more up-to-date information. A somewhat more detailed explanation of the
rolling merge process is presented in Sect. 4, where concurrency and
recovery designs are considered.

It is an import emciency consideration of the LSM-tree that when the
rolling merge process on a particular level of the Cl tree passes
through nodes at a relatively high rate, all reads and writes are in
multi-page blocks. By eliminating seek time and rotational latency, we
expect to gain a large advantage over random page I/O involved in normal
B-tree entry insertion. (This advantage is analyzed below, in Sect.
3.2.) The idea of always writing multi-page blocks to new locations was
inspired by the log-structured file system devised by Rosenblum and
Ousterhout {[}23{]}, from which the log-structured merge-tree takes its
name. Note that the continuous use of new disk space for fresh
multi-page block writes implies that the area of disk being written will
wrap, and old discarded blocks must be reused. This bookkeeping can be
done in a memory table; old multi-page blocks are invalidated and reused
as single units, and recovery is guaranteed by the checkpoint. In the
log-structured file system, the reuse of old blocks involves significant
I/O because blocks are typically only partially freed up, so reuse
requires a block read and block write. In the LSM-tree, blocks are
totally freed up on the trailing edge of the rolling merge, so no extra
I/O is involved.


\hypertarget{finds-in-the-lsm-tree-index}{%
\subsection{Finds in the LSM-tree
index}\label{finds-in-the-lsm-tree-index}}


When an exact-match find or range find requiring immediate response is
performed through the LSM-tree index, first the Co tree and then the Cl
tree is searched for the value or values desired. This may imply a
slight CPU overhead compared to the B-tree case, since two directories
may need to be searched. In LSM-trees with more than two components,
there may also be an I/O overhead. To anticipate Sect. 3 somewhat, we
define a multi component LSM-tree as having components
\includegraphics[width=0.47358in,height=0.12671in]{./media/image4.jpg}.
, and CK, indexed tree structures of increasing size, where Co is memory
resident and all other components are disk resident. There are
asynchronous rolling merge processes in train between all component
pairs (Ci--- 1, CD that move entries out from the smaller to the larger
component each time the smaller component, Ci--- 1, exceeds its
threshold size. As a rule, in order to guarantee that all entries in the
LSM-tree have been examined, it is necessary for an exact-match find or
range find to access each component Ci through its index structure.
However, there are a number of possible optimizations where this search
can be limited to an initial subset of the components.

First, where unique index values are guaranteed by the logic of
generation, as when timestamps are guaranteed to be distinct, a matching
indexed find is complete if it locates the desired value in an early Ci
component. As another example, we could limit our search when the find
criterion uses recent timestamp values so that the entries sought could
not yet have migrated out to the largest components. As the merge cursor
circulates through the (G, Ci+l) pairs, we will often have reason to
retain entries in Ci that have been inserted in the recent past (in the
last ti seconds), allowing only the older entries to go out to Ci+ 1. In
cases where the most frequent find references are to recently inserted
values, many finds can be completed in the Co tree, and so the Co tree
fulfills a valuable memory buffering function. This point was made also
in {[}23{]}, and represents an important effciency consideration. For
example, indexes to short-term transaction UNDO logs accessed in the
event of an abort will have a large proportion of accesses in a
relatively short time-span after creation, and we can expect most of
these indexes to remain memory resident. By keeping track of the
start-time for each transaction we can guarantee that all logs for a
transaction started in the last to seconds, for example, will be found
in component Co, without recourse to disk components.


\hypertarget{deletes-updates-and-long-latency-finds-in-the-lsm-tree}{%
\subsection{Deletes, updates and long-latency finds in the
LSM-tree}\label{deletes-updates-and-long-latency-finds-in-the-lsm-tree}}


We note that deletes can share with inserts the valuable properties of
deferral and batching. When an indexed row is deleted, if a key value
entry is not found in the appropriate position in the Co tree, a delete
node entry can be placed in that position, also indexed by the key
value, but noting an entry row ID (RID) to delete. The actual delete can
be done at a later time during the rolling merge process, when the
actual index entry is encountered: we say the delete node entry migrates
out to larger components during merge and annihilates the associated
entry when it is encountered. In the meantime, find requests must be
filtered through delete node entries so as to avoid returning references
to deleted records. This filtering is easily performed during the search
for the relevant keyvalue, since the delete node entry will be located
in the appropriate keyvalue position of an earlier component than the
entry itself, and in many cases this filter will reduce the overhead of
determining an entry is deleted. Updates of records that cause changes
to indexed values are unusual in any kind of applications, but such
updates can be handled by LSM-trees in a deferred manner if we view an
update as a delete followed by an insert.

We sketch another type of operation for effcient index modification. A
process known as predicate deletion provides a means of performing batch
deletes by simply asserting a predicate, for example the predicate that
all index values with timestamps more than 20 days old are to be
deleted. When the affected entries in the oldest (largest) component
become resident during the normal course of the rolling merge, this
assertion causes them simply to be dropped during the merge process. Yet
another type of operation, a long-latency find, provide an effcient
means of responding to a query where the results can wait for the
circulation period of the slowest cursor. A find note entry is inserted
in component Co, and the find is actually performed over an extended
period of time as it migrates out to later components. Once the find
note entry has circulated out to the appropriate region of the largest
relevant component of the LSM-tree, the accumulated list of RIDs for the
long-latency find is complete.


\hypertarget{cost-performance-and-the-multi-component-lsm-tree}{%
\section{Cost-performance and the multi-component
LSM-tree}\label{cost-performance-and-the-multi-component-lsm-tree}}


In this section we analyze the cost-performance of an LSM-tree, starting
with an LSM-tree of two components. We analyze the LSM-tree by analogy
with a B-tree providing the same indexing capabilities, comparing the
I/O resources utilized for a high volume of new insertions. As we will
argue in Sect. 5, other disk-based access methods are comparable to the
B-tree in I/O cost for inserts of new index entries. The most important
reason for the comparison of the LSM-tree and B-tree that we perform
here is that these two structures are easily comparable, both containing
an entry for each row indexed in collation sequence at a leaf level,
with upper level directory information that channels access along a path
of page-sized nodes. The analysis of I/O advantage for new entry inserts
to the LSM-tree is effectively illustrated by analogy to the less
emcient but well understood behavior of the B-tree.

In Sect. 3.2 following, we compare the I/O insert costs and demonstrate
that the small ratio of cost for an LSM-tree of two components to that
of a B-tree is a product of two factors. The first factor, COSTC/COSTp,
corresponds to the advantage gained in the LSM-tree by performing all
I/O in multi-page blocks, thus utilizing disk arms much more effciently
by saving a great deal of seek and rotational latency time. The COSTz
term represents the disk arm cost of reading or writing a page on disk
as part of a multi-page block, and COSTp represents the cost of reading
or writing a page at random. The second factor that determines I/O cost
ratio between the LSM-tree and the B-tree is given as I/M, representing
the batching effciency to be gained during a merge step. M is the
average number of entries merged from Co into a page-sized leaf node of
Cl . Inserting multiple entries per leaf is an advantage over a (large)
B-tree where each entry inserted normally requires two I/Os to read and
write the leaf node on which it resides. Because of the Five minute
rule, it is unlikely in Example 1.2 that a leaf page read in from a
B-tree will be re-referenced for a second insert during the short time
it remains in buffer. Thus there is no batching effect in a B-tree
index: each leaf node is read in, an insert of a new entry is performed,
and it is written out again. In an LSM-tree however, there will be an
important batching effect as long as the Co component is suffciently
large in comparison to the Cl component. For example, with 16 byte index
entries, we can expect 250 entries in a fully packed 4 KByte node. If
the Co component is 1/25 the size of the Cl component, we will expect
(about) 10 new entries entering each new Cl node of 250 entries during a
node I/O. It is clear that the LSM-tree has an emciency advantage over
the B-tree because of these two factors, and the "rolling merge" process
is fundamental to gaining this advantage.

The factor COSTN/COSTp corresponding to the ratio of emciency of
multipage block over single page I/O is a constant, and we can do
nothing with the LSM-tree structure to have any effect on it. However
the batching efficiency I/M of a merge step is proportional to the ratio
in size between the Co and the Cl components; the larger the Co
component in comparison to the Cl component, the more emciency is gained
in the merge; up to a certain point, this means that we can save
additional money on disk arm cost by using a larger Co component, but
this entails a larger memory cost to contain the Co component. There is
an optimal mix of sizes to minimize the total cost of disk arms and
memory capacity, but the solution can be quite expensive in terms of
memory for a large Co. It is this consideration that motivates the need
for a multi-component LSM-tree, which is investigated in Sect. 3.3. A
three component LSM-tree has memory resident component Co and disk
resident components Ci and C2, where the components increase in size
with increasing subscript. There is a rolling merge processes in train
between Co and Cl as well as a separate rolling merge between Cl and C2
that move entries out from the smaller to the larger component each time
the smaller component exceeds its threshold size. The advantage of an
LSM-tree of three components is that batching emciency can be
geometrically improved by choosing Cl to optimize the combined ratio of
size between Co and Cl and between Cl and C2. As a result, the size of
the Co memory component can be made much smaller in proportion to the
total index, with a significant improvement in cost.

Section 3.4 derives a mathematical procedure for arriving at the optimal
relative sizes of the different components of a multi-component LSM-tree
to minimize total cost for memory and disk.


\hypertarget{the-disk-model}{%
\subsection{The disk model}\label{the-disk-model}}


The advantage of the LSM-tree over the B-tree lies mainly in the area of
reduced cost for I/O (although disk components that are 100\% full offer
a capacity cost advantage as well over other known flexible disk
structures). Part of this I/O cost advantage for the LSM-tree is the
fact that a page I/O can be amortized along with many other pages of a
multi-page block.

Definition 3.1.1. I/O costs and data temperature As we store data of a
particular kind on disk, rows in a table or entries in an index, we find
that as we increase the amount ofdata stored, the disk arms see more and
more utilization under normal use in a given application environment. We
are paying for two things when we buy a disk: first, disk capacity, and
second, disk I/O rate. Usually one of these two will be a limitingfactor
in any kind ofuse. Ifcapacity is the limiting factor, we willfill up the
disks and find that the disk arms that provide the I/Os are
onlyfractionally utilized by the application; on the other hand we
mayfind that as we add data the disk arms reach theirfull utilization
rate when the disk is onlyfractionallyfull, and this means that the I/O
rate is the limiting factor.

A random page I/O during peak use has a cost, COSTp, which is based on a
fair rent for the disk arm, whereas the cost of a disk page I/O as part
of the large multi-page block I/O will be represented as COSTr, and this
quantity is a good deal smaller because it amortizes seek time and
rotational latency over multiple pages. We adopt the following
nomenclature for storage costs: COSTd = cost of 1 MByte of disk storage,

COSTm = cost of 1 MByte of memory storage,

COSTp = disk arm cost to provide 1 page/second I/O rate, for random
pages,

COST„ = disk arm cost to provide 1 page/second I/O rate, as part of
multi-page block I/O.

Given an application referencing a body of data with S MBytes of storage
and H random pages per second of I/O transfer (we assume no data is
buffered), the rent for disk arms is given by H COSTp and the rent for
disk media is given by S COSTd. Depending on which cost is the limiting
factor the other comes for free, so the calculated cost for accessing
this disk resident data, COST-D, is given by COST-D = max (S COSTd, H
COSTp).

COST-D will also be the total cost for supporting data access for this
application, COST-TOT, under the assumption given that none of the disk
pages are buffered in memory. In this case, the total cost increases
linearly with the random I/O rate H even while the total storage
requirement S remains constant. Now the point of memory buffering is to
replace disk I/O with memory buffers at a certain point of increasing
I/O rate to the same total storage S. If we assume under these
circumstances that memory buffers can be populated in advance to support
the random I/O requests, the cost for disk drops to the cost for disk
media alone, so the calculated cost of accessing this buffer resident
data, COST-B, is simply the cost of memory plus the cost of disk media:

COST-B = S COSTm + SCOSTd.

Now the total cost for supporting data access for this application is
the minimum of these two calculated costs:

COST-TOT =
\includegraphics[width=0.65702in,height=0.14005in]{./media/image5.jpg}COSTd,
H COSTp), S COSTm + S COSTd).

There are three cost regimes in the graph of COST-TOT as the page access
rate H increases for a given volume of data S. See Fig. 3, where we
graph COSTTOT/MByte vs. H/S, or accesses per second per megabyte. If H/S
is small, COSTTOT is limited by the cost of disk medium, S COSTd, a
constant for fixed S. As H/S increases, the cost comes to be dominated
by disk arm use, H COSTp, and is proportional to increasing H/S for
fixed S. Finally, at the point where the Five Minute rule dictates
memory residence, the dominant factor becomes S COSTm +

S COSTd, which is dominated by the memory term for present prices, COSTm
\textgreater{} COSTd. Following Copeland et al. {[}6{]}, we define the
temperature of a body of data as HIS, and we name these three cost
regimes cold, warm, and hot. Hot data has a high enough access rate H,
and thus temperature H/S, to justify memory buffer residence {[}6{]}. At
the other extreme, cold data is disk capacity limited; the disk volume
that it must occupy comes with enough disk arms to satisfy the I/O rate.
In between is warm data, whose access requirements must be met by
limiting the data capacity used under each disk arm, so that disk arms
are the limit of use. These ranges are divided as follows:

Tf = COSTd/COSTp = temperature division point between cold and warm data
("freezing"),

Tb = COSTm/COST = temperature division point between warm and hot data
("boiling").

Similarly-defined ranges exist for the multi-page block access case
using COSTn. The division between the warm and hot regions is a
generalization of the five minute rule {[}13{]}.

As stressed in {[}6{]}, it is straightforward to calculate the
temperature of a database table when it is accessed uniformly. However,
the relevance of this temperature depends on the access method: the
temperature that is relevant involves the actual disk access rate, not
the logical insert rate (including batched buffered inserts). One way to
express what an LSM-tree achieves is to say that it reduces the actual
disk accesses and thus lowers the effective temperature of the indexed
data. This idea is revisited in the conclusions of Sect. 6.

\includegraphics[width=3.64193in,height=1.42052in]{./media/image6.jpg}

Cdd Data

Fig. 3. Graph of cost of access per MByte vs. temperature

Multi-page block I/O advantage. The advantage to be gained by multi-page
block I/O is central to several earlier access methods, such as bounded
disorder files {[}18{]},

SB-trees {[}21{]}, and log structured files {[}23{]}. A 1989 IBM
publication analyzing DB2 utility performance on IBM 3380 disk {[}29{]}
gave the following analysis:
"\includegraphics[width=0.12006in,height=\textheight]{./media/image8.jpg}

The time to complete a {[}read of a single page{]} could be estimated to
be about 20 ms (assumes 10 ms seek, 8.3 ms rotational delay, 1.7 ms
read) ... The time to perform a sequential prefetch read {[}of 64
contiguous pages{]} could be estimated to be about 125 ms (assumes 10 ms
seek, 8.3 ms rotational delay, 106.9 ms read of 64 records {[}pages{]}),
or about 2 ms per page." Thus the ratio of 2 ms per page for multi-page
block I/O to 20 ms for random I/O implies a ratio of rental costs for
the disk arm, COSTN/COSTp, equal to about 1/10. An analysis of a more
recent SCSI-2 disk read of a 4 KByte page gives us a 9 ms seek, 5.5 ms
rotational delay, and 1.2 ms read, totalling 16 ms. Reading 64
contiguous 4 KByte pages requires a 9 ms seek, 5.5 ms rotational delay,
and 80 ms read for 64 pages, or a total of 95 ms, about 1.5 ms/page.
Once again COST,t/COSTp is equal to about 1/10.

We analyze a workstation server system with SCSI-2 disks holding one
GByte and costing about \$1000, and a peak rate of approximately
60---701/Os per second. The nominal usable I/O rate to avoid long I/O
queues is lower, about 40 1/Os per second. The multi-block I/O advantage
is significant.

Typical workstation costs, 1995:

COSTm = \$100/MByte, COSTd = \$1/MByte,

COSTp = \$25/(10s/s),

COSTr = \$2.5/(10s/s),

Tf COSTd/COSTp = 0.04 IOS/s Mbyte) ("freezing point"),

Tb = COSTm/COSTp = 4 10s/s MByte) ("boiling point").

We use the Tb value to derive the reference interval t for the five
minute rule, which asserts that data sustaining an I/O rate of one page
every t seconds is incurring the same cost as the memory needed to hold
it. That common cost is:

(IF) • COSTp = pagesize. COSTm.

Solving for t, we see = (1/pagesize)((COSTp/COSTm) = l/(pagesize Tb),
and for the values given above, with a page of 0.004 MBytes, we have =
1/(0.0044) 62.5 s/10.

Example 3.1. To achieve a rate of 1000 TPS in the TPC-A application of
Example 1.1, there will be H = 20001/Os per second to the Account table,
itself consisting of 100000 000 rows of 100 bytes, a total of S = 10
GBytes. The disk storage cost here is SCOSTd = \$10000 whereas the disk
1/0 cost is H COSTp = \$50000. The temperature T = H/S = 2000/10000 =
0.2, well above freezing (a factor of 5), but also well below the
boiling point. This warm data uses only 1/5 of its disk capacity for
data storage. We are paying for the disk arms and not for the capacity.
The situation is similar when we consider the 20 day Acct-ID Il
Timestamp index to the History table of Example 1.2. Such a B-tree
index, as we calculated in Example 1.2, requires about 9.2 GBytes of
leaf-level entries. Given that a growing tree is only about 70\% full,
the entire tree will require 13.8 GBytes, but it has the same I/O rate
(for inserts alone) as the Account table, which implies a comparable
temperature.


\hypertarget{comparison-of-lsm-tree-and-b-tree-io-costs}{%
\subsection{Comparison of LSM-tree and B-tree I/O
costs}\label{comparison-of-lsm-tree-and-b-tree-io-costs}}


We will be considering I/O costs of index operations which we call
mergeable: inserts, deletes, updates, and long-latency finds. The
following discussion presents an analysis to compare an LSM-tree to a
B-tree.

B-tree insert cost formula. Consider the disk arm rental cost of
performing a B-tree insert. We must first access the position in the
tree where the entry should be placed, and this entails a search down
nodes of the tree. We assume that successive inserts to the tree are to
random positions at the leaf level, so that node pages in the path of
access will not be consistently buffer resident because of past inserts.
A succession of inserts of ever increasing key-values, an
insert-on-the-right situation, is a relatively common case that does not
obey this assumption. We note that such an insert-on-the-right situation
can already be quite efficiently handled by the B-tree data structure,
since there is little I/O as the B-tree grows consistently to the right;
indeed this is the basic situation in which a B-tree load takes place.
There are a number of other proposed structures to deal with indexing
log records by ever-increasing value {[}8{]}.

In {[}21{]}, the effective depth of a B-tree, symbolized by De, was
defined to be the average number of pages not found in buffer during a
random key-value search down the directory levels of a B-tree. For
B-trees of the size used to index account-IDII Timestamp in Example 1.2,
the value for De is typically about 2.

To perform an insert to a B-tree, we perform a key-value search to a
leaf level page (De I/Os), update it, and (in the steady state) write
out a corresponding dirty leaf page (1 1/O). We can show that the
relatively infrequent node splits have an insignificant effect on our
analysis, and therefore ignore them. The pages read and written in this
process are all random access, with cost COSTp, so the total I/O cost
for a B-tree insert, COSTB.ins is given by:


COSTB ins = COSTp (De + 1). (3.1)


LSM-tree insert cost formula. To evaluate the cost of an insert into the
LSM-tree, we need to think in terms of amortization of multiple inserts,
since a single insert to the memory component Co only occasionally has
any I/O effect. As we explained at the beginning of this section, the
performance advantage an LSM-tree has over a B-tree is based on two
different batching effects. The first is the already mentioned reduced
cost of a page I/O, COSTr. The second is based on the idea that the
delay in merging newly inserted entries into the Cl tree usually allows
time for numerous entries to accumulate in Co; thus several entries will
get merged into each Cl tree leaf page during its trip from disk to
memory and back. By contrast, we have been assuming that the B-tree leaf
pages are too infrequently referenced in memory for more than one entry
insert to take place.

Definition 3.2.1. The batch-merge parameter M To quantify this
multiple-entriesper-leaf batching effect, define the parameter M for a
given LSM-tree as the average number of entries in the Co tree inserted
into each single page leafnode of the Cl tree during the rolling merge.
We assert that the parameter M is a relatively stable value
characterizing an LSM-tree. In fact, the value for M is determined by
index entry size and the ratio in size between the leaf level of the Cl
tree and that of the Co tree. We define the following new size
parameters: Se = entry (index entry) size in bytes,

Sp = page size in bytes,

S = size in MBytes of Co component leaf level,

Sl = size in MBytes of Cl component leaf level.

Then the number of entries to a page is approximately Sp/Se, and the
fraction of entries of the LSM-tree sitting in component Co is So/(So +
SD, so the parameter M is given by:


\includegraphics[width=1.58417in,height=0.14338in]{./media/image9.jpg}
(3.2)


Note that the larger the component Co in comparison to Cl, the larger
will be the parameter M. Typical implementations might have Sl = 40So
and the number of entries per disk page, Sp/Se, of 200, so that M = 5.
Given the parameter M, we can now give a rough formula for the cost
COSTISM-ins of an entry insert into the LSM-tree. We simply amortize the
per-page cost ofbringing the Cl tree leafnode into memory and writing it
out again, 2 COSTR , over the M inserts that are merged into an Cl tree
leaf node during this time.


COSTLSM-ins ¯- 2COST„/M. (3.3)


Note that we have ignored the relatively insignificant costs associated
with I/Osfor index updates in both the LSM-tree and B-tree cases.

A comparison ofLSM-tree and B-tree insert costs. If we compare the cost
formulas (3.1) and (3.3) for inserts to the two data structures, we see
the ratio:


COSTLSM-ins/COSTB*ins ¯
\includegraphics[width=1.62753in,height=0.14338in]{./media/image10.jpg}
(3.4)


where Kl is a (near) constant, 2/(De + 1), with a value of approximately
0.67 for index sizes we have been considering. This formula shows that
the cost ratio of an insert into the LSM-tree to one in the B-tree is
directly proportional to each of two batching effects we have discussed;
COSTR/COSTp, a small fraction corresponding to the ratio of cost for a
page I/O in a multi-page block to a random page I/O, and I/M, where M is
the number of entries batched per page during the rolling merge.
Typically the product of the two ratios will give a cost ratio
improvement of nearly two orders of magnitude. Naturally, such
improvement will only be possible in regimes where the index has a
relatively high temperature as a B-tree, so that it is possible to
greatly reduce the number of disks when moving to an LSM-tree index.

Example 3.2. If we assume that an index of the kind in Example 1.2 takes
up 1 GByte of disk space but is required to sit on 10 GBytes to achieve
necessary disk arm access rates, then there is certainly room for
improvement in saving money on disk arm costs. If the ratio of insert
costs given in (3.4) is 0.02 = 1/50, then we can shrink the index and
disk cost: the LSM-tree will need to take up only 0.7 GBytes on disk
because of closely packed entries and reduced disk arm utilization.

However, we see that the more efficient LSM-tree can only reduce cost
down to what is needed for disk capacity. If we had started with a 1
GByte B-tree which was constrained to sit on 35 GBytes to receive needed
disk arm service, the ratio of cost improvement of 1/50 could have been
fully realized.


\hypertarget{multi-component-lsm-trees}{%
\subsection{Multi-component LSM-trees}\label{multi-component-lsm-trees}}


The parameter M for a given LSM-tree was defined as the average number
of entries in the Co tree inserted into each single page leaf node of
the Cl tree during the rolling merge. We have been thinking of the
quantity M as being greater than 1 because of the delay period during
which new entries can accumulate in the Co tree before being merged into
nodes of the Cl tree. However, it should be clear from (3.2) that if the
Cl tree were extremely large in comparison to the Co tree, or entries
were extremely large and fit only a small number to a page, the quantity
M might be less than 1. Such a value for M means that on the average
more than one Cr tree page must be brought in and out of memory for each
entry which is merged in from the Co tree. In the case where M is
extremely small in terms of formula (3.4), specifically if M \textless{}
Kl COST„/COSTp, this could even cancel the batching effect of multi-page
disk reads, so we would do better to use a normal B-tree for inserts in
place of an LSM-tree.


To avoid a small value for M the only course with a two-component
LSM-tree is to increase the size of the Co component relative to that of
Cl. Consider a two-component LSM-tree of given total leaf entry size S
(S = So + Sl, an approximately stable value), and assume we have a
constant rate R in bytes per second of new entry inserts into Co. For
simplicity, we assume that no entries inserted into Co are deleted
before they get out to component Cl, and therefore entries must migrate
out to component Cl through the rolling merge at the same rate that they
are inserted into Co to keep the size of Co near its threshold size.
(Given that the total size S is approximately stable, this also implies
that the insertion rate into Co must be balanced by a constant deletion
rate from Cl, possibly using a succession of predicate deletes.) As we
vary the size of Co, we affect the circulation speed of the merge
cursor. A constant migration rate out to Cl in bytes per second requires
that the rolling merge cursor move through entries of Co at a constant
rate in bytes per second, and therefore as the size of Co decreases the
circulation rate from smallest to largest index values in Co will
increase; as a result, the I/O rate for multi-page blocks in Cl to
perform the rolling merge must also increase. If a Co size of a single
entry were possible, at this conceptual extreme point we would require a
circulation through all multi-page blocks of Cl for each newly inserted
entry, an immense demand on I/O. The approach of merging Co and Cl ,
rather than accessing relevant nodes of Cl for each newly inserted entry
as is done with the B-tree, would become a millstone around our necks.
By comparison, larger size Co components will slow down the circulation
of the merge cursor and decrease the I/O cost of inserts. However, this
will increase the cost of the memory-resident component Co.


Now there is a canonical size for Co determined by the point at which
the total cost of the LSM-tree, memory cost for Co plus media/disk arm
cost for the Cl component, is minimized. To arrive at this balance, we
start with a large Co component and pack the Cl component closely on
disk media. If the Co component is sumciently large, we will have a very
small I/O rate to Cl . We can now decrease the size of Co, trading off
expensive memory for inexpensive disk space, until the I/O rate to
service Cl increases to a point where the disk arms sitting over the C 1
component media are running at full rate. At this point, further savings
in memory cost for Co will result in increased media cost, as we are
required to spread out the Cl component over fractionally full disks to
reduce the disk arm load, and at some point as we continue to shrink Co
we will reach a minimum cost point. Now it is common in the two
component LSM-tree that the canonical size we determine for Co will
still be quite expensive in terms of memory use. An alternative is to
consider adopting an LSM-tree of three or more components. Conceptually,
if the size of the Co component is so large that the memory cost is a
significant factor, then we consider creating another intermediate size
disk based component between the two extremes. This will permit us to
limit the cost of disk arms while reducing the size of the Co component.

\includegraphics[width=3.19836in,height=0.86698in]{./media/image11.jpg}


\hypertarget{disk-memory}{%
\subsection{ Disk Memory}\label{disk-memory}}


Fig. 4. An LSM-tree of K + 1 components

In general, an LSM-tree of K + 1 components has components Co, Cl,
\includegraphics[width=0.3802in,height=0.12338in]{./media/image13.jpg},
K-1 and CK, which are indexed tree structures of increasing size; the Co
component tree is memory resident and all other components are disk
resident (but with popular pages buffered in memory as with any disk
resident access tree). Under pressure from inserts, there are
asynchronous rolling merge processes in train between all component
pairs (Ci- 1, CD, that move entries out from the smaller to the larger
component each time the smaller component, Ci- 1 , exceeds its threshold
size. During the life of a long-lived entry inserted in an LSM-tree, it
starts in the Co tree and eventually migrates out to the CK, through a
series of K asynchronous rolling merge steps.

The spotlight here is on performance under insert tramc because we are
assuming that the LSM-tree exists in an insert-mostly environment.
LSM-tree finds of three or more components suffer somewhat in
performance, typically by one extra page I/O per disk component.


\hypertarget{lsm-trees-component-sizes}{%
\subsection{LSM-trees: Component
sizes}\label{lsm-trees-component-sizes}}


In the current section, we derive a formula for the I/O cost for inserts
into an LSM-tree of several components and demonstrate mathematically
how to choose optimal threshold sizes for the various components. An
extended Example 3.3 illustrates the system cost for a B-tree, the
improved system cost for an LSM-tree of two components, and the greater
savings to be had with an LSM-tree of three components.

We define the size of an LSM-tree component, S(Ci), as the number of
bytes of entries it contains at the leaf level; the size of component Ci
is denoted by LSi, S(Ci) = LSi, and S is the total size of all leaf
level entries in all components, S = EiSi.

We assume there is some relatively steady rate R of insertion, in bytes
per second, to component Co of the LSM-tree, and for simplicity that all
newly inserted entries live to circulate out to component CK by a
succession of rolling merge steps. We also assume that each of the
components, Co, Cl, . . , K-1, has a size close to a maximum threshold
size to be determined by the current analysis. The component CK is
assumed to have a relatively stable size, because of deletes balancing
inserts over some standard time period. The deletes from component CK
can be thought of as taking place without any addition to the rate of
insertion R to component Co.

Given an LSM-tree of K components with a fixed total size S and memory
component size So, the tree is totally described by the variables ri, i
\includegraphics[width=0.42689in,height=0.12338in]{./media/image14.jpg}representating
size ratios between adjacent pairs of components, ri = Si/Si- . As
detailed below, the total page I/O rate to perform all ongoing merge
operations between component pairs (Ci- 1, CD can be expressed as a
function of R, the rate of insertions into Co, and the ratios "i. We
assume that blocks of the different components are striped across
different disk arms in a mixed way to achieve a balance in utilization,
so that minimizing H is the same as minimizing the total disk arm cost
(at least in any range where disk arms rather than media capacity
constitute the gating cost). It is a standard calculus minimization
problem to find the values for which minimize the total I/O rate H for a
given R. It turns out that the assumption that the total size S is fixed
leads to a rather diffcult problem with a somewhat complex recurrence
relation between the ri values. However, if we make the comparable
assumption that the largest component size SK is fixed (along with the
memory size So), as we will show in Theorem 3.1, this minimization
problem is solved when all of the values ri are equal to a single
constant value r. We show in Theorem 3.2 the slightly more precise
solution relating the ri values where the total size S is held constant,
and argue that the constant value r for ri gives similar results in all
areas of real interest. Assuming such a constant value r for all ri
factors, we have Si = r\textsuperscript{t} So. Thus the total size S is
given by the sum of the individual component sizes, S = So + rSo + r
\textsuperscript{2} So + + r\textsuperscript{x} So, and we can solve for
r in terms of S and So.

Thus in Theorem 3.1 we show that to minimize the total I/O rate H of a
multi-component LSM-tree, with fixed SK, So, and insertion rate R, we
size intermediate components in a geometric progression between the
smallest and largest. We will see, as in the case of a two-component
LSM-tree, that if we allow So to vary while R and SK remain constant,
and express H as a function of So, then H increases with decreasing So.
We can now minimize the total cost of the LSM-tree, memory plus disk arm
cost, by varying the size of So. The appropriate process to arrive at an
optimal total cost for a given number of components is illustrated below
in Example 3.3. The only remaining free variable in the total cost is
the number of components, K + 1. We discuss the tradeoffs for this value
at the end of the current section.

Theorem 3.1 Given an LSM-tree of K + 1 components, with a fixed
largest-component size SK, insert rate R, and memory component size So,
the total page I/O rate H to perform all merges is minimized when the
ratios ri = Si/Si-1 are all equal to a common value r. Thus the total
size S is given by the sum ofthe individual component sizes,


\includegraphics[width=2.19116in,height=0.14005in]{./media/image15.jpg}
(3.5)


and we can solve for r in terms of S and So. Similarly, the total page
I/O rate H is given by

H =
\includegraphics[width=0.71038in,height=0.14338in]{./media/image16.jpg}
+ r) --- 1/2), (3.6) where Sp is the number of bytes per page.

\includegraphics{./media/image17.jpg}Proof. Since we have assumed that
entries are never deleted until they arrive at component CK, it is clear
in the steady state that the rate R in bytes per second of inserts to Co
is the same as the rate with which entries migrate by rolling merge out
from component Ci- to component Ci, for all i, () \textless{} i
\textless{} K. Consider the case where the component Ci-l is disk
resident. Then the merge from Cito Ci entails multi-page block reads
from component Ci-l at a rate of R/S pages per second, where Sp is the
number of bytes per page (we derive this from the rate R in bytes per
second that entries migrate out from Ci- 1, assuming that 100\% of all
entries encountered are deleted from Ci- 1 ; other assumptions are
possible in a general case). The merge also entails multi-page reads
from Ci at a rate riR/Sp pages per second (this follows from the fact
that the rolling merge cursor passes over ri Si/Si-1 times as many pages
belonging to Ci as it does pages of Ci- 1). Finally, the merge entails
multi-page disk writes at a rate of (ri + 1)R/Sp pages per second to
write out newly merged data belonging to q. Note that here we are taking
into account the enlarged size to the Ci component resulting from the
merge. Summing over all disk resident components Ci, we have a total
rate H of multi-page I/Os in pages per second given by


H = ((2r1 + 2) + (\&2 + 2) + + (\&K ---1 + 2) + (2"K + (3.7)


where each term of the form (2n + k) represents all I/O on component G:
riR/Sp to read in pages in Ci for the merge from Ci- 1 to Ci, (ri +
l)R/Sp to write out pages in Ci for that same merge, and R/Sp to read in
pages in Ci for the merge from Ci to Ci+ 1. Clearly there is no term for
Co and the term for component CK does not have this final addition.
Equation (3.7) can be rewritten as:


\includegraphics[width=1.76093in,height=0.3768in]{./media/image18.jpg}
(3.8)


We wish to minimize the value of this function under the condition that:
\includegraphics[width=0.20011in,height=0.16339in]{./media/image20.jpg}ri
= (SK/So) = C, a constant. To solve this problem, we minimize E f ri,
with the term rK replaced by CFI \textsuperscript{K}1 ra
\textsuperscript{l} . Taking partial derivatives by each of the free
variables rj, J = 1, , K --- 1, and equating them to zero, we arrive at
a set of identical equations of the form:

\includegraphics[width=1.32737in,height=0.3768in]{./media/image21.jpg}

which is clearly solved when all rj (including rK) are equal to CM f -
\textsuperscript{I} ri---1 or C 11K

\includegraphics{./media/image23.jpg}

Theorem 3.2 We vary the assumptions of Theorem 3.1 to fix the total size
S rather than the size SK of the largest component. This minimization
problem is much more dificult, but can be done using Lagrange
multipliers. The results are a sequence of

formulas for each ri in terms of higher-indexed ri:

\textsuperscript{r}K-1 = + 1, \textsuperscript{r}K-2 =
\textsuperscript{r}K-1 + \textsuperscript{l}/\textsuperscript{r}x-l,
\textsuperscript{r}K-3 = rK-2 + 1/(rK-ÜK-2).

We omit the proof.

As we will see, useful values of ri are fairly large, say 20 or more, so
the size of the largest component, SK, dominates the total size S. Note
that in Theorem 3.2 therefore, each ri normally differs by only a small
fraction from its higher neighbor
\includegraphics[width=0.28682in,height=\textheight]{./media/image24.jpg}In
what follows, we base our examples on the approximation of Theorem 3.1

Minimizing total cost. From Theorem 3.1, it can be seen that if we allow
So to vary while R and SK remain constant and express the total I/O rate
H as a function of So, then since r increases with decreasing So by
(3.5), and H is proportional to r by (3.6), clearly H increases with
decreasing So. We can now minimize the total cost of the LSM-tree as in
the two component case by trading off expensive memory for inexpensive
disk. If we calculate the disk media needed to store the LSM-tree and
the total I/O rate H that keeps these disk arms fully utilized, this
becomes a starting point in our calculation to determine the size for So
that minimizes cost. From this point as we further decrease the size of
Co the cost of disk media goes up in inverse proportion, since we have
entered the region where disk arm cost is the limiting factor. Example
3.3, below, is a numerically based illustration of this process for a
two and three component LSM-tree. Prior to this example, we offer an
analytical derivation for the two component case.

The total cost is the sum of memory cost, COSTmSo, and disk cost, itself
a maximum over disk storage and I/O costs, here based on multi-page
block access rate H in pages per second:

COSTt0t = COSTmS0 + max {[}COSTdSl , COSTCH{]}.

Consider the case of two components, so that in (3.6), K = 1, r = SI/SO.
Let s =
\includegraphics[width=1.44076in,height=0.13672in]{./media/image25.jpg}=
cost of memory relative to storage cost for Sl data.

\includegraphics[width=3.18169in,height=0.14339in]{./media/image26.jpg}

C = COSTt0t/(COSTdS1) = total cost relative to storage cost for Sl data,

then, substituting (3.6) and simplifying, assuming So/S1 small, we
arrive at a close approximation:

C s + max(l, t/s).

The relative cost C is a function of two variables t and s; the variable
t is a kind of normalized temperature measuring the basic multi-page
block I/O rate required by the application. The variable s represents
how much memory we decide to use to implement the LSM-tree. To decide
the size of So, the simplest rule would be to follow the line s = t, on
which C = s + 1 and the disk storage and I/O capacities are fully
utilized. This rule is cost-minimal for t \textless{} 1, but for t
\textgreater{} 1, the locus of minimal-C follows the curve s = t
\textsuperscript{i} /\textsuperscript{2} on which C = 2t
\textsuperscript{112} . Putting the result back in dimensional form we
obtain, for t 1:


COSTmin - 2 {[}(COSTmSl
\includegraphics[width=0.9405in,height=0.15005in]{./media/image27.jpg}
\textsuperscript{1} /\textsuperscript{2} (3.9)


Thus the total cost of the LSM-tree (for t 1) is seen to be twice the
geometric mean of the (very high) cost of enough memory to hold all the
data in the LSM-tree and the (extremely low) cost of disk required to
support the multi-page block I/O needed to write its inserts to disk in
the cheapest way. Half of this total cost is used for memory for So, the
other half for disk for I/O access to Sl The cost of disk storage does
not show up because t 1 ensures that the data is warm enough to make
disk I/O predominate over disk storage at the minimum point. Note that
asymptotically, the cost goes as R \textsuperscript{1}
/\textsuperscript{2} compared to R for the B-tree, as R ---+ 00.

In the case that t \textless{} 1, the cooler case, the minimum cost
occurs along s = t, where C = t + 1 \textless{} 2. This means that the
total cost in this case is always less than twice the basic cost of
storing Sl on disk. In this case we size disk by its storage
requirements, and then use all its I/O capacity to minimize memory use.

Example 3.3. We consider the Account-ID Il Timestamp index detailed in
Example 3.1. The following analysis calculates costs for inserts only,
with an insertion rate R of 16 000 bytes per second to the index (1000
16 byte index entries, not counting overhead), resulting in an index of
576 million entries for 20 days of data, or 9.2 GBytes of data.

Using a B-tree to support the index, the disk I/O will be the limiting
factor as we saw in Example 3.1 --- the leaf-level data is warm. We are
required to use enough disk space to provide H = 2000 random I/Os per
second to update random pages at the leaf level (this assumes all
directory nodes are memory resident). Using the typical value COSTp =
\$25 from the table of Section 3.1, we find the cost for I/O is HCOSTp =
\$50 000. We calculate the cost to buffer upper-level nodes in memory as
follows. Assume leaf nodes that are 70\% full, 0.7(4K/16) = 180 entries
per leaf node, and therefore the level above the leaf contains about 576
million/180 = 3.2 million entries pointing to subordinate leaves. If we
grant some prefix compression so that we can fit 200 entries to a node
at this level, this implies about 16 000 pages of 4 KBytes each, or 64
MBytes, at a cost for memory, COSTm, of \$100 per MByte, or \$6400. We
ignore the relatively insignificant cost of node buffering at levels
above this, and say the total cost of a B-tree is \$50 000 for disk plus
\$6400 for memory, or a total cost of \$56 400.

With an LSM-tree of two components, Co and Cl , we need an Sl of 9.2
GBytes of disk to store the entries, at a cost of COSTdS1 = \$9200. We
pack this data closely on disk and calculate the total I/O rate H
supported by an equal cost in disk arms using multi-page block I/O, as H
= 9200/COST,t = 3700 pages per second. Now in (3.6) we solve for r after
setting the total I/O rate H as above, the rate R to 16 000 bytes/s, and
Sp to 4K. From the resulting ratio r = Sl/So = 460 and the fact that Sl
= 9.2 GBytes, we calculate 20 MBytes of memory for Co, costing \$2000.
This is the simple s = t solution, with total cost \$11200 and full
utilization of disk capacity and I/O capability. Since t = 0.22 is less
than 1, this is the optimal solution. We add \$200 for 2 MBytes of
memory to contain merging blocks, and arrive at a total cost of \$11
400. This is a significant improvement over the B-tree cost.

Here is a full explanation of the solution. The insert rate of R = 16000
bytes/s is turned into 4 pages/s that need to be merged from Co to Cl .
Since Cl is 460 times larger than Co, the new entries from Co are on the
average merged into positions 460 entries apart in Cl. Thus merging a
page from Co requires reading and writing 460 pages of Cl, a total of
3680 pages per second. But this is exactly what 9.2 disks provide in
multiblock I/O capacity, with each providing 400 pages/s, 10 times the
nominal random I/O rate of 40 pages/s.

Since this example shows full utilization of disk resources with two
components, we have no reason to explore the three-component LSM-tree
here. A more complete analysis would consider how occasional finds must
be performed in the index, and would consider utilizing more disk arms.
The following example shows a case where three components provide an
improved cost for a pure insert workload.

Example 3.4. Consider Example 3.3, with R increased by a factor of 10.
Note that the B-tree solution now costs \$500 000 for 500 GBytes of disk
to support an I/O rate H = 20000 1/Os per second; of this 491 GBytes
will be unutilized. But the B-tree is the same size and we still pay
\$6400 to buffer the directory in memory, for a total cost of \$506400.
In the LSM-tree analysis, the increase of R by a factor of 10 means that
t increases by the same factor, to 2.2. Since this t is greater than 1,
the best 2-component solution will not utilize all the disk capacity. We
use (3.8) to calculate the minimum cost of \$27 000 for a two-component
LSM-tree, half of which pays for 13.5 Gbytes of disk and half for 135
Mbytes of memory. Here 4.3 Gbytes of disk are unutilized. With 2 Mbytes
of memory for buffers, the total cost is \$27200.

Here is a full explanation of the two-component solution. The insert
rate R = 160 000 bytes/s is turned into 40 pages/s that need to be
merged from Co to Cl . Since Cr is 68 times larger than Co, merging a
page from Co requires 68 page reads and 68 writes to Cl , a total of
5450 pages per second. But this is exactly what 13.5 disks provide in
multiblock I/O capacity.

With an LSM-tree of three components for the R = 160000 bytes/s case,
the cost of the largest disk component and a cost-balanced I/O rate are
calculated as for two components. With Si/Si-1 = r for i = 1, 2, by
Theorem 3.1, we calculate r = 23 and So = 17 MBytes (for memory cost of
\$1700) for fully occupied disk arms. The smaller disk component costs
just 1/23 of the larger. Now increasing the memory size from this point
has no good cost effect, and decreasing the memory size will result in a
corresponding factor, squared, increase in the cost of disk. Since the
cost for disk is currently a good deal higher than the cost of memory,
we do not gain cost effectiveness by memory size reduction. Thus we have
an analogous s
\includegraphics[width=0.18677in,height=\textheight]{./media/image28.jpg}solution
in the three-component case. Allowing an additional 4 MBytes of memory
for buffering, costing \$400, for the two rolling merge operations, the
total cost for a 3 component LSM-tree is therefore \$9200 for disk plus
\$2100 for memory, or a total cost of \$11300, a further significant
improvement over the cost of a 2component LSM-tree.

Here is a full explanation of the three-component solution. The
in-memory component Co has 17 Mbytes, the smaller disk component Cl is
23 times larger, at 400 Mbytes, and C2 is 23 times larger than Cl, at
9.2 Gbytes. Each page of the 40 pages/s of data that must be merged from
Co to Cl entails 23 pages of reading and 23 of writing, or 1840 pages
per second. Similarly, 40 pages/s are being merged from Cl to C2, each
of which requires 23 pages of reads and writes of C2. The total of the
two I/O rates is 3680, exactly the multiblock I/O capacity of the 9.2 G
of disk.

An LSM-tree of two or three components will require more I/O for find
operations than the simple B-tree. The largest component in either case
will look very much like the corresponding simple B-tree, but in the
LSM-tree case we have not paid the \$6400 for memory for buffering nodes
just above the leaf level in the index. Nodes even higher in the tree
are relatively so few as to be negligible, and we can assume they are
buffered. Clearly we would be willing to pay for buffering all directory
nodes if queries to find entries were suffciently frequent to justify
this cost. In the three-component case, we need to consider the Cl
component as well. Since it is 23 times smaller than the largest
component, we can easily afford to buffer all of its non-leaf nodes, and
this cost should be added in the analysis. The unbuffered leaf access in
Cl entails another additional read for the find in cases where an entry
in Cl is being sought, and there is a decision to be made whether to
buffer the directory of C2. Thus for the three-component case, there may
be a few additional page reads over the two I/Os needed for finds in the
simple B-tree (counting one I/O for a page write of a leaf node). For
the two-component case, there may be one additional read. If we do buy
the memory for the buffering of nodes above leaf level of the LSM-tree
components, we can meet the B-tree speed in the two-component case and
pay for one extra read only in some cases in the three-component case.
The total cost to add buffering in the three-component case would then
be \$17 700, still far less than the B-tree. But it may well be better
to use this money in other ways: a full analysis should minimize total
cost over the workload, including both updates and retrievals.

We have minimized the total I/O needed for merge operations with given
So by varying the size ratios ri, with the result of Theorem 3.1, and
then minimized the total cost by choosing So to achieve best disk arm
and media cost. The only remaining variation possible in the LSM-tree is
the total number, K + 1, of components provided. It turns out that as we
increase the number of components the size of So continues to decrease
until the point is reached where the ratio r between component sizes
reaches the value e = 2.71 or until we reach the cold-data regime.
However, we can see from Example 3.4 that successively smaller So
components as the number of components increases make less and less
difference to total cost; in an LSM-tree of three components, the memory
size So has already been reduced to 17 MBytes. Further, there are costs
associated with increasing the number of components: a CPU cost to
perform the additional rolling merges and a memory cost to buffer the
nodes of those merges (which will actually swamp the memory cost of Co
in common cost regimes). In addition, indexed finds requiring immediate
response will sometimes have to perform retrieval from all component
trees. These considerations put a strong constraint on the appropriate
number of components, and three components are probably the most that
will be seen in practice.


\hypertarget{concurrency-and-recovery-in-the-lsm-tree}{%
\section{Concurrency and recovery in the
LSM-tree}\label{concurrency-and-recovery-in-the-lsm-tree}}


In the current section we investigate the approaches to be used to
provide concurrency and recovery for the LSM-tree. To accomplish this,
we need to sketch a more detailed level of design for the rolling merge
process. We leave a formal demonstration of correctness of the
concurrency and recovery algorithms for a later work, and try here
simply to motivate the design proposed.


\hypertarget{concurrency-in-the-lsm-tree}{%
\subsection{Concurrency in the
LSM-tree}\label{concurrency-in-the-lsm-tree}}


In general, we are given an LSM-tree of K + 1 components, Co, C C • • •
, K-1 and CK, of increasing size, where the Co component tree is memory
resident and all other components are disk resident. There are
asynchronous rolling merge processes in train between all component
pairs (Ci-l, CD that move entries out from the smaller to the larger
component each time the smaller component, Ci- 1, exceeds its threshold
size. Each disk resident component is constructed of pagesized nodes in
a B-tree type structure, except that multiple nodes in key sequence
order at all levels below the root sit on multi-page blocks. Directory
information in upper levels of the tree channels access down through
single page nodes and also indicates which sequence of nodes sits on a
multi-page block, so that a read or write of such a block can be
performed all at once. Under most circumstances, each multi-page block
is packed full with single page nodes, but as we will see there are a
few situations where a smaller number of nodes exist in such a block. In
that case, the active nodes of the LSM-tree will fall on a contiguous
set of pages of the multi-page block, though not necessarily the initial
pages of the block. Apart from the fact that such contiguous pages are
not necessarily the initial pages on the multi-page block, the structure
of an LSM-tree component is identical to the structure of the SB-tree
presented in {[}21{]}, to which the reader is referred for supporting
details.

A node of a disk-based component Ci can be individually resident in a
single page memory buffer, as when equal match finds are performed, or
it can be memory resident within its containing multi-page block. A
multi-page block will be buffered in memory as a result of a long range
find or else because the rolling merge cursor is passing through the
block in question at a high rate. In any event, all non-locked nodes of
the Ci component are accessible to directory lookup at all times, and
disk access will perform lookaside to locate any node in memory, even if
it is resident as part of a multi-page block taking part in the rolling
merge. Given these considerations, a concurrency approach for the
LSM-tree must mediate three distinct types of physical conflict.


\begin{enumerate}
\def\labelenumi{\roman{enumi}.}
\item
  A find operation should not access a node of a disk-based component at
  the same time that a different process performing a rolling merge is
  modifying the contents of the node.
\item
  A find or insert into the Co component should not access the same part
  of the tree that a different process is simultaneously altering to
  perform a rolling merge out to Cl.
\item
  The cursor for the rolling merge from Ci-l out to Ci will sometimes
  need to move past the cursor for the rolling merge from Ci out to
  Ci+l, since the rate of migration out from the component Ci-l is
  always at least as great as the rate of migration out from Ci and this
  implies a faster rate of circulation of the cursor attached to the
  smaller component Ci-l. Whatever concurrency method is adopted must
  permit this passage to take place without one process (migration out
  to CD being blocked behind the other at the point of intersection
  (migration out from CD.
\end{enumerate}


Nodes are the unit of locking used in the LSM-tree to avoid physical
conflict during concurrent access to disk based components. Nodes being
updated because of rolling merge are locked in write mode and nodes
being read during a find are locked in read mode; methods of directory
locking to avoid deadlocks are well understood (see, for example,
{[}3{]}). The locking approach taken in Co is dependent on the data
structure used. In the case of a (2-3)-tree, for example, we could write
lock a subtree falling below a single (2-3)-directory node that contains
all entries in the range affected during a merge to a node of Cl ;
simultaneously, find operations would lock all (2-3)-nodes on their
access path in read mode so that one type of access will exclude
another. Note that we are only considering concurrency at the lowest
physical level of multi-level locking, in the sense of {[}28{]}. We
leave to others the question of more abstract locks, such as key range
locking to preserve transactional isolation, and avoid for now the
problem of phantom updates; see {[}4, 14{]} for a discussion. Thus
read-locks are released as soon as the entries being sought at the leaf
level have been scanned. Write locks for (all) nodes under the cursor
are released following each node merged from the larger component. This
gives an opportunity for a long range find or for a faster cursor to
pass a relatively slower cursor position, and thus addresses point (iii)
above.

Now assume we are performing a rolling merge between two disk based
components, migrating entries from Ci- 1, which we refer to as the inner
component of this rolling merge, out to Ci, which we refer to as the
outer component. The cursor always has a well-defined inner component
position within a leaf-level node of Ci- 1 , pointing to the next entry
it is about to migrate out to Ci, and simultaneously a position in each
of the higher directory levels of Ci-l along the path of access to the
leaf level node position. The cursor also has an outer component
position in Ci, both at the leaf level and at upper levels along the
path of access, corresponding to an entry it is about to consider in the
merge process. As the merge cursor progresses through successive entries
of the inner and outer components, new leaf nodes of Ci created by the
merge are immediately placed in left-to-right sequence in a new buffer
resident multi-page block. Thus the nodes of the Ci component
surrounding the current cursor position will in general be split into
two partially full multi-page block buffers in memory: the "emptying"
block whose entries have been depleted but which retains information not
yet reached by the merge cursor, and the "filling" block which reflects
the result of the merge up to this moment but is not yet full enough to
write on disk. For concurrent access purposes, both the emptying block
and the filling block contain an integral number of page-sized nodes of
the Cl tree which simply happen to be buffer resident. During merge step
operations restructuring individual nodes, the nodes involved are locked
in write mode, blocking other types of concurrent access to the entries.

In the most general approach to a rolling merge, we may wish to retain
certain entries in the component Ci-l rather than migrating all entries
out to Ci as the cursor passes over them. In this case, the nodes in the
Ci- 1 component surrounding the merge cursor will also be split into two
buffer resident multi-page blocks, the "emptying" block that contains
nodes of Ci-l that the merge cursor has not yet reached, and the
"filling" block with nodes, placed left-to-right, that contain entries
recently passed over by the merge cursor and retained in component Ci---
1. In this most general case then, the merge cursor position is
affecting four different nodes at any one time: the inner and outer
component nodes in the emptying blocks where the merge is about to occur
and the inner and outer component nodes in the filling blocks where new
information is being written as the cursor progresses. Clearly these
four nodes may all be less than completely full at any moment, and the
same is true of the containing blocks. We take write locks on all four
nodes during the time the merge is actually modifying the node
structures and release these locks at quantized instants to allow a
faster cursor to pass by; we choose to release locks each time a node in
the emptying block in the outer component has been completely depleted,
but the other three nodes will generally be less than full at that time.
This is all right, since we can perform all operations of access on a
tree with nodes that are less than completely full as well as blocks
that are less than completely full with nodes. The case where one cursor
passes another requires particularly careful thought, because in general
the cursor position of the rolling merge being bypassed will be
invalidated on its inner component, and provision must be made to
reorient the cursor. Note that all of the above considerations also
apply at various directory levels of both components where changes occur
because of the moving cursor. High level directory nodes will not
normally be memory resident in a multi-page block buffer, however, so a
somewhat different algorithm must be used, but there will still be a
"filling" node and an "emptying" node at every instant. We leave such
complex considerations for later work, after an implementation of the
LSM-tree has provided additional experience.

Up to now we have not taken any special account of the situation where
the rolling merge under consideration is directed from the inner
component Co to the outer Cl component. In fact, this is a relatively
simple situation by comparison with a disk-based inner component. As
with all such merge steps, one CPU should be totally dedicated to this
task so that other accesses are excluded by write locks for a short a
time as possible. The range of Co entries to be merged should be
pre-calculated and a write lock taken on this entry range in advance by
the method already explained. Following this, CPU time is saved by
deleting entries from the Co component in a batch fashion, without
attempts to rebalance after each individual entry delete; the Co tree
can be fully rebalanced after the merge step is complete.


\hypertarget{recovery-in-the-lsm-tree}{%
\subsection{Recovery in the LSM-tree}\label{recovery-in-the-lsm-tree}}


As new entries are inserted into the Co component of the LSM-tree, and
the rolling merge processes migrates entry information out to
successively larger components, this work takes place in memory buffered
multi-page blocks. As with any such memory buffered changes, the work is
not resistant to system failure until it has been written to disk. We
are faced with a classical recovery problem: to reconstruct work that
has taken place in memory after a crash occurs and memory is lost. As we
mentioned at the beginning of Sect. 2, we do not need to create special
logs to recover index entries on newly created records: transactional
insert logs for these new records are written out to a sequential log
file in the normal course of events, and it is a simple matter to treat
these insert logs (which normally contain all field values together with
the RID where the inserted record has been placed) as a logical base for
reconstructing the index entries. This new approach to recover an index
must be built into the system recovery algorithm, and may have the
effect of extending the time before storage reclamation for such
transactional History insert logs can take place, but this is a minor
consideration.

To demonstrate recovery of the LSM-tree index, it is important that we
carefully define the form of a checkpoint and demonstrate that we know
where to start in the sequential log file, and how to apply successive
logs, so as to deterministically replicate updates to the index that
need to be recovered. The scheme we use is as follows. When a checkpoint
is requested at time To, we complete all merge steps in operation so
that node locks are released, then postpone all new entry inserts to the
LSM-tree until the checkpoint completes; at this point we create an
LSM-tree checkpoint with the following actions.


\begin{itemize}
\item
  We write the contents of component Co to a known disk location;
  following this, entry inserts to Co can begin again, but merge steps
  continue to be deferred.
\item
  We flush to disk all dirty memory buffered nodes of disk based
  components.
\item
  We create a special checkpoint log with the following information:

  \begin{itemize}
  \item

    the-log sequence number, LSNo, of the last inserted indexed row at
    time

  \item

    the disk addresses of the roots of all components,

  \item

    the location of all merge cursors in the various components,

  \item

    the current information for dynamic allocation of new multi-page
    blocks.

  \end{itemize}
\end{itemize}


Once this checkpoint information has been placed on disk, we can resume
regular operations of the LSM-tree. In the event of a crash and
subsequent restart, this checkpoint can be located and the saved
component Co loaded back into memory, together with the buffered blocks
of other components needed to continue rolling merges. Then logs
starting with the first LSN after LSNo are read into memory and have
their associated index entries entered into the LSM-tree. As of the time
of the checkpoint, the positions of all disk-based components containing
all indexing information were recorded in component directories starting
at the roots, whose locations are known from the checkpoint log. None of
this information has been wiped out by later writes of multi-page disk
blocks since these writes are always to new locations on disk until
subsequent checkpoints make outmoded multi-page blocks unnecessary. As
we recover logs of inserts for indexed rows, we place new entries into
the Co component; now the rolling merge starts again, overwriting any
multi-page blocks written since the checkpoint, but recovering all new
index entries, until the most recently inserted row has been indexed and
recovery is complete.

This recovery approach clearly works, and its only drawback is that
there is a possibly large pause while various disk writes take place
during the checkpoint process. This pause is not terribly significant,
however, since we can write the Co component to disk in a short period
and then resume inserts to the Co component while the rest of the writes
to disk complete; this will simply result in a longer than usual latency
period during which index entries newly inserted to Co are not merged
out to larger disk-based components. Once the checkpoint is complete,
the rolling merge process can catch up on work it has missed. Note that
the last piece of information mentioned in the checkpoint log list above
was the current information for dynamic allocation of new multi-page
blocks. In the case of a crash, we will need to figure out in recovery
what multi-page blocks are available in our dynamic disk storage
allocation algorithm. This is clearly not a diffcult problem; in fact a
more difficult problem of garbage collecting fragmented information
within such a block had to be solved in {[}23{]}.

Another detail of recovery has to do with directory information. Note
that as the rolling merge progresses, each time a multi-page block or a
higher level directory node is brought in from disk to be emptied it
must immediately be assigned a new disk position in case a checkpoint
occurs before the emptying is completed and remaining buffered
information must be forced out to disk. This means that the directory
entries pointing down to the emptying nodes must be immediately
corrected to point to the new node locations. Similarly we must
immediately assign a disk position for newly created nodes so that
directory entries in the tree will be able to point immediately to the
appropriate position on disk. At every point we need to take care that
directory nodes containing pointers to lower-level nodes buffered by a
rolling merge are also buffered; only in this way can we make all
necessary modifications quickly so that a checkpoint will not be held up
waiting for I/Os to correct directories. Furthermore, after a checkpoint
occurs and the multi-page blocks are read back into memory buffers to
continue the rolling merge, all the blocks involved must be assigned to
a new disk position, and thus all directory pointers to subsidiary nodes
must be corrected. If this sounds like a great deal of work the reader
should recall that there is no additional I/O necessary and the number
of pointers involved is probably only about 64 for each block buffered.
Furthermore these changes should be amortized over a large number of
merged nodes, assuming that the checkpoints are only taken frequently
enough to keep recovery time from growing beyond a few minutes; this
implies a few minutes of I/O between checkpoints.


\hypertarget{cost-performance-comparisons-with-other-access-methods}{%
\section{Cost-performance comparisons with other access
methods}\label{cost-performance-comparisons-with-other-access-methods}}


In our introductory Example 1.2, we considered a B-tree for the Acct-ID
Il Timestamp index on the History file because it is the most common
disk-based access method used in commercial systems. What we wish to
show now is that no other disk indexing structure consistently gives
superior I/O performance. To motivate this statement, we argue as
follows.

Assume we are dealing with an arbitrary indexing structure. Recall that
we calculated the number of entries in the Acct-ID" Timestamp index by
assuming they were generating 1000 entries per second over a 20 day
period of accumulation with eight hour days. Given index entries 16
bytes in length (4 bytes for the Acct-ID, 8 bytes for the timestamp, and
4 bytes for the History row RID) this implies 9.2 GBytes of entries or
about 2.3 million 4 KByte pages of index, even if there is no wasted
space. None of these conclusions are subject to change because of the
specific choice of index method. A B-tree will have a leaf level with a
certain amount of wasted space together with upper level directory
nodes, whereas an extendible hash table will have a somewhat different
amount of wasted space and no directory nodes, but both structures must
contain 9.2 GBytes of entries as calculated above. Now to perform an
insert of a new index entry into an index structure, we need to
calculate the page on which the entry is to be inserted and make sure
that page is memory resident. The question naturally arises: Are newly
inserted entries generally placed in an arbitrary position among all 9.2
GBytes of index entries that are already present? The answer, for most
classical access method structures, is Yes.

Definition 5.1. We say that the index structure ofa disk based access
method has the property of being a continuum structure if the indexing
scheme providesfor immediate placement of a newly inserted index entry
in its ultimate collation order, based on key-value, with all other
entries already present.

Recall that successive transactions in the TPC benchmark application
have Acct-ID values generated at random from each of 100 000 000
possible values. By Definition 1.1, each new entry insert of an Acct-ID
Il Timestamp index will be placed in a pretty much random position on
one of 2.3 million pages of entries that already exist. In a B-tree, for
example, the 576 000 000 accumulated entries will contain on the average
5.76 entries for each Acct-ID; presumably each entry with the same
Acct-ID has a distinct Timestamp. Each new entry insert will therefore
be placed on the right of all entries with the same Acct-ID. But this
still leaves 100000 000 points of insert randomly chosen, which
certainly implies that each new insert will be on a random one of the
2.3 million pages of existing entries. In an extendible hashing scheme
{[}9{]}, by contrast, new entries have a collation order calculated as a
hash value from the Acct-ID Il Timestamp key-value, and clearly any
placement of a new entry in sequence with all entries already present is
equally likely.

Now 2.3 million pages is the minimum number on which the 9.2 GBytes of
entries of a Continuum Structure can sit, and given 1000 inserts per
second, each page of such a Structure is accessed for a new insert about
once every 2300 s; by the five minute rule it is uneconomical to keep
all these pages buffered. If we consider larger nodes to hold the
entries as in the bounded disorder file {[}18{]}, this provides no
advantage, for although there is a greater frequency of reference, the
cost of memory to buffer the node is also greater and the two effects
cancel. In general, then, a page is read into memory buffer for an entry
insert and must later be dropped from buffer to make room for other
pages. In transactional systems that update disk pages in place before
dropping them from buffer, this update requires a second I/O for each
index insert. Thus we are able to state that a continuum structure that
does not defer updates will require at least two I/Os for each index
insert, approximately the same as a B-tree.

\includegraphics{./media/image29.jpg}Most existing disk-based access
methods are continuum structures, including B-trees {[}5{]} and its
large number of variants such as SB-trees {[}21{]}, bounded disorder
files {[}18{]}, various types of hashing schemes such as extendible
hashing {[}9{]}, and a myriad others. However, there are a few access
methods which migrate their entries from one segment to another: MD/OD
R-trees of Kolovson and Stonebraker {[}15{]} and time-split B-trees of
Lomet and Salzberg {[}16, 17{]}. The differential file approach {[}25{]}
also collects up changes in a small component, later performing updates
to the full-sized structure. We will consider these structures in a bit
more depth.

First of all we should analyze exactly why the LSM-tree beats the
continuum structure in terms of I/O performance, reducing the disk arm
load as much as two orders of magnitude in certain situations. In its
most general formulation the advantage of the LSM-tree enjoys results
from two factors: (1) the ability to keep component Co memory resident,
and (2) careful deferred placement. It is crucial that the original
insert be made to a memory based component. Inserts of new entries in
continuum structures requires two I/Os for exactly this reason: that the
size of the index in which they must be placed cannot economically be
buffered in memory. If the assured memory residence of component Co in
the LSM-tree were not assured, if this were merely a probabilistic
concomitant of buffering a relatively small disk resident structure,
there would presumably be circumstances where the memory-resident
property would deteriorate, and this would lead to serious deterioration
in LSM-tree performance as a growing fraction of new entry inserts led
to additional I/Os. Given the guarantee that the initial insert will not
cause an I/O, the second factor supporting high performance in the
LSM-tree, a careful deferred placement in the larger continuum of the
index, is important to guarantee that component Co will not grow without
control in the expensive memory medium. Indeed the multi-component
LSM-tree provides for a sequence of deferred placements to minimize our
total cost. It will turn out that with the special structures considered
that are not Continuum Structures, that while deferred placement in the
final position of newly inserted entries is provided for, this is not
carefully done to guarantee that the initial component for new inserts
remains memory resident. Instead this component is seen as disk resident
in the defining papers, although a large proportion may be buffered in
memory. But because there is no control of this factor, the component
can grow to be predominantly disk resident, so that the I/O performance
will degrade to a point where each new insert requires at least two
I/Os, just like a B-tree.

Time-split B-tree. To begin with, we consider the time-split B-tree or
TSB-tree of Lomet and Salzberg {[}16, 17{]}. The TSB-tree is a
two-dimensional search structure to locate records by dimensions of
timestamp and keyvalue. It is assumed that each time a record with a
given key value is inserted, the old one becomes outmoded; however, a
permanent history of all records, outmoded or not, is kept indexed. When
a new entry is inserted in a (current) node of a TSB-tree that has no
room to accept it, the node can be split either by key-value or by time,
depending on circumstance. If a node is split by time, t, all entries
with timestamp range less than t go to the history node of the split,
all entries with timestamp range crossing t go to the current node. The
object is to eventually migrate outmoded records out to a history
component of the TSB-tree on inexpensive write-once storage. All current
records and current nodes of the tree lie on disk.

We see the model for the TSB-tree is somewhat different from ours. We do
not assume our older history row is outmoded in any sense when a new
history row with the same Acct-ID has been written. It is indisputable
that the current node set of the TSB-tree forms a separate component
that defers updates to a longer-term component. However, there is no
attempt to keep this current tree in memory as with the Co component of
the LSM-tree. Indeed, the current tree is presented as being disk
resident while the history tree is resident on write-once storage. There
is no claim that the TSB-tree accelerates insert performance; the intent
of the design is rather to provide a history index to all records
generated over time. Without a guaranteed memory resident component to
which new inserts are performed, we are back to the situation of two
I/Os for each entry insert.

MD/OD R-tree. The MD/OD R-tree of Kolovson and Stonebraker {[}15{]} is
comparable to the TSB-tree, in that it uses a two dimensional access
method (R-tree) variant to cluster and index historical records by
timestamp range and keyvalue. The important R-tree variation introduced
in the MD/OD R-tree is that the structure is meant to span magnetic disk
(MD) and optical disk (OD); the ultimate object, as with the TSB-tree,
is to eventually migrate outmoded records to an archive R-tree with leaf
pages and appropriate directory pages contained on inexpensive
write-once optical storage. This migration occurs by means of a vacuum
cleaner process (VCP). Whenever the R-tree index on magnetic disk
reaches a threshold size, the VCP moves some fraction of the oldest leaf
pages to the archive R-tree on optical disk. Two different variations of
this process, involving the percentage to be vacuumed and whether the
archive and current R-trees are one or two structures, are investigated
in the paper (MD/OT-RT-I and MD/OTRT-2). As with the TSB-tree, the
current (MD R-tree) is represented as being disk resident while the
archive tree (OD R-tree) is resident on write-once storage, and there is
no claim that the MD/OD R-tree accelerates insert performance. Clearly
the OD target precludes the rolling merge technique. Without a
guaranteed memory resident component to which new inserts are performed,
we return to the situation of two I/Os for each entry insert. Indeed,
even with a small number of records used for simulation in {[}15{]},
their Fig. 4 shows that the average number of pages read per insert
never goes below two for the two variant structures investigated. There
is a rough correspondence between the LSM-tree and the MS/OD R-tree if
the latter is promoted up one level of the memory hierarchy to use
memory and disk, but most of the details are not the same because of the
differences in the features of the three media.

Differential file. The differential file approach {[}25{]} starts with a
main data file which remains unchanged over an extended period, while
newly added records are placed into a specific overflow area known as a
differential file. At some future point (not carefully specified) it is
assumed that the changes will be amalgamated with the main data file,
and a new differential file will be started. Much of the content of the
paper has to do with advantages of having a much smaller dynamic area
and methods to avoid double-accesses, find operations by unique record
identifier which need to look first in the differential file (through
some index) and then in the main data file (presumably through a
separate index). The concept of a Bloom filter is suggested as the main
mechanism to avoid such double accesses. Once again, as with access
methods defined above, the differential file makes no provision to keep
the differential file memory resident. It is suggested in Sect. 3.4 that
while the differential file is being dumped and later incorporated into
the main file, a "differential-differential" file could reasonably be
held in memory cache to permit online reorganization. This approach is
not analyzed further. It corresponds to the idea of maintaining a Co
component in memory while Cl is merged with C2, but the presentation
seems to assume relatively slow insert rates, confirmed by the example
given in Sect. 3.2 of a 10000000 record file with 100 changes per hour.
It is not suggested that a differential-differential file should be kept
memory resident at all times and no mention is made of I/O savings for
insert operations.

Selective deferred text index updates. The text index maintenance method
of Dadum et al. {[}7{]} is also designed to improve system performance
in index updates by deferring the actual disk writes. Index updates are
cached in memory until forced out by conflicts with queries or trickled
out by a background task. This being a text system, a conflict here
would be between the keywords associated with the document being updated
and those associated with the query. After the update, the query runs by
using the index on disk. Thus the memory cache is not part of the
authoritative index, unlike the LSM-tree. The deferral method allows
some batching of updates in both the forced and trickled cases. However
the pattern of updates still looks like that of Continuum Structure.


\hypertarget{conclusions-and-suggested-extensions}{%
\section{Conclusions and suggested
extensions}\label{conclusions-and-suggested-extensions}}


A B-tree, because it has popular directory nodes buffered in memory, is
really a hybrid data structure which combines the low cost of disk media
storage for the majority of the data with the high cost of memory
accessibility for the most popular data. The LSM-tree extends this
hierarchy to more than one level and incorporates the advantage of merge
I/O in performing multi-page disk reads.

In Fig. 5, we expand on Fig. 3, graphing "cost of access per MByte"
against "rate of access per MByte", i.e., data temperature, for data
access through a B-tree and through an LSM-tree of two components, i.e.,
number of disk components K = 1. Starting at the lowest access rate,
"cold" data has a cost proportional to the disk media on which it sits.
In terms of the typical cost figures, up to 0.04 1/Os per second per
MByte, the "freezing point", disk access costs \$1 per MByte. The "Warm
data" region begins at the freezing point, when disk arms become the
limiting factor in access and the media is underutilized. In terms of
Example 3.3, 1 page I/O per second per MByte would cost \$25 per MByte.
Finally, we have "Hot data" when the access is so frequent that
B-tree-accessed data should remain in memory buffers; at \$100 per MByte
of memory, the cost of this access rate will be \$100 per MByte, and
this implies a rate of at least 4 1/Os per second per MByte, the
"boiling point".

The effect of buffering on a B-tree is to flatten the graph as the rate
of access enters the Hot Data region, so that more frequent access does
not result in ever higher costs extending the slope of the rising line
for Warm Data. With a bit of thought, it can be seen that the effect of
the LSM-tree is to reduce the cost of access, for any realistic rate of
access for mergeable operations such as insert and delete, strongly
towards that of cold data. Further, many cases of access rate that would
indicate memory residence of the B-tree, the cases labeled "hot data" in
Fig. 5, can be accommodated mostly on disk with the LSM-tree. In these
cases, the data is hot in terms of logical access rate (inserts/sec) but
only warm in terms of physical disk access rate because of the batching
effect of the LSM tree. This is an extremely significant advantage for
applications that have a great preponderance of mergeable operations.


\hypertarget{extensions-of-lsm-tree-application}{%
\subsection{Extensions of LSM-tree
application}\label{extensions-of-lsm-tree-application}}


To begin with, it should be clear that the LSM-tree entries could
themselves contain records rather than RIDs pointing to records
elsewhere on disk. This means that the records themselves can be
clustered by their keyvalue. The cost for this is larger entries and a
concomitant acceleration of the rate of insert R in bytes per second and
therefore of cursor movement and total I/O rate H. However, as we saw in
Example 3.3 a three component LSM-tree should be able to provide the
necessary circulation at a cost of the disk media to store the records
and index, and all of this disk media would be needed in any event to
store the rows in a nonclustered manner.

Cost/Mbyte

\includegraphics[width=2.71144in,height=1.38717in]{./media/image30.jpg}Insert
Temperature

(inserts/sec/Mbyte)

Cdd Data

Fig. 5. Graph of cost of access per MByte vs. insert temperature

Advantages of clustering might have quite important performance
implications. For example, consider the Escrow transactional method
{[}19{]}, which serves as a good layer to support work-flow management
because of the non-blocking nature of long-lived updates. In the Escrow
method, a number of incremental changes to various aggregate Escrow
fields can be generated by a long-lived transaction. The approach used
is to set aside the incremental amount requested (Escrow quantity) and
unlock the aggregate record for concurrent requests. We need to keep
logs for these Escrow quantities, and we can think of two possible
clustering indexes for these logs: Transaction ID (TID) of the
generating transaction, and field ID (FID) of the field on which the
Escrow quantity was taken. We might easily have twenty Escrow logs with
a single TID in existence over an extended period (extended enough so
that the logs can no longer be memory resident in classical log
structures). and clustering by TID would be important up until the time
when the transaction performs a commit or abort, which determines the
ultimate effect these logs will have. In the event of a commit, the
quantity taken out of the field would be permanent and the log can
simply be forgotten, but in the event of an abort we would like to
return the quantity to the field specified by the log's FID. A certain
amount of speed is called for. In processing an abort, the logs of an
aborted transaction should be accessed (clustering by TID is an
important advantage) and fields with corresponding FID should be
corrected. However, if the field is not memory resident, rather than
read in the containing record the log can be reinverted (placed in a
different LSM-tree) clustered by its FID. Then when an Escrow field is
read back into memory, we will try to access all logs clustered by FID
that might have some update to perform; again there might be a large
number of logs accessed, and clustering these logs in an LSM-tree is an
important saving. Using LSM-trees to cluster Escrow logs first by TID,
then by FID when the associated field is not in memory, will save a
large number of I/Os where long-lived transactions make large numbers of
updates to cold or warm data. This approach is an improvement over the
"extended field" concept of {[}19{]}.

Another possible variation to the LSM-tree algorithm mentioned at the
end of Sect. 2.2 is the possibility of retaining recent entries
(generated in the last seconds) in component Ci rather then letting them
migrate out to Ci+ 1. A number of alternatives are suggested by this
idea. One variation suggests that during cursor circulation, a time-key
index such as that provided by the TSB-tree might be generated. The
rolling merge can be used to provide great effciency for new version
inserts, and the multi-component structure suggests a final component
migration to write-once storage, with a good deal of control over
archival time-key indexing. This approach clearly deserves further
study, and has been the subject of a conference paper {[}22{]}.

Other ideas for further research include the following.


\begin{enumerate}
\def\labelenumi{(\arabic{enumi})}
\item
  Extend the cost analysis approach of Theorem 3.1 and Example 3.3 to
  situations where some proportion of find operations must be balanced
  with the merge for purposes of I/O balancing. Because of the added
  load on the disks, it will no longer be possible to assign all of the
  disk I/O capacity to rolling merge operations and optimize for that
  case. Some proportion of the disk capacity will have to be set aside
  for the find operation workload. Other ways to extend the cost
  analysis are to allow for deletions prior to migration to component CK
  and consider retaining some proportion of recent entries in the inner
  component Ci- 1 during the (Ci- 1, CD merge.
\item
  It is clear that we can offoad the CPU work to maintain the LSM-tree
  so that this does not have to be done by the CPU that produces the log
  records. We merely need to communicate the logs to the other CPU and
  then communicate later find requests as well. In cases where there is
  shared memory, it is possible that finds can be done almost without
  added latency. The design for such distributed work needs to be
  carefully thought out.
\end{enumerate}

\hypertarget{acknowledgements.}{%
\section{Acknowledgements.}\label{acknowledgements.}}


The authors would like to acknowledge the assistance of Jim Gray and
Dave Lomet, both of whom read an early version of this paper and made
valuable suggestions for improvement. In addition, the reviewers for
this journal article made many valuable suggestions.


\hypertarget{references}{%
\section{References}\label{references}}

\begin{enumerate}
\def\labelenumi{\arabic{enumi}.}
\item
  Aho, A. V., Hopcroft, J. E., Ullman, J. D.: The design and analysis of
  computer algorithms. Reading, MA, Addison-Wesley
\item
  Anon et al.: A measure of transaction processing power. In:
  Stonebraker, M. (ed.) Readings in database systems, 2nd. edn., pp.
  442---454. San Mateo, CA, Morgan Kaufmann, 1988
\item
  Bayer, R., Schkolnick, M.: Concurrency of operations on B-trees. In:
  Stonebraker, M. (ed.) Readings in database systems, pp. 129---139, San
  Mateo, CA, Morgan Kaufmann, 1988
\item
  Bernstein, P. A., Hadzilacos, V., Goodman, N.: Concurrency control and
  recovery in database systems. Reading, MA, Addison-Wesley 1987
\item
  Corner, D.: The ubiquitous B-tree. Comput. Surv. 11, 121---137 (1979)
\item
  Copeland, G., Keller, T., Smith, M.: Database buffer and disk
  configuring and the battle of the bottlenecks. Proc. 4th International
  Workshop High Performance Transaction Systems, September 1991
\item
  Dadam, P., Lum, V., Praedel, U., Shlageter, G.: Selective deferred
  index maintenance \& concurrency control in integrated information
  systems. Proc. 1 Ith International VLDB Conference, pp. 142---150,
  August 1985
\item
  Daniels, D. S., Spector, A. Z., Thompson, D. S.: Distributed logging
  for transaction processing. ACM SIGMOD Transactions pp. 82-96, (1987)
\item
  Fagin, R., Nievergelt, J., Pippenger, N., Strong, H. R.: Extendible
  hashing --- a fast access method for dynamic files. ACM Trans.
  Database Systems, 4 (N3) 315---344 (1979)
\item
  Garcia-Molina, H., Salem, K.: sagas. ACM SIGMOD Transactions, pp.
  249-259 (1987)
\item
  Garcia-Molina, H., Gawlick, D., Klein, J., Kleissner, K., Salem, K.:
  Coordinating multitransactional activities. Princeton University
  Report, CS-TR-247-90, February 1990.
\item
  Garcia-Molina, H.: Modelling long-running activities as nested sagas.
  IEEE Data Engineering 14 (No 1) 14-18 (1991)
\item
  Gray, J., Putzolu, F.: The five minute rule for trading memory for
  disk accessess and the 10 Byte rule for trading memory for CPU time.
  Proc. 1987 ACM SIGMOD Conference, pp.
\end{enumerate}


395-398


\begin{enumerate}
\def\labelenumi{\arabic{enumi}.}
\setcounter{enumi}{13}
\item
  Gray, J., Reuter, A.: transaction processing, concepts and techniques.
  San Mateo, CA, Morgan Kaufmann 1992
\item
  Kolovson, C. P., Stonebraker, M.: Indexing techniques for historical
  databases. Proc. 1989 IEEE Data Engineering Conference, pp. 138-147
\item
  Lomet, D., Salzberg, B.: Access methods for multiversion data. Proc.
  1989 ACM SIGMOD
\end{enumerate}


Conference, pp. 315---323


\begin{enumerate}
\def\labelenumi{\arabic{enumi}.}
\setcounter{enumi}{16}
\item
  Lomet, D., Salzberg, B.: The performance of a multiversion access
  method. Proc. 1990 ACM SIGMOD Conference, pp. 353-363.
\item
  Lomet, D. B.: A simple bounded disorder file organization with good
  performance. ACM Trans. on Database Systems 13 (4) 525-551 (1988)
\item
  O'Neil, P. E.: The escrow transactional method. TODS, 11 (No. 4)
  405-430 (1986)
\item
  O'Neil, P., Cheng, E., Gawlick, D., O'Neil, E.: The log-structured
  merge-tree (LSM-tree). UMass/Boston Math \& CS Dept Technical Report,
  91-6, November, 1991
\item
  O'Neil, P. E.: The SB-tree: An index-sequential structure for
  high-performance sequential acess. Acta Inf. 29, 241-265 (1992)
\item
  O'Neil, P., Weikum, G.: A log-structured history data access method
  (LHAM). Presented at the Fifth International Workshop on
  High-Performance Transaction Systems, September 1993
\item
  Rosemblum, M., Ousterhout, J. K.: The design and implementation of a
  log structured file system. ACM Trans. Comp. sys. 10 (No. 1) 26-52
  (1992)
\item
  Reuter, A.: Contracts: A means for controlling system activities
  beyond transactional boundaries. Proc. 3rd International Workshop on
  High Performance Transaction Systems, September 1989
\item
  Severance, D. G., Lohman, G. M.: Differential files: their application
  to the maintenance of large databases. ACM Trans. Database Systems 1
  (No 3) 256---267 (1976)
\item
  Transaction Processing Performance Council (TPC): TPC BENCHMARK A
  standard specification. The performance handbook: for database and
  transaction processing systems, 2nd edn. San Mateo, CA, Morgan
  Kauffman, 1993
\item
  Wächter, H.: Contracts: A means for improving reliability in
  distributed computing. IEEE Spring CompCon 91
\item
  Weikum, G.: Principles and realization strategies for multilevel
  transaction management. ACM Trans. Database Systems. 16 (No 1) 132-180
  (1991)
\item
  Wodnicki, J. M., Kurtz, S. C.: GPD performance evaluation lab database
  2 Version 2 Utility analysis, IBM Document Number GG09-1031-0,
  September 28, 1989
\end{enumerate}

\end{document}
