\PassOptionsToPackage{unicode=true}{hyperref} % options for packages loaded elsewhere
\PassOptionsToPackage{hyphens}{url}
%
\documentclass[a4paper,11pt,twoside,openright]{article}

\usepackage{ifxetex}
\ifxetex{}
\else
  \errmessage{Must be built with xelatex}
\fi

\usepackage{amssymb,amsmath}
\usepackage{fourier}
\usepackage{inconsolata}
% Math
\usepackage[binary-units]{siunitx}

\usepackage{caption}
\usepackage{authblk}
\usepackage{enumitem}
\usepackage{footnote}

% Table
\usepackage{tabu}
\usepackage{longtable}
\usepackage{booktabs}
\usepackage{multirow}

% Verbatim & Source code
\usepackage{fancyvrb}
\usepackage{minted}

% Beauty
\usepackage[protrusion]{microtype}
\usepackage[all]{nowidow}
\usepackage{upquote}
\usepackage{parskip}
\usepackage[strict]{changepage}

\usepackage{hyperref}

% Graph
\usepackage{graphicx}
\usepackage{grffile}
\usepackage{tikz}


\hypersetup{
  bookmarksnumbered,
  pdfborder={0 0 0},
  pdfpagemode=UseNone,
  pdfstartview=FitH,
  breaklinks=true}
\urlstyle{same}  % don't use monospace font for urls

\usetikzlibrary{arrows.meta,calc,shapes.geometric,shapes.misc}

\setminted{
  autogobble,
  breakbytokenanywhere,
  breaklines,
  fontsize=\footnotesize,
}
\setmintedinline{
  autogobble,
  breakbytokenanywhere,
  breaklines,
  fontsize=\footnotesize,
}

\makeatletter
\def\maxwidth{\ifdim\Gin@nat@width>\linewidth\linewidth\else\Gin@nat@width\fi}
\def\maxheight{\ifdim\Gin@nat@height>\textheight\textheight\else\Gin@nat@height\fi}
\makeatother

% Scale images if necessary, so that they will not overflow the page
% margins by default, and it is still possible to overwrite the defaults
% using explicit options in \includegraphics[width, height, ...]{}
\setkeys{Gin}{width=\maxwidth,height=\maxheight,keepaspectratio}
\setlength{\emergencystretch}{3em}  % prevent overfull lines
\setcounter{secnumdepth}{3}

% Redefines (sub)paragraphs to behave more like sections
\ifx\paragraph\undefined\else
\let\oldparagraph\paragraph{}
\renewcommand{\paragraph}[1]{\oldparagraph{#1}\mbox{}}
\fi
\ifx\subparagraph\undefined\else
\let\oldsubparagraph\subparagraph{}
\renewcommand{\subparagraph}[1]{\oldsubparagraph{#1}\mbox{}}
\fi

% set default figure placement to htbp
\makeatletter
\def\fps@figure{htbp}
\makeatother


\title{What's Really New with NewSQL?}
\author{Andrew Pavlo, Matthew Aslett}
\date{June 2016}

\begin{document}
\maketitle

\hypertarget{abstract}{%
\begin{abstract}\label{abstract}}

A new class of database management systems (DBMSs) called NewSQL tout
their ability to scale modern on-line transaction processing (OLTP)
workloads in a way that is not possible with legacy systems. The term
NewSQL was first used by one of the authors of this article in a 2011
business analysis report discussing the rise of new database systems as
challengers to these established vendors (Oracle, IBM, Microsoft). The
other author was working on what became one of the first examples of a
NewSQL DBMS. Since then several companies and research projects have
used this term (rightly and wrongly) to describe their systems.

Given that relational DBMSs have been around for over four decades, it
is justifiable to ask whether the claim of NewSQL's superiority is
actually true or whether it is simply marketing. If they are indeed able
to get better performance, then the next question is whether there is
anything scientifically new about them that enables them to achieve
these gains or is it just that hardware has advanced so much that now
the bottlenecks from earlier years are no longer a problem.

To do this, we first discuss the history of databases to understand how
NewSQL systems came about. We then provide a detailed explanation of
what the term NewSQL means and the different categories of systems that
fall under this definition.
\end{abstract}

\hypertarget{a-brief-history-of-dbmss}{%
\section{A BRIEF HISTORY OF DBMSS}\label{a-brief-history-of-dbmss}}

The first DBMSs came on-line in the mid-1960s. One of the first was
IBM's IMS that was built to keep track of the supplies and parts
inventory for the Saturn V and Apollo space exploration projects. It
helped introduce the idea that an application's code should be separate
from the data that it operates on. This allows developers to write
applications that only focus on the access and manipulation of data, and
not the complications and overhead associated with how to actually
perform these operations. IMS was later followed by the pioneering work
in the early 1970s on the first relational DBMSs, IBM's System R and the
University of California's INGRES. INGRES was soon adopted at other
universities for their information systems and was subsequently
commercialized in the late 1970s. Around the same time, Oracle released
the first version of their DBMS that was similar to System R's design.
Other companies were founded in the early 1980s that sought to repeat
the success of the first commercial DBMSs, including Sybase and
Informix. Although IBM never made System R available to the public, it
later released a new relational DBMS (DB2) in 1983 that used parts of
the System R code base.

The late 1980s and early 1990s brought about a new class of DBMSs that
were designed to overcome the much touted impedance mismatch between the
relational model and object-oriented programming languages {[}65{]}.
These object-oriented DBMSs, however, never saw wide-spread market
adoption because they lacked a standard interface like SQL. But many of
the ideas from them were eventually incorporated in relational DBMSs
when the major vendors added object and XML support a decade later, and
then again in document-oriented NoSQL systems over two decades later.

The other notable event during the 1990s was the start of today's two
major open-source DBMS projects. MySQL was started in Sweden in 1995
based on the earlier ISAM-based mSQL system. PostgreSQL began in 1994
when two Berkeley graduate students forked the original QUEL-based
Postgres code from the 1980s to add support for SQL.

The 2000s brought the arrival of Internet applications that had more
challenging resource requirements than applications from previous years.
They needed to scale to support large number of concurrent users and had
to be on-line all the time. But the database for these new applications
was consistently found to be a bottleneck because the resource demands
were much greater than what DBMSs and hardware could support at the
time. Many tried the most obvious option of scaling their DBMS
vertically by moving the database to a machine with better hardware.
This, however, only improves performance so much and has diminishing
returns. Furthermore, moving the database from one machine to another is
a complex process and often requires significant downtime, which is
unacceptable for these Web-based applications. To overcome this problem,
some companies created custom \emph{middleware} to shard single-node
DBMSs over a cluster of less expensive machines. Such middleware
presents a single logical database to the application that is stored
across multiple physical nodes. When the application issues queries
against this database, the middleware redirects and/or rewrites them to
distribute their execution on one or more nodes in the cluster. The
nodes execute these queries and send the results back to the middleware,
which then coalesces them into a single response to the application. Two
notable examples of this middleware approach were eBay's Oracle-based
cluster {[}53{]} and Google's MySQL-based cluster {[}54{]}. This
approach was later adopted by Facebook for their own MySQL cluster that
is still used today.

Sharding middleware works well for simple operations like reading or
updating a single record. It is more difficult, however, to execute
queries that update more than one record in a transaction or join
tables. As such, these early middleware systems did not support these
types of operations. eBay's middleware in 2002, for example, required
their developers to implement all join operations in application-level
code.

Eventually some of these companies moved away from using middleware and
developed their own distributed DBMSs. The motivation for this was
three-fold. Foremost was that traditional DBMSs at that time were
focused on consistency and correctness at the expense of availability
and performance. But this trade-off was deemed inappropriate for
Web-based applications that need to be on-line all the time and have to
support a large number of concurrent operations. Secondly, it was
thought that there was too much overhead in using a full-featured DBMS
like MySQL as a ``dumb'' data store. Likewise, it was also thought that
the relational model was not the best way to represent an application's
data and that using SQL was an overkill for simple look-up queries.

These problems turned out to be the origin of the impetus for the
\emph{NoSQL}\footnote{The NoSQL community argues that the sobriquet
  should now be interpreted as ``Not Only SQL'', since some of these
  systems have since support some dialect of SQL.} movement in the mid
to late 2000s {[}22{]}. The key aspect of these NoSQL systems is that
they forgo strong transactional guarantees and the relational model of
traditional DBMSs in favor of eventual consistency and alternative data
models (e.g., key/value, graphs, documents). This is because it was
believed that these aspects of existing DBMSs inhibit their ability to
scale out and achieve the high availability that is needed to support
Web-based applications. The two most well-known systems that first
followed this creed are Google's BigTable {[}23{]} and Amazon's Dynamo
{[}26{]}. Neither of these two systems were available outside of their
respective company at first (although they are now as cloud services),
thus other organizations created their own open source clones of them.
These include Facebook's Cassandra (based on BigTable and Dynamo) and
PowerSet's HBase (based on BigTable). Other start-ups created their own
systems that were not necessarily copies of Google's or Amazon's systems
but still followed the tenets of the NoSQL philosophy; the most
well-known of these is MongoDB.

By the end of the 2000s, there was now a diverse set of scalable and
more affordable distributed DBMSs available. The advantage of using a
NoSQL system (or so people thought) was that developers could focus on
the aspects of their application that were more beneficial to their
business or organization, rather than having to worry about how to scale
the DBMS. Many applications, however, are unable to use these NoSQL
systems because they cannot give up strong transactional and consistency
requirements. This is common for enterprise systems that handle
high-profile data (e.g., financial and order processing systems). Some
organizations, most notably Google {[}24{]}, have found that NoSQL DBMSs
cause their developers to spend too much time writing code to handle
inconsistent data and that using transactions makes them more productive
because they provide a useful abstraction that is easier for humans to
reason about. Thus, the only options available for these organizations
were to either purchase a more powerful single-node machine and to scale
the DBMS vertically, or to develop their own custom sharding middleware
that supports transactions. Both approaches are prohibitively expensive
and are therefore not an option for many. It is in this environment that
brought about NewSQL systems.

\hypertarget{the-rise-of-newsql}{%
\section{THE RISE OF NEWSQL}\label{the-rise-of-newsql}}

Our definition of NewSQL is that they are a class of modern relational
DBMSs that seek to provide the same scalable performance of NoSQL for
OLTP read-write workloads while still maintaining ACID guarantees for
transactions. In other words, these systems want to achieve the same
scalability of NoSQL DBMSs from the 2000s, but still keep the relational
model (with SQL) and transaction support of the legacy DBMSs from the
1970--80s. This enables applications to execute a large number of
concurrent transactions to ingest new information and modify the state
of the database using SQL (instead of a proprietary API). If an
application uses a NewSQL DBMS, then developers do not have to write
logic to deal with eventually consistent updates as they would in a
NoSQL system. As we discuss below, this interpretation covers a number
of both academic and commercial systems.

We note that there are data warehouse DBMSs that came out in the
mid-2000s that some people think meet this criteria (e.g., Vertica,
Greenplum, Aster Data). These DBMSs target on-line analytical processing
(OLAP) workloads and should not be considered NewSQL systems. OLAP DBMSs
are focused on executing complex read-only queries (i.e., aggregations,
multiway joins) that take a long time to process large data sets (e.g.,
seconds or even minutes). Each of these queries can be significantly
different than the previous. The applications targeted by NewSQL DBMSs,
on the other hand, are characterized as executing read-write
transactions that (1) are short-lived (i.e., no user stalls), (2) touch
a small subset of data using index lookups (i.e., no full table scans or
large distributed joins), and (3) are repetitive (i.e., executing the
same queries with different inputs). Others have argued for a more
narrow definition where a NewSQL system's implementation has to use (1)
a lock-free concurrency control scheme and (2) a shared-nothing
distributed architecture {[}57{]}. All of the DBMSs that we classify as
NewSQL in Section 3 indeed share these properties and thus we agree with
this assessment.

\hypertarget{categorization}{%
\section{CATEGORIZATION}\label{categorization}}

Given the above definition, we now examine the landscape of today's
NewSQL DBMSs. To simplify this analysis, we will group systems based on
the salient aspects of their implementation. The three categories that
we believe best represent NewSQL systems are (1) novel systems that are
built from the ground-up using a new architecture, (2) middleware that
re-implement the same sharding infrastructure that was developed in the
2000s by Google and others, and (3) database-as-a-service offerings from
cloud computing providers that are also based on new architectures.

Both authors have previously included alternative storage engines for
existing single-node DBMSs in our categorization of NewSQL systems. The
most common examples of these are replacements for MySQL's default
InnoDB storage engine (e.g., TokuDB, ScaleDB, Akiban, deepSQL). The
advantage of using a new engine is that an organization can get better
performance without having to change anything in their application and
still leverage the DBMS's existing ecosystem (e.g., tools, APIs). The
most interesting of these was ScaleDB because it provided transparent
sharding underneath the system without using middleware by
redistributing execution between storage engines; the company, however,
has since pivoted to another problem domain. There have been other
similar extensions for systems other than MySQL. Microsoft's in-memory
Hekaton OLTP engine for SQL Server integrates almost seamlessly with the
traditional, disk-resident tables. Others use Postgres' foreign data
wrappers and API hooks to achieve the same type of integration but
target OLAP workloads (e.g., Vitesse, CitusDB).

We now assert that such storage engines and extensions for single-node
DBMSs are not representative of NewSQL systems and omit them from our
taxonomy. MySQL's InnoDB has improved significantly in terms of
reliability and performance, so the benefits of switching to another
engine for OLTP applications are not that pronounced. We acknowledge
that the benefits from switching from the row-oriented InnoDB engine to
a column-store engine for OLAP workloads are more significant (e.g.,
Infobright, InfiniDB). But in general, the MySQL storage engine
replacement business for OLTP workloads is the graveyard of failed
database projects.

\hypertarget{new-architectures}{%
\subsection{New Architectures}\label{new-architectures}}

This category contains the most interesting NewSQL systems for us
because they are new DBMSs built from scratch. That is, rather than
extending an existing system (e.g., Microsoft's Hekaton for SQL Server),
they are designed from a new codebase without any of the architectural
baggage of legacy systems. All of the DBMSs in this category are based
on distributed architectures that operate on shared-nothing resources
and contain components to support multi-node concurrency control, fault
tolerance through replication, flow control, and distributed query
processing. The advantage of using a new DBMS that is built for
distributed execution is that all parts of the system can be optimized
for multi-node environments. This includes things like the query
optimizer and communication protocol between nodes. For example, most
NewSQL DBMSs are able to send intra-query data directly between nodes
rather than having to route them to a central location like with some
middleware systems.

Every one of the DBMSs in this category (with the exception of Google
Spanner) also manages their own primary storage, either in-memory or on
disk. This means that the DBMS is responsible for distributing the
database across its resources with a custom engine instead of relying on
an off-the-shelf distributed filesystem (e.g., HDFS) or storage fabric
(e.g., Apache Ignite). This is an important aspect of them because it
allows the DBMS to ``send the query to the data'' rather than ``bring
the data to the query,'' which results in significantly less network
traffic since transmitting the queries is typically less network traffic
than having to transmit data (not just tuples, but also indexes and
materialized views) to the computation.

Managing their own storage also enables a DBMS to employ more
sophisticated replication schemes than what is possible with the
block-based replication scheme used in HDFS. In general, it allows these
DBMSs to achieve better performance than other systems that are layered
on top of other existing technologies; examples of this include the
``SQL on Hadoop'' systems like Trafodion {[}4{]} and Splice Machine
{[}16{]} that provide transactions on top of HBase. As such, we believe
that such systems should not be considered NewSQL.

But there are downsides to using a DBMS based on a new architecture.
Foremost is that many organizations are wary of adopting technologies
that are too new and un-vetted with a large installation base. This
means that the number of people that are experienced in the system is
much smaller compared to the more popular DBMS vendors. It also means
that an organization will potentially lose access to existing
administration and reporting tools. Some DBMSs, like Clustrix and
MemSQL, avoid this problem by maintaining compatibility with the MySQL
wire protocol.

\paragraph*{Examples:} Clustrix {[}6{]}, CockroachDB {[}7{]}, Google Spanner
{[}24{]}, H-Store {[}8{]}, HyPer {[}39{]}, MemSQL {[}11{]}, NuoDB
{[}14{]}, SAP HANA {[}55{]}, VoltDB {[}17{]}.

\hypertarget{transparent-sharding-middleware}{%
\subsection{Transparent Sharding
Middleware}\label{transparent-sharding-middleware}}

There are now products available that provide the same kind of sharding
middleware that eBay, Google, Facebook, and other companies developed in
the 2000s. These allow an organization to split a database into multiple
shards that are stored across a cluster of single-node DBMS instances.
Sharding is different than database federation technologies of the 1990s
because each node (1) runs the same DBMS, (2) only has a portion of the
overall database, and (3) is not meant to be accessed and updated
independently by separate applications.

The centralized middleware component routes queries, coordinates
transactions, as well as manages data placement, replication, and
partitioning across the nodes. There is typically a shim layer installed
on each DBMS node that communicates with the middleware. This component
is responsible for executing queries on behalf of the middleware at its
local DBMS instance and returning results. All together, these allow
middleware products to present a single logical database to the
application without needing to modify the underlying DBMS. The key
advantage of using a sharding middleware is that they are often a
drop-in replacement for an application that is already using an existing
single-node DBMS. Developers do not need to make any changes to their
application to use the new sharded database. The most common target for
middleware systems is MySQL. This means that in order to be MySQL
compatible, the middleware must support the MySQL wire protocol. Oracle
provides the MySQL Proxy {[}13{]} and Fabric {[}12{]} toolkits to do
this, but others have written their owning protocol handler library to
avoid GPL licensing issues. Although middleware makes it easy for an
organization to scale their database out across multiple nodes, such
systems still have to use a traditional DBMS on each node (e.g., MySQL,
Postgres, Oracle). These DBMSs are based on the disk-oriented
architecture that was developed in the 1970s, and thus they cannot use a
storage manager or concurrency control scheme that is optimized for
memory-oriented storage like in some of the NewSQL systems that are
built on new architectures. Previous research has shown that the legacy
components of disk-oriented architectures is a significant encumbrance
that prevents these traditional DBMSs from scaling up to take advantage
of higher CPU core counts and larger memory capacities {[}38{]}. The
middleware approach can also incur redundant query planning and
optimization on sharded nodes for complex queries (i.e., once at the
middleware and once on the individual DBMS nodes), but this does allow
each node to apply their own local optimizations on each query.

\paragraph*{Examples:} AgilData Scalable Cluster \footnote{Prior to 2015, AgilData
  Cluster was known as dbShards.} {[}1{]}, MariaDB MaxScale {[}10{]},
ScaleArc {[}15{]}, ScaleBase\footnote{ScaleBase was acquired by ScaleArc
  in 2015 and is no longer sold.}.

\hypertarget{database-as-a-service}{%
\subsection{Database-as-a-Service}\label{database-as-a-service}}

Lastly, there are cloud computing providers that offer NewSQL
database-as-a-service (DBaaS) products. With these services,
organizations do not have to maintain the DBMS on either their own
private hardware or on a cloud-hosted virtual machine (VM). Instead, the
DBaaS provider is responsible for maintaining the physical configuration
of the database, including system tuning (e.g., buffer pool size),
replication, and backups. The customer is provided with a connection URL
to the DBMS, along with a dashboard or API to control the system.

DBaaS customers pay according to their expected application's resource
utilization. Since database queries vary widely in how they use
computing resources, DBaaS providers typically do not meter query
invocations in the same way that they meter operations in block-oriented
storage services (e.g., Amazon's S3, Google's Cloud Storage). Instead,
customers subscribe to a pricing tier that specifies the maximum
resource utilization threshold (e.g., storage size, computation power,
memory allocation) that the provider will guarantee.

As in most aspects of cloud computing, the largest companies are the
major players in the DBaaS field due to the economies of scale. But
almost all of the DBaaSs just provide a managed instance of a
traditional, single-node DBMS (e.g., MySQL): notable examples include
Google Cloud SQL, Microsoft Azure SQL, Rackspace Cloud Database, and
Salesforce Heroku. We do not consider these to be NewSQL systems as they
use the same underlying disk-oriented DBMSs based on the 1970s
architectures. Some vendors, like Microsoft, retro-fitted their DBMS to
provide better support for multi-tenant deployments {[}21{]}.

We instead regard only those DBaaS products that are based on a new
architecture as NewSQL. The most notable examples is Amazon's Aurora for
their MySQL RDS. Its distinguishing feature over InnoDB is that it uses
a log-structured storage manager to improve I/O parallelism.

There are also companies that do not maintain their own data centers but
rather sell DBaaS software that run on top of these public cloud
platforms. ClearDB provides their own custom DBaaS that can be deployed
on all of the major cloud platforms. This has the advantage that it can
distribute a database across different providers in the same
geographical region to avoid downtimes due to service outages.

Aurora and ClearDB are the only two products available in this NewSQL
category as of 2016. We note that several companies in this space have
failed (e.g., GenieDB, Xeround), forcing their customers to scramble to
find a new provider and migrate their data out of those DBaaS before
they were shut down. We attribute their failure due to being ahead of
market demand and from being out-priced from the major vendors.

\paragraph*{Examples:} Amazon Aurora {[}3{]}, ClearDB {[}5{]}.

\hypertarget{the-state-of-the-art}{%
\section{THE STATE OF THE ART}\label{the-state-of-the-art}}

We next discuss the features of NewSQL DBMSs to understand what (if
anything) is novel in these systems. A summary of our analysis is shown
in Table 1.

\begin{table}
\tiny
\setlength{\extrarowsep}{1mm}
\begin{tabu} to 1.2\linewidth [htp] {
  l
  l
  X[1,c]
  X[1,c]
  X[1,c]
  X[1,c]
  X[1,c]
  X[3,j]}
  & & Year Released & Main Memory Storage & Partitioning & Concurrency Control & Replication & Summary \\\hline

\multirow{9}{*}{\rotatebox{90}{\textsc{New Architectures}}}
  & Clustrix {[}6{]} & 2006 & No & Yes & MVCC+2PL & Strong+Passive & MySQL-compatible DBMS that supports shared-nothing, distributed execution. \\
  & CockroachDB {[}7{]} & 2014 & No & Yes & MVCC & Strong+Passive & Built on top of distributed key/value store. Uses software hybrid clocks for WAN replication.\\
  & Google Spanner {[}24{]} & 2012 & No & Yes & MVCC+2PL & Strong+Passive & WAN-replicated, shared-nothing DBMS that uses special hardware for timestamp generation.\\
  & H-Store {[}8{]} & 2007 & Yes & Yes & TO & Strong+Active & Single-threaded execution engines per partition. Optimized for stored procedures.\\
  & HyPer {[}9{]} & 2010 & Yes & Yes & MVCC & Strong+Passive & HTAP DBMS that uses query compilation and memory efficient indexes.\\
  & MemSQL {[}11{]} & 2012 & Yes & Yes & MVCC & Strong+Passive & Distributed, shared-nothing DBMS using compiled queries. Supports MySQL wire protocol.\\
  & NuoDB {[}14{]} & 2013 & Yes & Yes & MVCC & Strong+Passive & Split architecture with multiple in-memory executor nodes and a single shared storage node.\\
  & SAP HANA {[}55{]} & 2010 & Yes & Yes & MVCC & Strong+Passive & Hybrid storage (rows + cols). Amalgamation of previous TREX, P*TIME, and MaxDB systems.\\
  & VoltDB {[}17{]} & 2008 & Yes & Yes & TO & Strong+Active & Single-threaded execution engines per partition. Supports streaming operators.\\\hline

\multirow{3}{*}{\rotatebox{90}{\textsc{Middleware}}}
  & AgilData {[}1{]} & 2007 & No & Yes & MVCC+2PL & Strong+Passive & Shared-nothing database sharding over single-node MySQL instances.\\
  & MariaDB MaxScale {[}10{]} & 2015 & No & Yes & MVCC+2PL & Strong+Passive & Query router that supports custom SQL rewriting. Relies on MySQL Cluster for coordination.\\
  & ScaleArc {[}15{]} & 2009 & No & Yes & Mixed & Strong+Passive & Rule-based query router for MySQL, SQL Server, and Oracle.\\\hline

\multirow{2}{*}{\rotatebox{90}{\textsc{DBaaS}}}
  & Amazon Aurora {[}3{]} & 2014 & No & No & MVCC & Strong+Passive & Custom log-structured MySQL engine for RDS.\\
  & ClearDB {[}5{]} & 2010 & No & No & MVCC+2PL & Strong+Active & Centralized router that mirrors a single-node MySQL instance in multiple data centers.\\

\end{tabu}
\caption{\textbf{NewSQL Systems} --- Summary of the system features described in Section 4 for the different DBMSs. Note that the year released is either when the project was announced publicly or when the company was first formed.}
\end{table}

\hypertarget{main-memory-storage}{%
\subsection{Main Memory Storage}\label{main-memory-storage}}

All of the major DBMSs use a disk-oriented storage architecture based on
the original DBMSs from the 1970s. In these systems, the primary storage
location of the database is assumed to be on a block-addressable durable
storage device, like an SSD or HDD. Since reading and writing to these
devices is slow, DBMSs use memory to cache blocks read from disk and to
buffer updates from transactions. This was necessary because
historically memory was much more expensive and had a limited capacity
compared to disks. We have now reached the point, however, where
capacities and prices are such that it is affordable to store all but
the largest OLTP databases entirely in memory. The benefit of this
approach is that it enables certain optimizations because the DBMS no
longer has to assume that a transaction could access data at any time
that is not in memory and will have to stall. Thus, these systems can
get better performance because many of the components that are necessary
to handle these cases, like a buffer pool manager or heavy-weight
concurrency control schemes, are not needed {[}38{]}.

There are several NewSQL DBMSs that are based on a main memory storage
architecture, including both academic (e.g., H-Store, HyPer) and
commercial (e.g., MemSQL, SAP HANA, VoltDB) systems. These systems
perform significantly better than disk-based DBMSs for OLTP workloads
because of this main memory orientation.

The idea of storing a database entirely in main memory is not a new one
{[}28, 33{]}. The seminal research at the University of
Wisconsin-Madison in the early 1980s established the foundation for many
aspects of main memory DBMSs {[}43{]}, including indexes, query
processing, and recovery algorithms. In that same decade, the first
distributed main-memory DBMSs, PRISMA/DB, was also developed {[}40{]}.
The first commercial main memory DBMSs appeared in 1990s; Altibase
{[}2{]}, Oracle's TimesTen {[}60{]}, and AT\&T's DataBlitz {[}20{]} were
early proponents of this approach.

One thing that is new with main memory NewSQL systems is the ability to
evict a subset of the database out to persistent storage to reduce its
memory footprint. This allows the DBMS to support databases that are
larger than the amount of memory available without having to switch back
to a disk-oriented architecture. The general approach is to use an
internal tracking mechanism inside of the system to identify which
tuples are not being accessed anymore and then chose them for eviction.
H-Store's \emph{anti-caching} component moves cold tuples to a
disk-resident store and then installs a ``tombstone'' record in the
database with the location of the original data {[}25{]}. When a
transaction tries to access a tuple through one of these tombstones, it
is aborted and then a separate thread asynchronously retrieves that
record and moves it back into memory. Another variant for supporting
larger-than-memory databases is an academic project from EPFL that uses
OS virtual memory paging in VoltDB {[}56{]}. To avoid false negatives,
all of these DBMSs retain the keys for evicted tuples in databases'
indexes, which inhibits the potential memory savings for those
applications with many secondary indexes. Although not a NewSQL DBMS,
Microsoft's \emph{Project Siberia} {[}29{]} for Hekaton maintains a
Bloom filter per index to reduce the in-memory storage overhead of
tracking evicted tuples.

Another DBMS that takes a different approach for larger-than-memory
databases is MemSQL where an administrator can manually instruct the
DBMS to store a table in a columnar format. MemSQL does not maintain any
in-memory tracking meta-data for these disk-resident tuples. It
organizes this data in log-structured storage to reduce the overhead of
updates, which are traditionally slow in OLAP data warehouses.

\hypertarget{partitioning-sharding}{%
\subsection{Partitioning / Sharding}\label{partitioning-sharding}}

The way that almost all of the distributed NewSQL DBMSs scale out is to
split a database up into disjoint subsets, called either partitions or
shards.

Distributed transaction processing on partitioned databases is not a new
idea. Many of the fundamentals of these systems came from the seminal
work by the great Phil Bernstein (and others) in the SDD-1 project in
the late 1970s {[}51{]}. In the early 1980s, the teams behind the two
pioneering, single-node DBMSs, System R and INGRES, both also created
distributed versions of their respective systems. IBM's R* was a
shared-nothing, disk-oriented distributed DBMS like SDD1 {[}63{]}. The
distributed version of INGRES is mostly remembered for its dynamic query
optimization algorithm that recursively breaks a distributed query into
smaller pieces {[}31{]}. Later, the GAMMA project {[}27{]} from the
University of Wisconsin-Madison explored different partitioning
strategies.

But these earlier distributed DBMSs never caught on for two reasons. The
first of these was that computing hardware in the 20th century was so
expensive that most organizations could not afford to deploy their
database on a cluster of machines. The second issue was that the
application demand for a high-performance distributed DBMS was simply
not there. Back then the expected peak throughput of a DBMS was
typically measured at tens to hundreds of transactions per second. We
now live in an era where both of these assumptions are no longer true.
Creating a large-scale, data-intensive application is easier now than it
ever has been, in part due to the proliferation of open-source
distributed system tools, cloud computing platforms, and affordable
mobile devices.

The database's tables are horizontally divided into multiple fragments
whose boundaries are based on the values of one (or more) of the table's
columns (i.e., the partitioning attributes). The DBMS assigns each tuple
to a fragment based on the values of these attributes using either range
or hash partitioning. Related fragments from multiple tables are
combined together to form a partition that is managed by a single node.
That node is responsible for executing any query that needs to access
data stored in its partition. Only the DBaaS systems (Amazon Aurora,
ClearDB) do not support this type of partitioning.

Ideally, the DBMS should be able to also distribute the execution of a
query to multiple partitions and then combine their results together
into a single result. All of the NewSQL systems except for ScaleArc that
support native partitioning provide this functionality.

The databases for many OLTP applications have a key property that makes
them amenable to partitioning. Their database schemas can be transposed
into a tree-like structure where descendants in the tree have a foreign
key relationship to the root {[}58{]}. The tables are then partitioned
on the attributes involved in these relationships such that all of the
data for a single entity are co-located together in the same partition.
For example, the root of the tree could be the customer table, and the
database is partitioned such that each customer, along with their order
records and account information, are stored together. The benefit of
this is that it allows most (if not all) transactions to only need to
access data at a single partition. This in turn reduces the
communication overhead of the system because it does not have to use an
atomic commitment protocol (e.g., two-phase commit) to make sure that
transactions finish correctly at different nodes.

The NewSQL DBMSs that deviate from the homogenous cluster node
architecture are NuoDB and MemSQL. For NuoDB, it designates one or more
nodes as storage managers (SM) that each store a partition of the
database. The SMs splits a database into blocks (called ``atoms'' in
NuoDB parlance). All other nodes in the cluster are designated as
transaction engines (TEs) that act as an in-memory cache of atoms. To
process a query, a TE node retrieves all of the atoms that it needs for
that query (either from the appropriate SMs or from other TEs). TEs
acquire write-locks on tuples and then broadcasts any changes to atoms
to the other TEs and the SM. To avoid atoms from moving back and forth
between nodes, NuoDB exposes load-balancing schemes to ensure that data
that is used together often reside at the same TE. This means that NuoDB
ends up with the same partitioning scheme as the other distributed DBMSs
but without having to pre-partition the database or identify the
relationships between tables.

MemSQL also uses a similar heterogeneous architecture comprised of
execution-only aggregator nodes and leaf nodes that store the actual
data. The difference between these two systems is in how they reduce the
amount of data that is pulled from the storage nodes to the execution
nodes. With NuoDB, the TEs cache atoms to reduce the amount data that
they read from the SMs. MemSQL's aggregator nodes do not cache any data,
but the leaf nodes execute parts of queries to reduce the amount of data
that is sent to the aggregator nodes; this is not possible in NuoDB
because the SMs are only a data store.

These two systems are able to add additional execution resources to the
DBMS's cluster (NuoDB's TE nodes, MemSQL's aggregator nodes) without
needing to re-partition the database. A research prototype of SAP HANA
also explored using this approach {[}36{]}. It remains to be seen,
however, whether such a heterogeneous architecture is superior to a
homogenous one (i.e., were each node both stores data and executes
queries) in terms of either performance or operational complexity.

Another aspect of partitioning in NewSQL systems that is new is that
some of them support live migration. This allows the DBMS to move data
between physical resources to re-balance and alleviate hotspots, or to
increase/decrease the DBMS's capacity without any interruption to
service. This is similar to re-balancing in NoSQL systems, but it is
more difficult because a NewSQL DBMS has to maintain ACID guarantees for
transactions during the migration {[}30{]}. There two approaches that
DBMSs use to achieve this. The first is to organize the database in many
coarse-grained ``virtual'' (i.e., logical) partitions that are spread
amongst the physical nodes {[}52{]}. Then when the DBMS needs to
re-balance, it moves these virtual partitions between nodes. This is the
approach used in Clustrix and AgilData, as well as in NoSQL systems like
Cassandra and DynamoDB. The other approach is for the DBMS to perform
more fine-grained re-balancing by redistributing individual tuples or
groups of tuples through range partitioning. This is akin to the
auto-sharding feature in the MongoDB NoSQL DBMS. It is used in systems
like ScaleBase and H-Store {[}30{]}.

\hypertarget{concurrency-control}{%
\subsection{Concurrency Control}\label{concurrency-control}}

Concurrency control scheme is the most salient and important
implementation detail of a transaction processing DBMS as it affects
almost all aspects of the system. Concurrency control permits end-users
to access a database in a multi-programmed fashion while preserving the
illusion that each of them is executing their transaction alone on a
dedicated system. It essentially provides the atomicity and isolation
guarantees in the system, and as such it influences the entire system's
behavior.

Beyond which scheme a system uses, another important aspect of the
design of a distributed DBMS is whether the system uses a centralized or
decentralized transaction coordination protocol. In a system with a
centralized coordinator, all transactions' operations have to go through
the coordinator, which then makes decisions about whether transactions
are allowed to proceed or not. This is the same approach used by the TP
monitors of the 1970--1980s (e.g., IBM CICS, Oracle Tuxedo). In a
decentralized system, each node maintains the state of transactions that
access the data that it manages. The nodes then have to coordinate with
each other to determine whether concurrent transactions conflict. A
decentralized coordinator is better for scalability but requires that
the clocks in the DBMS nodes are highly synchronized in order to
generate a global ordering of transactions {[}24{]}.

The first distributed DBMSs from the 1970--80s used two-phase locking
(2PL) schemes. SDD-1 was the first DBMS specifically designed for
distributed transaction processing across a cluster of shared-nothing
nodes managed by a centralized coordinator. IBM's R* was similar to
SDD-1, but the main difference was that the coordination of transactions
in R* was completely decentralized; it used distributed 2PL protocol
where transactions locked data items that they access directly at nodes.
The distributed version of INGRES also used decentralized 2PL with
centralized deadlock detection.

Almost all of the NewSQL systems based on new architectures eschew 2PL
because the complexity of dealing with deadlocks. Instead, the current
trend is to use variants of timestamp ordering (TO) concurrency control
where the DBMS assumes that transactions will not execute interleaved
operations that will violate serializable ordering. The most widely used
protocol in NewSQL systems is decentralized multi-version concurrency
control (MVCC) where the DBMS creates a new version of a tuple in the
database when it is updated by a transaction. Maintaining multiple
versions potentially allows transactions to still complete even if
another transaction updates the same tuples. It also allows for
long-running, read-only transactions to not block on writers. This
protocol is used in almost all of the NewSQL systems based on new
architectures, like MemSQL, HyPer, HANA, and CockroachDB. Although there
are engineering optimizations and tweaks that these systems use in their
MVCC implementations to improve performance, the basic concepts of the
scheme are not new. The first known work describing MVCC is a MIT PhD
dissertation from 1979 {[}49{]}, while the first commercial DBMSs to use
it were Digital's VAX Rdb and InterBase in the early 1980s. We note that
the architecture of InterBase was designed by Jim Starkey, who is also
the original designer of NuoDB and the failed Falcon MySQL storage
engine project.

Other systems use a combination of 2PL and MVCC together. With this
approach, transactions still have to acquire locks under the 2PL scheme
to modify the database. When a transaction modifies a record, the DBMS
creates a new version of that record just as it would with MVCC. This
scheme allows read-only queries to avoid having to acquire locks and
therefore not block on writing transactions. The most famous
implementation of this approach is MySQL's InnoDB, but it is also used
in both Google's Spanner, NuoDB, and Clustrix. NuoDB improves on the
original MVCC by employing a gossip protocol to broadcast versioning
information between nodes.

All of the middleware and DBaaS services inherit the concurrency control
scheme of their underlying DBMS architecture; since most of them use
MySQL, this makes them 2PL with MVCC systems.

We regard the concurrency control implementation in Spanner (along with
its descendants F1 {[}54{]} and SpannerSQL) as one of the most novel of
the NewSQL systems. The actual scheme itself is based on the 2PL and
MVCC combination developed in previous decades. But what makes Spanner
different is that it uses hardware devices (e.g., GPS, atomic clocks)
for high-precision clock synchronization. The DBMS uses these clocks to
assign timestamps to transactions to enforce consistent views of its
multi-version database over wide-area networks. CockroachDB also
purports to provide the same kind of consistency for transactions across
data centers as Spanner but without the use of atomic clocks. They
instead rely on a hybrid clock protocol that combines loosely
synchronized hardware clocks and logical counters {[}41{]}.

Spanner is also noteworthy because it heralds Google's return to using
transactions for its most critical services. The authors of Spanner even
remark that it is better to have their application programmers deal with
performance problems due to overuse of transactions, rather than writing
code to deal with the lack of transactions as one does with a NoSQL DBMS
{[}24{]}. Lastly, the only commercial NewSQL DBMS that is not using some
MVCC variant is VoltDB. This system still uses TO concurrency control,
but instead of interleaving transactions like in MVCC, it schedules
transactions to execute one-at-a-time at each partition. It also uses a
hybrid architecture where single-partition transactions are scheduled in
a decentralized manner, but multi-partition transactions are scheduled
with a centralized coordinator. VoltDB orders transactions based on
logical timestamps and then schedules them for execution at a partition
when it is their turn. When a transaction executes at a partition, it
has exclusive access to all of the data at that partition and thus the
system does not have to set fine-grained locks and latches on its data
structures. This allows transactions that only have to access a single
partition to execute efficiently because there is no contention from
other transactions. The downside of partition-based concurrency control
is that it does not work well if transactions span multiple partitions
because the network communication delays cause nodes to sit idle while
they wait for messages. This partition-based concurrency is not a new
idea. An early variant of it was first proposed in a 1992 paper by
Hector Garcia-Molina {[}34{]} and implemented in the kdb system in late
1990s {[}62{]} and in HStore (which is the academic predecessor of
VoltDB).

In general, we find that there is nothing significantly new about the
core concurrency control schemes in NewSQL systems other than laudable
engineering to make these algorithms work well in the context of modern
hardware and distributed operating environments.

\hypertarget{secondary-indexes}{%
\subsection{Secondary Indexes}\label{secondary-indexes}}

A secondary index contains a subset of attributes from a table that are
different than its primary key(s). This allows the DBMS to support fast
queries beyond primary key or partitioning key lookups. They are trivial
to support in a nonpartitioned DBMS because the entire database is
located on a single node. The challenge with secondary indexes in a
distributed DBMS is that they cannot always be partitioned in the same
manner as with the rest of the database. For example, suppose that the
tables of a database are partitioned based on the customer's table
primary key. But then there are some queries that want to do a reverse
look-up from the customer's email address to the account. Since the
tables are partitioned on the primary key, the DBMS will have to
broadcast these queries to every node, which is obviously inefficient.

The two design decisions for supporting secondary indexes in a
distributed DBMS are (1) where the system will store them and (2) how it
will maintain them in the context of transactions. In a system with a
centralized coordinator, like with sharding middleware, secondary
indexes can reside on both the coordinator node and the shard nodes. The
advantage of this approach is that there is only a single version of the
index in the entire system, and thus it is easier to maintain.

All of the NewSQL systems based on new architectures are decentralized
and use partitioned secondary indexes. This means that each node stores
a portion of the index, rather than each node having a complete copy of
it. The trade-off between partitioned and replicated indexes is that
with the former queries may need to span multiple nodes to find what
they are looking for but if a transaction updates an index it will only
have to modify one node. In a replicated index, the roles are reversed:
a look-up query can be satisfied by just one node in the cluster, but
any time a transaction modifies the attributes referenced in secondary
index's underlying table (i.e., the key or the value), the DBMS has to
execute a distributed transaction that updates all copies of the index.

An example of a decentralized secondary index that mixes both of these
concepts is in Clustrix. The DBMS first maintains a replicated,
coarse-grained (i.e., range-based) index at each node that maps values
to partitions. This mapping allows the DBMS to route queries to the
appropriate node using an attribute that is not the table's partitioning
attribute. These queries will then access a second partitioned index at
that node that maps exact values to tuples. Such a two-tier approach
reduces the amount of coordination that is needed to keep the replicated
index in sync across the cluster since it only maps ranges instead of
individual values.

The most common way that developers create secondary indexes when using
a NewSQL DBMS that does not support them is to deploy an index using an
in-memory, distributed cache, such as Memcached {[}32{]}. But using an
external system requires the application to maintain the cache since the
DBMSs will not automatically invalidate the external cache.

\hypertarget{replication}{%
\subsection{Replication}\label{replication}}

The best way that an organization can ensure high availability and data
durability for their OLTP application is to replicate their database.
All modern DBMSs, including NewSQL systems, support some kind of
replication mechanism. DBaaS have a distinct advantage in this area
because they hide all of the gritty details of setting of replication
from their customers. They make it easy to deploy a replicated DBMS
without the administrator having to worry about transmitting logs and
making sure that nodes are in sync.

There are two design decisions when it comes to database replication.
The first is how the DBMS enforces data consistency across nodes. In a
\emph{strongly consistent} DBMS, a transaction's writes must be
acknowledged and installed at all replicas before that transaction is
considered committed (i.e., durable). The advantage of this approach is
that replicas can serve read-only queries and still be consistent. That
is, if the application receives an acknowledgement that a transaction
has committed, then any modifications made by that transaction are
visible to any subsequent transaction in the future regardless of what
DBMS node they access. It also means that when a replica fails, there
are no lost updates because all the other nodes are synchronized. But
maintaining this synchronization requires the DBMS to use an atomic
commitment protocol (e.g, two-phase commit) to ensure that all replicas
agree with the outcome of a transaction, which has additional overhead
and can lead to stalls if a node fails or if there is a network
partition/delay. This is why NoSQL systems opt for a \emph{weakly
consistent} model (also called eventual consistency) where not all
replicas have to acknowledge a modification before the DBMS notifies the
application that the write succeeded.

All of the NewSQL systems that we are aware of support strongly
consistent replication. But there is nothing novel about how these
systems ensure this consistency. The fundamentals of state machine
replication for DBMSs were studied back in the 1970s {[}37, 42{]}.
NonStop SQL was one of the first distributed DBMSs built in the 1980s
using strongly consistency replication to provide fault tolerance in
this same manner {[}59{]}. In addition to the policy of when a DBMS
propagates updates to replicas, there are also two different execution
models for how the DBMS performs this propagation. The first, known as
\emph{active-active} replication, is where each replica node processes
the same request simultaneously. For example, when a transaction
executes a query, the DBMS executes that query in parallel at all of the
replicas. This is different from \emph{active-passive} replication where
a request is first processed at a single node and then the DBMS
transfers the resultant state to the other replicas. Most NewSQL DBMSs
implement this second approach because they use a non-deterministic
concurrency control scheme. This means that they cannot send queries to
replicas as they arrive on the master because they may get executed in a
different order on the replicas and the state of the databases will
diverge at each replica. This is because their execution order depends
on several factors, including network delays, cache stalls, and clock
skew.

Deterministic DBMSs (e.g., H-Store, VoltDB, ClearDB) on the other hand
do not perform these additional coordination steps. This is because the
DBMS guarantees that transactions' operations execute in the same order
on each replica and thus the state of the database is guaranteed to be
the same {[}44{]}. Both VoltDB and ClearDB also ensure that the
application does not execute queries that utilize sources of information
that are external to the DBMS that may be different on each replica
(e.g., setting a timestamp field to the local system clock).

One aspect of the NewSQL systems that is different than previous work
outside of academia is the consideration of replication over the
wide-area network (WAN). This is a byproduct of modern operating
environments where it is now trivial to deploy systems across multiple
data centers that are separated by large geographical differences. Any
NewSQL DBMS can be configured to provide synchronous updates of data
over the WAN, but this would cause significant slowdown for normal
operations. Thus, they instead provide asynchronous replication methods.
To the best of our knowledge, Spanner and CockroachDB are the only
NewSQL systems to provide a replication scheme that is optimized for
strongly consistent replicas over the WAN. They again achieve this
through a combination of atomic and GPS hardware clocks (in case of
Spanner {[}24{]}), or hybrid clocks (in the case of CockroachDB
{[}41{]}).

\hypertarget{crash-recovery}{%
\subsection{Crash Recovery}\label{crash-recovery}}

Another important feature of a NewSQL DBMS for providing fault tolerance
is its crash recovery mechanism. But unlike traditional DBMSs where the
main concern of fault tolerance is to ensure that no updates are lost
{[}47{]}, newer DBMSs must also minimize downtime. Modern web
applications are expected to be on-line all the time and site outages
are costly.

The traditional approach to recovery in a single-node system without
replicas is that when the DBMS comes back online after a crash, it loads
in the last checkpoint that it took from disk and then replays its
write-ahead log (WAL) to return the state of the database to where it
was at the moment of the crash. The canonical method of this approach,
known as ARIES {[}47{]}, was invented by IBM researchers in the 1990s.
All major DBMSs implement some variant of ARIES.

In a distributed DBMS with replicas, however, the traditional
single-node approach is not directly applicable. This is because when
the master node crashes, the system will promote one of the slave nodes
to be the new master. When the previous master comes back on-line, it
cannot just load in its last checkpoint and rerun its WAL because the
DBMS has continued to process transactions and therefore the state of
the database has moved forward. The recovering node needs to get the
updates from the new master (and potentially other replicas) that it
missed while it was down. There are two potential ways to do this. The
first is for the recovering node to load in its last checkpoint and WAL
from its local storage and then pull log entries that it missed from the
other nodes. As long as the node can process the log faster than new
updates are appended to it, the node will eventually converge to the
same state as the other replica nodes. This is possible if the DBMS uses
physical or physiological logging, since the time to apply the log
updates directly to tuples is much less than the time it takes to
execute the original SQL statement. To reduce the time it takes to
recover, the other option is for the recovering node to discard its
checkpoint and have system take a new one that the node will recover
from. One additional benefit of this approach is that this same
mechanism can also be used in the DBMS to add a new replica node.

The middleware and DBaaS systems rely on the built-in mechanisms of
their underlying single-node DBMSs but add additional infrastructure for
leader election and other management capabilities. The NewSQL systems
that are based on new architectures use a combination of off-the-shelf
components (e.g., ZooKeeper, Raft) and their own custom implementations
of existing algorithms (e.g., Paxos). All of these are standard
procedures and technologies that have been available in commercial
distributed systems since the 1990s.

\hypertarget{future-trends}{%
\section{FUTURE TRENDS}\label{future-trends}}

We foresee the next trend for database applications in the near future
is the ability to execute analytical queries and machine learning
algorithms on freshly obtained data. Such workloads, colloquially known
as ``real-time analytics'' or hybrid transaction-analytical processing
(HTAP), seek to extrapolate insights and knowledge by analyzing a
combination of historical data sets with new data {[}35{]}. This differs
from traditional business intelligence operations from the previous
decade that could only perform this analysis on historical data. Having
a shorter turnaround time is important in modern applications because
data has immense value as soon as it is created, but that value
diminishes over time.

There are three approaches to supporting HTAP pipelines in a database
application. The most common is to deploy separate DBMSs: one for
transactions and another for analytical queries. With this architecture,
the front-end OLTP DBMS stores all of the new information generated from
transactions. Then in the background, the system uses an
extract-transform-load utility to migrate data from this OLTP DBMS to a
second back-end data warehouse DBMS. The application executes all
complex OLAP queries in the back-end DBMS to avoid slowing down the OLTP
system. Any new information generated from the OLAP system is pushed
forward to front-end DBMS. Another prevailing system design, known as
the lambda architecture {[}45{]}, is to use a separate batch processing
system (e.g., Hadoop, Spark) to compute a comprehensive view on
historical data, while simultaneously using a stream processing system
(e.g., Storm {[}61{]}, Spark Streaming {[}64{]}) to provide views of
incoming data. In this split architecture, the batch processing system
periodically rescans the data set and performs a bulk upload of the
result to the stream processing system, which then makes modifications
based on new updates.

There are several problems inherent with the bifurcated environment of
these two approaches. Foremost is that the time it takes to propagate
changes between the separate systems is often measured in minutes or
even hours. This data transfer inhibits an application's ability to act
on data immediately when it is entered in the database. Second, the
administrative overhead of deploying and maintaining two different DBMSs
is non-trivial as personnel is estimated to be almost 50\% of the total
ownership cost of a large-scale database system {[}50{]}. It also
requires the application developer to write a query for multiple systems
if they want to combine data from different databases. Some systems that
try to achieve a single platform by hiding this split system
architecture; an example of this is Splice Machine {[}16{]}, but this
approach has other technical issues due to copying data from the OLTP
system (Hbase) before it can be used in the OLAP system (Spark).

The third (and in our opinion better) approach is to use a single HTAP
DBMS that supports the high throughput and low latency demands of OLTP
workloads, while also allowing for complex, longer running OLAP queries
to operate on both hot (transactional) and cold (historical) data. What
makes these newer HTAP systems different from legacy general-purpose
DBMSs is that they incorporate the advancements from the last decade in
the specialized OLTP (e.g., in-memory storage, lock-free execution) and
OLAP (e.g., columnar storage, vectorized execution) systems, but within
a single DBMS.

SAP HANA and MemSQL were the first NewSQL DBMSs to market themselves as
HTAP systems. HANA achieves this by using multiple execution engines
internally: one engine for row-oriented data that is better for
transactions and a different engine for column-oriented data that is
better for analytical queries. MemSQL uses two different storage
managers (one for rows, one for columns) but mixes them together in a
single execution engine. HyPer switched from a row-oriented system with
H-Store-style concurrency control that was focused on
OLTP to use an HTAP column-store architecture with MVCC to allow it
support more complex OLAP queries {[}48{]}. Even VoltDB has pivoted
their marketing strategy from pure OLTP performance to providing
streaming semantics. Similarly, the S-Store project seeks to add support
for stream processing operations on top of the H-Store architecture
{[}46{]}. It is likely that the specialized OLAP systems from the
mid-2000s (e.g., Greenplum) will start to add support for better OLTP.

We note, however, that the rise of HTAP DBMSs does mean the end of
giant, monolithic OLAP warehouses. Such systems will still be necessary
in the short-term as they stand to be the universal back-end database
for all of an organization's frontend OLTP silos. But eventually the
resurgence of database federation will allow organizations to execute
analytical queries that span multiple OLTP databases (including even
multiple vendors) without needing to move data around.

\hypertarget{conclusion}{%
\section{CONCLUSION}\label{conclusion}}

The main takeaway from our analysis is that NewSQL database systems are
not a radical departure from existing system architectures but rather
represent the next chapter in the continuous development of database
technologies. Most of the techniques that these systems employ have
existed in previous DBMSs from academia and industry. But many of them
were only implemented one-at-a-time in a single system and never all
together. What is therefore innovative about these NewSQL DBMSs is that
they incorporate these ideas into single platforms. Achieving this is by
no means a trivial engineering effort. They are by-products of a new era
where distributed computing resources are plentiful and affordable, but
at the same time the demands of applications is much greater.

It is also interesting to consider the potential impact and future
direction of NewSQL DBMSs in the marketplace. Given that the legacy DBMS
vendors are entrenched and well-funded, NewSQL systems have an uphill
battle to gain market share. In the last five years since we first
coined the term NewSQL {[}18{]}, several NewSQL companies have folded
(e.g., GenieDB, Xeround, Translattice) or pivoted to focus on other
problem domains (e.g., ScaleBase, ParElastic). Based on our analysis and
interviews with several companies, we have found that NewSQL systems
have had a relatively slow rate of adoption, especially compared to the
developer-driven NoSQL uptake. This is because NewSQL DBMSs are designed
to support the transactional workloads that are mostly found in
enterprise applications. Decisions regarding database choices for these
enterprise applications are likely to be more conservative than for new
Web application workloads. This is also evident from the fact that we
find that NewSQL DBMSs are used to complement or replace existing RDBMS
deployments, whereas
NoSQL are being deployed in new application workloads {[}19{]}.

Unlike with the OLAP DBMS start-ups from the 2000s, where almost all of
the vendors were acquired by major technology companies, up until now
there has been only one acquisition made of a NewSQL company. In March
2016, Tableau announced that it purchased the start-up formed for the
HyPer project. The two other possible exceptions to this are (1) Apple
acquiring FoundationDB in March 2015, but we exclude them because this
system was at its core a NoSQL key-value store with an inefficient SQL
layer grafted on top of it, and (2) ScaleArc acquiring ScaleBase, but
this was one competitor buying out another. None of these examples are
the same kind of acquisition where a legacy vendor purchasing an upstart
system (e.g., Teradata buying Aster Data Systems in 2011). We instead
see that the large vendors are choosing to innovate and improve their
own systems rather than acquire NewSQL start-ups. Microsoft added the
in-memory Hekaton engine to SQL Server in 2014 to improve OLTP
workloads. Oracle and IBM have been slightly slower to innovate; they
recently added column-oriented storage extensions to their systems to
compete with the rising popularity of OLAP DBMSs like HP Vertica and
Amazon Redshift. It is possible that they will add an in-memory option
for OLTP workloads in the future.

More long term, we believe that there will be a convergence of features
in the four classes of systems that we discussed here: (1) the older
DBMSs from the 1980-1990s, (2) the OLAP data warehouses from the 2000s,
(3) the NoSQL DBMSs from the 2000s, and (4) the NewSQL DBMSs from the
2010s. We expect that all of the key systems in these groups will
support some form of the relational model and SQL (if they do not
already), as well as both OLTP operations and OLAP queries together like
HTAP DBMSs. When this occurs, such labels will be meaningless.

\hypertarget{acknowledgments}{%
\section*{ACKNOWLEDGMENTS}\label{acknowledgments}}

The authors would like to thank the following people for their feedback:
Andy Grove (AgilData), Prakhar Verma (Amazon), Cashton Coleman
(ClearDB), Dave Anselmi (Clustrix), Spencer

Kimball (CockroachDB), Peter Mattis (CockroachDB), Ankur Goyal (MemSQL),
Seth Proctor (NuoDB), Anil Goel (SAP HANA), Ryan Betts (VoltDB). This
work was supported (in part) by the National Science Foundation (Award
CCF-1438955).

\textbf{For questions or comments about this paper, please call the CMU Database
Hotline at \texttt{+1-844-88-CMUDB}}.

\hypertarget{references}{%
\section{REFERENCES}\label{references}}

\begin{enumerate}[label={[}\arabic*{]}]
\item AgilData Scalable Cluster for MySQL.
\url{http://www.agildata.com/}.

\item Altibase. \url{http://altibase.com}.

\item Amazon Aurora. \url{https://aws.amazon.com/rds/aurora}.
\item
  Apache Trafodion. \url{http://trafodion.apache.org}.
\item
  ClearDB. \url{https://www.cleardb.com}.
\item
  Clustrix. \url{http://www.clustrix.com}.
\item
  CockroachDB. \url{https://www.cockroachlabs.com/}.
\item
  H-Store. \url{http://hstore.cs.brown.edu}.
\item
  HyPer. \url{http://hyper-db.de}.
\item MariaDB MaxScale. \url{https://mariadb.com/products/mariadb-maxscale}.
\item MemSQL. \url{http://www.memsql.com}.

\item MySQL Fabric. \url{https://www.mysql.com/products/enterprise/fabric.html}.
\item MySQL Proxy.
\url{http://dev.mysql.com/doc/mysql-proxy/en/}.

\item NuoDB. \url{http://www.nuodb.com}.
\item ScaleArc. \url{http://scalearc.com}.
\item
  Splice Machine. \url{http://www.splicemachine.com}.
\item
  VoltDB. \url{http://www.voltdb.com}.
\item
  M. Aslett. How will the database incumbents respond to NoSQL and
  NewSQL? The 451 Group, April 2011.
\item
  M. Aslett. MySQL vs. NoSQL and NewSQL:
2011-2015. The 451 Group, May 2012.
\item
  J. Baulier, P. Bohannon, S. Gogate, S. Joshi, C. Gupta, A. Khivesera,
  H. F. Korth, P. McIlroy, J. Miller, P. P. S.
Narayan, M. Nemeth, R. Rastogi, A. Silberschatz, and S. Sudarshan.
DataBlitz: A high performance main-memory storage manager. VLDB, pages
701--, 1998.
\item
  P. A. Bernstein, I. Cseri, N. Dani, N. Ellis, A. Kalhan,
G. Kakivaya, D. B. Lomet, R. Manne, L. Novik, and T. Talius. Adapting
microsoft SQL server for cloud computing. In \emph{ICDE}, pages
1255--1263, 2011.
\item
  R. Cattell. Scalable sql and nosql data stores. \emph{SIGMOD Rec.},
  39:12--27, 2011.
\item
  F. Chang, J. Dean, S. Ghemawat, W. C. Hsieh, D. A.
Wallach, M. Burrows, T. Chandra, A. Fikes, and R. E. Gruber. Bigtable: A
distributed storage system for structured data. \emph{ACM Trans. Comput.
Syst.}, 26:4:1--4:26, June 2008.
\item
  J. C. Corbett, J. Dean, M. Epstein, A. Fikes, C. Frost,
J. Furman, S. Ghemawat, A. Gubarev, C. Heiser,
P. Hochschild, W. Hsieh, S. Kanthak, E. Kogan, H. Li,
A. Lloyd, S. Melnik, D. Mwaura, D. Nagle, S. Quinlan,
R. Rao, L. Rolig, Y. Saito, M. Szymaniak, C. Taylor, R. Wang, and D.
Woodford. Spanner: Google's Globally-Distributed Database. In
\emph{OSDI}, 2012.
\item
  J. DeBrabant, A. Pavlo, S. Tu, M. Stonebraker, and S. B. Zdonik.
  Anti-caching: A new approach to database management system
  architecture. \emph{PVLDB}, 6(14):1942--1953, 2013.
\item
  G. DeCandia, D. Hastorun, M. Jampani, G. Kakulapati,
A. Lakshman, A. Pilchin, S. Sivasubramanian, P. Vosshall, and W. Vogels.
Dynamo: amazon's highly available key-value store. \emph{SIGOPS Oper.
Syst. Rev.}, 41:205--220, October 2007.
\item
  D. J. DeWitt, R. H. Gerber, G. Graefe, M. L. Heytens, K. B. Kumar, and
  M. Muralikrishna. GAMMA - a high performance dataflow database
  machine. In \emph{VLDB}, pages 228--237, 1986.
\item
  D. J. DeWitt, R. H. Katz, F. Olken, L. D. Shapiro, M. R. Stonebraker,
  and D. Wood. Implementation techniques for main memory database
  systems. \emph{SIGMOD Rec.}, 14(2):1--8, 1984.
\item
  A. Eldawy, J. Levandoski, and P.-\r{A}. Larson. Trekking through siberia:
  Managing cold data in a memory-optimized database. \emph{Proceedings
  of the VLDB Endowment}, 7(11):931--942, 2014.
\item
  A. J. Elmore, V. Arora, R. Taft, A. Pavlo, D. Agrawal, and A. E.
  Abbadi. Squall: Fine-grained live reconfiguration for partitioned main
  memory databases. SIGMOD, pages 299--313, 2015.
\item
  R. Epstein, M. Stonebraker, and E. Wong. Distributed query processing
  in a relational data base system. SIGMOD, pages 169--180, 1978.
\item
  B. Fitzpatrick. Distributed Caching with Memcached.
  \emph{Linux J.}, 2004(124):5--, Aug. 2004.
\item
  H. Garcia-Molina, R. J. Lipton, and J. Valdes. A massive memory
  machine. \emph{IEEE Trans. Comput.}, 33(5):391--399, May 1984.
\item
  H. Garcia-Molina and K. Salem. Main memory database systems: An
  overview. \emph{IEEE Trans. on Knowl. and Data Eng.}, 4(6):509--516,
  Dec. 1992.
\item
  Gartner. Hybrid Transaction/Analytical Processing Will Foster
  Opportunities for Dramatic Business Innovation.
  \url{https://www.gartner.com/doc/2657815/}, 2014.
\item A. K. Goel, J.
Pound, N. Auch, P. Bumbulis,
S. MacLean, F. F\"{a}rber, F. Gropengiesser, C. Mathis, T. Bodner, and W.
Lehner. Towards scalable real-time analytics: An architecture for
scale-out of olxp workloads. \emph{Proc. VLDB Endow.}, 8(12):1716--1727,
Aug. 2015.
\item
  J. Gray. \emph{Concurrency Control and Recovery in Database Systems},
  chapter Notes on data base operating systems, pages 393--481.
  Springer-Verlag, 1978.
\item
  S. Harizopoulos, D. J. Abadi, S. Madden, and M. Stonebraker. OLTP
  through the looking glass, and what we found there. In \emph{SIGMOD},
  pages 981--992, 2008.
\item
  A. Kemper and T. Neumann. HyPer: A hybrid OLTP\&OLAP main memory
  database system based on virtual memory snapshots. ICDE, pages
  195--206, 2011.
\item
  M. L. Kersten, P. M. Apers, M. A. Houtsma, E. J. Kuyk, and R. L. Weg.
  A distributed, main-memory database machine. In \emph{Database
  Machines and Knowledge Base Machines}, volume 43 of \emph{The Kluwer
  International Series in Engineering and Computer Science}, pages
  353--369. 1988.
\item
  S. Kimball. Living without atomic clocks.
  \url{https://www.cockroachlabs.com/blog/living-without-atomic-clocks/},
  February 2016.
\item
  L. Lamport. The implementation of reliable distributed multiprocess
  systems. \emph{Computer Networks}, 2:95--114, 1978.
\item
  T. J. Lehman. \emph{Design and performance evaluation of a main memory
  relational database system}. PhD thesis, University of
  Wisconsin--Madison, 1986.
\item
  N. Malviya, A. Weisberg, S. Madden, and M. Stonebraker. Rethinking
  main memory oltp recovery. In \emph{ICDE}, pages 604--615, 2014.
\item
  N. Marz and J. Warren. \emph{Big Data: Principles and best practices
  of scalable realtime data systems}. Manning Publications, 2013.
\item
  J. Meehan, N. Tatbul, S. Zdonik, C. Aslantas,
U. \c{C}etintemel, J. Du, T. Kraska, S. Madden, D. Maier,
A. Pavlo, M. Stonebraker, K. Tufte, and H. Wang. S-store: Streaming
meets transaction processing.
\emph{PVLDB}, 8(13):2134--2145, 2015.
\item
  C. Mohan, D. Haderle, B. Lindsay, H. Pirahesh, and P. Schwarz. ARIES:
  a transaction recovery method supporting fine-granularity locking and
  partial rollbacks using write-ahead logging. \emph{ACM Trans. Database
  Syst.}, 17(1):94--162, 1992.
\item
  T. Neumann, T. M\"{u}hlbauer, and A. Kemper. Fast serializable
  multi-version concurrency control for main-memory database systems.
  SIGMOD, pages 677--689, 2015.
\item
  D. P. Reed. \emph{Naming and synchronization in a decentralized
  computer system}. PhD thesis, MIT, 1979.
\item
  A. Rosenberg. Improving query performance in data warehouses.
  \emph{Business Intelligence Journal}, 11, Jan. 2006.
\item
  J. B. Rothnie, Jr., P. A. Bernstein, S. Fox, N. Goodman, M. Hammer, T.
  A. Landers, C. Reeve, D. W. Shipman, and E. Wong. Introduction to a
  system for distributed databases (SDD-1). \emph{ACM Trans. Database
  Syst.}, 5(1):1--17, Mar. 1980.
\item
  M. Serafini, E. Mansour, A. Aboulnaga, K. Salem, T. Rafiq, and U. F.
  Minhas. Accordion: Elastic scalability for database systems supporting
  distributed transactions. \emph{Proc. VLDB Endow.}, 7(12):1035--1046,
  Aug. 2014.
\item
  R. Shoup and D. Pritchett. The ebay architecture. SD Forum, November
  2006.
\item
  J. Shute, R. Vingralek, B. Samwel, B. Handy,
  C. Whipkey, E. Rollins, M. Oancea, K. Littlefield,
  D. Menestrina, S. Ellner, J. Cieslewicz, I. Rae,T. Stancescu, and H.
    Apte. F1: A distributed sql database that scales. \emph{Proc. VLDB
    Endow.}, 6(11):1068--1079, Aug. 2013.
\item
  V. Sikka, F. F\"{a}rber, W. Lehner, S. K. Cha, T. Peh, and C. Bornhövd.
  Efficient transaction processing in sap hana database: The end of a
  column store myth. SIGMOD, pages 731--742, 2012.
\item
  R. Stoica and A. Ailamaki. Enabling efficient os paging for
  main-memory OLTP databases. In \emph{DaMon}, 2013.
\item
  M. Stonebraker. New sql: An alternative to nosql and old sql for new
  oltp apps. BLOG@CACM, June 2011.
\item
  M. Stonebraker, S. Madden, D. J. Abadi,
S. Harizopoulos, N. Hachem, and P. Helland. The end of an architectural
era: (it's time for a complete rewrite). In \emph{VLDB}, pages
1150--1160, 2007.
\item
  Tandem Database Group. NonStop SQL, a distributed, high-performance,
  high-availability implementation of sql. Technical report, Tandem,
  Apr. 1987.
\item
  T. Team. In-memory data management for consumer transactions the
  timesten approach. SIGMOD '99, pages 528--529, 1999.
\item
  A. Toshniwal, S. Taneja, A. Shukla, K. Ramasamy, J. M.
Patel, S. Kulkarni, J. Jackson, K. Gade, M. Fu,
J. Donham, N. Bhagat, S. Mittal, and D. Ryaboy.
Storm@twitter. In \emph{SIGMOD}, pages 147--156, 2014.
\item
  A. Whitney, D. Shasha, and S. Apter. High Volume
Transaction Processing Without Concurrency Control, Two Phase Commit,
SQL or C++. In \emph{HPTS}, 1997.
\item
  R. Williams, D. Daniels, L. Haas, G. Lapis, B. Lindsay, P. Ng, R.
  Obermarck, P. Selinger, A. Walker, P. Wilms, and R. Yost. Distributed
  systems, vol. ii: distributed data base systems. chapter R*: an
  overview of the architecture, pages 435--461. 1986.
\item
  M. Zaharia, T. Das, H. Li, T. Hunter, S. Shenker, and I. Stoica.
  Discretized streams: Fault-tolerant streaming computation at scale. In
  \emph{SOSP}, 2013.\enlargethispage{\baselineskip}
\item
  S. B. Zdonik and D. Maier, editors. \emph{Readings in Object-Oriented
  Database Systems}. Morgan Kaufmann, 1990.
\end{enumerate}

\end{document}
