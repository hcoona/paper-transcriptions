\PassOptionsToPackage{unicode=true}{hyperref} % options for packages loaded elsewhere
\PassOptionsToPackage{hyphens}{url}
%
\documentclass[a4paper,11pt,twoside,openright]{article}

\usepackage{ifxetex}
\ifxetex{}
\else
  \errmessage{Must be built with xelatex}
\fi

\usepackage{amssymb,amsmath}
\usepackage{fourier}
\usepackage{inconsolata}
\usepackage{enumitem}
\usepackage{footnote}

% Table
\usepackage{tabu}
\usepackage{longtable}
\usepackage{booktabs}
\usepackage{multirow}

% Verbatim & Source code
\usepackage{fancyvrb}
\usepackage{minted}

% Beauty
\usepackage[protrusion]{microtype}
\usepackage[all]{nowidow}
\usepackage{upquote}
\usepackage{parskip}
\usepackage[strict]{changepage}

\usepackage{hyperref}

% Graph
\usepackage{graphicx}
\usepackage{grffile}
\usepackage{tikz}


\hypersetup{
  bookmarksnumbered,
  pdfborder={0 0 0},
  pdfpagemode=UseNone,
  pdfstartview=FitH,
  breaklinks=true}
\urlstyle{same}  % don't use monospace font for urls

\usetikzlibrary{arrows.meta,shapes.geometric,shapes.misc}

\newminted{java}{%
  autogobble,
  breakbytokenanywhere,
  breaklines,
  fontsize=\footnotesize,
}
\newmintinline{java}{%
  autogobble,
  breakbytokenanywhere,
  breaklines,
  fontsize=\footnotesize,
}

\makeatletter
\def\maxwidth{\ifdim\Gin@nat@width>\linewidth\linewidth\else\Gin@nat@width\fi}
\def\maxheight{\ifdim\Gin@nat@height>\textheight\textheight\else\Gin@nat@height\fi}
\makeatother

% Scale images if necessary, so that they will not overflow the page
% margins by default, and it is still possible to overwrite the defaults
% using explicit options in \includegraphics[width, height, ...]{}
\setkeys{Gin}{width=\maxwidth,height=\maxheight,keepaspectratio}
\setlength{\emergencystretch}{3em}  % prevent overfull lines
\setcounter{secnumdepth}{3}

% Redefines (sub)paragraphs to behave more like sections
\ifx\paragraph\undefined\else
\let\oldparagraph\paragraph
\renewcommand{\paragraph}[1]{\oldparagraph{#1}\mbox{}}
\fi
\ifx\subparagraph\undefined\else
\let\oldsubparagraph\subparagraph
\renewcommand{\subparagraph}[1]{\oldsubparagraph{#1}\mbox{}}
\fi

% set default figure placement to htbp
\makeatletter
\def\fps@figure{htbp}
\makeatother


\captionsetup{font=footnotesize,labelfont=bf}

\title{The Dangers of Replication and a Solution}
\author{Jim Gray, Pat Helland, Patrick O'Neil, and Dennis Shasha}
\date{January 1996}

\begin{document}

\maketitle

\begin{abstract}

Update anywhere-anytime-anyway transactional
replication has unstable behavior as the workload scales up: a ten-fold
increase in nodes and traffic gives a thousand fold increase in
deadlocks or reconciliations. Master copy replication (primary copy)
schemes reduce this problem. A simple analytic model demonstrates these
results. A new two-tier replication algorithm is proposed that allows
mobile (disconnected) applications to propose tentative update
transactions that are later applied to a master copy. Commutative update
transactions avoid the instability of other replication schemes.

\end{abstract}

\section{Introduction}

Data is replicated at multiple network nodes for performance and
availability. \emph{\textbf{Eager replication}} keeps all replicas
exactly synchronized at all nodes by updating all the replicas as part
of one atomic transaction. Eager replication gives serializable
execution --- there are no concurrency anomalies. But, eager replication
reduces update performance and increases transaction response times
because extra updates and messages are added to the transaction.

Eager replication is not an option for mobile applications where most
nodes are normally disconnected. Mobile applications require
\emph{\textbf{lazy replication}} algorithms that asynchronously
propagate replica updates to other nodes after the updating transaction
commits. Some continuously connected systems use lazy replication to
improve response time.

Lazy replication also has shortcomings, the most serious being stale
data versions. When two transactions read and write data concurrently,
one transaction's updates should be serialized after the other's. This
avoids concurrency anomalies. Eager replication typically uses a locking
scheme to detect and regulate concurrent execution. Lazy replication
schemes typically use a multi-version concurrency control scheme to
detect non-serializable behavior {[}Bernstein, Hadzilacos, Goodman{]},
{[}Berenson, et. al.{]}. Most multi-version isolation schemes provide
the transaction with the most recent committed value. Lazy replication
may allow a transaction to see a very old committed value. Committed
updates to a local value may be ``in transit'' to this node if the
update strategy is ``lazy''.

Eager replication delays or aborts an uncommitted transaction if
committing it would violate serialization. Lazy replication has a more
difficult task because some replica updates have already been committed
when the serialization problem is first detected. There is usually no
automatic way to reverse the committed replica updates, rather a program
or person must \emph{\textbf{reconcile}} conflicting transactions.

To make this tangible, consider a joint checking account you share with
your spouse. Suppose it has \$1,000 in it. This account is replicated in
three places: your checkbook, your spouse's checkbook, and the bank's
ledger.

Eager replication assures that all three books have the same account
balance. It prevents you and your spouse from writing checks totaling
more than \$1,000. If you try to overdraw your account, the transaction
will fail.

Lazy replication allows both you and your spouse to write checks
totaling \$1,000 for a total of \$2,000 in withdrawals. When these
checks arrived at the bank, or when you communicated with your spouse,
someone or something reconciles the transactions that used the virtual
\$1,000.

It would be nice to automate this reconciliation. The bank does that by
rejecting updates that cause an overdraft. This is a master replication
scheme: the bank has the master copy and only the bank's updates really
count. Unfortunately, this works only for the bank. You, your spouse,
and your creditors are likely to spend considerable time reconciling the
``extra'' thousand dollars worth of transactions. In the meantime, your
books will be inconsistent with the bank's books. That makes it
difficult for you to perform further banking operations.

The database for a checking account is a single number, and a log of
updates to that number. It is the simplest database. In reality,
databases are more complex and the serialization issues are more subtle.

\emph{The theme of this paper is that update-anywhere-anytime-anyway
replication is unstable. }

\begin{enumerate}
\def\labelenumi{\arabic{enumi}.}
\item
  \emph{If the number of checkbooks per account increases by a factor of
  ten, the deadlock or reconciliation rates rises by a factor of a
  thousand.}
\item
  \emph{Disconnected operation and message delays mean lazy replication
  has more frequent reconciliation.}
\end{enumerate}

\begin{figure}
  \centering
  \includegraphics[width=0.6\columnwidth]{fig-1.png}
  \caption{When replicated, a simple single-node transaction may
  apply its updates remotely either as part of the same transaction
  (\emph{eager}) or as separate transactions (\emph{lazy}). In either
  case, if data is replicated at \(N\) nodes, the transaction does
  \(N\) times as much work}
\end{figure}

Simple replication works well at low loads and with a few nodes. This
creates a \emph{\textbf{scaleup pitfall}}. A prototype system
demonstrates well. Only a few transactions deadlock or need
reconciliation when running on two connected nodes. But the system
behaves very differently when the application is scaled up to a large
number of nodes, or when nodes are disconnected more often, or when
message propagation delays are longer. Such systems have higher
transaction rates. Suddenly, the deadlock and reconciliation rate is
astronomically higher (cubic growth is predicted by the model). The
database at each node diverges further and further from the others as
reconciliation fails. Each reconciliation failure implies differences
among nodes. Soon, the system suffers \textbf{\emph{system delusion}}
--- the database is inconsistent and there is no obvious way to repair
it {[}Gray \& Reuter, pp. 149-150{]}.

This is a bleak picture, but probably accurate. Simple replication
(transactional update-anywhere-anytime-anyway) cannot be made to work
with global serializability.

In outline, the paper gives a simple model of replication and a
closed-form average-case analysis for the probability of waits,
deadlocks, and reconciliations. For simplicity, the model ignores many
issues that would make the predicted behavior even worse. In particular,
it ignores the message propagation delays needed to broadcast replica
updates. It ignores ``true'' serialization, and assumes a weak
multi-version form of committed-read serialization (no read locks)
{[}Berenson{]}. The paper then considers object master replication.
Unrestricted lazy master replication has many of the instability
problems of eager and group replication.

A restricted form of replication avoids these problems:
\emph{\textbf{two-tier replication}} has \emph{\textbf{base nodes}} that
are always connected, and \emph{\textbf{mobile nodes}} that are usually
disconnected.

\begin{enumerate}
\def\labelenumi{\arabic{enumi}.}
\item
  Mobile nodes propose tentative update transactions to objects owned by
  other nodes. Each mobile node keeps two object versions: a local
  version and a best known master version.
\item
  Mobile nodes occasionally connect to base nodes and propose tentative
  update transactions to a master node. These proposed transactions are
  re-executed and may succeed or be rejected. To improve the chances of
  success, tentative transactions are designed to commute with other
  transactions. After exchanges the mobile node's database is
  synchronized with the base nodes. Rejected tentative transactions are
  reconciled by the mobile node owner who generated the transaction.
\end{enumerate}

Our analysis shows that this scheme supports lazy replication and mobile
computing but avoids system delusion: tentative updates may be rejected
but the base database state remains consistent.

\section{Replication Models}

Figure 1 shows two ways to propagate updates to replicas:

\begin{enumerate}
\def\labelenumi{\arabic{enumi}.}
\item
  \emph{\textbf{Eager}}: Updates are applied to all replicas of an
  object as part of the original transaction.
\item
  \emph{\textbf{Lazy:}} One replica is updated by the originating
  transaction. Updates to other replicas propagate asynchronously,
  typically as a separate transaction for each node.
\end{enumerate}

\begin{figure}
  \centering
  \includegraphics[width=0.6\columnwidth]{fig-2.png}
  \caption{Updates may be controlled in two ways. Either all updates
emanate from a master copy of the object, or updates may emanate from
any. Group ownership has many more chances for conflicting updates.}
\end{figure}

Figure 2 shows two ways to regulate replica updates:

\begin{enumerate}
\def\labelenumi{\arabic{enumi}.}
\item
  \emph{\textbf{Group}}: Any node with a copy of a data item can update
  it. This is often called \emph{update anywhere}.
\item
  \emph{\textbf{Master}}: Each object has a master node. Only the master
  can update the \emph{primary copy} of the object. All other replicas
  are read-only. Other nodes wanting to update the object request the
  master do the update.
\end{enumerate}

\begin{table}
  \begin{tabu} to \columnwidth {XXX}
    \toprule
    \rowfont[c]\bfseries
    Propagation & \multirow{3}{*}{Lazy} & \multirow{3}{*}{Eager}\\
    \rowfont[c]\bfseries
    vs. & &\\
    \rowfont[c]\bfseries
    Ownership & &\\
    \midrule
    \multirow{2}{*}{Group} & N transactions & one transaction\\
    & N object owners & N object owners\\
    \multirow{2}{*}{Master} & N transactions & one transaction\\
    & one object owner & one object owners\\
    \multirow{2}{*}{Two Tier} & \multicolumn{2}{l}{N+1 transactions, one object owner}\\
    & \multicolumn{2}{l}{tentative local updates, eager base updates}\\
    \bottomrule
  \end{tabu}
  \caption{A taxonomy of replication strategies contrasting
  propagation strategy (eager or lazy) with the ownership strategy (master
  or group).}
\end{table}

\begin{table}
  \begin{tabu} to \columnwidth {X[1,c]X[3,L]}
  \toprule
  \emph{DB\_Size} & number of distinct objects in the database\\
  \emph{Nodes} & number of nodes; each node replicates all objects\\
  \emph{Transactions} & number of concurrent transactions at a node. This
  is a derived value.\\
  \emph{TPS} & number of transactions per second originating at this
  node.\\
  \emph{Actions} & number of updates in a transaction\\
  \emph{Action\_Time} & time to perform an action\\
  \emph{Time\_Between\newline
  \_Disconnects} & mean time between network disconnect of a node.\\
  \emph{Disconnected\newline
  \_time} & mean time node is disconnected from network\\
  \emph{Message\_Delay} & time between update of an object and update of a
  replica (ignored)\\
  \emph{Message\_cpu} & processing and transmission time needed to send a
  replication message or apply a replica update (ignored)\\
  \bottomrule
  \end{tabu}
  \caption{Variables used in the model and analysis}
\end{table}

The analysis below indicates that group and lazy replication are more
prone to serializability violations than master and eager replication

The model assumes the database consists of a fixed set of objects. There
are a fixed number of nodes, each storing a replica of all objects. Each
node originates a fixed number of transactions per second. Each
transaction updates a fixed number of objects. Access to objects is
equi-probable (there are no hotspots). Inserts and deletes are modeled
as updates.

Reads are ignored. Replica update requests have a transmit delay and
also require processing by the sender and receiver. These delays and
extra processing are ignored; only the work of sequentially updating the
replicas at each node is modeled. Some nodes are mobile and disconnected
most of the time. When first connected, a mobile node sends and receives
deferred replica updates. Table 2 lists the model parameters.

One can imagine many variations of this model. Applying eager updates in
parallel comes to mind. Each design alternative gives slightly different
results. The design here roughly characterizes the basic alternatives.
We believe obvious variations will not substantially change the results
here.

Each node generates \emph{TPS} transactions per second. Each transaction
involves a fixed number of actions. Each action requires a fixed time to
execute. So, a transaction's duration is \emph{Actions x Action\_Time}.
Given these two observations, the number of concurrent transactions
originating at a node is:

\begin{equation}
\text{\emph{Transactions}} = \text{\emph{TPS}} \times \text{\emph{Actions}} \times \text{\emph{Action\_Time}}
\end{equation}

\begin{figure}
  \centering
  \includegraphics[width=0.6\columnwidth]{fig-3.png}
  \caption{Systems can grow by (1) \emph{scaleup}: buying a bigger
machine, (2) \emph{partitioning}: dividing the work between two
machines, or (3) \emph{replication}: placing the data at two machines
and having each machine keep the data current. This simple idea is key
to understanding the \(N^2\) growth. Notice that
each of the replicated servers at the lower right of the illustration is
performing 2 TPS and the aggregate rate is 4 TPS. Doubling the users
increased the total workload by a factor of four. Read-only transactions
need not generate any additional load on remote nodes.}
\end{figure}

A more careful analysis would consider that fact that, as system load
and contention rises, the time to complete an action increases. In a
scaleable server system, this \emph{time-dilation} is a second-order
effect and is ignored here.

In a system of \(N\) nodes, \(N\) times as many transactions will
be originating per second. Since each update transaction must replicate
its updates to the other \((N - 1)\) nodes, it is easy to see that the
transaction size for eager systems grows by a factor of \(N\) and the
node update rate grows by \(N^2\). In lazy systems,
each \emph{user} update transaction generates \(N - 1\) lazy replica
updates, so there are \(N\) times as many concurrent transactions,
and the node update rate is \(N^2\) higher. This
non-linear growth in node update rates leads to unstable behavior as the
system is scaled up.

\section{Eager Replication}

Eager replication updates all replicas when a transaction updates any
instance of the object. There are no serialization anomalies
(inconsistencies) and no need for reconciliation in eager systems.
Locking detects potential anomalies and converts them to waits or
deadlocks.

With eager replication, reads at connected nodes give current data.
Reads at disconnected nodes may give stale (out of date) data. Simple
eager replication systems prohibit updates if any node is disconnected.
For high availability, eager replication systems allow updates among
members of the quorum or cluster {[}Gifford{]}, {[}Garcia-Molina{]}.
When a node joins the quorum, the quorum sends the new node all replica
updates since the node was disconnected. We assume here that a quorum or
fault tolerance scheme is used to improve update availability.

Even if all the nodes are connected all the time, updates may fail due
to deadlocks that prevent serialization errors. The following simple
analysis derives the wait and deadlock rates of an eager replication
system. We start with wait and deadlock rates for a single-node system.

In a single-node system the ``other'' transactions have about
\(\frac{\text{\emph{Transactions}} \times \text{\emph{Actions}}}{2}\) resources
locked (each is about half way complete). Since objects are chosen
uniformly from the database, the chance that a request by one
transaction will request a resource locked by any other transaction is:
\(\frac{\text{\emph{Transactions}} \times \text{\emph{Actions}}}{2 \times \text{\emph{DB\_Size}}}\).
A transaction makes \emph{Actions} such requests, so the chance that
it will wait sometime in its lifetime is approximately {[}Gray et.
al.{]}, {[}Gray \& Reuter pp. 428{]}:

\begin{equation}
  \text{\emph{PW}} \approx 1 - \left( 1 - \frac{\text{\emph{Transactions}} \times \text{\emph{Actions}}}{2 \times \text{\emph{DB\_Size}}} \right)^\text{\emph{Actions}}
     \approx \frac{\text{\emph{Transactions}} \times \text{\emph{Actions}}^2}{2 \times \text{\emph{DB\_Size}}}
\end{equation}

A deadlock consists of a cycle of transactions waiting for one another.
The probability a transaction forms a cycle of length two is
\(\text{\emph{PW}}^2\) divided by the number of transactions.
Cycles of length \(j\) are proportional to \(\text{\emph{PW}}^j\)and so are even less likely
if \(\text{\emph{PW}} \ll 1\).
Applying equation (1), the probability that the transaction deadlocks is
approximately:

\begin{equation}
\text{\emph{PD}} \approx \frac{\text{\emph{PW}}^2}{\text{\emph{Transactions}}} \frac{\text{\emph{Transactions}} \times \text{\emph{Actions}}^4}{4 \times \text{\emph{DB\_Size}}^2}
   = \frac{\text{\emph{TPS}} \times \text{\emph{Action\_Time}} \times \text{\emph{Actions}}^5}{4 \times \text{\emph{DB\_Size}}^2}
\end{equation}

Equation (3) gives the deadlock hazard for a transaction. The deadlock
rate for a transaction is the probability it deadlock's in the next
second. That is \text{\emph{PD}} divided by the transaction lifetime
(\(\text{\emph{Actions}} \times \text{\emph{Action\_Time}}\)).

\begin{equation}
\text{\emph{Trans\_Deadlock\_rate}} \approx \frac{\text{\emph{TPS}} \times \text{\emph{Actions}}^4}{4 \times \text{\emph{DB\_Size}}^2}
\end{equation}

Since the node runs \emph{Transactions} concurrent transactions, the
deadlock rate for the whole node is higher. Multiplying equation (4) and
equation (1), the node deadlock rate is:

\begin{equation}
\text{\emph{Node\_Deadlock\_rate}} \approx \frac{\text{\emph{TPS}}^2 \times \text{\emph{Action\_Time}} \times \text{\emph{Actions}}^5}{4 \times \text{\emph{DB\_Size}}^2}
\end{equation}

Suppose now that several such systems are replicated using eager
replication --- the updates are done immediately as in Figure 1. Each
node will initiate its local load of \emph{TPS} transactions per
second\footnote{The assumption that transaction arrival rate per node
  stays constant as nodes are replicated assumes that nodes are lightly
  loaded. As the replication workload increases, the nodes must grow
  processing and IO power to handle the increased load. Growing power at
  an \(N^2\) rate is problematic.}. The transaction
size, duration, and aggregate transaction rate for eager systems is:

\begin{equation}
\begin{aligned}
\text{\emph{Transaction\_Size}} &= \text{\emph{Actions}} \times \text{\emph{Nodes}}\\
\text{\emph{Transaction\_Duration}} &= \text{\emph{Actions}} \times \text{\emph{Nodes}} \times \text{\emph{Action\_Time}}\\
\text{\emph{Total\_TPS}} &= \text{\emph{TPS}} \times \text{\emph{Nodes}}
\end{aligned}
\end{equation}

Each node is now doing its own work and also applying the updates
generated by other nodes. So each update transaction actually performs
many more actions (\emph{Nodes x Actions}) and so has a much longer
lifetime --- indeed it takes at least \emph{Nodes} times
longer\footnote{An alternate model has eager actions broadcast the
  update to all replicas in one instant. The replicas are updated in
  parallel and the elapsed time for each action is constant (independent
  of \(N\)). In our model, we attempt to capture message handing
  costs by serializing the individual updates. If one follows this
  model, then the processing at each node rises quadraticly, but the
  number of concurrent transactions stays constant with scaleup. This
  model avoids the polynomial explosion of waits and deadlocks if the
  total TPS rate is held constant.}. As a result the total number of
transactions in the system rises quadratically with the number of nodes:

Total\_Transactions = TPS x Actions x Action\_Time x
Nodes\textsuperscript{2} (7)

This rise in active transactions is due to eager transactions taking
\(N\)-Times longer and due to lazy updates generating \(N\)-times
more transactions. The action rate also rises very fast with \(N\).
Each node generates work for all other nodes. The eager work rate,
measured in actions per second is:

Action\_Rate = Total\_TPS x Transaction\_Size

= TPS x Actions x Nodes\textsuperscript{2} (8)

It is surprising that the action rate and the number of active
transactions is the same for eager and lazy systems. Eager systems have
fewer-longer transactions. Lazy systems have more and shorter
transactions. So, although equations (6) are different for lazy systems,
equations (7) and (8) apply to both eager and lazy systems.

Ignoring message handling, the probability a transaction waits can be
computed using the argument for equation (2). The transaction makes
\emph{Actions} requests while the other Total\_Transactions have
\emph{Actions/2} objects locked. The result is approximately:

(9)

This is the probability that one transaction waits. The wait rate (waits
per second) for the entire system is computed as:

(10)

As with equation (4), The probability that a particular transaction
deadlocks is approximately:

(11)

The equation for a single-transaction deadlock implies the total
deadlock rate. Using the arguments for equations (4) and (5), and using
equations (7) and (11):

(12)

If message delays were added to the model, then each transaction would
last much longer, would hold resources much longer, and so would be more
likely to collide with other transactions. Equation (12) also ignores
the ``second order'' effect of two transactions racing to update the
same object at the same time (it does not distinguish between
\emph{Master} and \emph{Group} replication). If \emph{DB\_Size
\textgreater{}\textgreater{} Node,} such conflicts will be rare.

This analysis points to some serious problems with eager replication.
Deadlocks rise as the third power of the number of nodes in the network,
and the fifth power of the transaction size. Going from one-node to ten
nodes increases the deadlock rate a thousand fold. A ten-fold increase
in the transaction size increases the deadlock rate by a factor of
100,000.

To ameliorate this, one might imagine that the database size grows with
the number of nodes (as in the checkbook example earlier, or in the
TPC-A, TPC-B, and TPC-C benchmarks). More nodes, and more transactions
mean more data. With a scaled up database size, equation (12) becomes:

(13)

Now a ten-fold growth in the number of nodes creates \emph{only} a
ten-fold growth in the deadlock rate. This is still an unstable
situation, but it is a big improvement over equation (12)

Having a master for each object helps eager replication avoid deadlocks.
Suppose each object has an owner node. Updates go to this node first and
are then applied to the replicas. If, each transaction updated a single
replica, the object-master approach would eliminate all deadlocks.

In summary, eager replication has two major problems:

\begin{enumerate}
\def\labelenumi{\arabic{enumi}.}
\item
  Mobile nodes cannot use an eager scheme when disconnected.
\item
  The probability of deadlocks, and consequently failed transactions
  rises very quickly with transaction size and with the number of nodes.
  A ten-fold increase in nodes gives a thousand-fold increase in failed
  transactions (deadlocks).
\end{enumerate}

We see no solution to this problem. If replica updates were done
concurrently, the action time would not increase with \(N\) then the
growth rate would \emph{only} be quadratic.

\textbf{4. Lazy Group Replication}

Lazy group replication allows any node to update any local data. When
the transaction commits, a transaction is sent to every other node to
apply the root transaction's updates to the replicas at the destination
node (see Figure 4). It is possible for two nodes to update the same
object and race each other to install their updates at other nodes. The
replication mechanism must detect this and reconcile the two
transactions so that their updates are not lost.

Timestamps are commonly used to detect and reconcile lazy-group
transactional updates. Each object carries the timestamp of its most
recent update. Each replica update carries the new value and is tagged
with the old object timestamp. Each node detects incoming replica
updates that would overwrite earlier committed updates. The node tests
if the local replica's timestamp and the update's old timestamp are
equal. If so, the update is safe. The local replica's timestamp advances
to the new transaction's timestamp and the object value is updated. If
the current timestamp of the local replica does not match the old
timestamp seen by the root transaction, then the update may be
``dangerous''. In such cases, the node rejects the incoming transaction
and submits it for \emph{reconciliation}.

Figure 4: A lazy transaction has a root execution that updates either
master or local copies of data. Then subsequent transactions update
replicas at remote nodes --- one lazy transaction per remote replica
node. The lazy updates carry timestamps of each original object. If the
local object timestamp does not match, the update may be dangerous and
some form of reconciliation is needed.

Transactions that would wait in an eager replication system face
reconciliation in a lazy-group replication system. Waits are much more
frequent than deadlocks because it takes two waits to make a deadlock.
Indeed, if waits are a rare event, then deadlocks are very rare
\emph{(rare\textsuperscript{2}).} Eager replication waits cause delays
while deadlocks create application faults. With lazy replication, the
much more frequent waits are what determines the reconciliation
frequency. So, the system-wide lazy-group reconciliation rate follows
the transaction wait rate equation (Equation 10):

(14)

As with eager replication, if message propagation times were added, the
reconciliation rate would rise. Still, having the reconciliation rate
rise by a factor of a thousand when the system scales up by a factor of
ten is frightening.

The really bad case arises in mobile computing. Suppose that the typical
node is disconnected most of the time. The node accepts and applies
transactions for a day. Then, at night it connects and downloads them to
the rest of the network. At that time it also accepts replica updates.
It is as though the message propagation time was 24 hours.

If any two transactions at any two different nodes update the same data
during the disconnection period, then they will need reconciliation.
What is the chance of two disconnected transactions colliding during the
\emph{Disconnected\_Time}?

If each node updates a small fraction of the database each day then the
number of distinct \emph{outbound} pending object updates at reconnect
is approximately:

(15)

Each of these updates applies to all the replicas of an object. The
pending \emph{inbound updates} for this node from the rest of the
network is approximately \emph{(Nodes-1)} times larger than this.

\emph{(16)}

If the inbound and outbound sets overlap, then reconciliation is needed.
The chance of an object being in both sets is approximately:

(17)

Equation (17) is the chance one node needs reconciliation during the
\emph{Disconnect\_Time} cycle. The rate for all nodes is:

(18)

The quadratic nature of this equation suggests that a system that
performs well on a few nodes with simple transactions may become
unstable as the system scales up.

\textbf{5. Lazy Master Replication}

Master replication assigns an owner to each object. The owner stores the
object's correct current value. Updates are first done by the owner and
then propagated to other replicas. Different objects may have different
owners.

When a transaction wants to update an object, it sends an RPC (remote
procedure call) to the node owning the object. To get serializability, a
read action should send read-lock RPCs to the masters of any objects it
reads.

To simplify the analysis, we assume the node originating the transaction
broadcasts the replica updates to all the slave replicas after the
master transaction commits. The originating node sends one slave
transaction to each slave node (as in Figure 1). Slave updates are
timestamped to assure that all the replicas converge to the same final
state. If the record timestamp is newer than a replica update timestamp,
the update is ``stale'' and can be ignored. Alternatively, each master
node sends replica updates to slaves in sequential commit order.

Lazy-Master replication is not appropriate for mobile applications. A
node wanting to update an object must be connected to the object owner
and participate in an atomic transaction with the owner.

As with eager systems, lazy-master systems have no reconciliation
failures; rather, conflicts are resolved by waiting or deadlock.
Ignoring message delays, the deadlock rate for a lazy-master replication
system is similar to a single node system with much higher transaction
rates. Lazy master transactions operate on master copies of objects.
But, because there are \emph{Nodes} times more users, there are
\emph{Nodes} times as many concurrent master transactions and
approximately \emph{Nodes\textsuperscript{2}} times as many replica
update transactions. The replica update transactions do not really
matter, they are background housekeeping transactions. They can abort
and restart without affecting the user. So the main issue is how
frequently the master transactions deadlock. Using the logic of equation
(5), the deadlock rate is approximated by:

(19)

This is better behavior than lazy-group replication. Lazy-master
replication sends fewer messages during the base transaction and so
completes more quickly. Nevertheless, all of these replication schemes
have troubling deadlock or reconciliation rates as they grow to many
nodes.

In summary, lazy-master replication requires contact with object masters
and so is not useable by mobile applications. Lazy-master replication is
slightly less deadlock prone than eager-group replication primarily
because the transactions have shorter duration.

\textbf{6. Non-Transactional Replication Schemes}

The equations in the previous sections are facts of nature --- they help
explain another fact of nature. They show why there are no
high-update-traffic replicated databases with globally serializable
transactions.

Certainly, there are replicated databases: bibles, phone books, check
books, mail systems, name servers, and so on. But updates to these
databases are managed in interesting ways --- typically in a lazy-master
way. Further, updates are not record-value oriented; rather, updates are
expressed as transactional transformations such as ``Debit the account
by \$50'' instead of ``change account from \$200 to \$150''.

One strategy is to abandon serializabilty for the
\emph{\textbf{convergence property}}: if no new transactions arrive, and
if all the nodes are connected together, they will all converge to the
same replicated state after exchanging replica updates. The resulting
state contains the committed appends, and the most recent replacements,
but updates may be lost.

Lotus Notes gives a good example of convergence {[}Kawell{]}. Notes is a
lazy group replication design (update anywhere, anytime, anyhow). Notes
provides convergence rather than an ACID transaction execution model.
The database state may not reflect any particular serial execution, but
all the states will be identical. As explained below, timestamp schemes
have the lost-update problem.

Lotus Notes achieves convergence by offering lazy-group replication at
the transaction level. It provides two forms of update transaction:

\begin{enumerate}
\def\labelenumi{\arabic{enumi}.}
\item
  \emph{\textbf{Append}} adds data to a Notes file. Every appended note
  has a timestamp. Notes are stored in timestamp order. If all nodes are
  in contact with all others, then they will all converge on the same
  state.
\item
  \emph{\textbf{Timestamped}} \emph{\textbf{replace a value}} replaces a
  value with a newer value. If the current value of the object already
  has a timestamp greater than this update's timestamp, the incoming
  update is discarded.
\end{enumerate}

If convergence were the only goal, the timestamp method would be
sufficient. But, the timestamp scheme may lose the effects of some
transactions because it just applies the most recent updates. Applying a
timestamp scheme to the checkbook example, if there are two concurrent
updates to a checkbook balance, the highest timestamp value wins and the
other update is discarded as a ``stale'' value. Concurrency control
theory calls this the \emph{lost update problem}. Timestamp schemes are
vulnerable to lost updates.

Convergence is desirable, but the converged state should reflect the
effects of all committed transactions. In general this is not possible
unless global serialization techniques are used.

In certain cases transactions can be designed to commute, so that the
database ends up in the same state no matter what transaction execution
order is chosen. Timestamped Append is a kind of commutative update but
there are others (e.g., adding and subtracting constants from an integer
value). It would be possible for Notes to support a third form of
transaction:

\begin{enumerate}
\def\labelenumi{\arabic{enumi}.}
\setcounter{enumi}{2}
\item
  \emph{\textbf{Commutative}} \emph{\textbf{updates}} that are
  incremental transformations of a value that can be applied in any
  order.
\end{enumerate}

Lotus Notes, the Internet name service, mail systems, Microsoft Access,
and many other applications use some of these techniques to achieve
convergence and avoid delusion.

Microsoft Access offers convergence as follows. It has a single design
master node that controls all schema updates to a replicated database.
It offers update-anywhere for record instances. Each node keeps a
version vector with each replicated record. These version vectors are
exchanged on demand or periodically. The most recent update wins each
pairwise exchange. Rejected updates are reported {[}Hammond{]}.

The examples contrast with a simple update-anywhere-anytime-anyhow
lazy-group replication offered by some systems. If the transaction
profiles are not constrained, lazy-group schemes suffer from unstable
reconciliation described in earlier sections. Such systems degenerate
into system delusion as they scale up.

Lazy group replication schemes are emerging with specialized
reconciliation rules. Oracle 7 provides a choice of twelve
reconciliation rules to merge conflicting updates {[}Oracle{]}. In
addition, users can program their own reconciliation rules. These rules
give priority certain sites, or time priority, or value priority, or
they merge commutative updates. The rules make some transactions
commutative. A similar, transaction-level approach is followed in the
two-tier scheme described next.

\begin{enumerate}
\def\labelenumi{\arabic{enumi}.}
\setcounter{enumi}{6}
\item
  \textbf{Two-Tier Replication }
\end{enumerate}

An ideal replication scheme would achieve four goals:

\textbf{Availability and scaleability}: Provide high availability and
scaleability through replication, while avoiding instability.

\textbf{Mobility}: Allow mobile nodes to read and update the database
while disconnected from the network.

\textbf{Serializability}: Provide single-copy serializable transaction
execution.

\textbf{Convergence}: Provide convergence to avoid system delusion.

The safest transactional replication schemes, (ones that avoid system
delusion) are the eager systems and lazy master systems. They have no
reconciliation problems (they have no reconciliation). But these systems
have other problems. As shown earlier:

\begin{enumerate}
\def\labelenumi{\arabic{enumi}.}
\item
  Mastered objects cannot accept updates if the master node is not
  accessible. This makes it difficult to use master replication for
  mobile applications.
\item
  Master systems are unstable under increasing load. Deadlocks rise
  quickly as nodes are added.
\item
  Only eager systems and lazy master (where reads go to the master) give
  ACID serializability.
\end{enumerate}

Circumventing these problems requires changing the way the system is
used. We believe a scaleable replication system must function more like
the check books, phone books, Lotus Notes, Access, and other replication
systems we see about us.

Lazy-group replication systems are prone to reconciliation problems as
they scale up. Manually reconciling conflicting transactions is
unworkable. One approach is to \emph{undo} all the work of any
transaction that needs reconciliation --- backing out all the updates of
the transaction. This makes transactions atomic, consistent, and
isolated, but not durable --- or at least not durable until the updates
are propagated to each node. In such a lazy group system, every
transaction is tentative until all its replica updates have been
propagated. If some mobile replica node is disconnected for a very long
time, all transactions will be tentative until the missing node
reconnects. So, an undo-oriented lazy-group replication scheme is
untenable for mobile applications.

The solution seems to require a modified mastered replication scheme. To
avoid reconciliation, each object is mastered by a node --- much as the
bank owns your checking account and your mail server owns your mailbox.
Mobile agents can make tentative updates, then connect to the base nodes
and immediately learn if the tentative update is acceptable.

The \emph{\textbf{two-tier replication}} scheme begins by assuming there
are two kinds of nodes:

\emph{\textbf{Mobile}} \emph{\textbf{nodes}} are disconnected much of
the time. They store a replica of the database and may originate
tentative transactions. A mobile node may be the master of some data
items.

\emph{\textbf{Base nodes}} are always connected. They store a replica of
the database. Most items are mastered at base nodes.

Replicated data items have two versions at mobile nodes:

\emph{\textbf{Master Version}}: The most recent value received from the
object master. The version at the object master is \emph{the} master
version, but disconnected or lazy replica nodes may have older versions.

\emph{\textbf{Tentative Version}}: The local object may be updated by
tentative transactions. The most recent value due to local updates is
maintained as a tentative value.

Similarly, there are two kinds of transactions:

\emph{\textbf{Base Transaction}}: Base transactions work only on master
data, and they produce new master data. They involve at most one
connected-mobile node and may involve several base nodes.

\emph{\textbf{Tentative Transaction}:} Tentative transactions work on
local tentative data. They produce new tentative versions. They also
produce a base transaction to be run at a later time on the base nodes.

Tentative transactions must follow a \emph{\textbf{scope rule}}: they
may involve objects mastered on base nodes and mastered at the mobile
node originating the transaction (call this the transaction's
\emph{scope}). The idea is that the mobile node and all the base nodes
will be in contact when the tentative transaction is processed as a
``real'' base transaction --- so the real transaction will be able to
read the master copy of each item in the scope.

Local transactions that read and write \emph{only} local data can be
designed in any way you like. They cannot read-or write any tentative
data because that would make them tentative.

Figure 5: The two-tier-replication scheme. Base nodes store replicas of
the database. Each object is mastered at some node. Mobile nodes store a
replica of the database, but are usually disconnected. Mobile nodes
accumulate tentative transactions that run against the tentative
database stored at the node. Tentative transactions are reprocessed as
base transactions when the mobile node reconnects to the base. Tentative
transactions may fail when reprocessed.

The base transaction generated by a tentative transaction may fail or it
may produce different results. The base transaction has an
\emph{\textbf{acceptance criterion}}: a test the resulting outputs must
pass for the slightly different base transaction results to be
acceptable. To give some sample acceptance criteria:

\begin{itemize}
\item
  The bank balance must not go negative.
\item
  The price quote can not exceed the tentative quote.
\item
  The seats must be aisle seats.
\end{itemize}

If a tentative transaction fails, the originating node and person who
generated the transaction are informed it failed and why it failed.
Acceptance failure is equivalent to the reconciliation mechanism of the
lazy-group replication schemes. The differences are (1) the master
database is always converged --- there is no system delusion, and (2)
the originating node need only contact a base node in order to discover
if a tentative transaction is acceptable.

To continue the checking account analogy, the bank's version of the
account is the master version. In writing checks, you and your spouse
are creating tentative transactions which result in tentative versions
of the account. The bank runs a base transaction when it clears the
check. If you contact your bank and it clears the check, then you know
the tentative transaction is a real transaction.

Consider the two-tier replication scheme's behavior during connected
operation. In this environment, a two-tier system operates much like a
lazy-master system with the additional restriction that no transaction
can update data mastered at more than one mobile node. This restriction
is not really needed in the connected case.

Now consider the disconnected case. Imagine that a mobile node
disconnected a day ago. It has a copy of the base data as of yesterday.
It has generated tentative transactions on that base data and on the
local data mastered by the mobile node. These transactions generated
tentative data versions at the mobile node. If the mobile node queries
this data it sees the tentative values. For example, if it updated
documents, produced contracts, and sent mail messages, those tentative
updates are all visible at the mobile node.

When a mobile node connects to a base node, the mobile node:

\begin{enumerate}
\def\labelenumi{\arabic{enumi}.}
\item
  Discards its tentative object versions since they will soon be
  refreshed from the masters,
\item
  Sends replica updates for any objects mastered at the mobile node to
  the base node ``hosting'' the mobile node,
\item
  Sends all its tentative transactions (and all their input parameters)
  to the base node to be executed in the order in which they committed
  on the mobile node,
\item
  Accepts replica updates from the base node (this is standard
  lazy-master replication), and
\item
  Accepts notice of the success or failure of each tentative
  transaction.
\end{enumerate}

The ``host'' base node is the other tier of the two tiers. When
contacted by a mobile note, the host base node:

\begin{enumerate}
\def\labelenumi{\arabic{enumi}.}
\item
  Sends delayed replica update transactions to the mobile node.
\item
  Accepts delayed update transactions for mobile-mastered objects from
  the mobile node.
\item
  Accepts the list of tentative transactions, their input messages, and
  their acceptance criteria. Reruns each tentative transaction in the
  order it committed on the mobile node. During this reprocessing, the
  base transaction reads and writes object master copies using a
  lazy-master execution model. The scope-rule assures that the base
  transaction only accesses data mastered by the originating mobile node
  and base nodes. So master copies of all data in the transaction's
  scope are available to the base transaction. If the base transaction
  fails its acceptance criteria, the base transaction is aborted and a
  diagnostic message is returned to the mobile node. If the acceptance
  criteria requires the base and tentative transaction have identical
  outputs, then subsequent transactions reading tentative results
  written by T will fail too. On the other hand, weaker acceptance
  criteria are possible.
\item
  After the base node commits a base transaction, it propagates the lazy
  replica updates as transactions sent to all the other replica nodes.
  This is standard lazy-master.
\item
  When all the tentative transactions have been reprocessed as base
  transactions, the mobile node's state is converged with the base
  state.
\end{enumerate}

The key properties of the two-tier replication scheme are:

\begin{enumerate}
\def\labelenumi{\arabic{enumi}.}
\item
  Mobile nodes may make tentative database updates.
\item
  Base transactions execute with single-copy serializability so the
  master base system state is the result of a serializable execution.
\item
  A transaction becomes durable when the base transaction completes.
\item
  Replicas at all connected nodes converge to the base system state.
\item
  If all transactions commute, there are no reconciliations.
\end{enumerate}

This comes close to meeting the four goals outlined at the start of this
section.

When executing a base transaction, the two-tier scheme is a lazy-master
scheme. So, the deadlock rate for base transactions is given by equation
(19). This is still an \(N^2\) deadlock rate. If a
base transaction deadlocks, it is resubmitted and reprocessed until it
succeeds, much as the replica update transactions are resubmitted in
case of deadlock.

The reconciliation rate for base transactions will be zero if all the
transactions commute. The reconciliation rate is driven by the rate at
which the base transactions fail their acceptance criteria.

Processing the base transaction may produce results different from the
tentative results. This is acceptable for some applications. It is fine
if the checking account balance is different when the transaction is
reprocessed. Other transactions from other nodes may have affected the
account while the mobile node was disconnected. But, there are cases
where the changes may not be acceptable. If the price of an item has
increased by a large amount, if the item is out of stock, or if aisle
seats are no longer available, then the salesman's price or delivery
quote must be reconciled with the customer.

These acceptance criteria are application specific. The replication
system can do no more than detect that there is a difference between the
tentative and base transaction. This is probably too pessimistic a test.
So, the replication system will simply run the tentative transaction. If
the tentative transaction completes successfully and passes the
acceptance test, then the replication system assumes all is well and
propagates the replica updates as usual.

Users are aware that all updates are tentative until the transaction
becomes a base transaction. If the base transaction fails, the user may
have to revise and resubmit a transaction. The programmer must design
the transactions to be commutative and to have acceptance criteria to
detect whether the tentative transaction agrees with the base
transaction effects.

Figure 6: Executing tentative and base transactions in two-tier
replication.

Thinking again of the checkbook example of an earlier section. The check
is in fact a tentative update being sent to the bank. The bank either
honors the check or rejects it. Analogous mechanisms are found in forms
flow systems ranging from tax filing, applying for a job, or subscribing
to a magazine. It is an approach widely used in human commerce.

This approach is similar to, but more general than the Data Cycle
architecture {[}Herman{]} which has a single master node for all
objects.

The approach can be used to obtain pure serializability if the base
transaction only reads and writes master objects (current versions).

\textbf{8. Summary}

Replicating data at many nodes and letting anyone update the data is
problematic. Security is one issue, performance is another. When the
standard transaction model is applied to a replicated database, the size
of each transaction rises by the degree of replication. This, combined
with higher transaction rates means dramatically higher deadlock rates.

It might seem at first that a lazy replication scheme will solve this
problem. Unfortunately, lazy-group replication just converts waits and
deadlocks into reconciliations. Lazy-master replication has slightly
better behavior than eager-master replication. Both suffer from
dramatically increased deadlock as the replication degree rises. None of
the master schemes allow mobile computers to update the database while
disconnected from the system.

The solution appears to be to use semantic tricks (timestamps, and
commutative transactions), combined with a two-tier replication scheme.
Two-tier replication supports mobile nodes and combines the benefits of
an eager-master-replication scheme and a local update scheme.

\textbf{9. Acknowledgments}

Tanj (John G.) Bennett of Microsoft and Alex Thomasian of IBM gave some
very helpful advice on an earlier version of this paper. The anonymous
referees made several helpful suggestions to improve the presentation.
Dwight Joe pointed out a mistake in the published version of equation
19.

\textbf{10. References}

Bernstein, P.A., V. Hadzilacos, N. Goodman, Concurrency Control and
Recovery in Database Systems, Addison Wesley, Reading MA., 1987.

Berenson, H., Bernstein, P.A., Gray, J., Jim Melton, J., O'Neil, E.,
O'Neil, P., ``A Critique of ANSI SQL Isolation Levels,'' Proc. ACM
SIGMOD 95, pp. 1-10, San Jose CA, June 1995.

Garcia Molina, H. ``Performance of Update Algorithms for Replicated Data
in a Distributed Database,'' TR STAN-CS-79-744, CS Dept., Stanford U.,
Stanford, CA., June 1979.

Garcia Molina, H., Barbara, D., ``How to Assign Votes in a Distributed
System,'' J. ACM, 32(4). Pp. 841-860, October, 1985.

Gifford, D. K., ``Weighted Voting for Replicated Data,'' Proc. ACM
SIGOPS SOSP, pp: 150-159, Pacific Grove, CA, December 1979.

Gray, J., Reuter, A., \emph{Transaction Processing: Concepts and
Techniques,} Morgan Kaufmann, San Francisco, CA. 1993.

Gray, J., Homan, P, Korth, H., Obermarck, R., ``A Strawman Analysis of
the Probability of Deadlock,'' IBM RJ 2131, IBM Research, San Jose, CA.,
1981.

Hammond, Brad, ``Wingman, A Replication Service for Microsoft Access and
Visual Basic'', Microsoft White Paper, bradha@microsoft.com

Herman, G., Gopal, G, Lee, K., Weinrib, A., ``The Datacycle Architecture
for Very High Throughput Database Systems,'' Proc. ACM SIGMOD, San
Francisco, CA. May 1987.

Kawell, L.., Beckhardt, S., Halvorsen, T., Raymond Ozzie, R., Greif,
I.,"Replicated Document Management in a Group Communication System,"
Proc. Second Conference on Computer Supported Cooperative Work, Sept.
1988.

Oracle, "Oracle7 Server Distributed Systems: Replicated Data," Oracle
part number A21903.March 1994, Oracle, Redwood Shores, CA. Or
http://www.oracle.com/products/oracle7/
server/whitepapers/replication/html/index

\end{document}
