\PassOptionsToPackage{unicode=true}{hyperref} % options for packages loaded elsewhere
\PassOptionsToPackage{hyphens}{url}
\documentclass[a4paper,11pt,notitlepage,twoside,openright]{article}

\usepackage{ifxetex}
\ifxetex
\else
\errmessage{Must be built with XeLaTeX}
\fi

\usepackage{amssymb,amsmath}
\usepackage{fourier}
\usepackage{inconsolata}
\usepackage{enumitem}
\usepackage{footnote}

% Table
\usepackage{tabu}
\usepackage{longtable}
\usepackage{booktabs}
\usepackage{multirow}

% Verbatim & Source code
\usepackage{fancyvrb}
\usepackage{minted}

% Beauty
\usepackage[protrusion]{microtype}
\usepackage[all]{nowidow}
\usepackage{upquote}
\usepackage{parskip}
\usepackage[strict]{changepage}

\usepackage{hyperref}

% Graph
\usepackage{graphicx}
\usepackage{grffile}
\usepackage{tikz}


\hypersetup{
  bookmarksnumbered,
  pdfborder={0 0 0},
  pdfpagemode=UseNone,
  pdfstartview=FitH,
  breaklinks=true}
\urlstyle{same}  % don't use monospace font for urls

\usetikzlibrary{arrows.meta,shapes.geometric,shapes.misc}

\newminted{java}{%
  autogobble,
  breakbytokenanywhere,
  breaklines,
  fontsize=\footnotesize,
}
\newmintinline{java}{%
  autogobble,
  breakbytokenanywhere,
  breaklines,
  fontsize=\footnotesize,
}

\makeatletter
\def\maxwidth{\ifdim\Gin@nat@width>\linewidth\linewidth\else\Gin@nat@width\fi}
\def\maxheight{\ifdim\Gin@nat@height>\textheight\textheight\else\Gin@nat@height\fi}
\makeatother

% Scale images if necessary, so that they will not overflow the page
% margins by default, and it is still possible to overwrite the defaults
% using explicit options in \includegraphics[width, height, ...]{}
\setkeys{Gin}{width=\maxwidth,height=\maxheight,keepaspectratio}
\setlength{\emergencystretch}{3em}  % prevent overfull lines
\setcounter{secnumdepth}{3}

% Redefines (sub)paragraphs to behave more like sections
\ifx\paragraph\undefined\else
\let\oldparagraph\paragraph
\renewcommand{\paragraph}[1]{\oldparagraph{#1}\mbox{}}
\fi
\ifx\subparagraph\undefined\else
\let\oldsubparagraph\subparagraph
\renewcommand{\subparagraph}[1]{\oldsubparagraph{#1}\mbox{}}
\fi

% set default figure placement to htbp
\makeatletter
\def\fps@figure{htbp}
\makeatother


\title{End-To-End Arguments in System Design}
\author{J. H. SALTZER, D. P. REED, and D. D. CLARK}
\date{Received February 1983; accepted June 1983}

\begin{document}

\maketitle

\begin{abstract}
This paper presents a design principle that helps guide placement of
functions among the modules of a distributed computer system. The
principle, called the end-to-end argument, suggests that functions
placed at low levels of a system may be redundant or of little value
when compared with the cost of providing them at that low level.
Examples discussed in the paper include bit-error recovery, security
using encryption, duplicate message suppression, recovery from system
crashes, and delivery acknowledgment. Low-level mechanisms to support
these functions are justified only as performance enhancements.
\end{abstract}

\hypertarget{introduction}{%
  \section{INTRODUCTION}\label{introduction}}

Choosing the proper boundaries between functions is perhaps the primary
activity of the computer system designer. Design principles that provide
guidance in this choice of function placement are among the most
important tools of a system designer. This paper discusses one class of
function placement argument that has been used for many years with
neither explicit recognition nor much conviction. However, the
emergence of the data communication network as a computer system
component has sharpened this line of function placement argument by
making more apparent the situations in which and the reasons why it
applies. This paper articulates the argument explicitly, so as to
examine its nature and to see how general it really is. The argument
appeals to application requirements and provides a rationale for moving
a function upward in a layered system closer to the application that
uses the function. We begin by considering the communication network
version of the argument.

In a system that includes communications, one usually draws a modular
boundary around the communication subsystem and defines a firm interface
between it and the rest of the system. When doing so, it becomes
apparent that there is a list of functions each of which might be
implemented in any of several ways: by the communication subsystem, by
its client, as a joint venture, or perhaps redundantly, each doing its
own version. In reasoning about this choice, the requirements of the
application provide the basis for the following class of arguments:

\begin{quote}
\it
The function in question can completely and correctly be
implemented only with the knowledge and help of the application standing
at the endpoints of the communication system. Therefore, providing that
questioned function as a feature of the communication system itself is
not possible. (Sometimes an incomplete version of the function provided
by the communication system may be useful as a performance
enhancement.)
\end{quote}

We call this line of reasoning against low-level function implementation
the \emph{end-to-end argument.} The following sections examine the
end-to-end argument in detail, first with a case study of a typical
example in which it is used --- the function in question is reliable
data transmission --- and then by exhibiting the range of functions to
which the same argument can be applied. For the case of the data
communication system, this range includes encryption, duplicate message
detection, message sequencing, guaranteed message delivery, detecting
host crashes, and delivery receipts. In a broader context, the argument
seems to apply to many other functions of a computer operating system,
including its file system. Examination of this broader context will be
easier, however, if we first consider the more specific data
communication context.

\hypertarget{careful-file-transfer}{%
\section{CAREFUL FILE TRANSFER}\label{careful-file-transfer}}

\hypertarget{end-to-end-caretaking}{%
\subsection{End-to-End Caretaking}\label{end-to-end-caretaking}}


Consider the problem of \emph{careful file transfer.} A file is stored
by a file system in the disk storage of computer A. Computer A is linked
by a data communication network with computer B, which also has a file
system and a disk store. The object is to move the file from computer
A's storage to computer B's storage without damage, keeping in mind that
failures can occur at various points along the way. The application
program in this case is the file transfer program, part of which runs at
host A and part at host B. In order to discuss the possible threats to
the file's integrity in this transaction, let us assume that the
following specific steps are involved:


\begin{enumerate}[label=(\arabic*)]
\item

  At host A the file transfer program calls upon the file system to read
  the file from the disk, where it resides on several tracks, and the
  file system passes it to the file transfer program in fixed-size
  blocks chosen to be disk format independent.

\item

  Also at host A, the file transfer program asks the data communication
  system to transmit the file using some communication protocol that
  involves splitting the data into packets. The packet size is typically
  different from the file block size and the disk track size.

\item

  The data communication network moves the packets from computer A to
  computer B.

\item

  At host B, a data communication program removes the packets from the
  data communication protocol and hands the contained data to a second
  part of the file transfer application that operates within host B.

\item

  At host B, the file transfer program asks the file system to write the
  received data on the disk of host B.

\end{enumerate}

With this model of the steps involved, the following are some of the
threats to the transaction that a careful designer might be concerned
about:

\begin{enumerate}[label=(\arabic*)]
\item

  The file, though originally written correctly onto the disk at host A,
  if read now may contain incorrect data, perhaps because of hardware
  faults in the disk storage system.

\item

  The software of the file system, the file transfer program, or the
  data communication system might make a mistake in buffering and
  copying the data of the file, either at host A or host B.

\item

  The hardware processor or its local memory might have a transient
  error while doing the buffering and copying, either at host A or host
  B.

\item

  The communication system might drop or change the bits in a packet or
  deliver a packet more than once.

\item

  Either of the hosts may crash part way through the transaction after
  performing an unknown amount (perhaps all) of the transaction.

\end{enumerate}

How would a careful file transfer application then cope with this list
of threats? One approach might be to reinforce each of the steps along
the way using duplicate copies, time-out and retry, carefully located
redundancy for error detection, crash recovery, etc. The goal would be
to reduce the probability of each of the individual threats to an
acceptably small value. Unfortunately, systematic countering of threat
(2) requires writing correct programs, which is quite difficult. Also,
not all the programs that must be correct are written by the file
transfer-application programmer. If we assume further that all these
threats are relatively low in probability --- low enough for the system
to allow useful work to be accomplished --- brute force
countermeasures, such as doing everything three times, appear
uneconomical.


The alternate approach might be called \emph{end-to-end check and
retry.} Suppose that as an aid to coping with threat (1), stored with
each file is a checksum that has sufficient redundancy to reduce the
chance of an undetected error in the file to an acceptably negligible
value. The application program follows the simple steps above in
transferring the file from A to B. Then, as a final additional step, the
part of the file transfer application residing in host B reads the
transferred file copy back from its disk storage system into its own
memory, recalculates the checksum, and sends this value back to host A,
where it is compared with the checksum of the original. Only if the two
checksums agree does the file transfer application declare the
transaction committed. If the comparison fails, something has gone
wrong, and a retry from the beginning might be attempted.


If failures are fairly rare, this technique will normally work on the
first try; occasionally a second or even third try might be required.
One would probably consider two or more failures on the same file
transfer attempt as indicating that some part of this system is in need
of repair.


Now let us consider the usefulness of a common proposal, namely, that
the communication system provide, internally, a guarantee of reliable
data transmission. It might accomplish this guarantee by providing
selective redundancy in the form of packet checksums, sequence number
checking, and internal retry mechanisms, for example. With sufficient
care, the probability of undetected bit errors can be reduced to any
desirable level. The question is whether or not this attempt to be
helpful on the part of the communication system is useful to the careful
file transfer application.

The answer is that threat (4) may have been eliminated, but the careful
file transfer application must still counter the remaining threats; so
it should still provide its own retries based on an end-to-end checksum
of the file. If it does, the extra effort expended in the communication
system to provide a guarantee of reliable data transmission is only
reducing the frequency of retries by the file transfer application; it
has no effect on inevitability or correctness of the outcome, since
correct file transmission is ensured by the end-to-end checksum and
retry whether or not the data transmission system is especially
reliable.

Thus, the argument: In order to achieve careful file transfer, the
application program that performs the transfer must supply a
file-transfer-specific, end-to-end reliability guarantee --- in this
case, a checksum to detect failures and a retry-commit plan. For the
data communication system to go out of its way to be extraordinarily
reliable does not reduce the burden on the application program to ensure
reliability.


\hypertarget{a-too-real-example}{%
\subsection{A Too-Real Example}\label{a-too-real-example}}

An interesting example of the pitfalls that one can encounter turned up
recently at the Massachusetts Institute of Technology. One network
system involving several local networks connected by gateways used a
packet checksum on each hop from one gateway to the next, on the
assumption that the primary threat to correct communication was
corruption of bits during transmission. Application programmers, aware
of this checksum, assumed that the network was providing reliable
transmission, without realizing that the transmitted data were
unprotected while stored in each gateway. One gateway computer
developed a transient error: while copying data from an input to an
output buffer a byte pair was interchanged, with a frequency of about
one such interchange in every million bytes passed. Over a period of
time many of the source files of an operating system were repeatedly
transferred through the defective gateway. Some of these source files
were corrupted by byte exchanges, and their owners were forced to the
ultimate end-to-end error check: manual comparison with and correction
from old listings.

\hypertarget{performance-aspects}{%
\subsection{Performance Aspects}\label{performance-aspects}}

However, it would be too simplistic to conclude that the lower levels
should play no part in obtaining reliability. Consider a network that is
somewhat unreliable, dropping one message of each hundred messages sent.
The simple strategy outlined above, transmitting the file and then
checking to see that the file has arrived correctly, would perform more
poorly as the length of the file increased. The probability that all
packets of a file arrive correctly decreases exponentially
with the file length, and thus the expected time to transmit the file
grows exponentially with file length. Clearly, some effort at the lower
levels to improve network reliability can have a significant effect on
application performance. But the key idea here is that the lower levels
need not provide ``perfect'' reliability.

Thus the amount of effort to put into reliability measures within the
data communication system is seen to be an engineering trade-off based
on performance, rather than a requirement for correctness. Note that
performance has several aspects here. If the communication system is too
unreliable, the file transfer application performance will suffer
because of frequent retries following failures of its end-to-end
checksum. If the communication system is beefed up with internal
reliability measures, those measures also have a performance cost, in
the form of bandwidth lost to redundant data and added delay from
waiting for internal consistency checks to complete before delivering
the data. There is little reason to push in this direction very far,
when it is considered that \emph{the end-to-end check of the file
transfer application must still be implemented no matter how reliable
the communication system becomes.} The \emph{proper} trade-off requires
careful thought. For example, one might start by designing the
communication system to provide only the reliability that comes with
little cost and engineering effort, and then evaluate the residual error
level to ensure that it is consistent with an acceptable retry frequency
at the file transfer level. It is probably not important to strive for a
negligble error rate at any point below the application level.


Using performance to justify placing functions in a low-level subsystem
must be done carefully. Sometimes, by examining the problem thoroughly,
the same or better performance enhancement can be achieved at the high
level. Performing a function at a low level may be more efficient, if
the function can be performed with a minimum perturbation of the
machinery already included in the low-level subsystem. But the opposite
situation can occur --- that is, performing the function at the lower
level may cost more --- for two reasons. First, since the lower level
subsystem is common to many applications, those applications that do not
need the function will pay for it anyway. Second, the low-level
subsystem may not have as much information as the higher levels, so it
cannot do the job as efficiently.

Frequently, the performance trade-off is quite complex. Consider again
the careful file transfer on an unreliable network. The usual technique
for increasing packet reliability is some sort of per-packet error check
with a retry protocol. This mechanism can be implemented either in the
communication subsystem or in the careful file transfer application. For
example, the receiver in the careful file transfer can periodically
compute the checksum of the portion of the file thus far received and
transmit this back to the sender. The sender can then restart by
retransmitting any portion that has arrived in error.

The end-to-end argument does not tell us where to put the early checks,
since either layer can do this performance-enhancement job. Placing the
early retry protocol in the file transfer application simplifies the
communication system but may increase overall cost, since the
communication system is shared by other applications and each
application must now provide its own reliability enhancement. Placing
the early retry protocol in the communication system may be more
efficient, since it may be performed inside the network on a hop-by-hop
basis, reducing the delay involved in correcting a failure. At the same
time there may be some application that finds the cost of the
enhancement is not worth the result, but it now has no choice in the
matter.\footnote{For example, real-time transmission of speech has tighter constraints on message delay than on bit-error rate. Most retry schemes significantly increase the variability of delay.} A
great deal of information about system implementation is needed to make
this choice intelligently.


\hypertarget{other-examples-of-the-end-to-end-argument}{%
\section{OTHER EXAMPLES OF THE END-TO-END
ARGUMENT}\label{other-examples-of-the-end-to-end-argument}}

\hypertarget{delivery-guarantees}{%
\subsection{Delivery Guarantees}\label{delivery-guarantees}}


The basic argument that a lower level subsystem that supports a
distributed application may be wasting its effort in providing a
function that must, by nature, be implemented at the application level
anyway can be applied to a variety of functions in addition to reliable
data transmission. Perhaps the oldest and most widely known form of the
argument concerns acknowledgment of delivery. A data communication
network can easily return an acknowledgment to the sender for every
message delivered to a recipient. The ARPANET , for example,
returns a packet known as \emph{Request For Next Message} (RFNM) {[}1{]}
whenever it delivers a message. Although this acknowledgment may be
useful within the network as a form of congestion control (originally
the ARPANET refused to accept another message to the same target
until the previous RFNM had returned), it was never found to be very
helpful for applications using the ARPANET . The reason is that
knowing for sure that the message was delivered to the target host is
not very important. What the application wants to know is whether or
not the target host acted on the message; all manner of disaster might
have struck after message delivery but before completion of the action
requested by the message. The acknowledgment that is really desired is
an end-to-end one, which can be originated only by the target
application --- ``I did it,'' or ``I didn't.''


Another strategy for obtaining immediate acknowledgments is to make the
target host sophisticated enough that when it accepts delivery of a
message it also accepts responsibility for guaranteeing that the message
is acted upon by the target application. This approach can eliminate the
need for an end-to-end acknowledgment in some, but not all,
applications. An end-to-end acknowledgment is still required for
applications in which the action requested of the target host should be
done only if similar actions requested of other hosts are successful.
This kind of application requires a two-phase commit protocol {[}5, 10,
15{]}, which is a sophisticated end-to-end acknowledgment. Also, if the
target application either fails or refuses to do the requested action,
and thus a negative acknowledgment is a possible outcome, an end-to-end
acknowledgment may still be a requirement.

\hypertarget{secure-transmission-of-data}{%
\subsection{Secure Transmission of
Data}\label{secure-transmission-of-data}}

Another area in which an end-to-end argument can be applied is that of
data encryption. The argument here is threefold. First, if the data
transmission system performs encryption and decryption, it must be
trusted to securely manage the required encryption keys. Second, the
data will be in the clear and thus vulnerable
as they pass into the target node and are fanned out to the target
application. Third, the \emph{authenticity} of the message must still be
checked by the application. If the application performs end-to-end
encryption, it obtains its required authentication check and can handle
key management to its satisfaction, and the data are never exposed
outside the application.

Thus, to satisfy the requirements of the application, there is no need
for the communication subsystem to provide for automatic encryption of
all traffic. Automatic encryption of all traffic by the communication
subsystem may be called for, however, to ensure something else --- that a misbehaving user or application program does not deliberately
transmit information that should not be exposed. The automatic
encryption of all data as they are put into the network is one more
firewall the system designer can use to ensure that information does not
escape outside the system. Note however, that this is a different
requirement from authenticating access rights of a system user to
specific parts of the data. This network-level encryption can be quite
unsophisticated --- the same key can be used by all hosts, with
frequent changes of the key. No per-user keys complicate the key
management problem. The use of encryption for application-level
authentication and protection is complementary. Neither mechanism can
satisfy both requirements completely.

\hypertarget{duplicate-message-suppression}{%
\subsection{Duplicate Message
Suppression}\label{duplicate-message-suppression}}

A more sophisticated argument can be applied to duplicate message
suppression. A property of some communication network designs is that
a message or a part of a message may be delivered twice, typically as
a result of time-out-triggered failure detection and retry mechanisms
operating within the network. The network can watch for and suppress
any such duplicate messages, or it can simply deliver them. One might
expect that an application would find it very troublesome to cope with
a network that may deliver the same message twice; indeed, it is
troublesome. Unfortunately, even if the network suppresses duplicates,
the application itself may accidentally originate duplicate requests
in its own failure/retry procedures. These application-level
duplications look like different messages to the communication
system, so it cannot suppress them; suppression must be accomplished
by the application itself with knowledge of how to detect
its own duplicates.

A common example of duplicate suppression that must be handled at a high
level is when a remote system user, puzzled by lack of response,
initiates a new login to a time-sharing system. Another example is that
most communication applications involve a provision for coping with a
system crash at one end of a multisite transaction: reestablish the
transaction when the crashed system comes up again. Unfortunately,
reliable detection of a system crash is problematical: the problem may
just be a lost or long-delayed acknowledgment. If so, the retried
request is now a duplicate, which only the application can discover.
Thus, the end-to-end argument again: If the application level has to
have a duplicate-suppressing mechanism anyway, that mechanism can also
suppress any duplicates generated inside the communication network; therefore, the
function can be omitted from that lower level. The same basic reasoning
applies to completely omitted messages, as well as to duplicated ones.


\hypertarget{guaranteeing-fifo-message-delivery}{%
\subsection{Guaranteeing FIFO Message
Delivery}\label{guaranteeing-fifo-message-delivery}}

Ensuring that messages arrive at the receiver in the same order in which
they are sent is another function usually assigned to the communication
subsystem. The mechanism usually used to achieve such first-in,
first-out (FIFO) behavior guarantees FIFO ordering among messages sent
on the same virtual circuit. Messages sent along independent virtual
circuits, or through intermediate processes outside the communication
subsystem, may arrive in a different order from the order sent. A
distributed application in which one node can originate requests that
initiate actions at several sites cannot take advantage of the FIFO
ordering property to guarantee that the actions requested occur in the
correct order. Instead, an independent mechanism at a higher level than
the communication subsystem must control the ordering of actions.

\hypertarget{transaction-management}{%
\subsection{Transaction Management}\label{transaction-management}}

We have now applied the end-to-end argument in the construction of the
SWALLOW distributed data storage system {[}15{]}, where it leads to
significant reduction in overhead. SWALLOW provides data storage servers
called repositories that can be used remotely to store and retrieve
data. Accessing data at a repository is done by sending it a message
specifying the object to be accessed, the version, and type of access
(read/write), plus a value to be written if the access is a write. The
underlying message communication system does not suppress duplicate
messages, since (a) the object identifier plus the version information
suffices to detect duplicate writes, and (b) the effect of a duplicate
read-request message is only to generate a duplicate response, which is
easily discarded by the originator. Consequently, the low-level message
communication protocol is significantly simplified.

The underlying message communication system does not provide delivery
acknowledgment either. The acknowledgment that the originator of a write
request needs is that the data were stored safely. This acknowledgment
can be provided only by high levels of the SWALLOW system. For read
requests, a delivery acknowledgment is redundant, since the response
containing the value read is sufficient acknowledgment. By eliminating
delivery acknowledgments, the number of messages transmitted is halved.
This message reduction can have a significant effect on both host load
and network load, improving performance. This same line of reasoning has
also been used in development of an experimental protocol for remote
access to disk records {[}6{]}. The resulting reduction in path length
in lower level protocols has been important in maintaining good
performance on remote disk access.

\hypertarget{identifying-the-ends}{%
\section{IDENTIFYING THE ENDS}\label{identifying-the-ends}}

Using the end-to-end argument sometimes requires subtlety of analysis of
application requirements. For example, consider a computer
communication network that carries some packet voice connections, that
is, conversations between digital telephone instruments. For those
connections that carry voice packets, an unusually strong version of
the end-to-end argument applies: If low levels of the communication
system try to accomplish bit-perfect communication, they will probably
introduce uncontrolled delays in packet delivery, for example, by requesting retransmission of damaged packets and holding up delivery of
later packets until earlier ones have been correctly retransmitted. Such
delays are disruptive to the voice application, which needs to feed data
at a constant rate to the listener. It is better to accept slightly
damaged packets as they are, or even to replace them with silence, a
duplicate of the previous packet, or a noise burst. The natural
redundancy of voice, together with the high-level error correction
procedure in which one participant says ``excuse me, someone dropped a
glass. Would you please say that again?'' will handle such dropouts, if
they are relatively infrequent.

However, this strong version of the end-to-end argument is a property of
the specific application --- two people in real-time conversation ---
rather than a property, say, of speech in general. If, instead, one
considers a speech message system, in which the voice packets are stored
in a file for later listening by the recipient, the arguments suddenly
change their nature. Short delays in delivery of packets to the storage
medium are not particularly disruptive, so there is no longer any
objection to low-level reliability measures that might introduce delay
in order to achieve reliability. More important, it is actually helpful
to this application to get as much accuracy as possible in the recorded
message, since the recipient, at the time of listening to the recording,
is not going to be able to ask the sender to repeat a sentence. On the
other hand, with a storage system acting as the receiving end of the
voice communication, an end-to-end argument does apply to packet
ordering and duplicate suppression. Thus the end-to-end argument is not
an absolute rule, but rather a guideline that helps in application and
protocol design analysis; one must use some care to identify the
endpoints to which the argument should be applied.

\hypertarget{history-and-application-to-other-system-areas}{%
\section{HISTORY, AND APPLICATION TO OTHER SYSTEM
AREAS}\label{history-and-application-to-other-system-areas}}


The individual examples of end-to-end arguments cited in this paper are
not original; they have accumulated over the years. The first example of
questionable intermediate delivery acknowledgments noticed by the
authors was the ``wait'' message of the Massachusetts Institute of
Technology Compatible Time-Sharing System, which the system printed on
the user's terminal whenever the user entered a command {[}3{]}. (The
message had some value in the early days of the system, when crashes and
communication failures were so frequent that intermediate
acknowledgments provided some needed reassurance that all was well.)


The end-to-end argument relating to encryption was first publicly
discussed by Branstad in a 1973 paper {[}2{]}; presumably the military
security community held classified discussions before that time. Diffie
and Hellman {[}4{]} and Kent {[}8{]} developed the arguments in more
depth, and Needham and Schroeder {[}11{]} devised improved protocols for
the purpose.


The two-phase-commit data update protocols of Gray {[}5{]}, Lampson and
Sturgis {[}10{]} and Reed {[}13{]} all use a form of end-to-end argument to justify
their existence; they are end-to-end protocols that do not depend for
correctness on reliability, FIFO sequencing, or duplicate suppression
within the communication system, since all of these problems may also
be introduced by other system component failures as well. Reed makes
this argument explicitly in the second chapter of his Ph.D.
dissertation on decentralized atomic actions {[}14{]}.

End-to-end arguments are often applied to error control and correctness
in application systems. For example, a banking system usually provides
high-level auditing procedures as a matter of policy and legal
requirement. Those high-level auditing procedures will uncover not only
high-level mistakes, such as performing a withdrawal against the wrong
account, but they will also detect low-level mistakes such as
coordination errors in the underlying data management system. Therefore,
a costly algorithm that absolutely eliminates such coordination errors
may be arguably less appropriate than a less costly algorithm that just
makes such errors very rare. In airline reservation systems, an agent
can be relied upon to keep trying through system crashes and delays
until a reservation is either confirmed or refused. Lower level recovery
procedures to guarantee that an unconfirmed request for a reservation
will survive a system crash are thus not vital. In telephone exchanges,
a failure that could cause a single call to be lost is considered not
worth providing explicit recovery for, since the caller will probably
replace the call if it matters {[}7{]}. All of these design approaches
are examples of the end-to-end argument being applied to automatic
recovery.

Much of the debate in the network protocol community over datagrams,
virtual circuits, and connectionless protocols is a debate about
end-to-end arguments. A modularity argument prizes a reliable, FIFO
sequenced, duplicate-suppressed stream of data as a system component
that is easy to build on, and that argument favors virtual circuits. The
end-to-end argument claims that centrally provided versions of each of
those functions will be incomplete for some applications, and those
applications will find it easier to build their own version of the
functions starting with datagrams.

A version of the end-to-end argument in a noncommunication application
was developed in the 1950s by system analysts whose responsibility
included reading and writing files on large numbers of magnetic tape
reels. Repeated attempts to define and implement a \emph{reliable tape
subsystem} repeatedly foundered, as flaky tape drives, undependable
system operators, and system crashes conspired against all narrowly
focused reliability measures. Eventually, it became standard practice
for every application to provide its own application-dependent checks
and recovery strategy, and to assume that lower level error detection
mechanisms, at best, reduced the frequency with which the higher level
checks failed. As an example, the Multics file backup system {[}17{]},
even though it is built on a foundation of magnetic tape subsystem
format that provides very powerful error detection and correction
features, provides its own error control in the form of record labels
and multiple copies of every file.


The arguments that are used in support of reduced instruction set
computer (RISC) architecture are similar to end-to-end arguments. The
RISC argument is that the client of the architecture will get better
performance by implementing exactly the instructions needed from
primitive tools; any attempt by the computer designer to anticipate the
client's requirements for an esoteric feature will probably miss the
target slightly and the client will end up reimplementing that feature
anyway. (We are indebted to M. Satyanarayanan for pointing out this
example.)

Lampson, in his arguments supporting the \emph{open operating system,}
{[}9{]} uses an argument similar to the end-to-end argument as a
justification. Lampson argues
against making any function a permanent fixture of lower level modules;
the function may be provided by a lower level module, but it should
always be replaceable by an application's special version of the
function. The reasoning is that for any function that can be thought of,
at least some applications will find that, of necessity, they must
implement the function themselves in order to meet correctly their own
requirements. This line of reasoning leads Lampson to propose an ``open''
system in which the entire operating system consists of replaceable
routines from a library. Such an approach has only recently become
feasible in the context of computers dedicated to a single application.
It may be the case that the large quantity of fixed supervisor functions
typical of large-scale operating systems is only an artifact of economic
pressures that have demanded multiplexing of expensive, hardware and
therefore a protected supervisor. Most recent system ``kernelization''
projects have, in fact, focused at least in part on getting function out
of low system levels {[}12, 16{]}. Though this function movement is
inspired by a different kind of correctness argument, it has the side
effect of producing an operating system that is more flexible for
applications, which is exactly the main thrust of the end-to-end
argument.


\hypertarget{conclusions}{%
\section{CONCLUSIONS}\label{conclusions}}

End-to-end arguments are a kind of ``Occam's razor'' when it comes to
choosing the functions to be provided in a communication subsystem.
Because the communication subsystem is frequently specified before
applications that use the subsystem are known, the designer may be
tempted to ``help'' the users by taking on more function than necessary.
Awareness of end-to-end arguments can help to reduce such temptations.

It is fashionable these days to talk about \emph{layered} communication
protocols, but without clearly defined criteria for assigning functions
to layers. Such layerings are desirable to enhance modularity.
End-to-end arguments may be viewed as part of a set of rational
principles for organizing such layered systems. We hope that our
discussion will help to add substance to arguments about the ``proper''
layering.

\hypertarget{acknowledgments}{%
\section*{ACKNOWLEDGMENTS}\label{acknowledgments}}

Many people have read and commented on an earlier draft of this paper,
including David Cheriton, F. B. Schneider, and Liba Svobodova. The
subject was also discussed at the ACM Workshop in Fundamentals of
Distributed Computing, in Fallbrook, Calif., December 1980. Those
comments and discussions were quite helpful in clarifying the arguments.

\hypertarget{references}{%
\section*{REFERENCES}\label{references}}

\begin{enumerate}
\def\labelenumi{\arabic{enumi}.}
\item

  BOLT BERANEK AND NEWMAN INC. Specifications for the interconnection of a
  host and an IMP. Tech. Rep. 1822. Bolt Beranek and Newman Inc.
  Cambridge,Mass. Dec. 1981.

\item

  BRANSTAD, D.K. Security aspects of computer networks. AAIA Paper 73-427,
  AIAA Computer Network Systems Conference, Huntsville, Ala. Apr. 1973.

\item

  CORBATO, F.J., DAGGETT, M.M., DALEY, R.C., CREASY, R.J., HELLIWIG,
  J.D., ORENSTEIN, R.H., AND KORN, L.K. \emph{The Compatible Time-Sharing System, A Programmer's
Guide.} Massachusetts
Institute of Technology Press, Cambridge, Mass. 1963, p. 10.

\item

  DIFFIE, W., AND HELLMAN, M.E. New directions in cryptography.
  \emph{IEEE Trans. Inf. Theory} \emph{IT-22,} 6 (Nov. 1976), 644-654.

\item

  GRAY, J.N. \emph{Notes on database operating systems.} Operating
  Systems: An Advanced Course. Lecture Notes on Computer Science, vol.
  60. Springer-Verlag, New York. 1978. 393-481.

\item

  GREENWALD, M. Remote virtual disk protocol specifications. Tech. Memo.
  Massachusetts Institute of Technology Laboratory for Computer
  Science, Cambridge, Mass. In preparation.

\item

  KEISTER,W., KETCHLEDGE,R.W., AND VAUGHAN,H.E. No. 1 ESS: System
  organization and objectives. \emph{Bell Syst. Tech. J. 53,} 5, Pt 1,
  (Sept. 1964), 1841.

\item

  KENT, S.W. Encryption-based protection protocols for interactive
  user-computer communication. S.M. thesis, Dept. of Electrical
  Engineering and Computer Science, Massachusetts Institute of
  Technology, Cambridge, Mass., May 1976. Also available as Tech. Rep.
  TR-162. Massachusetts Institute of Technology Laboratory for Computer
  Science, May 1976.

\item

  LAMPSON,B.W., AND SPROULL, R.F. An open operating system for a
  single-user machine. In \emph{Proceedings of the 7th Symposium on
  Operating Systems Principles, (Pacific Grove, Calif. Dec. 10-12).}
  ACM, New York, 1979, pp. 98-105.

\item

  LAMPSON,S., AND STURGIS,n . Crash recovery in a distributed data
  storage system. Working paper, Xerox PARC, Palo Alto, Calif. Nov. 1976
  and Apr. 1979. Submitted for publication.

\item

  NEEDHAM,R.M., ANDSCHROEDER,M.D. Using encryption for authentication in
  large networks of computers. \emph{Commun. ACM 21,} 12 (Dec. 1978),
  993-999.

\item

  POPEK, G.J., et al. UCLA data secure unix. In \emph{Proceedings of the
  1979 National Computer} \emph{Conference,} vol. AFIPS Press, Reston,
  Va., pp. 355-364.

\item

  REED, D.P. Implementing atomic actions on decentralized data.
  \emph{ACM Trans. Comput. Syst. 1,} 1 (Feb. 1983), 3-23.

\item
REED, D.P. Naming and synchronization in a decentralized computer
system. Ph.D. dissertation, Massachusetts Institute of Technology,
Dept. of Electrical Engineering and Computer Science, Cambridge, Mass.
September 1978. Also available as Massachusetts Institute of Technology
Laboratory for Computer Science Tech. Rep. TR-205, Sept., 1978.


\item

  REED, D.P., AND SVOBODOVA,L. SWALLOW. A distributed data storage
  system for a local network. A. West, and P. Janson, Eds. In
  \emph{Local Networks for Computer Communications,} \emph{Proceedings
  of the IFIP Working Group 6.4 International Workshop on Local
  Networks} (Zurich, Aug 27-29 1980), North-Holland, Amsterdam, 1981,
  pp. 355-373.

\item

  SCHROEDER, M.D., CLARK, D.D., AND SALTZER, J.H. The multics kernel
  design project. In Proceedings 6th Symposium on Operating Systems
  Principles. \emph{Oper. Syst. Rev. 11,} 5 (Nov. 1977), 43-56.

\item

  STERN, J.A. Backup and recovery of on-line information in a computer
  utility. S.M. thesis, Department of Electrical Engineering and
  Computer Science, Massachusetts Institute of Technology, Cambridge,
  Mass. Aug. 1973. Available as Project MAC Tech. Rep. TR-116,
  Massachusetts Institute of Technology, Jan. 1974.

\end{enumerate}

\end{document}
