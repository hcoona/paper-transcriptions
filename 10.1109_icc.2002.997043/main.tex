\PassOptionsToPackage{unicode=true}{hyperref} % options for packages loaded elsewhere
\PassOptionsToPackage{hyphens}{url}
%
\documentclass[a4paper,11pt,notitlepage,twoside,openright]{article}

\usepackage{ifxetex}
\ifxetex{}
\else
  \errmessage{Must be built with xelatex}
\fi

\usepackage{amssymb,amsmath}
\usepackage{fourier}
\usepackage{inconsolata}
\usepackage{enumitem}
\usepackage{footnote}

% Table
\usepackage{tabu}
\usepackage{longtable}
\usepackage{booktabs}
\usepackage{multirow}

% Verbatim & Source code
\usepackage{fancyvrb}
\usepackage{minted}

% Beauty
\usepackage[protrusion]{microtype}
\usepackage[all]{nowidow}
\usepackage{upquote}
\usepackage{parskip}
\usepackage[strict]{changepage}

\usepackage{hyperref}

% Graph
\usepackage{graphicx}
\usepackage{grffile}
\usepackage{tikz}


\hypersetup{
  bookmarksnumbered,
  pdfborder={0 0 0},
  pdfpagemode=UseNone,
  pdfstartview=FitH,
  breaklinks=true}
\urlstyle{same}  % don't use monospace font for urls

\usetikzlibrary{arrows.meta,shapes.geometric,shapes.misc}

\newminted{java}{%
  autogobble,
  breakbytokenanywhere,
  breaklines,
  fontsize=\footnotesize,
}
\newmintinline{java}{%
  autogobble,
  breakbytokenanywhere,
  breaklines,
  fontsize=\footnotesize,
}

\makeatletter
\def\maxwidth{\ifdim\Gin@nat@width>\linewidth\linewidth\else\Gin@nat@width\fi}
\def\maxheight{\ifdim\Gin@nat@height>\textheight\textheight\else\Gin@nat@height\fi}
\makeatother

% Scale images if necessary, so that they will not overflow the page
% margins by default, and it is still possible to overwrite the defaults
% using explicit options in \includegraphics[width, height, ...]{}
\setkeys{Gin}{width=\maxwidth,height=\maxheight,keepaspectratio}
\setlength{\emergencystretch}{3em}  % prevent overfull lines
\setcounter{secnumdepth}{3}

% Redefines (sub)paragraphs to behave more like sections
\ifx\paragraph\undefined\else
\let\oldparagraph\paragraph
\renewcommand{\paragraph}[1]{\oldparagraph{#1}\mbox{}}
\fi
\ifx\subparagraph\undefined\else
\let\oldsubparagraph\subparagraph
\renewcommand{\subparagraph}[1]{\oldsubparagraph{#1}\mbox{}}
\fi

% set default figure placement to htbp
\makeatletter
\def\fps@figure{htbp}
\makeatother

\def\sectionautorefname{Section}
\def\subsectionautorefname{Subsection}

\title{A critical review of ``End-to-end arguments in system design''}
\author{Tim Moors}
\date{}

\begin{document}
\maketitle

\begin{abstract}
The end-to-end arguments raised by Saltzer,
Reed and Clark in the early 1980s are amongst the most influential of
all communication protocol design guides. However, they have recently
been challenged by the advent of firewalls, caches, active networks,
NAT, multicasting and network QOS. This paper reviews the end-to-end
arguments, highlighting their subtleties, and provides additional
arguments for and against end-to-end implementations. It shows the
importance of trust as a criterion for deciding whether to implement a
function locally or end-to-end, and how end-to-end implementations can
help robustness, scalability, ease of deployment, and the provision of
appropriate service. It focuses on the performance implications of
end-to-end or localized functionality, and argues against end-to-end
congestion control of the form used by TCP.
\end{abstract}

\hypertarget{introduction}{%
\section{Introduction}}

The paper ``End-to-end arguments in system design'' {[}1{]} (henceforth
called ``The Paper'') has had a profound impact since it was published
in 1984. For example, the literature contains numerous comments saying
that the end-to-end arguments are ``one of the most widely applied rules
of system design'' {[}2{]} and one of few general architectural
principles of the Internet {[}3{]}. Yet the very success of The Paper
has often led to people accepting the principle dogmatically, or
applying it without considering the attendant subtleties. Furthermore,
new networking products and architectures such as firewalls, caches and
Network Address Translators, active networks, multicasting and network
Quality Of Service all challenge the end-to-end arguments. At the same
time, the recent success of peer-to-peer networking exemplifies the
benefits of end-to-end implementations.

This paper reviews The Paper, and highlights new arguments for (and
against) end-to-end implementations that have become pronounced in the
20 years since The Paper was published. The authors of The Paper have
themselves revisited the original principle in a modern context,
evaluating active networking in terms of end-to-end arguments {[}4{]},
one author (Reed) has written on how end-to-end arguments remain
pertinent today {[}5{]}, and another (Clark) has written about their
role in the context of the changing requirements of the Internet
{[}6{]}.

This paper first describes the primary end-to-end argument and the
``careful file transfer'' case study, as presented in The Paper. It then
considers the performance implications of end-to-end implementations,
and describes additional end-to-end arguments. Finally, it considers how
responsibility and trust affect the applicability of the end-to-end
arguments, and evaluates the suitability of end-to-end implementations
of error control, security, routing and congestion control.

\hypertarget{the-end-to-end-arguments}{%
\section{The end-to-end arguments}\label{the-end-to-end-arguments}}

The ``end-to-end arguments'' guide the placement of functions in a
communication network. Certain functions can be implemented in both end
systems and the network, e.g. error control, security, and routing. The
end-to-end arguments suggest that those functions should be implemented
at the endpoints. While The Paper presents multiple end-to-end
arguments, it emphasizes one relating to correctness of function.
``{[}T{]}he end-to-end argument'' states that certain functions ``can
completely and correctly be implemented only with the knowledge and help
of the application standing at the endpoints of the communication
system. Therefore, providing {[}such{]} function{[}s completely{]} as a
feature of the communication system itself is not possible. (Sometimes
an incomplete version of the function provided by the communication
system may be useful as a performance enhancement.)'' {[}1, p. 278{]}.
(We will return to the other end-to-end arguments in \autoref{additional-end-to-end-arguments}.)

\hypertarget{an-example}{%
\subsection{An example}}

The Paper includes a case study of ``careful file transfer'' as an
example of the application of end-to-end argument. As illustrated in
\autoref{fig1}, the ``careful file transfer'' involves transferring a file from
the disk on a source computer to the disk on a destination computer. The
file may become corrupted at various points on the end-to-end path, e.g.
on the communication channel ($\alpha$), in intermediaries such as the router
($\beta$), or during disk access ($\chi$). For examples of the causes of errors,
the reader should refer to an error-control text such as Lin and
Costello {[}7{]} for discussion of link errors, and to Stone and
Partridge {[}8{]} for a discussion of router and end-system errors. The
authors argue that only a check made at the endpoints (i.e. from
information stored on the disks) can ``completely and correctly'' ensure
that no error has been introduced.

\begin{figure}
  \centering
  \includegraphics{fig1.png}
  \caption{Errors can occur at different points ($\alpha$, $\beta$, and $\chi$) on the
end-to-end path.\label{fig1}}
\end{figure}

\hypertarget{defining-correctness}{%
\subsection{Defining correctness}}

It is important to scrutinize what is meant by ``complete and correct''
implementation of a function. The service of secrecy can be implemented
completely and correctly by the applications in \autoref{fig1} using
one-time-pads, in that the destination application can decrypt the file
being transferred, but at no time do entities at other points in the
network (e.g. $\alpha$, $\beta$, and $\chi$) have sufficient information to decrypt the
file. On the other hand, there is no way to provide similar absolute
guarantees for integrity: To enhance integrity, the source adds to the
payload certain information (e.g. a CRC) that produces redundancy within
the transmitted information. The destination then checks that the
redundancy remains in the received information. It is always possible,
albeit perhaps improbable, that a modification can change the
transmitted information to a new value that contains the redundancy of
the form expected by the destination. Thus, we argue that while it makes
sense to discuss integrity being provided over the \emph{complete path}
that the information traverses, it is not possible to completely ensure
integrity due to the probabilistic nature of integrity checks.

An extreme form of lack of correctness is total failure. The Internet
architecture was influenced by military desire for robust operation when
network elements become unavailable {[}9{]}, which led to displacing
functions that relied on state information for proper operation to the
end systems. This is based on the ``fate-sharing model {[}that{]}
suggests that it is acceptable to lose the state information associated
with an entity if, at the same time, the entity itself is lost''
{[}9{]}. In a sense, fate can be viewed as being an extreme form of
correctness: An end that does not trust the fate of an intermediate
system may also not trust the ability of the intermediate system to
perform other functions correctly.

\hypertarget{carefully-identifying-the-ends}{%
\subsection{Carefully identifying the ends
}\label{carefully-identifying-the-ends}}

To apply the end-to-end arguments, it is not surprising that one must
identify the communication endpoints. The ends are the points where the
information is \emph{originally} generated or \emph{ultimately}
consumed. The Paper points out that in interactive speech, the ends are
not the telephones, but rather the people. This highlights an important,
but subtle point: that for payload information the end is usually the
uppermost entity in the system, such as a human user.

Returning to the careful file transfer example, The Paper suggests that
file transfer applications are the endpoints, and that after the
destination has stored the file to disk, it should read the file off
disk and calculate an integrity check, which it returns to the source
application, which decides whether the transfer was ``correct''. This
use of application-layer checks is also supported by a recent study of
the sources of data transfer errors {[}8{]}. However, it is a popular
belief (e.g. {[}2, p. 289{]}{[}10{]}) that the \emph{transport} layer
(e.g. TCP) is the endpoint for applications such as file transfers. The
transport layer merely resides in the end-\emph{system}. The transport
layer checks will not protect against errors introduced as the
information is written to disk. (Note that while natural noise and
potential design errors are omnipresent, physical security measures
(e.g. electromagnetic shielding) can isolate a system from security
threats. Thus, if the user trusts the end system then the transport
layer of the end system may constitute a \emph{security} endpoint.)
\emph{According to the end-to-end arguments, applications, not transport
layers, should check integrity}.

So why do most applications entrust transport layers such as TCP to
provide reliable transfer? There are two reasons for this: the first
reason is the high \emph{cost} of the application performing the error
checking, and the second reason is the low \emph{benefit} from this
activity.

The first reason is that application writers do not wish to be burdened
with reliable transfer considerations. Reliable transfer protocols are
complicated, and the endpoint may not have the \emph{capacity} to
implement the service (e.g. dumb terminals such as telephones), although
progress in semiconductor technology outlined by Moore's Law is
circumventing this. Alternatively, the endpoint designer may not
\emph{understand} how to implement the complicated function correctly,
although this can be avoided with the Protocol Organs technique of
making protocol functions available through a library {[}11{]}. Reliable
transfer protocols also involve significant overhead in order to check
the information that has been delivered {[}12{]}, e.g. disk access at
the destination of the careful file transfer example is doubled by the
requirement of reading the file off the disk to check it. Furthermore,
if the ultimate endpoint is a human, the system designers generally do
not want to make the human user solely responsible for the menial task
of checking integrity. Finally, for the application to check for errors
within the computer, it must also account for the possibility of itself,
and not the data, being in error. Such totally self-checking systems are
non-trivial {[}13{]}.

The second reason is that for \emph{most} applications, the error rate
within the end-system is negligible. For example, when applications
store or move information within the computing system, they generally
assume that the system will not alter the information. Low-level
hardware is often designed with this assumption in mind, e.g. providing
error control to protect memory from soft DRAM errors. It is only when
the information has an appreciable chance of being corrupted, e.g. when
traversing an unreliable network, that the application seeks integrity
checks. The crux of the issue is how does the application know whether
its information will experience appreciable errors? It is relatively
easy for the application to know about the reliability of its local
system, but more difficult for it to know about the reliability of
intermediaries that its data must pass through as it crosses a network.
Thus, \emph{the decision to implement reliable transfer in the transport
layer is not justified on the basis of end-to-end arguments, but rather
on the basis of trust}. We will return to this issue of trust in \autoref{responsibility-and-trust}.

It is important that the application be able to disable integrity
checking by the transport layer (and this is not possible with the most
popular reliable transport protocol, TCP). This is because while
\emph{most} applications can neglect the chance of errors in the local
system, \emph{some} applications will be concerned about errors in the
local system (e.g. when writing to disk), and may implement their own,
truly end-to-end checks. By the end-to-end argument, such end-to-end
checks render lower-layer checks redundant and useful (or detrimental)
only in terms of performance. Also, interactive applications implicitly
provide positive acknowledgements (though not error recovery) of receipt
of requests by sending the corresponding reply. Again, for such
applications, transport layer mechanisms for providing reliable transfer
may be redundant and only impede performance.

\hypertarget{performance}{%
\subsection{Performance}\label{performance}}

The Paper notes that local implementations of functions may enhance
performance above that achievable using end-to-end implementations
alone. In this section, we first consider how local implementations may
enhance performance, and then consider how they may degrade performance.

\hypertarget{performance-benefits-of-localized-implementations}{%
\subsubsection{Performance benefits of localized implementations
}\label{performance-benefits-of-localized-implementations}}

We will first consider the function of error control, and then extend to
other functions such as multicasting and QOS.

Localized error detection reduces the network load by reducing the
distance that erroneous packets will propagate through the network.
Localized error recovery also reduces the network load by reducing the
distance that retransmitted packets must propagate. Localized error
recovery can also reduce the delay in delivering the packet with
integrity, because of these reduced distances that the erroneous packet
and retransmission must propagate, and because the propagation delay (as
used to dimension retransmission timers) across a hop is likely to be
smaller and less variable than the end-to-end propagation delay.
(Smaller propagation delays also benefit localized flow control and
congestion control.)

If routing is performed locally within the network, as is common, then
terminals may not know which links their traffic will traverse. The
efficiency of retransmission-based error control depends on the bit
error rate of the link, and can be improved by choosing a segment size
that is appropriate for the bit error rate (e.g. small segments when the
BER is high). However, this requires segmentation within the network,
and separate error control within the network. Thus, again, localized
error control can offer improved efficiency.

Multicasting can be implemented end-to-end (e.g. by a source sending
separate copies of the information to each destination), or locally
(e.g. by the network copying the information as it branches towards the
destinations), or a combination of both. Localized multicast branching
can reduce the transmission load on the source, and the network near the
source. Similarly, if nodes in the multicast tree leading from the
source root to destination leaves can aggregate flow control or
acknowledgement information from the destinations, or even retransmit
information as needed, then they can reduce the load on and near the
source. Caching provides similar benefits to localized multicasting in
that both techniques expand information at points within the network,
reducing the distance that it must travel. Content Distribution Networks
(CDNs), most notably that run by Akamai, also improve performance by the
source pushing information into caches within the network, rather than
delivering information end-to-end.

All of the performance improvements described above result from
localized implementations improving the \emph{efficiency} with which
communication resources are used. This distinction between efficiency
and performance is important when we consider the implementation of
``Quality of Service'', in particular the assurance of delay
requirements. Traffic tends to experience appreciable delay through a
network when it passes through points of congestion, where traffic is
buffered until it can be forwarded (rather than being discarded). The
localized technique for ensuring delay requirements (exemplified by ATM
and the Integrated Services protocols of the Internet, e.g. RSVP) is for
endpoints to reserve resources within the network before communication,
so nodes within the network know which traffic should be served first to
meet its delay requirements, and which traffic can be delayed. This
technique is necessary when the network can be overloaded by
delay-constrained traffic, since excess traffic of this type must be
``blocked''. The end-to-end technique for ensuring delay requirements
(exemplified by the Differentiated Services protocols of the Internet)
is for endpoints to label their traffic to indicate its delay
requirements, and for nodes within the network to use this information
to decide the service order. This technique cannot prevent
delay-constrained traffic from congesting the network, but when this
traffic occupies only a fraction of the network capacity (as is common
in modern fixed infrastructure networks which are dominated by data
rather than voice and video traffic), then it can assure the
delay-constrained traffic priority over the other traffic. That is, the
localized approach is suitable when efficiency is needed because the
load of delay-constrained traffic approaches the network capacity,
whereas the end-to-end approach is suitable when this is not the case.

If communicating endpoints are not available simultaneously, then
communication may progress faster if the endpoints can communicate with
an intermediary, rather than directly communicate end-to-end. For
example, sending email allows asynchronous communication, avoiding the
problems of telephone tag. Generally, a client sends email to their
server using a reliable transfer protocol such as TCP, their server then
forwards the email to the destination's server (again using TCP), and
the destination client eventually transfers the mail from their server
(again using TCP) in order to read it. The servers may still drop the
mail, so an end-to-end acknowledgement (e.g. between humans) may still
be needed, but the localized error control between servers allows the
mail to be transferred rapidly even when the endpoints are not available
simultaneously.

The final performance benefit of localized implementations stems from
sharing and economies of scale. An extreme example of this is the
function of multiplexing and switching: A network could be constructed
as a broadcast medium, and with only the endpoints involved in
multiplexing. Some early all-optical networks were based on such a
broadcast-and-select architecture. However, such a network would be
inefficient since unicast traffic would be distributed to many endpoints
that had no need to receive it. It is more efficient (in terms of
communication capacity) for intermediaries (routers) to participate in
the switching and route traffic only to the endpoints that need it.

The economies of scale from sharing also extend into end-systems. For
example, multiple applications (endpoints) sharing a security function
will not need to duplicate functions such as key management and random
number generation. The Paper gives another example: An end-to-end
implementation of reliable transfer ``may increase overall cost, since
\ldots{} each application must now provide its own reliability
enhancement'' {[}1, p. 281{]}. The end-to-end argument against sharing
is that it is fairer if the user pays (\autoref{additional-end-to-end-arguments}).

\hypertarget{performance-benefits-of-end-to-end-implementations}{%
\subsubsection{Performance benefits of end-to-end implementations
}\label{performance-benefits-of-end-to-end-implementations}}

Having considered the performance benefits of localized implementations,
we now consider the performance benefits of end-to-end implementations.

The principle performance benefit of end-to-end implementations is that
such implementations tend to reduce the amount of processing required in
the network, allowing the network to operate at higher speed when
processing is the bottleneck (as is currently common with optical
transmission technology). While the end-to-end arguments do not
\emph{require} that the network offer limited functionality and be
simple {[}14{]}, simple or ``stupid'' networks {[}15{]} are often a
consequence of applying end-to-end arguments. Simple networks are more
\emph{scalable} than complicated networks (since there is less
processing to extend as the network expands), and this is an important
contributor to the recent success of peer-to-peer networking (e.g.
Gnutella and Napster). However, it is important to note that some
techniques for improving scalability such as NAT (for address scaling),
and caching require \emph{more} processing within the network, and are
contrary to the end-to-end arguments.

A second performance benefit of simple networks (which follow end-to-end
arguments) are that they are easier to design and change, and this short
design turnaround time allows them to track improvements in
implementation technologies. The network will also not duplicate a
function (and the costs of implementing that function) that is
implemented end-to-end for reasons of correctness, or because the
endpoint design was unaware of the function being available in the
network (e.g. because the endpoint was designed to be portable, or
because the network description was too complicated).

Finally, end-to-end functions need only be encountered once (at the
endpoints), whereas localized functions may be encountered multiple
times, e.g. once for each hop that the traffic takes through the
network. This repeated processing can also degrade performance. For
example, when bridges operate in a store-and-forward mode, only
forwarding packets that were received without errors, then each packet
will be delayed by at least its transmission time in each bridge. If
instead, the bridges allow erroneous packets to pass, then they can
start forwarding them as soon as they enter the bridge, and the decision
to discard (and the consequent delay) will only be made at the
destination endpoint. Several bridges offer adaptive forwarding to merge
the benefits of end-to-end and local implementation: They check the
integrity of incoming frames, and when frames have a low error rate,
they forward frames directly, in a cut-through manner, without
buffering, leading to low delay, whereas if the error rate increases,
they store-and-forward the frames, preventing erroneous frames from
propagating.

Given that end-to-end implementations can both benefit and hinder
performance, it is not possible to generalize the performance
implications of end-to-end implementations.

\hypertarget{additional-end-to-end-arguments}{%
\subsection{Additional end-to-end arguments
}\label{additional-end-to-end-arguments}}

While The Paper often refers to the singular ``end-to-end argument''
concerning correctness of function, and this is the most famous
end-to-end argument, there are also other end-to-end arguments. This
section reviews these arguments, relating to appropriate service,
network transparency, ease of deployment, and decentralism.

A corollary of the correctness argument is that if a function is
implemented end-to-end for reasons of correctness, then any local
implementation may be redundant. A second end-to-end argument is that
local implementations may be redundant because certain applications
never need the function to be implemented, \emph{anywhere}. This
end-to-end argument states that a function or service should be carried
out within a network layer only if all clients of that layer need it.
(The authors of The Paper include this aspect in their definition of
``\emph{the} end-to-end principle'' (italics added) in {[}4{]}.)
Generally, the end-system has better knowledge than the network of what
type of service it needs, and has more information to provide the
service. For example, this argues against the inclusion of error control
(either error protection, detection or correction) within the network,
since applications such as uncompressed voice do not need high
integrity, and are sensitive to the delays that error recovery can
introduce. This also argues that endpoints should implement compression
since they know the type of information being transferred, and so can
apply an appropriate compression technique.

Another expression of this end-to-end argument is that end-to-end
implementations lead to a ``user pays'' system. In contrast, functions
provided in the network add a cost that is borne by all users,
irrespective of whether they use the function. A counterargument is that
the total price paid by all users may be lower than the user-pays
system, since the implementations can be shared as described in \autoref{performance}.

A corollary of this argument is that end-to-end implementations lead to
a network that is more \emph{flexible}, since it does not incorporate
features that are required by specific applications. For example,
loading coils were introduced into telephony cabling to improve the
quality of voice calls, but now interfere with the provision of data
services over that cabling {[}16{]}. Features designed into the network
leave as a legacy not just themselves, but also other applications that
come to be designed to assume that the feature will exist in the network
{[}15{]}, and this installed base of applications creates inertia that
makes it difficult to change the network. As the authors of The Paper
later wrote {[}4{]}, the key to network flexibility ``is the idea that a
lower layer of a system should support the widest possible variety of
services and functions, so as to permit applications that cannot be
anticipated.'' The difficulty arises in determining \emph{which} network
functions support the widest variety of services and functions when the
future services and functions are unknown. The end-to-end arguments lead
to a minimalist approach, making the network flexible by virtue of the
fact that it doesn't contain any functionality that might interfere with
new services. However, new services, such as active networking, might
need new functionality within the network. When the authors of The Paper
revisited the end-to-end arguments in the context of active networks
{[}4{]}, they pointed out the tradeoff between flexibility, and another
end-to-end argument: that end-to-end implementations allow the network
to be simpler, and more transparent, and hence easier to describe,
model, and predict. This is particularly important when networks scale,
as the number and complexity of potential interactions between endpoints
rises.

Allowing endpoints to implement services that are appropriate to their
needs (e.g. web browser plug-ins) also aids the deployment of new
services, since it can proceed without the protracted process of
changing the network. Consequently, services implemented at endpoints in
the Internet (e.g. streaming media) have progressed much faster than
both those implemented in the core of the Internet (e.g. the Multicast
Backbone) and new services introduced by the telephone industry.
Endpoint implementations allow a service to be trialed by a minority of
people who like to be on the cutting edge of technology, or have
particular need for the service, and then for the service to spread from
this installed base. In contrast, in order for a service to be installed
within the network, it needs to already have an appreciable customer
base; something that hard to produce because of network externalities
(also known as Metcalfe's Law) {[}17{]}. Implementing services within
the network relies on the foresight of the network provider, whereas the
fate of services implemented at endpoints is more market driven.

Finally, the end-to-end arguments are popular with advocates of
decentralism (e.g. see {[}18{]}), since both views question the ability
of a central control to predict what will be important in the future.

\hypertarget{responsibility-and-trust}{%
\section{Responsibility and trust}\label{responsibility-and-trust}}

In this section, we consider how responsibility and trust influence the
end-to-end arguments. We start by considering the function of reliable
transfer, and then extend to the other functions of security, routing,
and congestion control.

Section II.B explained why integrity checks can never be complete, and \S
II.C discussed how applications are often satisfied with transport layer
error control, even though it is not truly end-to-end, because they
trust the end-system. Trust has been defined as ``the firm belief in the
competence of an entity to act dependably, securely, and reliably within
a specified context'' {[}19{]}. In the case of integrity checks, the
endpoint has an interest in the application of the service, and so takes
responsibility for ensuring that the service meets its requirements,
supplementing the network's service if necessary. If the endpoint could
trust the service of the network, then it would not need to supplement
the network's service. Trust of the network is not a characteristic of
current Internet protocols, since they originated in a military
environment {[}9{]} and grew in a research and development environment,
where network components often fail, are compromised, or perform
erratically. However, the Internet is now becoming a mature operational
commercial environment, and this need for end-to-end checks may be
weakened. Furthermore, while end-to-end checks may help detect that
\emph{a} component is misbehaving, they often lack the selectivity of
localized checks needed to determine \emph{which} component is
misbehaving and should be replaced.

Security is another service where the endpoint may be interested in the
service, and so may implement that service. However, there are also
cases in which the endpoint is not \emph{responsible} for ensuring
security. For example, military systems often seek the performance
offered by the technological advances in ``Commercial Off The Shelf''
end-systems, but also need assurances that these COTS systems will not
divulge secret information. A common approach is to implement encryption
at the link level (e.g. see {[}20{]}) so that the endpoints can use COTS
technology that need not be proven to be trustworthy, while still
maintaining security. Thus, the value of end-to-end implementation of
security depends on which entity is \emph{responsible} for security.

Routing is another function that may be implemented either at the
endpoints (``source routing'') or within the network (``transparent
routing''). It is instructive to consider why source routing (which the
authors of The Paper supported {[}21{]}) has now generally fallen from
favor in preference for transparent routing. The main reason for route
computation within the network is that we now consider the network to be
\emph{responsible} for routing information to its destination(s). A
second reason is that routing within the network may be more responsive
to link failures or congestion that is localized within the network, and
may be better able to provide a hierarchy to handle the complexity of
the Internet.

The final function that we will consider is congestion control.
Congestion can occur in networks when the offered load exceeds the
capacity of the network. Congestion control involves predicting or
detecting congestion, and responding by reducing the offered load. In
today's Internet, congestion control is primarily implemented in
end-systems: Most traffic is carried by TCP, which employs a Slow Start
algorithm {[}22{]} to try to avoid congestion, uses the rate of
acknowledgement return to estimate the permissible transmission rate,
and interprets packet loss as indicating congestion that requires that
the source throttle its transmissions. The only network support is some
Random Early Discard devices that reinforce TCP's behavior by signaling
the onset of congestion by discarding packets. However, congestion
control is not amenable to end-to-end implementation for the following
reasons: First, like routing, congestion is a phenomenon of the network,
and since multiple endpoints share the network, it is \emph{the network}
that is responsible for isolating endpoints that offer excessive traffic
so that they do not interfere with the ability of \emph{the network} to
provide its service to other endpoints. Second, it is naive in today's
commercial Internet to expect endpoints to act altruistically,
sacrificing the performance that they receive from the network in order
to help the network limit congestion. The end-to-end arguments that
enable the success of peer-to-peer applications also allow the rapid
proliferation of applications that do not behave in a ``TCP friendly''
manner. It is cavalier to allow the commercially valuable Internet to be
susceptible to such risks. The requirement that the transport layer
implement congestion control also prevents the use of active networking
to make transport layers configurable {[}23{]}. Summarizing these first
two reasons: even though the network is \emph{responsible} for
controlling congestion, it has no reason to \emph{trust} that endpoints
will cooperate in controlling congestion.

A third argument against endpoint implementation of congestion control
is that it is inappropriate for certain networks, leading to an
unnecessary performance penalty. For example, Slow Start unnecessarily
impedes sources that are transmitting on optical circuits (which don't
congest), Media Access Control protocols already provide congestion
control for traffic that is local to a LAN, and the assumption that
packet loss indicates congestion is invalid for wireless networks in
which appreciable loss may also occur due to noise. Fourth, the
transport layer lacks the innate ability to detect that congestion is
imminent; it can only detect the possible presence of congestion, e.g.
through observing packet loss. Schemes such as RED may signal imminent
congestion, but they do so by unnecessarily discarding traffic for which
the network has already spent resources partially delivering. Fifth,
endpoints that implement congestion control separately must
independently re-learn the network state, leading to excessively
cautious behavior. Finally, while the endpoint may know \emph{how} it
would like to adapt to congestion, it is the network that knows when and
where adaptation is needed {[}24{]}, and should be responsible for
ensuring that adaptation occurs.

Thus, congestion control is one function that is not well suited to
end-to-end implementation.

\hypertarget{conclusion}{%
\section{Conclusion}}

The end-to-end arguments are a valuable guide for placing functionality
in a communication system. End-to-end implementations are supported by
the need for correctness of implementation, their ability to ensure
appropriate service, and to facilitate network transparency, ease of
deployment, and decentralism. Care must be taken in identifying the
endpoints, and end-to-end implementations can have a mixed impact on
performance and scalability. To determine if the end-to-end arguments
are applicable to a certain service, it is important to consider what
entity is \emph{responsible} for ensuring that service, and the extent
to which that entity can \emph{trust} other entities to maintain that
service. The end-to-end arguments are insufficiently compelling to
outweigh other criteria for certain functions such as routing and
congestion control. So we must conclude by quoting The Paper: ``A great
deal of information about system implementation is needed to make this
choice {[}of end-to-end or local implementation{]} intelligently'' {[}1,
p. 282{]}.

\hypertarget{references}{%
\section*{References}\label{references}}

\begin{enumerate}[label={[}\arabic*{]}]
\item
  J. Saltzer, D. Reed and D. Clark: ``End-to-end arguments in system
  design'', \emph{ACM Trans. Comp. Sys.}, 2(4):277-88, Nov. 1984
\item
  L. Peterson and B. Davie: \emph{Computer Networks: A Systems
  Approach}. Morgan Kaufmann, 1996
\item
  B. Carpenter: ``Architectural principles of the Internet'', IETF, RFC
  1958, Jun. 1996
\item
  D. Reed, J. Saltzer and D. Clark: ``Active networking and end-to-end
  arguments'', \emph{IEEE Net. Mag.}, 12(3):69-71, May/Jun. 1998
\item
  D. Reed: "The end of the end-to-end argument",
  \url{http://www.reed.com/Papers/endofendtoend.html}, 2000
\item
  D. Clark and M. S. Blumenthal: ``Rethinking the design of the
  Internet: The end to end arguments vs. the brave new world''; Workshop
  on The Policy Implications of
  End-to-End, Dec. 2000
\item
  S. Lin and D. Costello, Jr.: \emph{Error Control Coding: Fundamentals
  and Applications}, Prentice-Hall, 1983
\item
  J. Stone and C. Partridge: ``When the CRC and TCP checksum disagree'',
  \emph{Proc. SIGCOMM}, pp. 309-19, 2000
\item
  D. Clark: ``The design philosophy of the DARPA Internet protocols'',
  \emph{Proc. SIGCOMM}, pp. 106-14, Aug. 1988
\item
  V. Jacobson: ``Compressing TCP/IP headers for lowspeed serial links'',
  IETF, RFC 1144, Feb. 1990
\item
  T. Moors: ``Protocol Organs: Modularity should reflect function, not
  timing'', \emph{Proc. OPENARCH}, pp. 91-100, Apr. 1998
\item
  B. Lampson: ``Hints for computer system design'', \emph{ACM Operating
  Systems Review}, 15(5):33-48, Oct. 1983
\item
  D. Pradhan: \emph{Fault-Tolerant Computer System Design}; Prentice
  Hall, 1996
\item
  D. Reed: email to the end-to-end mailing list, May 23, 2001
\item
  D. Isenberg: ``The rise of the stupid network''; \emph{Computer
  Telephony}, Aug. 1997, pp. 16-26
\item
  R. Lucky: ``When is dumb smart?'', \emph{IEEE Spectrum}, 34(11):21,
  Nov. 1997
\item
  S. Liebowitz and S. Margolis: ``Network externality'', \emph{The New
  Palgraves Dictionary of Economics and the Law}, MacMillan, 1998
\item
  N. Negroponte: ``The future of phone companies'', \emph{Wired}, 4(9),
  Sep. 1996
\item
  T. Grandison and M. Sloman: ``A survey of trust in Internet
  applications'', \emph{IEEE Comm. Surveys}, Q4, 2000
\item
  M. Anderson, C. North, J. Griffin, R. Milner, J. Yesberg, K. Yiu:
  ``Starlight: Interactive link''; \emph{Proc. 12th Annual Comp.
  Security Applications Conf.}, pp. 55-63; Dec. 1996
\item
  J. H. Saltzer, D. P. Reed, D. D. Clark: ``Source routing for
  campus-wide Internet transport'', \emph{Proc. IFIP WG 6.4 Int'l
  Workshop on Local Networks}, pp. 1-23, Aug. 1980
\item
  V. Jacobson: ``Congestion avoidance and control''; \emph{Proc.
  SIGCOMM}; pp. 314-29, Aug. 1988
\item
  C. Partridge, W. Strayer, B. Schwartz, and A. Jackson, ``Commentaries
  on `Active Networking and End-to-End Arguments''', \emph{IEEE
  Network}, 12(3), May/Jun. 1998
\item
  S. Bhattacharjee, K. Calvert and E. Zegura: ``Active networking and
  the end-to-end argument'', \emph{Proc. Int'l Conf. on Network
  Protocols,} pp. 220-8, Oct. 1997
\end{enumerate}

\end{document}
